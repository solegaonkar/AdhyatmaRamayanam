

\chapt{सुन्दरकाण्डः}


\sect{प्रथमः सर्गः}

\textbf{श्रीमहादेव उवाच}

\twolineshloka
{शतयोजनविस्तीर्णं समुद्रं मकरालयम्}
{लिलङ्घयिषुरानन्दसन्दोहो मारुतात्मजः} %1-1

\twolineshloka
{ध्यात्वा रामं परात्मानमिदं वचनमब्रवीत्}
{पश्यन्तु वानराः सर्वे गच्छन्तं मां विहायसा} %1-2

\twolineshloka
{अमोघं रामनिर्मुक्तं महाबाणमिवाखिलाः}
{पश्याम्यद्यैव रामस्य पत्नीं जनकनन्दिनीम्} %1-3

\twolineshloka
{कृतार्थोऽहं कृतार्थोऽहं पुनः पश्यामि राघवम्}
{प्राणप्रयाणसमये यस्य नाम सकृत्स्मरन्} %1-4

\twolineshloka
{नरस्तीर्त्वा भवाम्भोधिमपारं याति तत्पदम्}
{किं पुनस्तस्य दूतोऽहं तदङ्गाङ्गुलिमुद्रिकः} %1-5

\twolineshloka
{तमेव हृदये ध्यात्वा लङ्घयाम्यल्पवारिधिम्}
{इत्युक्त्वा हनुमान् बाहू प्रसार्यायतवालधिः} %1-6

\twolineshloka
{ऋजुग्रीवोर्ध्वदृष्टिः सन्नाकुञ्चितपदद्वयः}
{दक्षिणाभिमुखस्तूर्णं पुप्लुवेऽनिलविक्रमः} %1-7

\twolineshloka
{आकशात्त्वरितं देवैर्वीक्ष्यमाणो जगाम सः}
{दृष्ट्वानिलसुतं देवा गच्छन्तं वायुवेगतः} %1-8

\twolineshloka
{परीक्षणार्थं सत्त्वस्य वानरस्येदमब्रुवन्}
{गच्छत्येष महासत्त्वो वानरो वायुविक्रमः} %1-9

\twolineshloka
{लङ्कां प्रवेष्टुं शक्तो वा न वा जानीमहे बलम्}
{एवं विचार्य नागानां मातरं सुरसाभिधाम्} %1-10

\twolineshloka
{अब्रवीद्देवतावृन्दः कौतूहलसमन्वितः}
{गच्छ त्वं वानरेन्द्रस्य किञ्चिद्विघ्नं समाचर} %1-11

\twolineshloka
{ज्ञात्वा तस्य बलं बुद्धिं पुनरेहि त्वरान्विता}
{इत्युक्ता सा ययौ शिघ्रं हनुमद्विघ्नकारणात्} %1-12

\twolineshloka
{आवृत्य मार्गं पुरतः स्थित्वा वानरमब्रवीत्}
{एहि मे वदनं शीघ्रं प्रविशस्व महामते} %1-13

\twolineshloka
{देवैस्त्वं कल्पितो भक्ष्यः क्षुधासम्पीडितात्मनः}
{तामाह हनुमान् मातरहं रामस्य शासनात्} %1-14

\twolineshloka
{गच्छामि जानकीं द्रष्टुं पुनरागम्य सत्वरः}
{रामाय कुशलं तस्याः कथयित्वा त्वदाननम्} %1-15

\twolineshloka
{निवेक्ष्ये देहि मे मार्गं सुरसायै नमोऽस्तु ते}
{इत्युक्ता पुनरेवाह सुरसा क्षुधितास्म्यहम्} %1-16

\twolineshloka
{प्रविश्य गच्छ मे वक्त्रं नो चेत्त्वां भक्षयाम्यहम्}
{इत्युक्तो हनुमानाह मुखं शीघ्रं विदारय} %1-17

\twolineshloka
{प्रविश्य वदनं तेऽद्य गच्छामि त्वरयान्वितः}
{इत्युक्त्वा योजनायामदेहो भूत्वा पुरः स्थितः} %1-18

\twolineshloka
{दृष्ट्वा हनूमतो रूपं सुरसा पञ्चयोजनम्}
{मुखं चकार हनुमान् द्विगुणं रूपमादधत्} %1-19

\twolineshloka
{ततश्चकार सुरसा योजनानां च विंशतिम्}
{वक्त्रं चकार हनुमांस्त्रिंशद्योजनसम्मितम्} %1-20

\twolineshloka
{ततश्चकार सुरसा पञ्चाशद्योजनायतम्}
{वक्त्रं तदा हनूमांस्तु बभूवाङ्गुष्ठसन्निभः} %1-21

\twolineshloka
{प्रविश्य वदनं तस्याः पुनरेत्य पुरः स्थितः}
{प्रविष्टो निर्गतोऽहं ते वदनं देवि ते नमः} %1-22

\twolineshloka
{एवं वदन्तं दृष्ट्वा सा हनूमन्तमथाब्रवीत्}
{गच्छ साधय रामस्य कार्यं बुद्धिमतां वर} %1-23

\twolineshloka
{देवैः सम्प्रेषिताहं ते बलं जिज्ञासुभिः कपे}
{दृष्ट्वा सीतां पुनर्गत्वा रामं द्रक्ष्यसि गच्छ भोः} %1-24

\twolineshloka
{इत्युक्त्वा सा ययौ देवलोकं वायुसुतः पुनः}
{जगाम वायुमार्गेण गरुत्मानिव पक्षिराट्} %1-25

\twolineshloka
{समुद्रोऽप्याह मैनाकं मणिकाञ्चनपर्वतम्}
{गच्छत्येष महासत्त्वो हनूमान्मारुतात्मजः} %1-26

\twolineshloka
{रामस्य कार्यसिद्ध्यर्थं तस्य त्वं सचिवो भव}
{सगरैर्वर्द्धितो यस्मात्पुराहं सागरोऽभवम्} %1-27

\twolineshloka
{तस्यान्वये बभूवासौ रामो दाशरथिः प्रभुः}
{तस्य कार्यार्थसिद्ध्यर्थं गच्छत्येष महाकपिः} %1-28

\twolineshloka
{त्वमुत्तिष्ठ जलात्तूर्णं त्वयि विश्रम्य गच्छतु}
{स तथेति प्रादुरभूज्जलमध्यान्महोन्नतः} %1-29

\twolineshloka
{नानामणिमयैः शृङ्गैस्तस्योपरि नराकृतिः}
{प्राह यान्तं हनूमन्तं मैनाकोऽहं महाकपे} %1-30

\twolineshloka
{समुद्रेण समादिष्टस्त्वद्विश्रामाय मारुते}
{आगच्छामृतकल्पानि जग्ध्वा पक्वफलानि मे} %1-31

\twolineshloka
{विश्रम्यात्र क्षणं पश्चाद्गमिष्यसि यथासुखम्}
{एवमुक्तोऽथ तं प्राह हनूमान्मारुतात्मजः} %1-32

\twolineshloka
{गच्छतो रामकार्यार्थं भक्षणं मे कथं भवेत्}
{विश्रामो वा कथं मे स्याद्गन्तव्यं त्वरितं मया} %1-33

\twolineshloka
{इत्युक्त्वा स्पृष्टशिखरः कराग्रेण ययौ कपिः}
{किञ्चिद्दूरं गतस्यास्य छायां छायाग्रहोऽग्रहीत्} %1-34

\twolineshloka
{सिंहिका नाम सा घोरा जलमध्ये स्थिता सदा}
{आकाशगामिनां छायामाक्रम्याकृष्य भक्षयेत्} %1-35

\twolineshloka
{तया गृहीतो हनुमांश्चिन्तयामास वीर्यवान्}
{केनेदं मे कृतं वेगरोधनं विघ्नकारिणा} %1-36

\twolineshloka
{दृश्यते नैव कोऽप्यत्र विस्मयो मे प्रजायते}
{एवं विचिन्त्य हनुमानधो दृष्टिं प्रसारयत्} %1-37

\twolineshloka
{तत्र दृष्ट्वा महाकायां सिंहिकां घोररूपिणीम्}
{पपात सलिले तूर्णं पद्भ्यामेवाहनद्रुषा} %1-38

\twolineshloka
{पुनरुत्प्लुत्य हनूमान् दक्षिणाभिमुखो ययौ}
{ततो दक्षिणमासाद्य कूलं नानाफलद्रुमम्} %1-39

\twolineshloka
{नानापक्षिमृगाकीर्णं नानापुष्पलतावृतम्}
{ततो ददर्श नगरं त्रिकूटाचलमूर्धनि} %1-40

\twolineshloka
{प्राकारैर्बहुभिर्युक्तं परिखाभिश्च सर्वतः}
{प्रवेक्ष्यामि कथं लङ्कामिति चिन्तापरोऽभवत्} %1-41

\twolineshloka
{रात्रौ वेक्ष्यामि सूक्ष्मोऽहं लङ्कां रावणपालिताम्}
{एवं विचिन्त्य तत्रैव स्थित्वा लङ्कां जगाम सः} %1-42

\twolineshloka
{धृत्वा सूक्ष्मं वपुर्द्वारं प्रविवेश प्रतापवान्}
{तत्र लङ्कापुरी साक्षाद्राक्षसीवेषधारिणी} %1-43

\twolineshloka
{प्रविशन्तं हनूमन्तं दृष्ट्वा लङ्का व्यतर्जयत्}
{कस्त्वं वानररूपेण मामनादृत्य लङ्किनीम्} %1-44

\twolineshloka
{प्रविश्य चोरवद्रात्रौ किं भवान् कर्तुमिच्छति}
{इत्युक्त्वा रोषताम्राक्षी पादेनाभिजघान तम्} %1-45

\twolineshloka
{हनुमानपि तां वाममुष्टिनावज्ञयाहनत्}
{तदैव पतिता भूमौ रक्तमुद्वमती भृशम्} %1-46

\twolineshloka
{उत्थाय प्राह सा लङ्का हनूमन्तं महाबलम्}
{हनूमन् गच्छ भद्रं ते जिता लङ्का त्वयानघ} %1-47

\twolineshloka
{पुराहं ब्रह्मणा प्रोक्ता ह्यष्टाविंशतिपर्यये}
{त्रेतायुगे दाशरथी रामो नारायणोऽव्ययः} %1-48

\twolineshloka
{जनिष्यते योगमाया सीता जनकवेश्मनि}
{भूभारहरणार्थाय प्रार्थितोऽयं मया क्वचित्} %1-49

\twolineshloka
{सभार्यो राघवो भ्रात्रा गमिष्यति महावनम्}
{तत्र सीतां महामायां रावणोऽपहरिष्यति} %1-50

\twolineshloka
{पश्चादरामेण साचिव्यं सुग्रीवस्य भविष्यति}
{सुग्रीवो जानकीं द्रष्टुं वानरान् प्रेषयिष्यति} %1-51

\twolineshloka
{तत्रैको वानरो रात्रावागमिष्यति तेऽन्तिकम्}
{त्वया च भर्त्सितः सोऽपि त्वां हनिष्यति मुष्टिना} %1-52

\twolineshloka
{तेनाहता त्वं व्यथिता भविष्यसि यदानघे}
{तदैव रावणस्यान्तो भविष्यति न संशयः} %1-53

\twolineshloka
{तस्मात् त्वया जिता लङ्का जितं सर्वं त्वयानघ}
{रावणान्तःपुरवरे क्रीडाकाननमुत्तमम्} %1-54

\twolineshloka
{तन्मध्येऽशोकवनिका दिव्यपादपसङ्कुला}
{अस्ति तस्यां महावृक्षः शिंशपा नाम मध्यगः} %1-55

\twolineshloka
{तत्रास्ते जानकी घोरराक्षसीभिः सुरक्षिता}
{दृष्ट्वैव गच्छ त्वरितं राघवाय निवेदय} %1-56

\fourlineindentedshloka
{धन्याहमप्यद्य चिराय राघव\-}
{स्मृतिर्ममासीद्भवपाशमोचिनी}
{तद्भक्तसङ्गोऽप्यतिदुर्लभो मम}
{प्रसीदतां दाशरथिः सदा हृदि} %1-57

\fourlineindentedshloka
{उल्लङ्घितेऽब्धौ पवनात्मजेन}
{धरासुतायाश्च दशाननस्य}
{पुस्फोर वामाक्षि भुजश्च तीव्रम्}
{रामस्य दक्षाङ्गमतीन्द्रियस्य} %1-58

{॥इति श्रीमदध्यात्मरामायणे उमामहेश्वरसंवादे सुन्दरकाण्डे
प्रथमः सर्गः ॥ १॥
}
%%%%%%%%%%%%%%%%%%%%



\sect{द्वितीयः सर्गः}

\textbf{श्रीमहादेव उवाच}

\twolineshloka
{ततो जगाम हनुमान् लङ्कां परमशोभनाम्}
{रात्रौ सूक्ष्मतनुर्भूत्वा बभ्राम परितः पुरीम्} %2-1

\twolineshloka
{सीतान्वेषणकार्यार्थी प्रविवेश नृपालयम्}
{तत्र सर्वप्रदेशेषु विविच्य हनुमान् कपिः} %2-2

\twolineshloka
{नापश्यज्जानकीं स्मृत्वा ततो लङ्काभिभाषितम्}
{जगाम हनुमान् शीघ्रमशोकवनिकां शुभाम्} %2-3

\twolineshloka
{सुरपादपसम्बाधां रत्नसोपानवापिकाम्}
{नानापक्षिमृगाकीर्णां स्वर्णप्रासादशोभिताम्} %2-4

\twolineshloka
{फलैरानम्रशाखाग्रपादपैः परिवारिताम्}
{विचिन्वन् जानकीं तत्र प्रतिवृक्षं मरुत्सुतः} %2-5

\twolineshloka
{ददर्शाभ्रंलिहं तत्र चैत्यप्रासादमुत्तमम्}
{दृष्ट्वा विस्मयमापन्नो मणिस्तम्भशतान्वितम्} %2-6

\twolineshloka
{समतीत्य पुनर्गत्वा किञ्चिद्दूरं स मारुतिः}
{ददर्श शिंशपावृक्षमत्यन्तनिबिडच्छदम्} %2-7

\twolineshloka
{अदृष्टातपमाकीर्णं स्वर्णवर्णविहङ्गमम्}
{तन्मूले राक्षसीमध्ये स्थितां जनकनन्दिनीम्} %2-8

\twolineshloka
{ददर्श हनुमान् वीरो देवतामिव भूतले}
{एकवेणीं कृशां दीनां मलिनाम्बरधारिणीम्} %2-9

\twolineshloka
{भुमौ शयानां शोचन्तीं रामरामेति भाषिणीम्}
{त्रातारं नाधिगच्छन्तीमुपवासकृशां शुभाम्} %2-10

\twolineshloka
{शाखान्तच्छदमध्यस्थो ददर्श कपिकुञ्जरः}
{कृतार्थोऽहं कृतार्थोऽहं दृष्ट्वा जनकनन्दिनीम्} %2-11

\twolineshloka
{मयैव साधितं कार्यं रामस्य परमात्मनः}
{ततः किलकिलाशब्दो बभूवान्तःपुराद्बहिः} %2-12

\twolineshloka
{किमेतदिति स।ण्ल्लीनो वृक्षपत्रेषु मारुतिः}
{आयान्तं रावणं तत्र स्त्रीजनैः परिवारितम्} %2-13

\twolineshloka
{दशास्यं विंशतिभुजं नीलाञ्जनचयोपमम्}
{दृष्ट्वा विस्मयमापन्नः पत्रखण्डेष्वलीयत} %2-14

\twolineshloka
{रावणो राघवेणाशु मरणं मे कथं भवेत्}
{सीतार्थमपि नायाति रामः किं कारणं भवेत्} %2-15

\twolineshloka
{इत्येवं चिन्तयन्नित्यं राममेव सदा हृदि}
{तस्मिन् दिनेऽपररात्रौ रावणो राक्षसाधिपः} %2-16

\twolineshloka
{स्वप्ने रामेण सन्दिष्टः कश्चिदागत्य वानरः}
{कामरूपधरः सूक्ष्मो वृक्षग्रस्थोऽनुपश्यति} %2-17

\twolineshloka
{इति दृष्ट्वाद्भुतं स्वप्नं स्वात्मन्येवानुचिन्त्य सः}
{स्वप्नः कदाचित्सत्यः स्यादेवं तत्र करोम्यहम्} %2-18

\twolineshloka
{जानकीं वाकःशरैर्विद्ध्वा दुःखितां नितरामहम्}
{करोमि दृष्ट्वा रामाय निवेदयतु वानरः} %2-19

\twolineshloka
{इत्येवं चिन्तयन् सीतासमीपमगमद्द्रुतम्}
{नूपुराणां किङ्किणीनां श्रुत्वा शिञ्जितमङ्गना} %2-20

\twolineshloka
{सीता भीता लीयमाना स्वात्मन्येव सुमध्यमा}
{अधोमुख्यश्रुनयना स्थिता रामार्पितान्तरा} %2-21

\twolineshloka
{रावणोऽपि तदा सीतामालोक्याह सुमध्यमे}
{मां दृष्ट्वा किं वृथा सुभ्रु स्वात्मन्येव विलीयसे} %2-22

\twolineshloka
{रामो वनचराणां हि मध्ये तिष्ठति सानुजः}
{कदाचिद्दृश्यते कैश्चित्कदाचिन्नैव दृश्यते} %2-23

\twolineshloka
{मया तु बहुधा लोकाः प्रेषितास्तस्य दर्शने}
{न पश्यन्ति प्रयत्नेन वीक्षमाणाः समन्ततः} %2-24

\twolineshloka
{किं करिष्यसि रामेण निःस्पृहेण सदा त्वयि}
{त्वया सदालिङ्गितोऽपि समीपस्थोऽपि सर्वदा} %2-25

\twolineshloka
{हृदयेऽस्य न च स्नेहस्त्वयि रामस्य जायते}
{त्वत्कृतान् सर्वभोगांश्च त्वद्गुणानपि राघवः} %2-26

\twolineshloka
{भुञ्जानोऽपि न जानाति कृतघ्नो निर्गुणोऽधमः}
{त्वमानीता मया साध्वी दुःखशोकसमाकुला} %2-27

\twolineshloka
{इदानीमपि नायाति भक्तिहीनः कथं व्रजेत्}
{निःसत्त्वो निर्ममो मानी मूढः पण्डितमानवान्} %2-28

\twolineshloka
{नराधमं त्वद्विमुखं किं करिष्यसि भामिनि}
{त्वय्यतीव समासक्तं मां भजस्वासुरोत्तमम्} %2-29

\twolineshloka
{देवगन्धर्वनागानां यक्षकिन्नरयोषिताम्}
{भविष्यसि नियोक्त्री त्वं यदि मां प्रतिपद्यसे} %2-30

\twolineshloka
{रावणस्य वचः श्रुत्वा सीताऽमर्षसमन्विता}
{उवाचाधोमुखी भूत्वा निधाय तृणमन्तरे} %2-31

\twolineshloka
{राघवाद्बिभ्यता नूनं भिक्षुरूपं त्वया धृतम्}
{रहिते राघवाभ्यां त्वं शुनीव हविरध्वरे} %2-32

\twolineshloka
{हृतवानसि मां नीच तत्फलं प्राप्स्यसेऽचिरात्}
{यदा रामशराघातविदारितवपुर्भवान्} %2-33

\twolineshloka
{ज्ञास्यसेऽमानुषं रामं गमिष्यसि यमान्तिकम्}
{समुद्रं शोषयित्वा वा शरैर्बद्ध्वाथ वारिधिम्} %2-34

\twolineshloka
{हन्तुं त्वां समरे रामो लक्ष्मणेन समन्वितः}
{आगमिष्यत्यसन्देहो द्रक्ष्यसे राक्षसाधम} %2-35

\twolineshloka
{त्वां सपुत्रं सहबलं हत्वा नेष्यति मां पुरम्}
{श्रुत्वा रक्षःपतिः क्रुद्धो जानक्याः परुषाक्षरम्} %2-36

\twolineshloka
{वाक्यं क्रोधसमाविष्टः खड्गमुद्यम्य सत्वरः}
{हन्तुं जनकराजस्य तनयां ताम्रलोचनः} %2-37

\twolineshloka
{मन्दोदरी निवार्याह पतिं पतिहिते रता}
{त्यजैनां मानुषीं दीनां दुःखितां कृपणां कृशाम्} %2-38

\twolineshloka
{देवगन्धर्वनागानां बह्व्यः सन्ति वराङ्गनाः}
{त्वामेव वरयन्त्युच्चैर्मदमत्तविलोचनाः} %2-39

\threelineshloka
{ततोऽब्रवीद्दशग्रीवो राक्षसीर्विकृताननाः}
{यथा मे वशगा सीता भविष्यति सकामना}
{तथा यतध्वं त्वरितं तर्जनादरणादिभिः} %2-40

\twolineshloka
{द्विमासाभ्यन्तरे सीता यदि मे वशगा भवेत्}
{तदा सर्वसुखोपेता राज्यं भोक्ष्यति सा मया} %2-41

\twolineshloka
{यदि मासद्वयादूर्ध्वं मच्छय्यां नाभिनन्दति}
{तदा मे प्रातराशाय हत्वा कुरुत मानुषीम्} %2-42

\twolineshloka
{इत्युक्त्वा प्रययौ स्त्रीभी रावणोऽन्तःपुरालयम्}
{राक्षस्यो जानकीमेत्य भीषयन्त्यः स्वतर्जनैः} %2-43

\twolineshloka
{तत्रैका जानकीमाह यौवनं ते वृथा गतम्}
{रावणेन समासाद्य सफलं तु भविष्यति} %2-44

\twolineshloka
{अपरा चाह कोपेन किं विलम्बेन जानकि}
{इदानीं छेद्यतामङ्गं विभज्य च पृथक् पृथक्} %2-45

\twolineshloka
{अन्या तु खड्गमुद्यम्य जानकीं हन्तुमुद्यता}
{अन्या करालवदना विदार्यास्यमभीषयत्} %2-46

\twolineshloka
{एवं तां भीषयन्तीस्ता राक्षसीर्विकृताननाः}
{निवार्य त्रिजटा वृद्धा राक्षसी वाक्यमब्रवीत्} %2-47

{शृणुध्वं दुष्टराक्षस्यो मद्वाक्यं वो हितं भवेत्॥४८॥} %2-48
\refstepcounter{shlokacount}


\twolineshloka
{न भीषयध्वं रुदतीं नमस्कुरुत जानकीम्}
{इदानीमेव मे स्वप्ने रामः कमललोचनः} %2-49

\twolineshloka
{आरुह्यैरावतं शुभ्रं लक्ष्मणेन समागतः}
{दग्ध्वा लङ्कापुरीं सर्वां हत्वा रावणमाहवे} %2-50

\twolineshloka
{आरोप्य जानकीं स्वाङ्के स्थितो दृष्टोऽगमूर्धनि}
{रावणो गोमयह्रदे तैलाभ्यक्तो दिगम्बरः} %2-51

\twolineshloka
{अगाहत्पुत्रपौत्रैश्च कृत्वा वदनमालिकाम्}
{विभीषणस्तु रामस्य सन्निधौ हृष्टमानसः} %2-52

\twolineshloka
{सेवां करोति रामस्य पादयोर्भक्तिसंयुतः}
{सर्वथा रावणं रामो हत्वा सकुलमञ्जसा} %2-53

\twolineshloka
{विभीषणायाधिपत्यं दत्त्वा सीतां शुभाननाम्}
{अङ्के निधाय स्वपुरीं गमिष्यति न संशयः} %2-54

\twolineshloka
{त्रिजटाया वचः श्रुत्वा भीतास्ता राक्षसस्त्रियः}
{तूष्णीमासंस्तत्र तत्र निद्रावशमुपागताः} %2-55

\twolineshloka
{तर्जिता राक्षसीभिः सा सीता भीतातिविह्वला}
{त्रातारं नाधिगच्छन्ती दुःखेन परिमूर्च्छिता} %2-56

\threelineshloka
{अश्रुभिः पूर्णनयना चिन्तयन्तीदमब्रवीत्}
{प्रभाते भक्षयिष्यन्ति राक्षस्यो मां न संशयः}
{इदानीमेव मरणं केनोपायेन मे भवेत्} %2-57

\fourlineindentedshloka
{एवं सुदुःखेन परिप्लुता सा}
{विमुक्तकण्ठं रुदती चिराय}
{आलम्ब्य शाखां कृतनिश्चया मृतौ}
{न जानती कश्चिदुपायमङ्गना} %2-58

{॥इति श्रीमदध्यात्मरामायणे उमामहेश्वरसंवादे सुन्दरकाण्डे
द्वितीयः सर्गः ॥ २॥
}
%%%%%%%%%%%%%%%%%%%%



\sect{तृतीयः सर्गः}

\textbf{श्रीमहादेव उवाच}

\twolineshloka
{उद्बन्धनेन वा मोक्ष्ये शरीरं राघवं विना}
{जीवितेन फलं किं स्यान्मम रक्षोऽधिमध्यतः} %3-1

\twolineshloka
{दीर्घा वेणी ममात्यर्थमुद्बन्धाय भविष्यति}
{एवं निश्चितबुद्धिं तां मरणायाथ जानकीम्} %3-2

\twolineshloka
{विलोक्य हनुमान् किञ्चिद्विचार्यैतदभाषत}
{शनैः शनैः सूक्ष्मरूपो जानक्याः श्रोत्रगं वचः} %3-3

\twolineshloka
{इक्ष्वाकुवंशसम्भूतो राजा दशरथो महान्}
{अयोध्याधिपतिस्तस्य चत्वारो लोकविश्रुताः} %3-4

\twolineshloka
{पुत्रा देवसमाः सर्वे लक्षणैरुपलक्षिताः}
{रामश्चलक्ष्मणश्चैव भरतश्चैव शत्रुहा} %3-5

\twolineshloka
{ज्येष्ठो रामः पितुर्वाक्याद्दण्डकारण्यमागतः}
{लक्ष्मणेन सह भ्रात्रा सीतया भार्यया सह} %3-6

\twolineshloka
{उवास गौतमीतीरे पञ्चवट्यां महामनाः}
{तत्र नीता महाभागा सीता जनकनन्दिनी} %3-7

\twolineshloka
{रहिते रामचन्द्रेण रावणेन दुरात्मना}
{ततो रामोऽतिदुःखार्तो मार्गमाणोऽथ जानकीम्} %3-8

\twolineshloka
{जटायुषं पक्षिराजमपश्यत्पतितं भुवि}
{तस्मै दत्त्वा दिवं शीघ्रमृष्यमूकमुपागमत्} %3-9

\twolineshloka
{सुग्रीवेण कृता मैत्री रामस्य विदितात्मनः}
{तद्भार्याहारिणं हत्वा वालिनं रघुनन्दनः} %3-10

\twolineshloka
{राज्येऽभिषिच्य सुग्रीवं मित्रकार्यं चकार सः}
{सुग्रीवस्तु समानाय्य वानरान् वानरप्रभुः} %3-11

\twolineshloka
{प्रेषयामास परितो वानरान् परिमार्गणे}
{सीतायास्तत्र चैकोऽहं सुग्रीवसचिवो हरिः} %3-12

\twolineshloka
{सम्पातिवचनाच्छीघ्रमुल्लङ्घ्य शतयोजनम्}
{समुद्रं नगरीं लङ्कां विचिन्वन् जानकीं शुभाम्} %3-13

\twolineshloka
{शनैरशोकवनिकां विचिन्वन् शिंशपातरुम्}
{अद्राक्षं जानकीमत्र शोचन्तीं दुःखसम्प्लुताम्} %3-14

\twolineshloka
{रामस्य महिषीं देवीं कृतकृत्योऽहमागतः}
{इत्युक्त्वोपररामाथ मारुतिर्बुद्धिमत्तरः} %3-15

\twolineshloka
{सीता क्रमेण तत्सर्वं श्रुत्वा विस्मयमाययौ}
{किमिदं मे श्रुतं व्योम्नि वायुना समुदीरितम्} %3-16

\twolineshloka
{स्वप्नो वा मे मनोभ्रान्तिर्यदि वा सत्यमेव तत्}
{निद्रा मे नास्ति दुःखेन जानाम्येतत्कुतो भ्रमः} %3-17

\twolineshloka
{येन मे कर्णपीयुषं वचनं समुदीरितम्}
{स दृश्यतां महाभागः प्रियवादी ममाग्रतः} %3-18

\twolineshloka
{श्रुत्वा तज्जानकीवाक्यं हनुमान् पत्रखण्डतः}
{अवतीर्य शनैः सीतापुरतः समवस्थितः} %3-19

\twolineshloka
{कलविङ्कप्रमाणाङ्गो रक्तास्यः पीतवानरः}
{ननाम शनकैः सीतां प्राञ्जलिः पुरतः स्थितः} %3-20

\twolineshloka
{दृष्ट्वा तं जानकी भीता रावणोऽयमुपागतः}
{मां मोहयितुमायातो मायया वानराकृतिः} %3-21

\twolineshloka
{इत्येवं चिन्तयित्वा सा तूष्णिमासीदधोमुखी}
{पुनरप्याह तां सीतां देवि यत्त्वं विशङ्कसे} %3-22

\twolineshloka
{नाहं तथाविधो मातस्त्यज शङ्कां मयि स्थिताम्}
{दासोऽहं कोसलेन्द्रस्य रामस्य परमात्मनः} %3-23

\twolineshloka
{सचिवोऽहं हरीन्द्रस्य सुग्रीवस्य शुभप्रदे}
{वायोः पुत्रोऽहमखिलप्राणभूतस्य शोभने} %3-24

\twolineshloka
{तच्छ्रुत्वा जानकी प्राह हनूमन्तं कृताञ्जलिम्}
{वानराणां मनुष्याणां सङ्गतिर्घटते कथम्} %3-25

\twolineshloka
{यथा त्वं रामचन्द्रस्य दासोऽहमिति भाषसे}
{तामाह मारुतिः प्रीतो जानकीं पुरतः स्थितः} %3-26

\twolineshloka
{ऋष्यमूकमगाद्रामः शबर्या नोदितः सुधीः}
{सुग्रीवो ऋष्यमूकस्थो दृष्टवान् रामलक्ष्मणौ} %3-27

\twolineshloka
{भीतो मां प्रेषयामास ज्ञातुं रामस्य हृद्गतम्}
{ब्रह्मचारिवपुर्धृत्वा गतोऽहं रामसन्निधिम्} %3-28

\twolineshloka
{ज्ञात्वा रामस्य सद्भावं स्कन्धोपरि निधाय तौ}
{नीत्वा सुग्रीवसामीप्यं सख्यं चाकरवं तयोः} %3-29

\twolineshloka
{सुग्रीवस्य हृता भार्या वालिना तं रघूत्तमः}
{जघानैकेन बाणेन ततो राज्येऽभ्यषेचयत्} %3-30

\twolineshloka
{सुग्रीवं वानराणां स प्रेषयामास वानरान्}
{दिग्भ्यो महाबलान् वीरान् भवत्याः परिमार्गणे} %3-31

{गच्छन्तं राघवो दृष्ट्वा मामभाषत सादरम्॥३२॥} %3-32
\refstepcounter{shlokacount}


\twolineshloka
{त्वयि कार्यमशेषं मे स्थितं मारुतनन्दन}
{ब्रूहि मे कुशलं सर्वं सीतायै लक्ष्मणस्य च} %3-33

\twolineshloka
{अङ्गुलीयकमेतन्मे परिज्ञानार्थमुत्तमम्}
{सीतायै दीयतां साधु मन्नामाक्षरमुद्रितम्} %3-34

\twolineshloka
{इत्युक्त्वा प्रददौ मह्यं कराग्रादङ्गुलीयकम्}
{प्रयत्नेन मयानीतं देवि पश्याङ्गुलीयकम्} %3-35

\twolineshloka
{इत्युक्त्वा प्रददौ देव्यै मुद्रिकां मारुतात्मजः}
{नमस्कृत्य स्थितो दूराद्बद्धाञ्जलिपुटो हरिः} %3-36

\twolineshloka
{दृष्ट्वा सीता प्रमुदिता रामनामाङ्कितां तदा}
{मुद्रिकां शिरसा धृत्वा स्रवदानन्दनेत्रजा} %3-37

\twolineshloka
{कपे मे प्राणदाता त्वं बुद्धिमानसि राघवे}
{भक्तोऽसि प्रियकारी त्वं विश्वासोऽस्ति तवैव हि} %3-38

\twolineshloka
{नो चेन्मत्सन्निधिं चान्यं पुरुषं प्रेषयेत्कथम्}
{हनूमन् दृष्टमखिलं मम दुःखादिकं त्वया} %3-39

\twolineshloka
{सर्वं कथय रामाय यथा मे जायते दया}
{मासद्वयावधि प्राणाः स्थास्यन्ति मम सत्तम} %3-40

\twolineshloka
{नागमिष्यति चेद्रामो भक्षयिष्यति मां खलः}
{अतः शीघ्रं कपीन्द्रेण सुग्रीवेण समन्वितः} %3-41

\twolineshloka
{वानरानीकपैः सार्धं हत्वा रावणमाहवे}
{सपुत्रं सबलं रामो यदि मां मोचयेत्प्रभुः} %3-42

\twolineshloka
{तत्तस्य सदृशं वीर्यं वीर वर्णय वर्णितम्}
{यथा मां तारयेद्रामो हत्वा शीघ्रं दशाननम्} %3-43

\twolineshloka
{तथा यतस्व हनुमन् वाचा धर्ममवाप्नुहि}
{हनुमानपि तामाह देवि दृष्टो यथा मया} %3-44

\twolineshloka
{रामः सलक्ष्मणः शीघ्रमागमिष्यति सायुधः}
{सुग्रीवेण ससैन्येन हत्वा दशमुखं बलात्} %3-45

\twolineshloka
{समानेष्यति देवि त्वामयोध्यां नात्र संशयः}
{तमाह जानकी रामः कथं वारिधिमाततम्} %3-46

\twolineshloka
{तीर्त्वायास्यत्यमेयात्मा वानरानीकपैः सह}
{हनूमानाह मे स्कन्धावारुह्य पुरुषर्षभौ} %3-47

\twolineshloka
{आयास्यतः ससैन्यश्च सुग्रीवो वानरेश्वरः}
{विहायसा क्षणेनैव तीर्त्वा वारिधिमाततम्} %3-48

\twolineshloka
{निर्दहिष्यति रक्षौघांस्त्वत्कृते नात्र संशयः}
{अनुज्ञां देहि मे देवि गच्छामि त्वरयान्वितः} %3-49

\twolineshloka
{द्रष्टुं रामं सह भ्रात्रा त्वरयामि तवान्तिकम्}
{देवि किञ्चिदभिज्ञानं देहि मे येन राघवः} %3-50

\twolineshloka
{विश्वसेन्मां प्रयत्नेन ततो गन्ता समुत्सुकः}
{ततः किञ्चिद्विचार्याथ सीता कमललोचना} %3-51

\twolineshloka
{विमुच्य केशपाशान्ते स्थितं चूडामणिं ददौ}
{अनेन विश्वसेद्रामस्त्वां कपीन्द्र सलक्ष्मणः} %3-52

\threelineshloka
{अभिज्ञानार्थमन्यच्च वदामि तव सुव्रत}
{चित्रकूटगिरौ पूर्वमेकदा रहसि स्थितः}
{मदङ्के शिर आधाय निद्राति रघुनन्दनः} %3-53

\twolineshloka
{ऐन्द्रः काकस्तदागत्य नखैस्तुण्डेन चासकृत्}
{मत्पादाङ्गुष्ठमारक्तं विददारामिषाशया} %3-54

\twolineshloka
{ततो रामः प्रबुद्ध्याथ दृष्ट्वा पादं कृतव्रणम्}
{केन भद्रे कृतं चैतद्विप्रियं मे दुरात्मना} %3-55

\twolineshloka
{इत्युक्त्वा पुरतोऽपश्यद्वायसं मां पुनः पुनः}
{अभिद्रवन्तं रक्ताक्तनखतुण्डं चुकोप ह} %3-56

\twolineshloka
{तृणमेकमुपादाय दिव्यास्त्रेणाभियोज्य तत्}
{चिक्षेप लीलया रामो वायसोपरि तज्ज्वलन्} %3-57

\twolineshloka
{अभ्यद्रवद्वायसश्च भीतो लोकान् भ्रमन् पुनः}
{इन्द्रब्रह्मादिभिश्चापि न शक्यो रक्षितुं तदा} %3-58

\twolineshloka
{रामस्य पादयोरग्रेऽपतद्भीत्या दयानिधेः}
{शरणागतमालोक्य रामस्तमिदमब्रवीत्} %3-59

\twolineshloka
{अमोघमेतदस्त्रं मे दत्वैकाक्षिमितो व्रज}
{सव्यं दत्त्वा गतः काक एवं पौरुषवानपि} %3-60

\twolineshloka
{उपेक्षते किमर्थं मामिदानीं सोऽपि राघवः}
{हनूमानपि तामाह श्रुत्वा सीतानुभाषितम्} %3-61

\twolineshloka
{देवि त्वां यदि जानाति स्थितामत्र रघूत्तमः}
{करिष्यति क्षणाद्भस्म लङ्कां राक्षसमण्डिताम्} %3-62

\twolineshloka
{जानकी प्राह तं वत्स कथं त्वं योत्स्यसेऽसुरैः}
{अतिसूक्ष्मवपुः सर्वे वानराश्च भवादृशाः} %3-63

\twolineshloka
{श्रुत्वा तद्वचनं देव्यै पूर्वरूपमदर्शयत्}
{मेरुमन्दरसङ्काशं रक्षोगणविभीषणम्} %3-64

\twolineshloka
{दृष्ट्वा सीता हनुमन्तं महापर्वतसन्निभम्}
{हर्षेण महताविष्टा प्राह तं कपिकुञ्जरम्} %3-65

\twolineshloka
{समर्थोऽसि महासत्त्व द्रक्ष्यन्ति त्वां महाबलम्}
{राक्षस्यस्ते शुभः पन्था गच्छ रामान्तिकं द्रुतम्} %3-66

\twolineshloka
{बुभुक्षितः कपिः प्राह दर्शनात्पारणं मम}
{भविष्यति फलैः सर्वैस्तव दृष्टौ स्थितैर्हि मे} %3-67

\threelineshloka
{तथेत्युक्तः स जानक्या भक्षयित्वा फलं कपिः}
{ततः प्रस्थापितोऽगच्छज्जानकीं प्रणिपत्य सः}
{किञ्चिद्दूरमथो गत्वा स्वात्मन्येवान्वचिन्तयत्} %3-68

\twolineshloka
{कार्यार्थमागतो दूतः स्वामिकार्याविरोधतः}
{अन्यत्किञ्चिदसम्पाद्य गच्छत्यधम एव सः} %3-69

\twolineshloka
{अतोऽहं किञ्चिदन्यच्च कृत्वा दृष्ट्वाथ रावणम्}
{सम्भाष्य च ततो रामदर्शनार्थं व्रजाम्यहम्} %3-70

\twolineshloka
{इति निश्चित्य मनसा वृक्षखण्डान् महाबलः}
{उत्पाट्याशोकवनिकां निर्वृक्षामकरोत्क्षणात्} %3-71

\twolineshloka
{सीताऽऽश्रयनगं त्यक्त्वा वनं शून्यं चकार सः}
{उत्पाटयन्तं विपिनं दृष्ट्वा राक्षसयोषितः} %3-72

{अपृच्छन् जानकीं कोऽसौ॥७३॥} %3-73
\refstepcounter{shlokacount}


\textbf{जानक्युवाच}

\twolineshloka
{भवत्य एव जानन्ति मायां राक्षसनिर्मिताम्}
{नाहमेनं विजानामि दुःखशोकसमाकुला} %3-74

\twolineshloka
{इत्युक्तास्त्वरितं गत्वा राक्षस्यो भयपीडिताः}
{हनूमता कृतं सर्वं रावणाय न्यवेदयन्} %3-75

\threelineshloka
{देव कश्चिन्महासत्त्वो वानराकृतिदेहभृत्}
{सीतया सह सम्भाष्य ह्यशोकवनिकां क्षणात्}
{उत्पाट्य चैत्यप्रासादं बभञ्जामितविक्रमः} %3-76

\twolineshloka
{प्रासादरक्षिणः सर्वान् हत्वा तत्रैव तस्थिवान्}
{तच्छ्रुत्वा तूर्णमुत्थाय वनभङ्गं महाऽप्रियम्} %3-77

\twolineshloka
{किङ्करान् प्रेषयामास नियुतं राक्षसाधिपः}
{निभग्नचैत्यप्रासादप्रथमान्तरसंस्थितः} %3-78

\twolineshloka
{हनुमान् पर्वताकारो लोहस्तम्भकृतायुधः}
{किञ्चिल्लाङ्गूलचलनो रक्तास्यो भीषणाकृतिः} %3-79

\twolineshloka
{आपतन्तं महासङ्घं राक्षसानां ददर्श सः}
{चकार सिंहनादं च श्रुत्वा ते मुमुहुर्भृशम्} %3-80

\twolineshloka
{हनुमन्तमथो दृष्ट्वा राक्षसा भीषणाकृतिम्}
{निर्जघ्नुर्विविधास्त्रौघैः सर्वराक्षसघातिनम्} %3-81

\twolineshloka
{तत उत्थाय हनुमान् मुद्गरेण समन्ततः}
{निष्पिपेष क्षणादेव मशकानिव यूथपः} %3-82

\twolineshloka
{निहतान् किङ्करान् श्रुत्वा रावणः क्रोधमूर्च्छितः}
{पञ्च सेनापतींस्तत्र प्रेषयामास दुर्मदान्} %3-83

\twolineshloka
{हनूमानपि तान् सर्वांल्लोहस्तम्भेन चाहनत्}
{ततः क्रुद्धो मन्त्रिसुतान् प्रेषयामास सप्त सः} %3-84

\twolineshloka
{आगतानपि तान् सर्वान् पूर्ववद्वानरेश्वरः}
{क्षणान्निःशेषतो हत्वा लोहस्तम्भेन मारुतिः} %3-85

\twolineshloka
{पूर्वस्थानमुपाश्रित्य प्रतीक्षन् राक्षसान् स्थितः}
{ततो जगाम बलवान् कुमारोऽक्षः प्रतापवान्} %3-86

\twolineshloka
{तमुत्पपात हनुमान् दृष्ट्वाकाशे समुद्गरः}
{गगनात्त्वरितो मूर्ध्नि मुद्गरेण व्यताडयत्} %3-87

{हत्वा तमक्षं निःशेषं बलं सर्वं चकार सः॥८८॥} %3-88
\refstepcounter{shlokacount}


\twolineshloka
{ततः श्रुत्वा कुमारस्य वधं राक्षसपुङ्गवः}
{क्रोधेन महताविष्ट इन्द्रजेतारमब्रवीत्} %3-89

\twolineshloka
{पुत्र गच्छाम्यहं तत्र यत्रास्ते पुत्रहा रिपुः}
{हत्वा तमथवा बद्ध्वा आनयिष्यामि} %3-90

\twolineshloka
{इन्द्रजित्पितरं प्राह त्यज शोकं महामते}
{मयि स्थिते किमर्थं त्वं भाषसे दुःखितं वचः} %3-91

\twolineshloka
{बद्ध्वाऽऽनेष्ये द्रुतं तात वानरं ब्रह्मपाशतः}
{इत्युक्त्वा रथमारुह्य राक्षसैर्बहुभिर्वृतः} %3-92

\twolineshloka
{जगाम वायुपुत्रस्य समीपं वीरविक्रमः}
{ततोऽतिगर्जितं श्रुत्वा स्तम्भमुद्यस्य वीर्यवान्} %3-93

\twolineshloka
{उत्पपात नभोदेशं गरुत्मानिव मारुतिः}
{ततो भ्रमन्तं नभसि हनूमन्तं शिलीमुखैः} %3-94

\twolineshloka
{विद्ध्वा तस्य शिरोभागमिषुभिश्चाष्टभिः पुनः}
{हृदयं पादयुगलं षड्भिरेकेन वालधिम्} %3-95

\twolineshloka
{भेदयित्वा ततो घोरं सिंहनादमथाकरोत्}
{ततोऽतिहर्षाद्धनुमान् स्तम्भमुद्यस्य वीर्यवान्} %3-96

\twolineshloka
{जघान सारथिं साश्वं रथं चाचूर्णयत्क्षणात्}
{ततोऽन्यं रथमादाय मेघनादो महाबलः} %3-97

\twolineshloka
{शीघ्रं ब्रह्मास्त्रमादाय बद्ध्वा वानरपुङ्गवम्}
{निनाय निकटं राज्ञो रावणस्य महाबलः} %3-98

\fourlineindentedshloka
{यस्य नाम सततं जपन्ति ये\-}
{ऽज्ञानकर्मकृतबंधनं क्षणात्}
{सद्य एव परिमुच्य तत्पदम्}
{यान्ति कोटिरविभासुरं शिवम्} %3-99

\fourlineindentedshloka
{तस्यैव रामस्य पदाम्बुजं सदा}
{हृत्पद्ममध्ये सुनिधाय मारुतिः}
{सदैव निर्मुक्तसमस्तबन्धनः}
{किं तस्य पाशैरितरैश्च बन्धनैः} %3-100

{॥इति श्रीमदध्यात्मरामायणे उमामहेश्वरसंवादे सुन्दरकाण्डे
तृतीयः सर्गः ॥ ३॥
}
%%%%%%%%%%%%%%%%%%%%



\sect{चतुर्थः सर्गः}

\textbf{श्रीमहादेव उवाच}

\fourlineindentedshloka
{यान्तं कपीन्द्रं धृतपाशबन्धनम्}
{विलोकयन्तं नगरं विभीतवत्}
{अताडयन्मुष्टितलैः सुकोपनाः}
{पौराः समन्तादनुयान्त ईक्षितुम्} %4-1

\fourlineindentedshloka
{ब्रह्मास्त्रमेनं क्षणमात्रसङ्गमम्}
{कृत्वा गतं ब्रह्मवरेण सत्वरम्}
{ज्ञात्वा हनूमानपि फल्गुरज्जुभि\-}
{र्धृतो ययौ कार्यविशेषगौरवात्} %4-2

\fourlineindentedshloka
{सभान्तरस्थस्य च रावणस्य तम्}
{पुरो निधायाह बलारिजित्तदा}
{बद्धो मया ब्रह्मवरेण वानरः}
{समागतोऽनेन हता महासुराः} %4-3

\fourlineindentedshloka
{यदुक्तमत्रार्य विचार्य मन्त्रिभि\-}
{र्विधीयतामेष न लौकिको हरिः}
{ततो विलोक्याह स राक्षसेश्वरः}
{प्रहस्तमग्रे स्थितमञ्जनाद्रिभम्} %4-4

\fourlineindentedshloka
{प्रहस्त पृच्छैनमसौ किमागतः}
{किमत्र कार्यं कुत एव वानरः}
{वनं किमर्थं सकलं विनाशितम्}
{हताः किमर्थं मम राक्षसा बलात्} %4-5

\fourlineindentedshloka
{ततः प्रहस्तो हनुमन्तमादरात्}
{पप्रच्छ केन प्रहितोऽसि वानर}
{भयं च ते मास्तु विमोक्ष्यसे मया}
{सत्यं वदस्वाखिलराजसन्निधौ} %4-6

\fourlineindentedshloka
{ततोऽतिहर्षात्पवनात्मजो रिपुम्}
{निरीक्ष्य लोकत्रयकण्टकासुरम्}
{वक्तुं प्रचक्रे रघुनाथसत्कथाम्}
{क्रमेण रामं मनसा स्मरन्मुहुः} %4-7

\fourlineindentedshloka
{शृणु स्फुटं देवगणाद्यमित्र हे}
{रामस्य दूतोऽहमशेषहृत्स्थितेः}
{यस्याखिलेशस्य हृताधुना त्वया}
{भार्या स्वनाशाय शुनेव सद्धविः} %4-8

\fourlineindentedshloka
{स राघवोऽभ्येत्य मतङ्गपर्वतम्}
{सुग्रीवमैत्रीमनलस्य सन्निधौ}
{कृत्वैकबाणेन निहत्य वालिनम्}
{सुग्रीवमेवाधिपतिं चकार तम्} %4-9

\fourlineindentedshloka
{स वानराणामधिपो महाबली}
{महाबलैर्वानरयूथकोटिभिः}
{रामेण सार्धं सह लक्ष्मणेन भोः}
{प्रवर्षणेऽमर्षयुतोऽवतिष्ठते} %4-10

\fourlineindentedshloka
{सञ्चोदितास्तेन महाहरीश्वरा}
{धरासुतां मार्गयितुं दिशो दश}
{तत्राहमेकः पवनात्मजः कपिः}
{सीतां विचिन्वन् शनकैः समागतः} %4-11

\fourlineindentedshloka
{दृष्टा मया पद्मपलाशलोचना}
{सीता कपित्वाद्विपिनं विनाशितम्}
{दृष्ट्वा ततोऽहं रभसा समागतान्}
{मां हन्तुकामान् धृतचापसायकान्} %4-12

\fourlineindentedshloka
{मया हतास्ते परिरक्षितं वपुः}
{प्रियो हि देहोऽखिलदेहिनां प्रभो}
{ब्रह्मास्त्रपाशेन निबध्य मां ततः}
{समागमन्मेघनिनादनामकः} %4-13

\fourlineindentedshloka
{स्पृष्ट्वैव मां ब्रह्मवरप्रभावतः}
{त्यक्त्वा गतं सर्वमवैमि रावण}
{तथाप्यहं बद्ध इवागतो हितम्}
{प्रवक्तुकामः करुणारसार्द्रधीः} %4-14

\fourlineindentedshloka
{विचार्य लोकस्य विवेकतो गतिम्}
{न राक्षसीं बुद्धिमुपैहि रावण}
{दैवीं गतिं संसृतिमोक्षहैतुकीम्}
{समाश्रयात्यन्तहिताय देहिनः} %4-15
{त्वं ब्रह्मणो ह्युत्तमवंशसम्भवः}
{पौलस्त्यपुत्रोऽसि कुबेरबान्धवः}
{देहात्मबुद्ध्यापि च पश्य राक्षसो}
{नास्यात्मबुद्ध्या किमु राक्षसो नहि} %4-16

\fourlineindentedshloka
{शरीरबुद्धीन्द्रियदुःखसन्ततिः}
{न ते न च त्वं तव निर्विकारतः}
{अज्ञानहेतोश्च तथैव सन्ततेः}
{असत्त्वमस्याः स्वपतो हि दृश्यवत्} %4-17

\fourlineindentedshloka
{इदं तु सत्यं तव नास्ति विक्रिया}
{विकारहेतुर्न च तेऽद्वयत्वतः}
{यथा नभः सर्वगतं न लिप्यते}
{तथा भवान् देहगतोऽपि सूक्ष्मकः}
{देहेन्द्रियप्राणशरीरसङ्गतः}
{त्वात्मेति बद्ध्वाखिलबन्धभाग्भवेत्} %4-18

\fourlineindentedshloka
{चिन्मात्रमेवाहमजोऽहमक्षरो}
{ह्यानन्दभावोऽहमिति प्रमुच्यते}
{देहोऽप्यनात्मा पृथिवीविकारजो}
{न प्राण आत्माऽनिल एष एव सः} %4-19

\fourlineindentedshloka
{मनोऽप्यहङ्कारविकार एव नो}
{न चापि बुद्धिः प्रकृतेर्विकारजा}
{आत्मा चिदानन्दमयोऽविकारवान्}
{देहादिसङ्घाद्व्यतिरिक्त ईश्वरः} %4-20

\fourlineindentedshloka
{निरञ्जनो मुक्त उपाधितः सदा}
{ज्ञात्वैवमात्मानमितो विमुच्यते}
{अतोऽहमात्यन्तिकमोक्षसाधनम्}
{वक्ष्ये शृणुष्वावहितो महामते} %4-21

\fourlineindentedshloka
{विष्णोर्हि भक्तिः सुविशोधनं धियः}
{ततो भवेज्ज्ञानमतीव निर्मलम्}
{विशुद्धतत्त्वानुभवो भवेत्ततः}
{सम्यग्विदित्वा परमं पदं व्रजेत्} %4-22

\fourlineindentedshloka
{अतो भजस्वाद्य हरिं रमापतिम्}
{रामं पुराणं प्रकृतेः परं विभुम्}
{विसृज्य मौर्ख्यं हृदि शत्रुभावनाम्}
{भजस्व रामं शरणागतप्रियम्}
{सीतां पुरस्कृत्य सपुत्रबान्धवो}
{रामं नमस्कृत्य विमुच्यसे भयात्} %4-23

\fourlineindentedshloka
{रामं परात्मानमभावयन् जनो}
{भक्त्या हृदिस्थं सुखरूपमद्वयम्}
{कथं परं तीरमवाप्नुयाज्जनो}
{भवाम्बुधेर्दुःखतरङ्गमालिनः} %4-24

\fourlineindentedshloka
{नो चेत्त्वमज्ञानमयेन वह्निना}
{ज्वलन्तमात्मानमरक्षितारिवत्}
{नयस्यधोऽधः स्वकृतैश्च पातकैः}
{विमोक्षशङ्का न च ते भविष्यति} %4-25

\fourlineindentedshloka
{श्रुत्वामृतास्वादसमानभाषितम्}
{तद्वायुसूनोर्दशकन्धरोऽसुरः}
{अमृष्यमाणोऽतिरुषा कपीश्वरम्}
{जगाद रक्तान्तविलोचनो} %4-26

\fourlineindentedshloka
{कथं ममाग्रे विलपस्यभीतवत्}
{प्लवङ्गमानामधमोऽसि दुष्टधीः}
{क एष रामः कतमो वनेचरो}
{निहन्मि सुग्रीवयुतं नराधमम्} %4-27

\fourlineindentedshloka
{त्वां चाद्य हत्वा जनकात्मजां ततो}
{निहन्मि रामं सहलक्ष्मणं ततः}
{सुग्रीवमग्रे बलिनं कपीश्वरम्}
{सवानरं हन्म्यचिरेण वानर}
{श्रुत्वा दशग्रीववचः स मारुतिः}
{विवृद्धकोपेन दहन्निवासुरम्} %4-28

\fourlineindentedshloka
{न मे समा रावणकोटयोऽधम}
{रामस्य दासोऽहमपारविक्रमः}
{श्रुत्वातिकोपेन हनूमतो वचो}
{दशाननो राक्षसमेवमब्रवीत्} %4-29

\fourlineindentedshloka
{पार्श्वे स्थितं मारय खण्डशः कपिम्}
{पश्यन्तु सर्वेऽसुरमित्रबान्धवाः}
{निवारयामास ततो विभीषणो}
{महासुरं सायुधमुद्यतं वधे}
{राजन् वधार्हो न भवेत्कथञ्चन}
{प्रतापयुक्तैः परराजवानरः} %4-30

\twolineshloka
{हतेऽस्मिन् वानरे दूते वार्ता को वा निवेदयेत्}
{रामाय त्वं यमुद्दिश्य वधाय समुपस्थितः} %4-31

\twolineshloka
{अतो वधसमं किञ्चिदन्यच्चिन्तय वानरे}
{सचिह्नो गच्छतु हरिर्यं दृष्ट्वाऽऽयास्यति द्रुतम्} %4-32

\twolineshloka
{रामः सुग्रीवसहितस्ततो युद्धं भवेत्तव}
{विभीषणवचः श्रुत्वा रावणोऽप्येतदब्रवीत्} %4-33

\twolineshloka
{वानराणां हि लाङ्गूले महामानो भवेत्किल}
{अतो वस्त्रादिभिः पुच्छं वेष्टयित्वा} %4-34

\twolineshloka
{वह्निना योजयित्वैनं भ्रामयित्वा पुरेऽभितः}
{विसर्जयत पश्यन्तु सर्वे वानरयूथपाः} %4-35

\twolineshloka
{तथेति शणपट्टैश्च वस्त्रैरन्यैरनेकशः}
{तैलाक्तैर्वेष्टयामासुर्लाङ्गूलं मारुतेर्दृढम्} %4-36

\twolineshloka
{पुच्छाग्रे किञ्चिदनलं दीपयित्वाथ राक्षसाः}
{रज्जुभिः सुदृढं बद्ध्वा धृत्वा तं बलिनोऽसुराः} %4-37

\twolineshloka
{समन्ताद्भ्रामयामासुश्चोरोऽयमिति वादिनः}
{तूर्यघोषैर्घोषयन्तस्ताडयन्तो मुहुर्मुहुः} %4-38

\twolineshloka
{हनूमतापि तत्सर्वं सोढं किञ्चिच्चिकीर्षुणा}
{गत्वा तु पश्चिमद्वारसमीपं तत्र मारुतिः} %4-39

\twolineshloka
{सूक्ष्मो बभूव बन्धेभ्यो निःसृतः पुनरप्यसौ}
{बभूव पर्वताकारस्तत उत्प्लुत्य गोपुरम्} %4-40

\twolineshloka
{तत्रैकं स्तम्भमादाय हत्वा तान् रक्षिणः क्षणात्}
{विचार्य कार्यशेषं स प्रासादाग्राद्गृहाद्गृहम्} %4-41

\twolineshloka
{उत्प्लुत्योप्लुत्य सन्दीप्तपुच्छेन महता कपिः}
{ददाह लङ्कामखिलां साट्टप्रासादतोरणाम्} %4-42

\twolineshloka
{हा तात पुत्र नाथेति क्रन्दमानाः समन्ततः}
{व्याप्ताः प्रासादशिखरेऽप्यारूढा दैत्ययोषितः} %4-43

\twolineshloka
{देवता इव दृश्यन्ते पतन्त्यः पावकेऽखिलाः}
{विभीषणगृहं त्यक्त्वा सर्वं भस्मीकृतं पुरम्} %4-44

\twolineshloka
{तत उत्प्लुत्य जलधौ हनूमान्मारुतात्मजः}
{लाङ्गूलं मज्जयित्वान्तः स्वस्थचित्तो बभूव सः} %4-45

\twolineshloka
{वायोः प्रियसखित्वाच्च सीतया प्रार्थितोऽनलः}
{न ददाह हरेः पुच्छं बभूवात्यन्तशीतलः} %4-46

\fourlineindentedshloka
{यन्नामसंस्मरणधूतसमस्तपापाः}
{तापत्रयानलमपीह तरन्ति सद्यः}
{तस्यैव किं रघुवरस्य विशिष्टदूतः}
{सन्तप्यते कथमसौ प्रकृतानलेन} %4-47

{॥इति श्रीमदध्यात्मरामायणे उमामहेश्वरसंवादे सुन्दरकाण्डे
चतुर्थः सर्गः ॥ ४॥
}
%%%%%%%%%%%%%%%%%%%%



\sect{पञ्चमः सर्गः}

\textbf{श्रीमहादेव उवाच}

\twolineshloka
{ततः सीतां नमस्कृत्य हनूमानब्रवीद्वचः}
{आज्ञापयतु मां देवि भवती रामसन्निधिम्} %5-1

\twolineshloka
{गच्छामि रामस्त्वां द्रष्टुमागमिष्यति सानुजः}
{इत्युक्त्वा त्रिःपरिक्रम्य जानकीं मारुतात्मजः} %5-2

\twolineshloka
{प्रणम्य प्रस्थितो गन्तुमिदं वचनमब्रवीत्}
{देवि गच्छामि भद्रं ते तूर्णं द्रक्ष्यसि} %5-3

\twolineshloka
{लक्ष्मणं च ससुग्रीवं वानरायुतकोटिभिः}
{ततः प्राह हनूमन्तं जानकी दुःखकर्शिता} %5-4

\twolineshloka
{त्वां दृष्ट्वा विस्मृतं दुःखमिदानीं त्वं गमिष्यसि}
{इतः परं कथं वर्ते रामवार्ताश्रुतिं विना} %5-5

\textbf{मारुतिरुवाच}

\twolineshloka
{यद्येवं देवि मे स्कन्धमारोह क्षणमात्रतः}
{रामेण योजयिष्यामि मन्यसे यदि जानकि} %5-6

\textbf{सीतोवाच}

\twolineshloka
{रामः सागरमाशोष्य बद्ध्वा वा शरपञ्जरैः}
{आगत्य वानरैः सार्धं हत्वा रावणमाहवे} %5-7

\twolineshloka
{मां नयेद्यदि रामस्य कीर्तिर्भवति शाश्वती}
{अतो गच्छ कथं चापि प्राणान् सन्धारयाम्यहम्} %5-8

\twolineshloka
{इति प्रस्थापितो वीरः सीतया प्रणिपत्य ताम्}
{जगाम पर्वतस्याग्रे गन्तुं पारं महोदधेः} %5-9

\twolineshloka
{तत्र गत्वा महासत्त्वः पादाभ्यां पीडयन् गिरिम्}
{जगाम वायुवेगेन पर्वतश्च महीतलम्} %5-10

\twolineshloka
{गतो महीसमानत्वं त्रिंशद्योजनमुच्छ्रितः}
{मारुतिर्गगनाअन्तःस्थो महाशब्दं चकार सः} %5-11

\twolineshloka
{तं श्रुत्वा वानराः सर्वे ज्ञात्वा मारुतिमागतम्}
{हर्षेण महताविष्टाः शब्दं चक्रुर्महास्वनम्} %5-12

\twolineshloka
{शब्देनैव विजानीमः कृतकार्यः समागतः}
{हनूमानेव पश्यध्वं वानरा वानरर्षभम्} %5-13

\twolineshloka
{एवं ब्रुवत्सु वीरेषु वानरेषु स मारुतिः}
{अवतीर्य गिरेर्मुर्ध्नि वानरानिदमब्रवीत्} %5-14

\twolineshloka
{दृष्टा सीता मया लङ्का धर्षिता च सकानना}
{सम्भाषितो दशग्रीवस्ततोऽहं पुनरागतः} %5-15

\twolineshloka
{इदानीमेव गच्छामो रामसुग्रीवसन्निधिम्}
{इत्युक्ता वानराः सर्वे हर्षेणालिङ्ग्य मारुतिम्} %5-16

\twolineshloka
{केचिच्चुचुम्बुर्लाङ्गूलं ननृतुः केचिदुत्सुकाः}
{हनूमता समेतास्ते जग्मुः प्रस्रवणं गिरिम्} %5-17

\twolineshloka
{गच्छन्तो ददृशुर्वीरा वनं सुग्रीवरक्षितम्}
{मधुसंज्ञं तदा प्राहुरङ्गदं वानरर्षभाः} %5-18

\twolineshloka
{क्षुधिताः स्मो वयं वीर देह्यनुज्ञां महामते}
{भक्षयामः फलान्यद्य पिबामोऽमृतवन्मधु} %5-19

{सन्तुष्टा राघवं द्रष्टुं गच्छामोऽद्यैव सानुजम्॥२०॥} %5-20
\refstepcounter{shlokacount}


\textbf{अङ्गद उवाच}

\twolineshloka
{हनूमान् कृतकार्योऽयं पिबतैतत्प्रसादतः}
{जक्षध्वं फलमूलानि त्वरितं हरिसत्तमाः} %5-21

\twolineshloka
{ततः प्रविश्य हरयः पातुमारेभिरे मधु}
{रक्षिणस्ताननादृत्य दधिवक्त्रेण नोदितान्} %5-22

\twolineshloka
{पिबतस्ताडयामासुर्वानरान् वानरर्षभाः}
{ततस्तान् मुष्टिभिः पादैश्चूर्णयित्वा पपुर्मधु} %5-23

\twolineshloka
{ततो दधिमुखः क्रुद्धः सुग्रीवस्य स मातुलः}
{जगाम रक्षिभिः सार्धं यत्र राजा कपीश्वरः} %5-24

\twolineshloka
{गत्वा तमब्रवीद्देव चिरकालाभिरक्षितम्}
{नष्टं मधुवनं तेऽद्य कुमारेण हनूमता} %5-25

\twolineshloka
{श्रुत्वा दधिमुखेनोक्तं सुग्रीवो हृष्टमानसः}
{दृष्ट्वागतो न सन्देहः सीतां पवननन्दनः} %5-26

\twolineshloka
{नो चेन्मधुवनं द्रष्टुं समर्थः को भवेन्मम}
{तत्रापि वायुपुत्रेण कृतं कार्यं न संशयः} %5-27

\twolineshloka
{श्रुत्वा सुग्रीववचनं हृष्टो रामस्तमब्रवीत्}
{किमुच्यते त्वया राजन् वचः सीताकथान्वितम्} %5-28

\twolineshloka
{सुग्रीवस्त्वब्रवीद्वाक्यं देव दृष्टावनीसुता}
{हनुमत्प्रमुखाः सर्वे प्रविष्टा मधुकाननम्} %5-29

\twolineshloka
{भक्षयन्ति स्म सकलं ताडयन्ति स्म रक्षिणः}
{अकृत्वा देवकार्यं ते द्रष्टुं मधुवनं मम} %5-30

\twolineshloka
{न समर्थास्ततो देवी दृष्टा सीतेति निश्चितम्}
{रक्षिणो वो भयं मास्तु गत्वा ब्रूत ममाज्ञया} %5-31

\twolineshloka
{वानरानङ्गदमुखानानयध्वं ममान्तिकम्}
{श्रुत्वा सुग्रीववचनं गत्वा ते वायुवेगतः} %5-32

\twolineshloka
{हनूमत्प्रमुखानूचुर्गच्छतेश्वरशासनात्}
{द्रष्टुमिच्छति सुग्रीवः सरामो लक्ष्मणान्वितः} %5-33

\twolineshloka
{युष्मानतीव हृष्टास्ते त्वरयन्ति महाबलाः}
{तथेत्यम्बरमासाद्य ययुस्ते वानरोत्तमाः} %5-34

\twolineshloka
{हनूमन्तं पुरस्कृत्य युवराजं तथाङ्गदम्}
{रामसुग्रीवयोरग्रे निपेतुर्भुवि सत्वरम्} %5-35

\twolineshloka
{हनूमान् राघवं प्राह दृष्टा सीता निरामया}
{साष्टाङ्गं प्रणिपत्याग्रे रामं पश्चाद्धरीश्वरम्} %5-36

\twolineshloka
{कुशलं प्राह राजेन्द्र जानकी त्वां शुचान्विता}
{अशोकवनिकामध्ये शिंशपामूलमाश्रिता} %5-37

\twolineshloka
{राक्षसीभिः परिवृता निराहारा कृशा प्रभो}
{हा राम राम रामेति शोचन्ती मलिनाम्बरा} %5-38

\twolineshloka
{एकवेणी मया दृष्टा शनैराश्वासिता शुभा}
{वृक्षशाखान्तरे स्थित्वा सूक्ष्मरूपेण ते कथाम्} %5-39

\twolineshloka
{जन्मारभ्य तवात्यर्थं दण्डकागमनं तथा}
{दशाननेन हरणं जानक्या रहिते त्वयि} %5-40

\twolineshloka
{सुग्रीवेण यथा मैत्री कृत्वा वालिनिबर्हणम्}
{मार्गणार्थं च वैदेह्या सुग्रीवेण विसर्जिताः} %5-41

\twolineshloka
{महाबला महासत्त्वा हरयो जितकाशिनः}
{गताः सर्वत्र सर्वे वै तत्रैकोऽहमिहागतः} %5-42

\twolineshloka
{अहं सुग्रीवसचिवो दासोऽहं राघवस्य हि}
{दृष्टा यज्जानकी भाग्यात्प्रयासः फलितोऽद्य मे} %5-43

\twolineshloka
{इत्युदीरितमाकर्ण्य सीता विस्फारितेक्षणा}
{केन वा कर्णपीयुषं श्रावितं मे शुभाक्षरम्} %5-44

\twolineshloka
{यदि सत्यं तदायातु मद्दर्शनपथं तु सः}
{ततोऽहं वानराकारः सूक्ष्मरूपेण जानकीम्} %5-45

\twolineshloka
{प्रणम्य प्राञ्जलिर्भूत्वा दूरादेव स्थितः प्रभो}
{पृष्टोऽहं सीतया कस्त्वमित्यादि बहुविस्तरम्} %5-46

\twolineshloka
{मया सर्वं क्रमेणैव विज्ञापितमरिन्दम}
{पश्चान्मयार्पितं देव्यै भवद्दत्ताङ्गुलीयकम्} %5-47

\twolineshloka
{तेन मामतिविश्वस्ता वचनं चेदमब्रवीत्}
{यथा दृष्टास्मि हनुमन् पीड्यमाना दिवानिशम्} %5-48

\twolineshloka
{राक्षसीनां तर्जनैस्तत्सर्वं कथय राघवे}
{मयोक्तं देवि रामोऽपि त्वच्चिन्तापरिनिष्ठितः} %5-49

\twolineshloka
{परिशोचत्यहोरात्रं त्वद्वार्तां नाधिगम्य सः}
{इदानीमेव गत्वाहं स्थितिं रामाय ते ब्रुवे} %5-50

\twolineshloka
{रामः श्रवणमात्रेण सुग्रीवेण सलक्ष्मणः}
{वानरानीकपैः सार्धमागमिष्यति तेऽन्तिकम्} %5-51

\twolineshloka
{रावणं सकुलं हत्वा नेष्यति त्वां स्वकं पुरम्}
{अभिज्ञां देहि मे देवि यथा मां विश्वसेद्विभुः} %5-52

\twolineshloka
{इत्युक्ता सा शिरोरत्नं चूडापाशे स्थितं प्रियम्}
{दत्त्वा काकेन यद्वृत्तं चित्रकूटगिरौ पुरा} %5-53

\twolineshloka
{तदप्याहाश्रुपूर्णाक्षी कुशलं ब्रूहि राघवम्}
{लक्ष्मणं ब्रूहि मे किञ्चिद्दुरुक्तं भाषितं पुरा} %5-54

\twolineshloka
{तत्क्षमस्वाज्ञभावेन भाषितं कुलनन्दन}
{तारयेन्मां यथा रामस्तथा कुरु कृपान्वितः} %5-55

\twolineshloka
{इत्युक्त्वा रुदती सीता दुःखेन महतावृता}
{मयाप्याश्वासिता राम वदता सर्वमेव ते} %5-56

\twolineshloka
{ततः प्रस्थापितो राम त्वत्समीपमिहागतः}
{तदागमनवेलायामशोकवनिकां प्रियाम्} %5-57

\twolineshloka
{उत्पाट्य राक्षसांस्तत्र बहून् हत्वा क्षणादहम्}
{रावणस्य सुतं हत्वा रावणेनाभिभाष्य च} %5-58

\twolineshloka
{लङ्कामशेषतो दग्ध्वा पुनरप्यागमं क्षणात्}
{श्रुत्वा हनूमतो वाक्यं रामोऽत्यन्तप्रहृष्टधीः} %5-59

\twolineshloka
{हनूमंस्ते कृतं कार्यं देवैरपि सुदुष्करम्}
{उपकारं न पश्यामि तव प्रत्युपकारिणः} %5-60

\twolineshloka
{इदानीं ते प्रयच्छामि सर्वस्वं मम मारुते}
{इत्यालिङ्ग्य समाकृष्य गाढं वानरपुङ्गवम्} %5-61

\twolineshloka
{सार्द्रनेत्रो रघुश्रेष्ठः परां प्रीतिमवाप सः}
{हनूमन्तमुवाचेदं राघवो भक्तवत्सलः} %5-62

\twolineshloka
{परिरम्भो हि मे लोके दुर्लभः परमात्मनः}
{अतस्त्वं मम भक्तोऽसि प्रियोऽसि हरिपुङ्गव} %5-63

\fourlineindentedshloka
{यत्पादपद्मयुगलं तुलसीदलाद्यैः}
{सम्पूज्य विष्णुपदवीमतुलां प्रयान्ति}
{तेनैव किं पुनरसौ परिरब्धमूर्ती}
{रामेण वायुतनयः कृतपुण्यपुञ्जः} %5-64

{॥इति श्रीमदध्यात्मरामायणे उमामहेश्वरसंवादे सुन्दरकाण्डे
पञ्चमः सर्गः ॥ ५॥
}
%%%%%%%%%%%%%%%%%%%%

इति श्रीमदध्यात्मरामायणे सुन्दरकाण्डः समाप्तः॥॥
