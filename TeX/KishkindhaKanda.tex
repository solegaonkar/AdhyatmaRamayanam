

\chapt{किष्किन्धाकाण्डः}


\sect{प्रथमः सर्गः}

\textbf{श्रीमहादेव उवाच}

\twolineshloka
{ततः सलक्ष्मणो रामः शनैः पम्पासरस्तटम्}
{आगत्य सरसां श्रेष्ठां दृष्ट्वा विस्मयमाययौ} %1-1

\twolineshloka
{क्रोशमात्रं सुविस्तीर्णमगाधामलशम्बरम्}
{उत्फुल्लाम्बुजकल्हारकुमुदोत्पलमण्डितम्} %1-2

\twolineshloka
{हंसकारण्डवाकीर्णं चक्रवाकादिशोभितम्}
{जलकुक्कुटकोयष्टिक्रौञ्चनादोपनादितम्} %1-3

\twolineshloka
{नानापुष्पलताकीर्णं नानाफलसमावृतम्}
{सतां मनःस्वच्छजलं पद्मकिञ्जल्कवासितम्} %1-4

\twolineshloka
{तत्रोपस्पृश्य सलिलं पीत्वा श्रमहरं विभुः}
{सानुजः सरसस्तीरे शीतलेन पथा ययौ} %1-5

\threelineshloka
{ऋष्यमूकगिरेः पार्श्वे गच्छन्तौ रामलक्ष्मणौ}
{धनुर्बाणकरौ दान्तौ जटावल्कलमण्डितौ}
{पश्यन्तौ विविधान् वृक्षान् गिरेः शोभां सुविक्रमौ} %1-6

\twolineshloka
{सुग्रीवस्तु गिरेर्मूर्ध्नि चतुर्भिः सह वानरैः}
{स्थित्वा ददर्श तौ यान्तावारुरोह गिरेः शिरः} %1-7

\twolineshloka
{भयादाह हनूमन्तं कौ तौ वीरवरौ सखे}
{गच्छ जानीहि भद्रं ते वटुर्भूत्वा द्विजाकृतिः} %1-8

\twolineshloka
{वालिना प्रेषितौ किंवा मां हन्तुं समुपागतौ}
{ताभ्यां सम्भाषणं कृत्वा जानीहि हृदयं तयोः} %1-9

\twolineshloka
{यदि तौ दुष्टहृदयौ संज्ञां कुरु कराग्रतः}
{विनयावनतो भूत्वा एवं जानीहि निश्चयम्} %1-10

\twolineshloka
{तथेति वटुरूपेण हनुमान् समुपागतः}
{विनयावनतो भूत्वा रामं नत्वेदमब्रवीत्} %1-11

\twolineshloka
{कौ युवां पुरुषव्याघ्रौ युवानौ वीरसम्मतौ}
{द्योतयन्तौ दिशः सर्वाः प्रभया भास्कराविव} %1-12

\twolineshloka
{युवां त्रैलोक्यकर्ताराविति भाति मनो मम}
{युवां प्रधानपुरुषौ जगद्धेतू जगन्मयौ} %1-13

\twolineshloka
{मायया मानुषाकारौ चरन्ताविव लीलया}
{भूभारहरणार्थाय भक्तानां पालनाय च} %1-14

\twolineshloka
{अवतीर्णाविह परौ चरन्तौ क्षत्रियाकृती}
{जगत्स्थितिलयौ सर्गं लीलया कर्तुमुद्यतौ} %1-15

\twolineshloka
{स्वतन्त्रौ प्रेरकौ सर्वहृदयस्थाविहेश्वरौ}
{नरनारायणौ लोके चरन्ताविति मे मतिः} %1-16

\twolineshloka
{श्रीरामो लक्ष्मणं प्राह पश्यैनं वटुरूपिणम्}
{शब्दशास्त्रमशेषेण श्रुतं नूनमनेकधा} %1-17

\twolineshloka
{अनेन भाषितं कृत्स्नं न किञ्चिदपशब्दितम्}
{ततः प्राह हनूमन्तं राघवो ज्ञानविग्रहः} %1-18

\twolineshloka
{अहं दाशरथी रामस्त्वयं मे लक्ष्मणोऽनुजः}
{सीतया भार्यया सार्धं पितुर्वचनगौरवात्} %1-19

\threelineshloka
{आगतस्तत्र विपिने स्थितोऽहं दण्डके द्विज}
{तत्र भार्या हृता सीता रक्षसा केनचिन्मम}
{तामन्वेष्टुमिहायातौ त्वं को वा कस्य वा वद} %1-20

\textbf{वटुरुवाच}

\twolineshloka
{सुग्रीवो नाम राजा यो वानराणां महामतिः}
{चतुर्भिर्मन्त्रिभिः सार्धं गिरिमूर्धनि तिष्ठति} %1-21

\twolineshloka
{भ्राता कनियान् सुग्रीवो वालिनः पापचेतसः}
{तेन निष्कासितो भार्या हृता तस्येह वालिना} %1-22

\twolineshloka
{तद्भयादृष्यमूकाख्यं गिरिमाश्रित्य संस्थितः}
{अहं सुग्रीवसचिवो वायुपुत्रो महामते} %1-23

\twolineshloka
{हनुमान्नाम विख्यातो ह्यञ्जनीगर्भसम्भवः}
{तेन सख्यं त्वया युक्तं सुग्रीवेण रघूत्तम} %1-24

\twolineshloka
{भार्यापहारिणं हन्तुं सहायस्ते भविष्यति}
{इदानीमेव गच्छाम आगच्छ यदि रोचते} %1-25

\textbf{श्रीराम उवाच}

\twolineshloka
{अहमप्यागतस्तेन सख्यं कर्तुं कपीश्वर}
{सख्युस्तस्यापि यत्कार्यं तत्करिष्याम्यसंशयम्} %1-26

\twolineshloka
{हनुमान् स्वस्वरूपेण स्थितो राममथाब्रवीत्}
{आरोहतां मम स्कन्धौ गच्छामः पर्वतोपरि} %1-27

\twolineshloka
{यत्र तिष्ठति सुग्रीवो मन्त्रिभिर्वालिनो भयात्}
{तथेति तस्यारुरोह स्कन्धं रामोऽथ लक्ष्मणः} %1-28

\twolineshloka
{उत्पपात गिरेर्मूर्ध्नि क्षणादेव महाकपिः}
{वृक्षच्छायां समाश्रित्य स्थितौ तौ रामलक्ष्मणौ} %1-29

\twolineshloka
{हनुमानपि सुग्रीवमुपगम्य कृताञ्जलिः}
{व्येतु ते भयमायातौ राजन् श्रीरामलक्ष्मणौ} %1-30

\twolineshloka
{शीघ्रमुत्तिष्ठ रामेण सख्यं ते योजितं मया}
{अग्निं साक्षिणमारोप्य तेन सख्यं द्रुतं कुरु} %1-31

\twolineshloka
{ततोऽतिहर्षात्सुग्रीवः समागम्य रघूत्तमम्}
{वृक्षशाखां स्वयं छित्वा विष्टराय ददौ मुदा} %1-32

\twolineshloka
{हनूमान्ल्लक्ष्मणायादात्सुग्रीवाय च लक्ष्मणः}
{हर्षेण महताऽऽविष्टाः सर्व एवावतस्थिरे} %1-33

\twolineshloka
{लक्ष्मणस्त्वब्रवीत्सर्वं रामवृत्तान्तमादितः}
{वनवासाभिगमनं सीताहरणमेव च} %1-34

\twolineshloka
{लक्ष्मणोक्तं वचः श्रुत्वा सुग्रीवो राममब्रवीत्}
{अहं करिष्ये राजेन्द्र सीतायाः परिमार्गणम्} %1-35

\twolineshloka
{साहाय्यमपि ते राम करिष्ये शत्रुघातिनः}
{शृणु राम मया दृष्टं किञ्चित्ते कथयाम्यहम्} %1-36

\twolineshloka
{एकदा मन्त्रिभिः सार्धं स्थितोऽहं गिरिमूर्धनि}
{विहायसा नीयमानां केनचित्प्रमदोत्तमाम्} %1-37

\twolineshloka
{क्रोशन्तीं रामरामेति दृष्ट्वाऽस्मान् पर्वतोपरि}
{आमुच्याभरणान्याशु स्वोत्तरीयेण भामिनी} %1-38

\twolineshloka
{निरीक्ष्याधः परित्यज्य क्रोशन्ती तेन रक्षसा}
{नीताहं भूषणान्याशु गुहायामक्षिपं प्रभो} %1-39

\twolineshloka
{इदानीमपि पश्य त्वं जानीहि तव वा न वा}
{इत्युक्त्वाऽऽनीय रामाय दर्शयामास वानरः} %1-40

\twolineshloka
{विमुच्य रामस्तद्दृष्ट्वा हा सीतेति मुहुर्मुहुः}
{हृदि निक्षिप्य तत्सर्वं रुरोद प्राकृतो यथा} %1-41

\threelineshloka
{आश्वास्य राघवं भ्राता लक्ष्मणो वाक्यमब्रवीत्}
{अचिरेणैव ते राम प्राप्यते जानकी शुभा}
{वानरेन्द्रसहायेन हत्वा रावणमाहवे} %1-42

\twolineshloka
{सुग्रीवोऽप्याह हे राम प्रतिज्ञां करवाणि ते}
{समरे रावणं हत्वा तव दास्यामि जानकीम्} %1-43

\twolineshloka
{ततो हनूमान् प्रज्वाल्य तयोरग्निं समीपतः}
{तावुभौ रामसुग्रीवावग्नौ साक्षिणि तिष्ठति} %1-44

\twolineshloka
{बाहू प्रसार्य चालिङ्ग्य परस्परमकल्मषौ}
{समीपे रघुनाथस्य सुग्रीवः समुपाविशत्} %1-45

\twolineshloka
{स्वोदन्तं कथयामास प्रणयाद्रघुनायके}
{सखे शृणु ममोदन्तं वालिना यत्कृतं पुरा} %1-46

\twolineshloka
{मयपुत्रोऽथ मायावी नाम्ना परमदुर्मदः}
{किष्किन्धां समुपागत्य वालिनं समुपाह्वयत्} %1-47

\twolineshloka
{सिंहनादेन महता वाली तु तदमर्षणः}
{निर्ययौ क्रोधताम्राक्षो जघान दृढमुष्टिना} %1-48

\twolineshloka
{दुद्राव तेन संविग्नो जगाम स्वगुहां प्रति}
{अनुदुद्राव तं वाली मायाविनमहं तथा} %1-49

\threelineshloka
{ततः प्रविष्टमालोक्य गुहां मायाविनं रुषा}
{वाली मामाह तिष्ठ त्वं बहिर्गच्छाम्यहं गुहाम्}
{इत्युक्त्वाऽऽविश्य स गुहां मासमेकं न निर्ययौ} %1-50

\twolineshloka
{मासादूर्ध्वं गुहाद्वारान्निर्गतं रुधिरं बहु}
{तद्दृष्ट्वा परितप्ताङ्गो मृतो वालीति दुःखितः} %1-51

\twolineshloka
{गुहाद्वारि शिलामेकां निधाय गृहमागतः}
{ततोऽब्रवं मृतो वाली गुहायां रक्षसा हतः} %1-52

\twolineshloka
{तच्छ्रुत्वा दुःखिताः सर्वे मामनिच्छन्तमप्युत}
{राज्येऽभिषेचनं चक्रुः सर्वे वानरमन्त्रिणः} %1-53

\twolineshloka
{शिष्टं तदा मया राज्यं किञ्चित्कालमरिन्दम}
{ततः समागतो वाली मामाह परुषं रुषा} %1-54

\twolineshloka
{बहुधा भर्त्सयित्वा मां निजघान च मुष्टिभिः}
{ततो निर्गत्य नगरादधावं परया भिया} %1-55

\twolineshloka
{लोकान् सर्वान् परिक्रम्य ऋष्यमूकं समाश्रितः}
{ऋषेः शापभयात्सोऽपि नायातीमं गिरिं प्रभो} %1-56

\twolineshloka
{तदादि मम भार्यां स स्वयं भुङ्क्ते विमूढधीः}
{अतो दुःखेन सन्तप्तो हृतदारो हृताश्रयः} %1-57

\twolineshloka
{वसाम्यद्य भवत्पादसंस्पर्शात्सुखितोऽस्म्यहम्}
{मित्रदुःखेन सन्तप्तो रामो राजीवलोचनः} %1-58

\twolineshloka
{हनिष्यामि तव द्वेष्यं शीघ्रं भार्यापहारिणम्}
{इति प्रतिज्ञामकरोत्सुग्रीवस्य पुरस्तदा} %1-59

\twolineshloka
{सुग्रीवोऽप्याह राजेन्द्र वाली बलवतां बली}
{कथं हनिष्यति भवान् देवैरपि दुरासदम्} %1-60

\twolineshloka
{शृणु ते कथयिष्यामि तद्बलं बलिनां वर}
{कदाचिद्दुन्दुभिर्नाम महाकायो महाबलः} %1-61

\twolineshloka
{किष्किन्धामगमद्राम महामहिषरूपधृक्}
{युद्धाय वालिनं रात्रौ समाह्वयत भीषणः} %1-62

\twolineshloka
{तच्छ्रुत्वाऽसहमानोऽसौ वाली परमकोपनः}
{महिषं शृङ्गयोर्धृत्वा पातयामास भूतले} %1-63

\twolineshloka
{पादेनैकेन तत्कायमाक्रम्यास्य शिरो महत्}
{हस्ताभ्यां भ्रामयंश्छित्त्वा तोलयित्वाऽक्षिपद्भुवि} %1-64

\twolineshloka
{पपात तच्छिरो राम मातङ्गाश्रमसन्निधौ}
{योजनात्पतितं तस्मान्मुनेराश्रममण्डले} %1-65

\twolineshloka
{रक्तवृष्टिः पपातोच्चैर्दृष्ट्वा तां क्रोधमूर्च्छितः}
{मातङ्गो वालिनं प्राह यद्यागन्तासि मे गिरिम्} %1-66

\twolineshloka
{इतः परं भग्नशिरा मरिष्यसि न संशयः}
{एवं शप्तस्तदारभ्य ऋष्यमूकं न यात्यसौ} %1-67

\twolineshloka
{एतज्ज्ञात्वाऽहमप्यत्र वसामि भयवर्जितः}
{राम पश्य शिरस्तस्य दुन्दुभेः पर्वतोपमम्} %1-68

\twolineshloka
{तत्क्षेपणे यदा शक्तः शक्तस्त्वं वालिनो वधे}
{इत्युक्त्वा दर्शयामास शिरस्तद्गिरिसन्निभम्} %1-69

\twolineshloka
{दृष्ट्वा रामः स्मितं कृत्वा पादाङ्गुष्ठेन चाक्षिपत्}
{दशयोजनपर्यन्तं तदद्भुतमिवाभवत्} %1-70

\twolineshloka
{साधु साध्विति सम्प्राह सुग्रीवो मन्त्रिभिः सह}
{पुनरप्याह सुग्रीवो रामं भक्तपरायणम्} %1-71

\twolineshloka
{एते ताला महासाराः सप्त पश्य रघूत्तम}
{एकैकं चालयित्वाऽसौ निष्पत्रान् कुरुतेऽञ्जसा} %1-72

\threelineshloka
{यदि त्वमेकबाणेन विद्ध्वा छिद्रं करोषि चेत्}
{हतस्त्वया तदा वाली विश्वासो मे प्रजायते}
{तथेति धनुरादाय सायकं तत्र सन्दधे} %1-73

\twolineshloka
{बिभेद च तदा रामः सप्त तालान् महाबलः}
{तालान् सप्त विनिर्भिद्य गिरिं भूमिं च सायकः} %1-74

\twolineshloka
{पुनरागत्य रामस्य तूणीरे पूर्ववत्स्थितः}
{ततोऽतिहर्षात्सुग्रीवो राममाहातिविस्मितः} %1-75

\twolineshloka
{देव त्वं जगतां नाथः परमात्मा न संशयः}
{मत्पूर्वकृतपुण्यौघैः सङ्गतोऽद्य मया सह} %1-76

\twolineshloka
{त्वां भजन्ति महात्मानः संसारविनिवृत्तये}
{त्वां प्राप्य मोक्षसचिवं प्रार्थयेऽहं कथं भवम्} %1-77

\twolineshloka
{दाराः पुत्रा धनं राज्यं सर्वं त्वन्मायया कृतम्}
{अतोऽहं देवदेवेश नाकाङ्क्षेऽन्यत्प्रसीद मे} %1-78

\twolineshloka
{आनन्दानुभवं त्वाऽद्य प्राप्तोऽहं भाग्यगौरवात्}
{मृदर्थं यतमानेन निधानमिव सत्पते} %1-79

\twolineshloka
{अनाद्यविद्यासंसिद्धं बन्धनं छिन्नमद्य नः}
{यज्ञदानतपःकर्मपूर्तेष्टादिभिरप्यसौ} %1-80

\twolineshloka
{न जीर्यते पुनर्दार्ढ्यं भजते संसृतिः प्रभो}
{त्वत्पाददर्शनात्सद्यो नाशमेति न संशयः} %1-81

\twolineshloka
{क्षणार्धमपि यच्चित्तं त्वयि तिष्ठत्यचञ्चलम्}
{तस्याज्ञानमनर्थानां मूलं नश्यति तत्क्षणात्} %1-82

{तत्तिष्ठतु मनो राम त्वयि नान्यत्र मे सदा॥८३॥} %1-83
\refstepcounter{shlokacount}


\twolineshloka
{रामरामेति यद्वाणी मधुरं गायति क्षणम्}
{स ब्रह्महा सुरापो वा मुच्यते सर्वपातकैः} %1-84

\twolineshloka
{न काङ्क्षे विजयं राम न च दारसुखादिकम्}
{भक्तिमेव सदा काङ्क्षे त्वयि बन्धविमोचनीम्} %1-85

\twolineshloka
{त्वन्मायाकृतसंसारस्त्वदंशोऽहं रघूत्तम}
{स्वपादभक्तिमादिश्य त्राहि मां भवसङ्कटात्} %1-86

\twolineshloka
{पूर्वं मित्रार्युदासीनास्त्वन्मायावृतचेतसः}
{आसन्मेऽद्य भवत्पाददर्शनादेव राघव} %1-87

\twolineshloka
{सर्वं ब्रह्मैव मे भाति क्व मित्रं क्व च मे रिपुः}
{यावत्त्वन्मायया बद्धस्तावद्गुणविशेषता} %1-88

\twolineshloka
{सा यावदस्ति नानात्वं तावद्भवति नान्यथा}
{यावन्नानात्वमज्ञानात्तावत्कालकृतं भयम्} %1-89

\threelineshloka
{अतोऽविद्यामुपास्ते यः सोऽन्धे तमसि मज्जति}
{मायामूलमिदं सर्वं पुत्रदारादिबन्धनम्}
{तदुत्सारय मायां त्वं दासीं तव रघूत्तम} %1-90

\fourlineindentedshloka
{त्वत्पादपद्मार्पितचित्तवृत्ति\-}
{स्त्वन्नामसङ्गीतकथासु वाणी}
{त्वद्भक्तसेवानिरतौ करौ मे}
{त्वदङ्गसङ्गं लभतां मदङ्गम्} %1-91

\fourlineindentedshloka
{त्वन्मूर्तिभक्तान् स्वगुरुं च चक्षुः}
{पश्यत्वजस्रं स शृणोतु कर्णः}
{त्वज्जन्मकर्माणि च पादयुग्मम्}
{व्रजत्वजस्रं तव मन्दिराणि} %1-92

\fourlineindentedshloka
{अङ्गानि ते पादरजोविमिश्र\-}
{तीर्थानि बिभ्रत्वहिशत्रुकेतो}
{शिरस्त्वदीयं भवपद्मजाद्यैर्\-}
{जुष्टं पदं राम नमत्वजस्रम्} %1-93

{॥इति श्रीमदध्यात्मरामायणे उमामहेश्वरसंवादे किष्किन्धाकाण्डे
प्रथमः सर्गः॥१॥
}
%%%%%%%%%%%%%%%%%%%%



\sect{द्वितीयः सर्गः}

\twolineshloka
{इत्थं स्वात्मपरिष्वङ्गनिर्धूताशेषकल्मषम्}
{रामः सुग्रीवमालोक्य सस्मितं वाक्यमब्रवीत्} %2-1

\twolineshloka
{मायां मोहकरीं तस्मिन् वितन्वन् कार्यसिद्धये}
{सखे त्वदुक्तं यत्तन्मां सत्यमेव न संशयः} %2-2

\twolineshloka
{किन्तु लोका वदिष्यन्ति मामेवं रघुनन्दनः}
{कृतवान् किं कपीन्द्राय सख्यं कृत्वाऽग्निसाक्षिकम्} %2-3

\twolineshloka
{इति लोकापवादो मे भविष्यति न संशयः}
{तस्मादाह्वय भद्रं ते गत्वा युद्धाय वालिनम्} %2-4

\twolineshloka
{बाणेनैकेन तं हत्वा राज्ये त्वामभिषेचये}
{तथेति गत्वा सुग्रीवः किष्किन्धोपवनं द्रुतम्} %2-5

\twolineshloka
{कृत्वा शब्दं महानादं तमाह्वयत वालिनम्}
{तच्छ्रुत्वा भ्रातृनिनदं रोषताम्रविलोचनः} %2-6

\twolineshloka
{निर्जगाम गृहाच्छीघ्रं सुग्रीवो यत्र वानरः}
{तमापतन्तं सुग्रीवः शीघ्रं वक्षस्यताडयत्} %2-7

\twolineshloka
{सुग्रीवमपि मुष्टिभ्यां जघान क्रोधमूर्छितः}
{वाली तमपि सुग्रीव एवं क्रुद्धौ परस्परम्} %2-8

\twolineshloka
{अयुद्ध्येतामेकरूपौ दृष्ट्वा रामोऽतिविस्मितः}
{न मुमोच तदा बाणं सुग्रीववधशङ्कया} %2-9

\twolineshloka
{ततो दुद्राव सुग्रीवो वमन् रक्तं भयाकुलः}
{वाली स्वभवनं यातः सुग्रीवो राममब्रवीत्} %2-10

\twolineshloka
{किं मां घातयसे राम शत्रुणा भ्रातृरूपिणा}
{यदि मद्धनने वाञ्छा त्वमेव जहि मां विभो} %2-11

\twolineshloka
{एवं मे प्रत्ययं कृत्वा सत्यवादिन् रघूत्तम}
{उपेक्षसे किमर्थं मां शरणागतवत्सल} %2-12

\twolineshloka
{श्रुत्वा सुग्रीववचनं रामः साश्रुविलोचनः}
{आलिङ्ग्य मा स्म भैषीस्त्वं दृष्ट्वा वामेकरूपिणौ} %2-13

\twolineshloka
{मित्रघातित्वमाशङ्क्य मुक्तवान् सायकं न हि}
{इदानीमेव ते चिह्नं करिष्ये भ्रमशान्तये} %2-14

\twolineshloka
{गत्वाऽऽह्वय पुनः शत्रुं हतं द्रक्ष्यसि वालिनम्}
{रामोऽहं त्वां शपे भ्रातर्हनिष्यामि रिपुं क्षणात्} %2-15

\twolineshloka
{इत्याश्वास्य स सुग्रीवं रामो लक्ष्मणमब्रवीत्}
{सुग्रीवस्य गले पुष्पमालामामुच्य पुष्पिताम्} %2-16

\twolineshloka
{प्रेषयस्व महाभाग सुग्रीवं वालिनं प्रति}
{लक्ष्मणस्तु तदा बद्ध्वा गच्छ गच्छेति सादरम्} %2-17

\twolineshloka
{प्रेषयामास सुग्रीवं सोऽपि गत्वा तथाऽकरोत्}
{पुनरप्यद्भुतं शब्दं कृत्वा वालिनमाह्वयत्} %2-18

\twolineshloka
{तच्छ्रुत्वा विस्मितो वाली क्रोधेन महताऽऽवृतः}
{बद्ध्वा परिकरं सम्यग्गमनायोपचक्रमे} %2-19

\twolineshloka
{गच्छन्तं वालिनं तारा गृहीत्वा निषिषेध तम्}
{न गन्तव्यं त्वयेदानीं शङ्का मेऽतीव जायते} %2-20

\twolineshloka
{इदानीमेव ते भग्नः पुनरायाति सत्वरः}
{सहायो बलवांस्तस्य कश्चिन्नूनं समागतः} %2-21

\twolineshloka
{वाली तामाह हे सुभ्रु शङ्का ते व्येतु तद्गता}
{प्रिये करं परित्यज्य गच्छ गच्छामि तं रिपुम्} %2-22

\twolineshloka
{हत्वा शीघ्रं समायास्ये सहायस्तस्य को भवेत्}
{सहायो यदि सुग्रीवस्ततो हत्वोभयं क्षणात्} %2-23

\twolineshloka
{आयास्ये मा शुचः शूरः कथं तिष्ठेद् गृहे रिपुम्}
{ज्ञात्वाऽप्याह्वयमानं हि हत्वाऽऽयास्यामि सुन्दरि} %2-24

\textbf{तारोवाच}

\twolineshloka
{मत्तोऽन्यच्छृणु राजेन्द्र श्रुत्वा कुरु यथोचितम्}
{आह मामङ्गदः पुत्रो मृगयायां श्रुतं वचः} %2-25

\twolineshloka
{अयोध्याधिपतिः श्रीमान् रामो दाशरथिः किल}
{लक्ष्मणेन सह भ्रात्रा सीतया भार्यया सह} %2-26

\twolineshloka
{आगतो दण्डकारण्यं तत्र सीता हृता किल}
{रावणेन सह भ्रात्रा मार्गमाणोऽथ जानकीम्} %2-27

\twolineshloka
{आगतो ऋष्यमूकाद्रिं सुग्रीवेण समागतः}
{चकार तेन सुग्रीवः सख्यं चानलसाक्षिकम्} %2-28

\twolineshloka
{प्रतिज्ञां कृतवान् रामः सुग्रीवाय सलक्ष्मणः}
{वालिनं समरे हत्वा राजानं त्वां करोम्यहम्} %2-29

\twolineshloka
{इति निश्चित्य तौ यातौ निश्चितं शृणु मद्वचः}
{इदानीमेव ते भग्नः कथं पुनरुपागतः} %2-30

\twolineshloka
{अतस्त्वं सर्वथा वैरं त्यक्त्वा सुग्रीवमानय}
{यौवराज्येऽभिषिञ्चाशु रामं त्वं शरणं व्रज} %2-31

\twolineshloka
{पाहि मामङ्गदं राज्यं कुलं च हरिपुङ्गव}
{इत्युक्त्वाऽश्रुमुखी तारा पादयोः प्रणिपत्य तम्} %2-32

\twolineshloka
{हस्ताभ्यां चरणौ धृत्वा रुरोद भयविह्वला}
{तामालिङ्ग्य तदा वाली सस्नेहमिदमब्रवीत्} %2-33

\twolineshloka
{स्त्रीस्वभावाद्बिभेषि त्वं प्रिये नास्ति भयं मम}
{रामो यदि समायातो लक्ष्मणेन समं प्रभुः} %2-34

\twolineshloka
{तदा रामेण मे स्नेहो भविष्यति न संशयः}
{रामो नारायणः साक्षादवतीर्णोऽखिलप्रभुः} %2-35

\twolineshloka
{भूभारहरणार्थाय श्रुतं पूर्वं मयाऽनघे}
{स्वपक्षः परपक्षो वा नास्ति तस्य परात्मनः} %2-36

\twolineshloka
{आनेष्यामि गृहं साध्वि नत्वा तच्चरणाम्बुजम्}
{भजतोऽनुभजत्येष भक्तिगम्यः सुरेश्वरः} %2-37

\twolineshloka
{यदि स्वयं समायाति सुग्रीवो हन्मि तं क्षणात्}
{यदुक्तं यौवराज्याय सुगीवस्याभिषेचनम्} %2-38

\twolineshloka
{कथमाहूयमानोऽहं युद्धाय रिपुणा प्रिये}
{शूरोऽहं सर्वलोकानां सम्मतः शुभलक्षणे} %2-39

\twolineshloka
{भीतभीतमिदं वाक्यं कथं वाली वदेत्प्रिये}
{तस्माच्छोकं परित्यज्य तिष्ठ सुन्दरि वेश्मनि} %2-40

\twolineshloka
{एवमाश्वास्य तारां तां शोचन्तीमश्रुलोचनाम्}
{गतो वाली समुद्युक्तः सुग्रीवस्य वधाय सः} %2-41

\twolineshloka
{दृष्ट्वा वालिनमायान्तं सुग्रीवो भीमविक्रमः}
{उत्पपात गले बद्धपुष्पमालो मतङ्गवत्} %2-42

\twolineshloka
{मुष्टिभ्यां ताडयामास वालिनं सोऽपि तं तथा}
{अहन्वाली च सुग्रीवं सुग्रीवो वालिनं तथा} %2-43

\twolineshloka
{रामं विलोकयन्नेव सुग्रीवो युयुधे युधि}
{इत्येवं युद्ध्यमानौ तौ दृष्ट्वा रामः प्रतापवान्} %2-44

\twolineshloka
{बाणमादाय तूणीरादैन्द्रे धनुषि सन्दधे}
{आकृष्य कर्णपर्यन्तमदृश्यो वृक्षखण्डगः} %2-45

\twolineshloka
{निरीक्ष्य वालिनं सम्यग्लक्ष्यं तद्धृदयं हरिः}
{उत्ससर्जाशनिसमं महावेगं महाबलः} %2-46

\twolineshloka
{बिभेद स शरो वक्षो वालिनः कम्पयन् महीम्}
{उत्पपात महाशब्दं मुञ्चन् स निपपात ह} %2-47

\threelineshloka
{तदा मुहूर्त्तं निःसंज्ञो भूत्वा चेतनमाप सः}
{ततो वाली ददर्शाग्रे रामं राजीवलोचनम्}
{धनुरालम्ब्य वामेन हस्तेनान्येन सायकम्} %2-48

\twolineshloka
{बिभ्राणं चीरवसनं जटामुकुटधारिणम्}
{विशालवक्षसं भ्राजद्वनमालाविभूषितम्} %2-49

\twolineshloka
{पीनचार्वायतभुजं नवदूर्वादलच्छविम्}
{सुग्रीवलक्ष्मणाभ्यां च पार्श्वयोः परिसेवितम्} %2-50

\twolineshloka
{विलोक्य शनकैः प्राह वाली रामं विगर्हयन्}
{किं मयाऽपकृतं राम तव येन हतोऽस्म्यहम्} %2-51

\twolineshloka
{राजधर्ममविज्ञाय गर्हितं कर्म ते कृतम्}
{वृक्षखण्डे तिरोभूत्वा त्यजता मयि सायकम्} %2-52

\twolineshloka
{यशः किं लप्स्यसे राम चोरवत्कृतसङ्गरः}
{यदि क्षत्रियदायादो मनोर्वंशसमुद्भवः} %2-53

\twolineshloka
{युद्धं कृत्वा समक्षं मे प्राप्यसे तत्फलं तदा}
{सुग्रीवेण कृतं किं ते मया वा न कृतं किमु} %2-54

\twolineshloka
{रावणेन हृता भार्या तव राम महावने}
{सुग्रीवं शरणं यातस्तदर्थमिति शुश्रुम} %2-55

\twolineshloka
{बत राम न जानीषे मद्बलं लोकविश्रुतम्}
{रावणं सकुलं बद्ध्वा ससीतं लङ्कया सह} %2-56

\twolineshloka
{आनयामि मुहूर्त्तार्द्धाद्यदि चेच्छामि राघव}
{धर्मिष्ठ इति लोकेऽस्मिन् कथ्यसे रघुनन्दन} %2-57

\twolineshloka
{वानरं व्याधवद्धत्वा धर्मं कं लप्स्यसे वद}
{अभक्ष्यं वानरं मांसं हत्वा मां किं करिष्यसि} %2-58

\twolineshloka
{इत्येवं बहु भाषन्तं वालिनं राघवोऽब्रवीत्}
{धर्मस्य गोप्ता लोकेऽस्मिंश्चरामि सशरासनः} %2-59

\twolineshloka
{अधर्मकारिणं हत्वा सद्धर्मं पालयाम्यहम्}
{दुहिता भगिनी भ्रातुर्भार्या चैव तथा स्नुषा} %2-60

\twolineshloka
{समा यो रमते तासामेकामपि विमूढधीः}
{पातकी स तु विज्ञेयः स वध्यो राजभिः सदा} %2-61

\twolineshloka
{त्वं तु भ्रातुः कनिष्ठस्य भार्यायां रमसे बलात्}
{अतो मया धर्मविदा हतोऽसि वनगोचर} %2-62

\twolineshloka
{त्वं कपित्वान्न जानीषे महान्तो विचरन्ति यत्}
{लोकं पुनानाः सञ्चारैरतस्तान्नातिभाषयेत्} %2-63

\twolineshloka
{तच्छ्रुत्वा भयसन्त्रस्तो ज्ञात्वा रामं रमापतिम्}
{वाली प्रणम्य रभसाद्रामं वचनमब्रवीत्} %2-64

\twolineshloka
{राम राम महाभाग जाने त्वां परमेश्वरम्}
{अजानता मया किञ्चिदुक्तं तत्क्षन्तुमर्हसि} %2-65

\twolineshloka
{साक्षात्त्वच्छरघातेन विशेषेण तवाग्रतः}
{त्यजाम्यसून् महायोगिदुर्लभं तव दर्शनम्} %2-66

\twolineshloka
{यन्नाम विवशो गृह्णन् म्रियमाणः परं पदम्}
{याति साक्षात्स एवाद्य मुमूर्षोर्मे पुरः स्थितः} %2-67

\twolineshloka
{देव जानामि पुरुषं त्वां श्रियं जानकीं शुभाम्}
{रावणस्य वधार्थाय जातं त्वां ब्रह्मणाऽर्थितम्} %2-68

\twolineshloka
{अनुजानीहि मां राम यान्तं त्वत्पदमुत्तमम्}
{मम तुल्यबले बाले अङ्गदे त्वं दयां कुरु} %2-69

\threelineshloka
{विशल्यं कुरु मे राम हृदयं पाणिना स्पृशन्}
{तथेति बाणमुद्धृत्य रामः पस्पर्श पाणिना}
{त्यक्त्वा तद्वानरं देहममरेन्द्रोऽभवत्क्षणात्} %2-70

\fourlineindentedshloka
{वाली रघूत्तमशराभिहतो विमृष्टो}
{रामेण शीतलकरेण सुखाकरेण}
{सद्यो विमुच्य कपिदेहमनन्यलभ्यम्}
{प्राप्तं पदं परमहंसगणैर्दुरापम्} %2-71

{॥इति श्रीमदध्यात्मरामायणे उमामहेश्वरसंवादे किष्किन्धाकाण्डे
द्वितीयः सर्गः॥२॥
}
%%%%%%%%%%%%%%%%%%%%



\sect{तृतीयः सर्गः}

\twolineshloka
{निहते वालिनि रणे रामेण परमात्मना}
{दुद्रुवुर्वानराः सर्वे किष्किन्धां भयविह्वलाः} %3-1

\twolineshloka
{तारामूचुर्महाभागे हतो वाली रणाजिरे}
{अङ्गदं परिरक्षाद्य मन्त्रिणः परिनोदय} %3-2

\twolineshloka
{चतुर्द्वारकपाटादीन् बद्ध्वा रक्षामहे पुरीम्}
{वानराणां तु राजानमङ्गदं कुरु भामिनि} %3-3

\twolineshloka
{निहतं वालिनं श्रुत्वा तारा शोकविमूर्छिता}
{अताडयत्स्वपाणिभ्यां शिरो वक्षश्च भूरिशः} %3-4

\twolineshloka
{किमङ्गदेन राज्येन नगरेण धनेन वा}
{इदानीमेव निधनं यास्यामि पतिना सह} %3-5

\twolineshloka
{इत्युक्त्वा त्वरिता तत्र रुदती मुक्तमूर्धजा}
{ययौ ताराऽतिशोकार्ता यत्र भर्तृकलेवरम्} %3-6

\twolineshloka
{पतितं वालिनं दृष्ट्वा रक्तैः पांसुभिरावृतम्}
{रुदती नाथनाथेति पतिता तस्य पादयोः} %3-7

\twolineshloka
{करुणं विलपन्ती सा ददर्श रघुनन्दनम्}
{राम मां जहि बाणेन येन वाली हतस्त्वया} %3-8

\twolineshloka
{गच्छामि पतिसालोक्यं पतिर्मामभिकाङ्क्षते}
{स्वर्गेऽपि न सुखं तस्य मां विना रघुनन्दन} %3-9

\twolineshloka
{पत्नीवियोगजं दुःखमनुभूतं त्वयाऽनघ}
{वालिने मां प्रयच्छाशु पत्नीदानफलं भवेत्} %3-10

\twolineshloka
{सुग्रीव त्वं सुखं राज्यं दापितं वालिघातिना}
{रामेण रुमया सार्धं भुङ्क्ष्व सापत्नवर्जितम्} %3-11

\twolineshloka
{इत्येवं विलपन्तीं तां तारां रामो महामनाः}
{सान्त्वयामास दयया तत्त्वज्ञानोपदेशतः} %3-12

\twolineshloka
{किं भीरु शोचसि व्यर्थं शोकस्याविषयं पतिम्}
{पतिस्तवायं देहो वा जीवो वा वद तत्त्वतः} %3-13

\twolineshloka
{पञ्चात्मको जडो देहस्त्वङ्मांसरुधिरास्थिमान्}
{कालकर्मगुणोत्पन्नः सोऽप्यास्तेऽद्यापि ते पुरः} %3-14

\twolineshloka
{मन्यसे जीवमात्मानं जीवस्तर्हि निरामयः}
{न जायते न म्रियते न तिष्ठति न गच्छति} %3-15

\threelineshloka
{न स्त्री पुमान्वा षण्ढो वा जीवः सर्वगतोऽव्ययः}
{एक एवाद्वितीयोऽयमाकाशवदलेपकः}
{नित्यो ज्ञानमयः शुद्धः स कथं शोकमर्हति} %3-16

\textbf{तारोवाच}

\twolineshloka
{देहोऽचित्काष्ठवद्राम जीवो नित्यश्चिदात्मकः}
{सुखदुःखादिसम्बन्धः कस्य स्याद्राम मे वद} %3-17

\textbf{श्रीराम उवाच}

\twolineshloka
{अहङ्कारादिसम्बन्धो यावद्देहेन्द्रियैः सह}
{संसारस्तावदेव स्यादात्मनस्त्वविवेकिनः} %3-18

\twolineshloka
{मिथ्यारोपितसंसारो न स्वयं विनिवर्तते}
{विषयान् ध्यायमानस्य स्वप्ने मिथ्यागमो यथा} %3-19

\twolineshloka
{अनाद्यविद्यासम्बन्धात्तत्कार्याहङ्कृतेस्तथा}
{संसारोऽपार्थकोऽपि स्याद्रागद्वेषादिसङ्कुलः} %3-20

\twolineshloka
{मन एव हि संसारो बन्धश्चैव मनः शुभे}
{आत्मा मनःसमानत्वमेत्य तद्गतबन्धभाक्} %3-21

\twolineshloka
{यथा विशुद्धः स्फटिकोऽलक्तकादिसमीपगः}
{तत्तद्वर्णयुगाभाति वस्तुतो नास्ति रञ्जनम्} %3-22

\twolineshloka
{बुद्धीन्द्रियादिसामीप्यादात्मनः संसृतिर्बलात्}
{आत्मा स्वलिङ्गं तु मनः परिगृह्य तदुद्भवान्} %3-23

\twolineshloka
{कामान् जुषन् गुणैर्बद्धः संसारे वर्ततेऽवशः}
{आदौ मनोगुणान् सृष्ट्वा ततः कर्माण्यनेकधा} %3-24

\twolineshloka
{शुक्ललोहितकृष्णानि गतयस्तत्समानतः}
{एवं कर्मवशाज्जीवो भ्रमत्याभूतसम्प्लवम्} %3-25

\twolineshloka
{सर्वोपसंहृतौ जीवो वासनाभिः स्वकर्मभिः}
{अनाद्यविद्यावशगस्तिष्ठत्यभिनिवेशतः} %3-26

\twolineshloka
{सृष्टिकाले पुनः पूर्ववासनामानसैः सह}
{जायते पुनरप्येवं घटीयन्त्रमिवावशः} %3-27

\twolineshloka
{यदा पुण्यविशेषेण लभते सङ्गतिं सताम्}
{मद्भक्तानां सुशान्तानां तदा मद्विषया मतिः} %3-28

\twolineshloka
{मत्कथाश्रवणे श्रद्धा दुर्लभा जायते ततः}
{ततः स्वरूपविज्ञानमनायासेन जायते} %3-29

\twolineshloka
{तदाऽऽचार्यप्रसादेन वाक्यार्थज्ञानतः क्षणात्}
{देहेन्द्रियमनःप्राणाहङ्कृतिभ्यः पृथक् स्थितम्} %3-30

\twolineshloka
{स्वात्मानुभवतः सत्यमानन्दात्मानमद्वयम्}
{ज्ञात्वा सद्यो भवेन्मुक्तः सत्यमेव मयोदितम्} %3-31

\twolineshloka
{एवं मयोदितं सम्यगालोचयति योऽनिशम्}
{तस्य संसारदुःखानि न स्पृशन्ति कदाचन} %3-32

\twolineshloka
{त्वमप्येतन्मया प्रोक्तमालोचय विशुद्धधीः}
{न स्पृश्यसे दुःखजालैः कर्मबन्धाद्विमोक्ष्यसे} %3-33

\twolineshloka
{पूर्वजन्मनि ते सुभ्रु कृता मद्भक्तिरुत्तमा}
{अतस्तव विमोक्षाय रूपं मे दर्शितं शुभे} %3-34

\twolineshloka
{ध्यात्वा मद्रूपमनिशमालोचय मयोदितम्}
{प्रवाहपतितं कार्यं कुर्वन्त्यपि न लिप्यसे} %3-35

\twolineshloka
{श्रीरामेणोदितं सर्वं श्रुत्वा ताराऽतिविस्मिता}
{देहाभिमानजं शोकं त्यक्त्वा नत्वा रघूत्तमम्} %3-36

\twolineshloka
{आत्मानुभवसन्तुष्टा जीवन्मुक्ता बभूव ह}
{क्षणसङ्गममात्रेण रामेण परमात्मना} %3-37

\twolineshloka
{अनादिबन्धं निर्धूय मुक्ता साऽपि विकल्मषा}
{सुग्रीवोऽपि च तच्छ्रुत्वा रामवक्त्रात्समीरितम्} %3-38

\twolineshloka
{जहावज्ञानमखिलं स्वस्थचित्तोऽभवत्तदा}
{ततः सुग्रीवमाहेदं रामो वानरपुङ्गवम्} %3-39

\twolineshloka
{भ्रातुर्ज्येष्ठस्य पुत्रेण यदुक्तं साम्परायिकम्}
{कुरु सर्वं यथान्यायं संस्कारादि ममाऽऽज्ञया} %3-40

\twolineshloka
{तथेति बलिभिर्मुख्यैर्वानरैः परिणीय तम्}
{वालिनं पुष्पके क्षिप्त्वा सर्वराजोपचारकैः} %3-41

\twolineshloka
{भेरीदुन्दुभिनिर्घोषैर्ब्राह्मणैर्मन्त्रिभिः सह}
{यूथपैर्वानरैः पौरैस्तारया चाङ्गदेन च} %3-42

\twolineshloka
{गत्वा चकार तत्सर्वं यथाशास्त्रं प्रयत्नतः}
{स्नात्वा जगाम रामस्य समीपं मन्त्रिभिः सह} %3-43

\twolineshloka
{नत्वा रामस्य चरणौ सुग्रीवः प्राह हृष्टधीः}
{राज्यं प्रशाधि राजेन्द्र वानराणां समृद्धिमत्} %3-44

\twolineshloka
{दासोऽहं ते पादपद्मं सेवे लक्ष्मणवच्चिरम्}
{इत्युक्तो राघवः प्राह सुग्रीवं सस्मितं वचः} %3-45

\twolineshloka
{त्वमेवाहं न सन्देहः शीघ्रं गच्छ ममऽऽज्ञया}
{पुरराज्याधिपत्ये त्वं स्वात्मानमभिषेचय} %3-46

\twolineshloka
{नगरं न प्रवेक्ष्यामि चतुर्दश समाः सखे}
{आगमिष्यति मे भ्राता लक्ष्मणः पत्तनं तव} %3-47

\twolineshloka
{अङ्गदं यौवराज्ये त्वमभिषेचय सादरम्}
{अहं समीपे शिखरे पर्वतस्य सहानुजः} %3-48

\twolineshloka
{वत्स्यामि वर्षदिवसांस्ततस्त्वं यत्नवान् भव}
{किञ्चित्कालं पुरे स्थित्वा सीतायाः परिमार्गणे} %3-49

\twolineshloka
{साष्टाङ्गं प्रणिपत्याह सुग्रीवो रामपादयोः}
{यदाज्ञापयसे देव तत्तथैव करोम्यहम्} %3-50

\twolineshloka
{अनुज्ञातश्च रामेण सुग्रीवस्तु सलक्ष्मणः}
{गत्वा पुरं तथा चक्रे यथा रामेण चोदितः} %3-51

\twolineshloka
{सुग्रिवेण यथान्यायं पूजितो लक्ष्मणस्तदा}
{आगत्य राघवं शीघ्रं प्रणिपत्योपतस्थिवान्} %3-52

\twolineshloka
{ततो रामो जगामाऽऽशु लक्ष्मणेन समन्वितः}
{प्रवर्षणगिरेरूर्ध्वं शिखरं भूरिविस्तरम्} %3-53

\threelineshloka
{तत्रैकं गह्वरं दृष्ट्वा स्फाटिकं दीप्तिमच्छुभम्}
{वर्षवातातपसहं फलमूलसमीपगम्}
{वासाय रोचयामास तत्र रामः सलक्ष्मणः} %3-54

\twolineshloka
{दिव्यमूलफलपुष्पसंयुते मौक्तिकोपमजलौघपल्वले}
{चित्रवर्णमृगपक्षिशोभिते पर्वते रघुकुलोत्तमोऽवसत्} %3-55

{॥इति श्रीमदध्यात्मरामायणे उमामहेश्वरसंवादे किष्किन्धाकाण्डे
तृतीयः सर्गः॥३॥
}
%%%%%%%%%%%%%%%%%%%%



\sect{चतुर्थः सर्गः}

\twolineshloka
{तत्र वार्षिकदिनानि राघवो लीलया मणिगुहासु सञ्चरन्}
{पक्वमूलफलभोगतोषितो लक्ष्मणेन सहितोऽवसत्सुखम्} %4-1

\twolineshloka
{वातनुन्नजलपूरितमेघानन्तरस्तनितवैद्युतगर्भान्}
{वीक्ष्य विस्मयमगाद्गजयूथान् यद्वदाहितसुकाञ्चनकक्षान्} %4-2

\twolineshloka
{नवघासं समास्वाद्य हृष्टपुष्टमृगद्विजाः}
{धावन्तो परितो रामं वीक्ष्य विस्फारितेक्षणाः} %4-3

\twolineshloka
{न चलन्ति सदा ध्याननिष्ठा इव मुनीश्वराः}
{रामं मानुषरूपेण गिरिकाननभूमिषु} %4-4

\twolineshloka
{चरन्तं परमात्मानं ज्ञात्वा सिद्धगणा भुवि}
{मृगपक्षिगणा भूत्वा राममेवानुसेविरे} %4-5

\twolineshloka
{सौमित्रिरेकदा राममेकान्ते ध्यानतत्परम्}
{समाधिविरमे भक्त्या प्रणयाद्विनयान्वितः} %4-6

\twolineshloka
{अब्रवीद्देव ते वाक्यात्पूर्वोक्ताद्विगतो मम}
{अनाद्यविद्यासम्भूतः संशयो हृदि संस्थितः} %4-7

\twolineshloka
{इदानीं श्रोतुमिच्छामि क्रियामार्गेण राघव}
{भवदाराधनं लोके यथा कुर्वन्ति योगिनः} %4-8

\twolineshloka
{इदमेव सदा प्राहुर्योगिनो मुक्तिसाधनम्}
{नारदोऽपि तथा व्यासो ब्रह्मा कमलसम्भवः} %4-9

\threelineshloka
{ब्रह्मक्षत्रादिवर्णानामाश्रमाणां च मोक्षदम्}
{स्त्रीशूद्राणां च राजेन्द्र सुलभं मुक्तिसाधनम्}
{तव भक्ताय मे भ्रात्रे ब्रूहि लोकोपकारकम्} %4-10

\textbf{श्रीराम उवाच}

\twolineshloka
{मम पूजाविधानस्य नान्तोऽस्ति रघुनन्दन}
{तथाऽपि वक्ष्ये सङ्क्षेपाद्यथावदनुपूर्वशः} %4-11

\twolineshloka
{स्वगृह्योक्तप्रकारेण द्विजत्वं प्राप्य मानवः}
{सकाशात्सद्गुरोर्मन्त्रं लब्ध्वा मद्भक्तिसंयुतः} %4-12

\twolineshloka
{तेन सन्दर्शितविधिर्मामेवाराधयेत्सुधीः}
{हृदये वाऽनले वार्चेत्प्रतिमादौ विभावसौ} %4-13

\twolineshloka
{शालग्रामशिलायां वा पूजयेन्मामतन्द्रितः}
{प्रातःस्नानं प्रकुर्वीत प्रथमं देहशुद्धये} %4-14

\twolineshloka
{वेदतन्त्रोदितैर्मन्त्रैर्मृल्लेपनविधानतः}
{सन्ध्यादि कर्म यन्नित्यं तत्कुर्याद्विधिना बुधः} %4-15

\twolineshloka
{सङ्कल्पमादौ कुर्वीत सिद्ध्यर्थं कर्मणां सुधीः}
{स्वगुरुं पूजयेद्भक्त्या मद्बुद्ध्या पूजको मम} %4-16

\twolineshloka
{शिलायां स्नपनं कुर्यात्प्रतिमासु प्रमार्जनम्}
{प्रसिद्धैर्गन्धपुष्पाद्यैर्मत्पूजा सिद्धिदायिका} %4-17

\twolineshloka
{अमायिकोऽनुवृत्त्या मां पूजयेन्नियतव्रतः}
{प्रतिमादिष्वलङ्कारः प्रियो मे कुलनन्दन} %4-18

\twolineshloka
{अग्नौ यजेत हविषा भास्करे स्थण्डिले यजेत्}
{भक्तेनोपहृतं प्रीत्यै श्रद्धया मम वार्यपि} %4-19

\twolineshloka
{किं पुनर्भक्ष्यभोज्यादि गन्धपुष्पाक्षतादिकम्}
{पूजाद्रव्याणि सर्वाणि सम्पाद्यैवं समारभेत्} %4-20

\twolineshloka
{चैलाजिनकुशैः सम्यगासनं परिकल्पयेत्}
{तत्रोपविश्य देवस्य सम्मुखे शुद्धमानसः} %4-21

\twolineshloka
{ततो न्यासं प्रकुर्वीत मातृकाबहिरान्तरम्}
{केशवादि ततः कुर्यात्तत्त्वन्यासं ततः परम्} %4-22

\twolineshloka
{मन्मूर्तिपञ्जरन्यासं मन्त्रन्यासं ततो न्यसेत्}
{प्रतिमादावपि तथा कुर्यान्नित्यमतन्द्रितः} %4-23

\twolineshloka
{कलशं स्वपुरो वामे क्षिपेत्पुष्पादि दक्षिणे}
{अर्घ्यपाद्यप्रदानार्थं मधुपर्कार्थमेव च} %4-24

\twolineshloka
{तथैवाचमनार्थं तु न्यसेत्पात्रचतुष्टयम्}
{हृत्पद्मे भानुविमले मत्कलां जीवसंज्ञिताम्} %4-25

\twolineshloka
{ध्यायेत्स्वदेहमखिलं तया व्याप्तमरिन्दम}
{तामेवावाहयेन्नित्यं प्रतिमादिषु मत्कलाम्} %4-26

\twolineshloka
{पाद्यार्घ्याचमनीयाद्यैः स्नानवस्त्रविभूषणैः}
{यावच्छक्योपचारैर्वा त्वर्चयेन्माममायया} %4-27

\twolineshloka
{विभवे सति कर्पूरकुङ्कुमागरुचन्दनैः}
{अर्चयेन्मन्त्रवन्नित्यं सुगन्धकुसुमैः शुभैः} %4-28

\twolineshloka
{दशावरणपूजां वै ह्यागमोक्तां प्रकारयेत्}
{नीराजनैर्धूपदीपैर्नैवेद्यैर्बहुविस्तरैः} %4-29

\twolineshloka
{श्रद्धयोपहरेन्नित्यं श्रद्धाभुगहमीश्वरः}
{होमं कुर्यात्प्रयत्नेन विधिना मन्त्रकोविदः} %4-30

\twolineshloka
{अगस्त्येनोक्तमार्गेण कुण्डेनागमवित्तमः}
{जुहुयान्मूलमन्त्रेण पुंसूक्तेनाथवा बुधः} %4-31

\twolineshloka
{अथवौपासनाग्नौ वा चरुणा हविषा तथा}
{तप्तजाम्बूनदप्रख्यं दिव्याभरणभूषितम्} %4-32

\twolineshloka
{ध्यायेदनलमध्यस्थं होमकाले सदा बुधः}
{पार्षदेभ्यो बलिं दत्त्वा होमशेषं समापयेत्} %4-33

\twolineshloka
{ततो जपं प्रकुर्वीत ध्यायेन्मां यतवाक् स्मरन्}
{मुखवासं च ताम्बूलं दत्त्वा प्रीतिसमन्वितः} %4-34

\twolineshloka
{मदर्थे नृत्यगीतादि स्तुतिपाठादि कारयेत्}
{प्रणमेद्दण्डवद्भूमौ हृदये मां निधाय च} %4-35

\twolineshloka
{शिरस्याधाय मद्दत्तं प्रसादं भावनामयम्}
{पाणिभ्यां मत्पदे मूर्ध्नि गृहीत्वा भक्तिसंयुतः} %4-36

\twolineshloka
{रक्ष मां घोरसंसारादित्युक्त्वा प्रणमेत्सुधीः}
{उद्वासयेद्यथापूर्वं प्रत्यग्ज्योतिषि संस्मरन्} %4-37

\twolineshloka
{एवमुक्तप्रकारेण पूजयेद्विधिवद्यदि}
{इहामुत्र च संसिद्धिं प्राप्नोति मदनुग्रहात्} %4-38

\twolineshloka
{मद्भक्तो यदि मामेवं पूजां चैव दिने दिने}
{करोति मम सारूप्यं प्राप्नोत्येव न संशयः} %4-39

\fourlineindentedshloka
{इदं रहस्यं परमं च पावनम्}
{मयैव साक्षात्कथितं सनातनम्}
{पठत्यजस्रं यदि वा शृणोति यः}
{स सर्वपूजाफलभाङ्न संशयः} %4-40

\twolineshloka
{एवं परात्मा श्रीरामः क्रियायोगमनुत्तमम्}
{पृष्टः प्राह स्वभक्ताय शेषांशाय महात्मने} %4-41

\twolineshloka
{पुनः प्राकृतवद्रामो मायामालम्ब्य दुःखितः}
{हा सीतेति वदन्नैव निद्रां लेभे कथञ्चन} %4-42

\twolineshloka
{एतस्मिन्नन्तरे तत्र किष्किन्धायां सुबुद्धिमान्}
{हनूमान् प्राह सुग्रीवमेकान्ते कपिनायकम्} %4-43

\twolineshloka
{शृणु राजन् प्रवक्ष्यामि तवैव हितमुत्तमम्}
{रामेण ते कृतः पूर्वमुपकारो ह्यनुत्तमः} %4-44

\twolineshloka
{कृतघ्नवत्त्वया नूनं विस्मृतः प्रतिभाति मे}
{त्वत्कृते निहतो वाली वीरस्त्रैलोक्यसम्मतः} %4-45

\twolineshloka
{राज्ये प्रतिष्ठितोऽसि त्वं तारां प्राप्तोऽसि दुर्लभाम्}
{स रामः पर्वतस्याग्रे भ्रात्रा सह वसन् सुधीः} %4-46

\twolineshloka
{त्वदागमनमेकाग्रमीक्षते कार्यगौरवात्}
{त्वं तु वानरभावेन स्त्रीसक्तो नावबुद्ध्यसे} %4-47

\twolineshloka
{करोमीति प्रतिज्ञाय सीतायाः परिमार्गणम्}
{न करोषि कृतघ्नस्त्वं हन्यसे वालिवद्द्रुतम्} %4-48

\twolineshloka
{हनूमद्वचनं श्रुत्वा सुग्रीवो भयविह्वलः}
{प्रत्युवाच हनूमन्तं सत्यमेव त्वयोदितम्} %4-49

\twolineshloka
{शीघ्रं कुरु ममाज्ञां त्वं वानराणां तरस्विनाम्}
{सहस्राणि दशेदानीं प्रेषयाऽऽशु दिशो दश} %4-50

\twolineshloka
{सप्तद्वीपगतान् सर्वान् वानरानानयन्तु ते}
{पक्षमध्ये समायान्तु सर्वे वानरपुङ्गवाः} %4-51

\twolineshloka
{ये पक्षमतिवर्तन्ते ते वध्या मे न संशयः}
{इत्याज्ञाप्य हनूमन्तं सुग्रीवो गृहमाविशत्} %4-52

\twolineshloka
{सुग्रीवाज्ञां पुरस्कृत्य हनूमान् मन्त्रिसत्तमः}
{तत्क्षणे प्रेषयामास हरीन् दश दिशः सुधीः} %4-53

\fourlineindentedshloka
{अगणितगुणसत्त्वान् वायुवेगप्रचारान्}
{वनचरगणमुख्यान् पर्वताकाररूपान्}
{पवनहितकुमारः प्रेषयामास दूता\-}
{नतिरभसतरात्मा दानमानादितृप्तान्} %4-54

{॥इति श्रीमदध्यात्मरामायणे उमामहेश्वरसंवादे किष्किन्धाकाण्डे
चतुर्थः सर्गः॥४॥
}
%%%%%%%%%%%%%%%%%%%%



\sect{पञ्चमः सर्गः}

\twolineshloka
{रामस्तु पर्वतस्याग्रे मणिसानौ निशामुखे}
{सीताविरहजं शोकमसहन्निदमब्रवीत्} %5-1

\twolineshloka
{पश्य लक्ष्मण मे सीता राक्षसेन हृता बलात्}
{मृताऽमृता वा निश्चेतुं न जानेऽद्यापि भामिनीम्} %5-2

\twolineshloka
{जीवतीति मम ब्रूयात्कश्चिद्वा प्रियकृत् स मे}
{यदि जानामि तां साध्वीं जीवन्तीं यत्र कुत्र वा} %5-3

\twolineshloka
{हठादेवाहरिष्यामि सुधामिव पयोनिधेः}
{प्रतिज्ञां शृणु मे भ्रातर्येन मे जनकात्मजा} %5-4

\twolineshloka
{नीता तं भस्मसात्कुर्यां सपुत्रबलवाहनम्}
{हे सीते चन्द्रवदने वसन्ती राक्षसालये} %5-5

\twolineshloka
{दुःखार्त्ता मामपश्यन्ती कथं प्राणान् धरिष्यसि}
{चन्द्रोऽपि भानुवद्भाति मम चन्द्राननां विना} %5-6

\twolineshloka
{चन्द्र त्वं जानकीं स्पृष्ट्वा करैर्मां स्पृश शीतलैः}
{सुग्रीवोऽपि दयाहीनो दुःखितं मां न पश्यति} %5-7

\twolineshloka
{राज्यं निष्कण्टकं प्राप्य स्त्रीभिः परिवृतो रहः}
{कृतघ्नो दृश्यते व्यक्तं पानासक्तोऽतिकामुकः} %5-8

\twolineshloka
{नाऽऽयाति शरदं पश्यन्नपि मार्गयितुं प्रियाम्}
{पूर्वोपकारिणं दुष्टः कृतघ्नो विस्मृतो हि माम्} %5-9

\twolineshloka
{हन्मि सुग्रीवमप्येवं सपुरं सहबान्धवम्}
{वाली यथा हतो मेऽद्य सुग्रीवोऽपि तथा भवेत्} %5-10

\twolineshloka
{इति रुष्टं समालोक्य राघवं लक्ष्मणोऽब्रवीत्}
{इदानीमेव गत्वाऽहं सुग्रीवं दुष्टमानसम्} %5-11

\twolineshloka
{मामाज्ञापय हत्वा तमायास्ये राम तेऽन्तिकम्}
{इत्युक्त्वा धनुरादाय स्वयं तूणीरमेव च} %5-12

\twolineshloka
{गन्तुमभ्युद्यतं वीक्ष्य रामो लक्ष्मणमब्रवीत्}
{न हन्तव्यस्त्वया वत्स सुग्रीवो मे प्रियः सखा} %5-13

\twolineshloka
{किन्तु भीषय सुग्रीवं वालिवत्त्वं हनिष्यसे}
{इत्युक्त्वा शीघ्रमादाय सुग्रीवप्रतिभाषितम्} %5-14

\twolineshloka
{आगत्य पश्चाद्यत्कार्यं तत्करिष्याम्यसंशयम्}
{तथेति लक्ष्मणोऽगच्छत्त्वरितो भीमविक्रमः} %5-15

\twolineshloka
{किष्किन्धां प्रति कोपेन निर्दहन्निव वानरान्}
{सर्वज्ञो नित्यलक्ष्मीको विज्ञानात्माऽपि राघवः} %5-16

\twolineshloka
{सीतामनुशुशोचार्त्तः प्राकृतः प्राकृतामिव}
{बुद्ध्यादिसाक्षिणस्तस्य मायाकार्यातिवर्तिनः} %5-17

\twolineshloka
{रागादिरहितस्यास्य तत्कार्यं कथमुद्भवेत्}
{ब्रह्मणोक्तमृतं कर्तुं राज्ञो दशरथस्य हि} %5-18

\twolineshloka
{तपसः फलदानाय जातो मानुषवेषधृक्}
{मायया मोहिताः सर्वे जना अज्ञानसंयुताः} %5-19

\twolineshloka
{कथमेषां भवेन्मोक्ष इति विष्णुर्विचिन्तयन्}
{कथां प्रथयितुं लोके सर्वलोकमलापहाम्} %5-20

\twolineshloka
{रामायणाभिधां रामो भूत्वा मानुषचेष्टकः}
{क्रोधं मोहं च कामं च व्यवहारार्थसिद्धये} %5-21

\twolineshloka
{तत्तत्कालोचितं गृह्णन् मोहयत्यवशाः प्रजाः}
{अनुरक्त इवाशेषगुणेषु गुणवर्जितः} %5-22

\twolineshloka
{विज्ञानमूर्तिर्विज्ञानशक्तिः साक्ष्यगुणान्वितः}
{अतः कामादिभिर्नित्यमविलिप्तो यथा नभः} %5-23

\threelineshloka
{विन्दन्ति मुनयः केचिज्जानन्ति जनकादयः}
{तद्भक्ता निर्मलात्मानः सम्यग्जानन्ति नित्यदा}
{भक्तचित्तानुसारेण जायते भगवानजः} %5-24

\twolineshloka
{लक्ष्मणोऽपि तदा गत्वा किष्किन्धानगरान्तिकम्}
{ज्याघोषमकरोत्तीव्रं भीषयन् सर्ववानरान्} %5-25

\twolineshloka
{तं दृष्ट्वा प्राकृतास्तत्र वानरा वप्रमूर्धनि}
{चक्रुः किलकिलाशब्दं धृतपाषाणपादपाः} %5-26

\twolineshloka
{तान् दृष्ट्वा क्रोधताम्राक्षो वानरान् लक्ष्मणस्तदा}
{निर्मूलान् कर्तुमुद्युक्तो धनुरानम्य वीर्यवान्} %5-27

{ततः शीघ्रं समाप्लुत्य ज्ञात्वा लक्ष्मणमागतम्॥२८॥} %5-28
\refstepcounter{shlokacount}


\twolineshloka
{निवार्य वानरान् सर्वानङ्गदो मन्त्रिसत्तमः}
{गत्वा लक्ष्मणसामीप्यं प्रणनाम स दण्डवत्} %5-29

\twolineshloka
{ततोऽङ्गदं परिष्वज्य लक्ष्मणः प्रियवर्धनः}
{उवाच वत्स गच्छ त्वं पितृव्याय निवेदय} %5-30

\twolineshloka
{मामागतं राघवेण चोदितं रौद्रमूर्तिना}
{तथेति त्वरितं गत्वा सुग्रीवाय न्यवेदयत्} %5-31

\twolineshloka
{लक्ष्मणः क्रोधताम्राक्षः पुरद्वारि बहिः स्थितः}
{तच्छ्रुत्वाऽतीव सन्त्रस्तः सुग्रीवो वानरेश्वरः} %5-32

\twolineshloka
{आहूय मन्त्रिणां श्रेष्ठं हनूमन्तमथाब्रवीत्}
{गच्छ त्वमङ्गदेनाशु लक्ष्मणं विनयान्वितः} %5-33

\twolineshloka
{सान्त्वयन् कोपितं वीरं शनैरानय सादरम्}
{प्रेषयित्वा हनूमन्तं तारामाह कपीश्वरः} %5-34

\twolineshloka
{त्वं गच्छ सान्त्वयन्ती तं लक्ष्मणं मृदुभाषितैः}
{शान्तमन्तःपुरं नीत्वा पश्चाद्दर्शय मेऽनघे} %5-35

\twolineshloka
{भवत्विति ततस्तारा मध्यकक्षं समाविशत्}
{हनुमानङ्गदेनैव सहितो लक्ष्मणान्तिकम्} %5-36

\twolineshloka
{गत्वा ननाम शिरसा भक्त्या स्वागतमब्रवीत्}
{एहि वीर महाभाग भवद्गृहमशङ्कितम्} %5-37

\twolineshloka
{प्रविश्य राजदारादीन् दृष्ट्वा सुग्रीवमेव च}
{यदाज्ञापयसे पश्चात्तत्सर्वं करवाणि भोः} %5-38

\twolineshloka
{इत्युक्त्वा लक्ष्मणं भक्त्या करे गृह्य स मारुतिः}
{आनयामास नगरमध्याद्राजगृहं प्रति} %5-39

\twolineshloka
{पश्यंस्तत्र महासौधान् यूथपानां समन्ततः}
{जगाम भवनं राज्ञः सुरेन्द्रभवनोपमम्} %5-40

\twolineshloka
{मध्यकक्षे गता तत्र तारा ताराधिपानना}
{सर्वाभरणसम्पन्ना मदरक्तान्तलोचना} %5-41

\twolineshloka
{उवाच लक्ष्मणं नत्वा स्मितपूर्वाभिभाषिणी}
{एहि देवर भद्रं ते साधुस्त्वं भक्तवत्सलः} %5-42

\twolineshloka
{किमर्थं कोपमाकार्षीर्भक्ते भृत्ये कपीश्वरे}
{बहुकालमनाश्वासं दुःखमेवानुभूतवान्} %5-43

\twolineshloka
{इदानीं बहुदुःखौघाद्भवद्भिरभिरक्षितः}
{भवत्प्रसादात्सुग्रीवः प्राप्तसौख्यो महामतिः} %5-44

\twolineshloka
{कामासक्तो रघुपतेः सेवार्थं नागतो हरिः}
{आगमिष्यन्ति हरयो नानादेशगताः प्रभो} %5-45

\twolineshloka
{प्रेषितो दशसाहस्रा हरयो रघुसत्तम}
{आनेतुं वानरान् दिग्भ्यो महापर्वतसन्निभान्} %5-46

\twolineshloka
{सुग्रीवः स्वयमागत्य सर्ववानरयूथपैः}
{वधयिष्यति दैत्यौघान् रावणं च हनिष्यति} %5-47

\twolineshloka
{त्वयैव सहितोऽद्यैव गन्ता वानरपुङ्गवः}
{पश्यान्तर्भवनं तत्र पुत्रदारसुहृद्वृतम्} %5-48

\twolineshloka
{दृष्ट्वा सुग्रीवमभयं दत्त्वा नय सहैव ते}
{ताराया वचनं श्रुत्वा कृशक्रोधोऽथ लक्ष्मणः} %5-49

\twolineshloka
{जगामान्तःपुरं यत्र सुग्रीवो वानरेश्वरः}
{रुमामालिङ्ग्य सुग्रीवः पर्यङ्के पर्यवस्थितः} %5-50

\twolineshloka
{दृष्ट्वा लक्ष्मणमत्यर्थमुत्पपातातिभीतवत्}
{तं दृष्ट्वा लक्ष्मणः क्रुद्धो मदविह्वलितेक्षणम्} %5-51

\twolineshloka
{सुग्रीवं प्राह दुर्वृत्त विस्मृतोऽसि रघूत्तमम्}
{वाली येन हतो वीरः स बाणोऽद्य प्रतीक्षते} %5-52

\twolineshloka
{त्वमेव वालिनो मार्गं गमिष्यसि मया हतः}
{एवमत्यन्तपरुषं वदन्तं लक्ष्मणं तदा} %5-53

\twolineshloka
{उवाच हनुमान् वीरः कथमेवं प्रभाषसे}
{त्वत्तोऽधिकतरो रामे भक्तोऽयं वानराधिपः} %5-54

\twolineshloka
{रामकार्यार्थमनिशं जागर्ति न तु विस्मृतः}
{आगताः परितः पश्य वानराः कोटिशः प्रभो} %5-55

\twolineshloka
{गमिष्यन्त्यचिरेणैव सीतायाः परिमार्गणम्}
{साधयिष्यति सुग्रीवो रामकार्यमशेषतः} %5-56

\twolineshloka
{श्रुत्वा हनुमतो वाक्यं सौमित्रिर्लज्जितोऽभवत्}
{सुग्रीवोऽप्यर्घ्यपाद्याद्यैर्लक्ष्मणं समपूजयत्} %5-57

\twolineshloka
{आलिङ्ग्य प्राह रामस्य दासोऽहं तेन रक्षितः}
{रामः स्वतेजसा लोकान् क्षणार्द्धेनैव जेष्यति} %5-58

\twolineshloka
{सहायमात्रमेवाहं वानरैः सहितः प्रभो}
{सौमित्रिरपि सुग्रीवं प्राह किञ्चिन्मयोदितम्} %5-59

\twolineshloka
{तत्क्षमस्व महाभाग प्रणयाद्भाषितं मया}
{गच्छामोऽद्यैव सुग्रीव रामस्तिष्ठति कानने} %5-60

\twolineshloka
{एक एवातिदुःखार्त्तो जानकीविरहात्प्रभुः}
{तथेति रथमारुह्य लक्ष्मणेन समन्वितः} %5-61

{वानरैः सहितो राजा राममेवान्वपद्यत॥६२॥} %5-62
\refstepcounter{shlokacount}


\twolineshloka
{भेरीमृदङ्गैर्बहुऋक्षवानरैः श्वेतातपत्रैर्व्यजनैश्च शोभितः}
{नीलाङ्गदाद्यैर्हनुमत्प्रधानैः समावृतो राघवमभ्यगाद्धरिः} %5-63

{॥इति श्रीमदध्यात्मरामायणे उमामहेश्वरसंवादे किष्किन्धाकाण्डे
पञ्चमः सर्गः॥५॥
}
%%%%%%%%%%%%%%%%%%%%



\sect{षष्ठः सर्गः}

\twolineshloka
{दृष्ट्वा रामं समासीनं गुहाद्वारि शिलातले}
{चैलाजिनधरं श्यामं जटामौलिविराजितम्} %6-1

\twolineshloka
{विशालनयनं शान्तं स्मितचारुमुखाम्बुजम्}
{सीताविरहसन्तप्तं पश्यन्तं मृगपक्षिणः} %6-2

\twolineshloka
{रथाद्दूरात्समुत्पत्य वेगात्सुग्रीवलक्ष्मणौ}
{रामस्य पादयोरग्रे पेततुर्भक्तिसंयुतौ} %6-3

\twolineshloka
{रामः सुग्रीवमालिङ्ग्य पृष्ट्वाऽनामयमन्तिके}
{स्थापयित्वा यथान्यायं पूजयामास धर्मवित्} %6-4

\twolineshloka
{ततोऽब्रवीद्रघुश्रेष्ठं सुग्रीवो भक्तिनम्रधीः}
{देव पश्य समायान्तीं वानराणां महाचमूम्} %6-5

\twolineshloka
{कुलाचलाद्रिसम्भूता मेरुमन्दरसन्निभाः}
{नानाद्वीपसरिच्छैलवासिनः पर्वतोपमाः} %6-6

\twolineshloka
{असङ्ख्याताः समायान्ति हरयः कामरूपिणः}
{सर्वे देवांशसम्भूताः सर्वे युद्धविशारदाः} %6-7

\twolineshloka
{अत्र केचिद्गजबलाः केचिद्दशगजोपमाः}
{गजायुतबलाः केचिदन्येऽमितबलाः प्रभो} %6-8

\twolineshloka
{केचिदञ्जनकूटाभाः केचित्कनकसन्निभाः}
{केचिद्रक्तान्तवदना दीर्घवालास्तथाऽपरे} %6-9

\twolineshloka
{शुद्धस्फटिकसङ्काशाः केचिद्राक्षससन्निभाः}
{गर्जन्तः परितो यान्ति वानरा युद्धकाङ्क्षिणः} %6-10

\twolineshloka
{त्वदाज्ञाकारिणः सर्वे फलमूलाशनाः प्रभो}
{ऋक्षाणामधिपो वीरो जाम्बवान्नाम बुद्धिमान्} %6-11

\twolineshloka
{एष मे मन्त्रिणां श्रेष्ठः कोटिभल्लूकवृन्दपः}
{हनूमानेष विख्यातो महासत्त्वपराक्रमः} %6-12

\twolineshloka
{वायुपुत्रोऽतितेजस्वी मन्त्री बुद्धिमतां वरः}
{नलो नीलश्च गवयो गवाक्षो गन्धमादनः} %6-13

\twolineshloka
{शरभो मैन्दवश्चैव गजः पनस एव च}
{वलीमुखो दधिमुखः सुषेणस्तार एव च} %6-14

\twolineshloka
{केसरी च महासत्त्वः पिता हनुमतो बली}
{एते ते यूथपा राम प्राधान्येन मयोदिताः} %6-15

\twolineshloka
{महात्मानो महावीर्याः शक्रतुल्यपराक्रमाः}
{एते प्रत्येकतः कोटिकोटिवानरयूथपाः} %6-16

\twolineshloka
{तवाज्ञाकारिणः सर्वे सर्वे देवांशसम्भवाः}
{एष वालिसुतः श्रीमानङ्गदो नाम विश्रुतः} %6-17

\twolineshloka
{वालितुल्यबलो वीरो राक्षसानां बलान्तकः}
{एते चान्ये च बहवस्त्वदर्थे त्यक्तजीविताः} %6-18

\twolineshloka
{योद्धारः पर्वताग्रैश्च निपुणाः शत्रुघातने}
{आज्ञापय रघुश्रेष्ठ सर्वे ते वशवर्तिनः} %6-19

\twolineshloka
{रामः सुग्रीवमालिङ्ग्य हर्षपूर्णाश्रुलोचनः}
{प्राह सुग्रीव जानासि सर्वं त्वं कार्यगौरवम्} %6-20

\twolineshloka
{मार्गणार्थं हि जानक्या नियुङ्क्ष्व यदि रोचते}
{श्रुत्वा रामस्य वचनं सुग्रीवः प्रीतमानसः} %6-21

\twolineshloka
{प्रेषयामास बलिनो वानरान् वानरर्षभः}
{दिक्षु सर्वासु विविधान् वानरान् प्रेष्य सत्वरम्} %6-22

\twolineshloka
{दक्षिणां दिशमत्यर्थं प्रयत्नेन महाबलान्}
{युवराजं जाम्बवन्तं हनूमन्तं महाबलम्} %6-23

\twolineshloka
{नलं सुषेणं शरभं मैन्दं द्विविदमेव च}
{प्रेषयामास सुग्रीवो वचनं चेदमब्रवीत्} %6-24

\twolineshloka
{विचिन्वन्तु प्रयत्नेन भवन्तो जानकीं शुभाम्}
{मासादर्वाङ्निवर्तध्वं मच्छासनपुरःसराः} %6-25

\twolineshloka
{सीतामदृष्ट्वा यदि वो मासादूर्ध्वं दिनं भवेत्}
{तदा प्राणान्तिकं दण्डं मत्तः प्राप्स्यथ वानराः} %6-26

\twolineshloka
{इति प्रस्थाप्य सुग्रीवो वानरान् भीमविक्रमान्}
{रामस्य पार्श्वे श्रीरामं नत्वा चोपविवेश सः} %6-27

\twolineshloka
{गच्छन्तं मारुतिं दृष्ट्वा रामो वचनमब्रवीत्}
{अभिज्ञानार्थमेतन्मे ह्यङ्गुलीयकमुत्तमम्} %6-28

\threelineshloka
{मन्नामाक्षरसंयुक्तं सीतायै दीयतां रहः}
{अस्मिन् कार्ये प्रमाणं हि त्वमेव कपिसत्तम}
{जानामि सत्त्वं ते सर्वं गच्छ पन्थाः शुभस्तव} %6-29

\twolineshloka
{एवं कपीनां राज्ञा ते विसृष्टाः परिमार्गणे}
{सीताया अङ्गदमुखा बभ्रमुस्तत्र तत्र ह} %6-30

\twolineshloka
{भ्रमन्तो विन्ध्यगहने ददृशुः पर्वतोपमम्}
{राक्षसं भीषणाकारं भक्षयन्तं मृगान् गजान्} %6-31

\twolineshloka
{रावणोऽयमिति ज्ञात्वा केचिद्वानरपुङ्गवाः}
{जघ्नुः किलकिलाशब्दं मुञ्चन्तो मुष्टिभिः क्षणात्} %6-32

\twolineshloka
{नायं रावण इत्युक्त्वा ययुरन्यन्महद्वनम्}
{तृषार्ता सलिलं तत्र नाविन्दन् हरिपुङ्गवाः} %6-33

\twolineshloka
{विभ्रमन्तो महारण्ये शुष्ककण्ठोष्ठतालुकाः}
{ददृशुर्गह्वरं तत्र तृणगुल्मावृतं महत्} %6-34

\twolineshloka
{आर्द्रपक्षान् क्रौञ्चहंसान्निःसृतान् ददृशुस्ततः}
{अत्रास्ते सलिलं नूनं प्रविशामो महागुहाम्} %6-35

\twolineshloka
{इत्युक्त्वा हनुमानग्रे प्रविवेश तमन्वयुः}
{सर्वे परस्परं धृत्वा बाहून् बाहुभिरुत्सुकाः} %6-36

\twolineshloka
{अन्धकारे महद्दूरं गत्वाऽपश्यन् कपीश्वराः}
{जलाशयान् मणिनिभतोयान् कल्पद्रुमोपमान्} %6-37

\twolineshloka
{वृक्षान् पक्वफलैर्नम्रान् मधुद्रोणसमन्वितान्}
{गृहान् सर्वगुणोपेतान् मणिवस्त्रादिपूरितान्} %6-38

\twolineshloka
{दिव्यभक्ष्यान्नसहितान् मानुषैः परिवर्जितान्}
{विस्मितास्तत्र भवने दिव्ये कनकविष्टरे} %6-39

\twolineshloka
{प्रभया दीप्यमानां तु ददृशुः स्त्रियमेककाम्}
{ध्यायन्तीं चीरवसनां योगिनीं योगमास्थिताम्} %6-40

\twolineshloka
{प्रणेमुस्तां महाभागां भक्त्या भीत्या च वानराः}
{दृष्ट्वा तान् वानरान् देवी प्राह यूयं किमागताः} %6-41

\twolineshloka
{कुतो वा कस्य दूता वा मत्स्थानं किं प्रधर्षथ}
{तच्छ्रुत्वा हनुमानाह शृणु वक्ष्यामि देवि ते} %6-42

\twolineshloka
{अयोध्याधिपतिः श्रीमान् राजा दशरथः प्रभुः}
{तस्य पुत्रो महाभागो ज्येष्ठो राम इति श्रुतः} %6-43

\twolineshloka
{पितुराज्ञां पुरस्कृत्य सभार्यः सानुजो वनम्}
{गतस्तत्र हृता भार्या तस्य साध्वी दुरात्मना} %6-44

\twolineshloka
{रावणेन ततो रामः सुग्रीवं सानुजो ययौ}
{सुग्रीवो मित्रभावेन रामस्य प्रियवल्लभाम्} %6-45

\twolineshloka
{मृगयध्वमिति प्राह ततो वयमुपागताः}
{ततो वनं विचिन्वन्तो जानकीं जलकाङ्क्षिणः} %6-46

\twolineshloka
{प्रविष्टा गह्वरं घोरं दैवादत्र समागताः}
{त्वं वा किमर्थमत्रासि का वा त्वं वद नः शुभे} %6-47

\twolineshloka
{योगिनी च तथा दृष्ट्वा वानरान् प्राह हृष्टधीः}
{यथेष्टं फलमूलानि जग्ध्वा पीत्वाऽमृतं पयः} %6-48

\twolineshloka
{आगच्छत ततो वक्ष्ये मम वृत्तान्तमादितः}
{तथेति भुक्त्वा पीत्वा च हृष्टास्ते सर्ववानराः} %6-49

\twolineshloka
{देव्याः समीपं गत्वा ते बद्धाञ्जलिपुटाः स्थिताः}
{ततः प्राह हनूमन्तं योगिनी दिव्यदर्शना} %6-50

\twolineshloka
{हेमा नाम पुरा दिव्यरूपिणी विश्वकर्मणः}
{पुत्री महेशं नृत्येन तोषयामास भामिनी} %6-51

\twolineshloka
{तुष्टो महेशः प्रददाविदं दिव्यपुरं महत्}
{अत्र स्थिता सा सुदती वर्षाणामयुतायुतम्} %6-52

\twolineshloka
{तस्या अहं सखी विष्णुतत्परा मोक्षकाङ्क्षिणी}
{नाम्ना स्वयम्प्रभा दिव्यगन्धर्वतनया पुरा} %6-53

\twolineshloka
{गच्छन्ती ब्रह्मलोकं सा मामाहेदं तपश्चर}
{अत्रैव निवसन्ती त्वं सर्वप्राणिविवर्जिते} %6-54

\twolineshloka
{त्रेतायुगे दाशरथिर्भूत्वा नारायणोऽव्ययः}
{भूभारहरणार्थाय विचरिष्यति कानने} %6-55

\twolineshloka
{मार्गन्तो वानरास्तस्य भार्यामायान्ति ते गुहाम्}
{पूजयित्वाऽथ तान् नत्वा रामं स्तुत्वा प्रयत्नतः} %6-56

\twolineshloka
{यातासि भवनं विष्णोर्योगिगम्यं सनातनम्}
{इतोऽहं गन्तुमिच्छामि रामं द्रष्टुं त्वरान्विता} %6-57

\twolineshloka
{यूयं पिदध्वमक्षीणि गमिष्यथ बहिर्गुहाम्}
{तथैव चक्रुस्ते वेगाद्गताः पूर्वस्थितं वनम्} %6-58

\twolineshloka
{साऽपि त्यक्त्वा गुहां शीघ्रं ययौ राघवसन्निधिम्}
{तत्र रामं ससुग्रीवं लक्ष्मणं च ददर्श ह} %6-59

\twolineshloka
{कृत्वा प्रदक्षिणं रामं प्रणम्य बहुशः सुधीः}
{आह गद्गदया वाचा रोमाञ्चिततनूरुहा} %6-60

\twolineshloka
{दासी तवाहं राजेन्द्र दर्शनार्थमिहाऽऽगता}
{बहुवर्षसहस्राणि तप्तं मे दुश्चरं तपः} %6-61

\twolineshloka
{गुहायां दर्शनार्थं ते फलितं मेऽद्य तत्तपः}
{अद्य हि त्वां नमस्यामि मायायाः परतः स्थितम्} %6-62

\twolineshloka
{सर्वभूतेषु चालक्ष्यं बहिरन्तरवस्थितम्}
{योगमायाजवनिकाच्छन्नो मानुषविग्रहः} %6-63

\twolineshloka
{न लक्ष्यसेऽज्ञानदृशां शैलूष इव रूपधृक्}
{महाभागवतानां त्वं भक्तियोगविधित्सया} %6-64

\twolineshloka
{अवतीर्णोऽसि भगवन् कथं जानामि तामसी}
{लोके जानातु यः कश्चित्तव तत्त्वं रघूत्तम} %6-65

\twolineshloka
{ममैतदेव रूपं ते सदा भातु हृदालये}
{राम ते पादयुगलं दर्शितं मोक्षदर्शनम्} %6-66

\threelineshloka
{अदर्शनं भवार्णानां सन्मार्गपरिदर्शनम्}
{धनपुत्रकलत्रादिविभूतिपरिदर्पितः}
{अकिञ्चनधनं त्वाऽद्य नाभिधातुं जनोऽर्हति} %6-67

{निवृत्तगुणमार्गाय निष्किञ्चनधनाय ते॥६८॥} %6-68
\refstepcounter{shlokacount}


\twolineshloka
{नमः स्वात्माभिरामाय निर्गुणाय गुणात्मने}
{कालरूपिणमीशानमादिमध्यान्तवर्जितम्} %6-69

\twolineshloka
{समं चरन्तं सर्वत्र मन्ये त्वां पुरुषं परम्}
{देव ते चेष्टितं कश्चिन्न वेद नृविडम्बनम्} %6-70

\twolineshloka
{न तेऽस्ति कश्चिद्दयितो द्वेष्यो वाऽपर एव च}
{त्वन्मायापिहितात्मानस्त्वां पश्यन्ति तथाविधम्} %6-71

\twolineshloka
{अजस्याकर्तुरीशस्य देवतिर्यङ्नरादिषु}
{जन्मकर्मादिकं यद्यत्तदत्यन्तविडम्बनम्} %6-72

\twolineshloka
{त्वामाहुरक्षरं जातं कथाश्रवणसिद्धये}
{केचित्कोसलराजस्य तपसः फलसिद्धये} %6-73

\twolineshloka
{कौसल्यया प्रार्थ्यमानं जातमाहुः परे जनाः}
{दुष्टराक्षसभूभारहरणायार्थितो विभुः} %6-74

\twolineshloka
{ब्रह्मणा नररूपेण जातोऽयमिति केचन}
{शृण्वन्ति गायन्ति च ये कथास्ते रघुनन्दन} %6-75

\twolineshloka
{पश्यन्ति तव पादाब्जं भवार्णवसुतारणम्}
{त्वन्मायागुणबद्धाहं व्यतिरिक्तं गुणाश्रयम्} %6-76

\threelineshloka
{कथं त्वां देव जानीयां स्तोतुं वाऽविषयं विभुम्}
{नमस्यामि रघुश्रेष्ठं बाणासनशरान्वितम्}
{लक्ष्मणेन सह भ्रात्रा सुग्रीवादिभिरन्वितम्} %6-77

\twolineshloka
{एवं स्तुतो रघुश्रेष्ठः प्रसन्नः प्रणताघहृत्}
{उवाच योगिनीं भक्तां किं ते मनसि काङ्क्षितम्} %6-78

\twolineshloka
{सा प्राह राघवं भक्त्या भक्तिं ते भक्तवत्सल}
{यत्र कुत्रापि जाताया निश्चलां देहि मे प्रभो} %6-79

\twolineshloka
{त्वद्भक्तेषु सदा सङ्गो भूयान्मे प्राकृतेषु न}
{जिह्वा मे रामरामेति भक्त्या वदतु सर्वदा} %6-80

\twolineshloka
{मानसं श्यामलं रूपं सीतालक्ष्मणसंयुतम्}
{धनुर्बाणधरं पीतवाससं मुकुटोज्ज्वलम्} %6-81

\twolineshloka
{अङ्गदैर्नूपुरैर्मुक्ताहारैः कौस्तुभकुण्डलैः}
{भान्तं स्मरतु मे राम वरं नान्यं वृणे प्रभो} %6-82

\textbf{श्रीराम उवाच}

\threelineshloka
{भवत्वेवं महाभागे गच्छ त्वं बदरीवनम्}
{तत्रैव मां स्मरन्ती त्वं त्यक्त्वेदं भूतपञ्चकम्}
{मामेव परमात्मानमचिरात्प्रतिपद्यसे} %6-83

\fourlineindentedshloka
{श्रुत्वा रघूत्तमवचोऽमृतसारकल्पम्}
{गत्वा तदैव बदरीतरुखण्डजुष्टम्}
{तीर्थं तदा रघुपतिं मनसा स्मरन्ती}
{त्यक्त्वा कलेवरमवाप परं पदं सा} %6-84

{॥इति श्रीमदध्यात्मरामायणे उमामहेश्वरसंवादे किष्किन्धाकाण्डे
षष्ठः सर्गः॥६॥}
%%%%%%%%%%%%%%%%%%%%



\sect{सप्तमः सर्गः}

\textbf{श्रीमहादेव उवाच}

\twolineshloka
{अथ तत्र समासीना वृक्षखण्डेषु वानराः}
{चिन्तयन्तो विमुह्यन्तः सीतामार्गणकर्शिताः} %7-1

\twolineshloka
{तत्रोवाचाङ्गदः कांश्चिद्वानरान् वानरर्षभः}
{भ्रमतां गह्वरेऽस्माकं मासो नूनं गतोऽभवत्} %7-2

\twolineshloka
{सीता नाधिगताऽस्माभिर्न कृतं राजशासनम्}
{यदि गच्छाम किष्किन्धां सुग्रीवोऽस्मान् हनिष्यति} %7-3

\twolineshloka
{विशेषतः शत्रुसुतं मां मिषान्निहनिष्यति}
{मयि तस्य कुतः प्रीतिरहं रामेण रक्षितः} %7-4

\twolineshloka
{इदानीं रामकार्यं मे न कृतं तन्मिषं भवेत्}
{तस्य मद्धनने नूनं सुग्रीवस्य दुरात्मनः} %7-5

\twolineshloka
{मातृकल्पां भ्रातृभार्यां पापात्माऽनुभवत्यसौ}
{न गच्छेयमतः पार्श्वं तस्य वानरपुङ्गवाः} %7-6

\twolineshloka
{त्यक्ष्यामि जीवितं चात्र येन केनापि मृत्युना}
{इत्यश्रुनयनं केचिद्दृष्ट्वा वानरपुङ्गवाः} %7-7

{व्यथिताः साश्रुनयना युवराजमथाब्रुवन्॥८॥} %7-8
\refstepcounter{shlokacount}


\twolineshloka
{किमर्थं तव शोकोऽत्र वयं ते प्राणरक्षकाः}
{भवामो निवसामोऽत्र गुहायां भयवर्जिताः} %7-9

\twolineshloka
{सर्वसौभाग्यसहितं पुरं देवपुरोपमम्}
{शनैः परस्परं वाक्यं वदतां मारुतात्मजः} %7-10

\twolineshloka
{श्रुत्वाऽङ्गदं समालिङ्ग्य प्रोवाच नयकोविदः}
{विचार्यते किमर्थं ते दुर्विचारो न युज्यते} %7-11

\twolineshloka
{राज्ञोऽत्यन्तप्रियस्त्वं हि तारापुत्रोऽतिवल्लभः}
{रामस्य लक्ष्मणात्प्रीतिस्त्वयि नित्यं प्रवर्धते} %7-12

\twolineshloka
{अतो न राघवाद्भीतिस्तव राज्ञो विशेषतः}
{अहं तव हिते सक्तो वत्स नान्यं विचारय} %7-13

\twolineshloka
{गुहावासश्च निर्भेद्य इत्युक्तं वानरैस्तु यत्}
{तदेतद्रामबाणानामभेद्यं किं जगत्त्रये} %7-14

\twolineshloka
{ये त्वां दुर्बोधयन्त्येते वानरा वानरर्षभ}
{पुत्रदारादिकं त्यक्त्वा कथं स्थास्यन्ति ते त्वया} %7-15

\twolineshloka
{अन्यद्गुह्यतमं वक्ष्ये रहस्यं शृणु मे सुत}
{रामो न मानुषो देवः साक्षान्नारायणोऽव्ययः} %7-16

\twolineshloka
{सीता भगवती माया जनसम्मोहकारिणी}
{लक्ष्मणो भुवनाधारः साक्षाच्छेषः फणीश्वरः} %7-17

\twolineshloka
{ब्रह्मणा प्रार्थिताः सर्वे रक्षोगणविनाशने}
{मायामानुषभावेन जाता लोकैकरक्षकाः} %7-18

\twolineshloka
{वयं च पार्षदाः सर्वे विष्णोर्वैकुण्ठवासिनः}
{मनुष्यभावमापन्ने स्वेच्छया परमात्मनि} %7-19

\twolineshloka
{वयं वानररूपेण जातास्तस्यैव मायया}
{वयं तु तपसा पूर्वमाराध्य जगतां पतिम्} %7-20

\twolineshloka
{तेनैवानुगृहीताः स्मः पार्षदत्वमुपागताः}
{इदानीमपि तस्यैव सेवां कृत्वैव मायया} %7-21

\twolineshloka
{पुनर्वैकुण्ठमासाद्य सुखं स्थास्यामहे वयम्}
{इत्यङ्गदमथाऽश्वास्य गता विन्ध्यं महाचलम्} %7-22

\twolineshloka
{विचिन्वन्तोऽथ शनकैर्जानकीं दक्षिणाम्बुधेः}
{तीरे महेन्द्राख्यगिरेः पवित्रं पादमाययुः} %7-23

\twolineshloka
{दृष्ट्वा समुद्रं दुष्पारमगाधं भयवर्धनम्}
{वानरा भयसन्त्रस्ताः किं कुर्म इति वादिनः} %7-24

\twolineshloka
{निषेदुरुदधेस्तीरे सर्वे चिन्तासमन्विताः}
{मन्त्रयामासुरन्योन्यमङ्गदाद्या महाबलाः} %7-25

\twolineshloka
{भ्रमतो मे वने मासो गतोऽत्रैव गुहान्तरे}
{न दृष्टो रावणो वाऽद्य सीता वा जनकात्मजा} %7-26

\twolineshloka
{सुग्रीवस्तीक्ष्णदण्डोऽस्मान्निहन्त्येव न संशयः}
{सुग्रीववधतोऽस्माकं श्रेयः प्रायोपवेशनम्} %7-27

\twolineshloka
{इति निश्चित्य तत्रैव दर्भानास्तीर्य सर्वतः}
{उपाविवेशुस्ते सर्वे मरणे कृतनिश्चयाः} %7-28

\twolineshloka
{एतस्मिन्नन्तरे तत्र महेन्द्राद्रिगुहान्तरात्}
{निर्गत्य शनकैरागाद्गृध्रः पर्वतसन्निभः} %7-29

\twolineshloka
{दृष्ट्वा प्रायोपवेशेन स्थितान् वानरपुङ्गवान्}
{उवाच शनकैर्गृध्रः प्राप्तो भक्ष्योऽद्य मे बहुः} %7-30

\twolineshloka
{एकैकशः क्रमात्सर्वान् भक्षयामि दिने दिने}
{श्रुत्वा तद्गृध्रवचनं वानरा भीतमानसाः} %7-31

\twolineshloka
{भक्षयिष्यति नः सर्वानसौ गृध्रो न संशयः}
{रामकार्यं च नास्माभिः कृतं किञ्चिद्धरीश्वराः} %7-32

\twolineshloka
{सुग्रीवस्यापि च हितं न कृतं स्वात्मनामपि}
{वृथाऽनेन वधं प्राप्ता गच्छामो यमसादनम्} %7-33

\twolineshloka
{अहो जटायुर्धर्मात्मा रामस्यार्थे मृतः सुधीः}
{मोक्षं प्राप दुरावापं योगिनामप्यरिन्दमः} %7-34

\twolineshloka
{सम्पातिस्तु तदा वाक्यं श्रुत्वा वानरभाषितम्}
{के वा यूयं मम भ्रातुः कर्णपीयूषसन्निभम्} %7-35

\twolineshloka
{जटायुरिति नामाद्य व्याहरन्तः परस्परम्}
{उच्यतां वो भयं मा भून्मत्तः प्लवगसत्तमाः} %7-36

\twolineshloka
{तमुवाचाङ्गदः श्रीमानुत्थितो गृध्रसन्निधौ}
{रामो दाशरथिः श्रीमान् लक्ष्मणेन समन्वितः} %7-37

\twolineshloka
{सीतया भार्यया सार्धं विचचार महावने}
{तस्य सीता हृता साध्वी रावणेन दुरात्मना} %7-38

\twolineshloka
{मृगयां निर्गते रामे लक्ष्मणे च हृता बलात्}
{रामरामेति क्रोशन्ती श्रुत्वा गृध्रः प्रतापवान्} %7-39

\twolineshloka
{जटायुर्नाम पक्षीन्द्रो युद्धं कृत्वा सुदारुणम्}
{रावणेन हतो वीरो राघवार्थं महाबलः} %7-40

\twolineshloka
{रामेण दग्धो रामस्य सायुज्यमगमत्क्षणात्}
{रामः सुग्रीवमासाद्य सख्यं कृत्वाऽग्निसाक्षिकम्} %7-41

\twolineshloka
{सुग्रीवचोदितो हत्वा वालिनं सुदुरासदम्}
{राज्यं ददौ वानराणां सुग्रीवाय महाबलः} %7-42

\twolineshloka
{सुग्रीवः प्रेषयामास सीतायाः परिमार्गणे}
{अस्मान् वानरवृन्दान् वै महासत्त्वान् महाबलः} %7-43

\twolineshloka
{मासादर्वाङ्निवर्तध्वं नो चेत्प्राणान् हरामि वः}
{इत्याज्ञया भ्रमन्तोऽस्मिन् वने गह्वरमध्यगाः} %7-44

\twolineshloka
{गतो मासो न जानीमः सीतां वा रावणं च वा}
{मर्तुं प्रायोपविष्टा स्मस्तीरे लवणवारिधेः} %7-45

\twolineshloka
{यदि जानासि हे पक्षिन् सीतां कथय नः शुभाम्}
{अङ्गदस्य वचः श्रुत्वा सम्पातिर्हृष्टमानसः} %7-46

\twolineshloka
{उवाच मत्प्रियो भ्राता जटायुः प्लवगेश्वराः}
{बहुवर्षसहस्रान्ते भ्रातृवार्ता श्रुता मया} %7-47

\twolineshloka
{वाक्साहाय्यं करिष्येऽहं भवतां प्लवगेश्वराः}
{भ्रातुः सलिलदानाय नयध्वं मां जलान्तिकम्} %7-48

\twolineshloka
{पश्चात्सर्वं शुभं वक्ष्ये भवतां कार्यसिद्धये}
{तथेति निन्युस्ते तीरं समुद्रस्य विहङ्गमम्} %7-49

\threelineshloka
{सोऽपि तत्सलिले स्नात्वा भ्रातुर्दत्त्वा जलाञ्जलिम्}
{पुनः स्वस्थानमासाद्य स्थितो नीतो हरीश्वरैः}
{सम्पातिः कथयामास वानरान् परिहर्षयन्} %7-50

\twolineshloka
{लङ्का नाम नगर्यास्ते त्रिकूटगिरिमूर्धनि}
{तत्राशोकवने सीता राक्षसीभिः सुरक्षिता} %7-51

\twolineshloka
{समुद्रमध्ये सा लङ्का शतयोजनदूरतः}
{दृश्यते मे न सन्देहः सीता च परिदृश्यते} %7-52

\twolineshloka
{गृध्रत्वाद्दूरदृष्टिर्मे नात्र संशयितुं क्षमम्}
{शतयोजनविस्तीर्णं समुद्रं यस्तु लङ्घयेत्} %7-53

\threelineshloka
{स एव जानकीं दृष्ट्वा पुनरायास्यति ध्रुवम्}
{अहमेव दुरात्मानं रावणं हन्तुमुत्सहे}
{भ्रातुर्हन्तारमेकाकी किन्तु पक्षविवर्जितः} %7-54

\twolineshloka
{यतध्वमतियत्नेन लङ्घितुं सरितां पतिम्}
{ततो हन्ता रघुश्रेष्ठो रावणं राक्षसाधिपम्} %7-55

\fourlineindentedshloka
{उल्लङ्घ्य सिन्धुं शतयोजनायतम्}
{लङ्कां प्रविश्याथ विदेहकन्यकाम्}
{दृष्ट्वा समाभाष्य च वारिधिं पुनः}
{तर्तुं समर्थः कतमो विचार्यताम्} %7-56

{॥इति श्रीमदध्यात्मरामायणे उमामहेश्वरसंवादे किष्किन्धाकाण्डे
सप्तमः सर्गः॥७॥}
%%%%%%%%%%%%%%%%%%%%



\sect{अष्टमः सर्गः}

\twolineshloka
{अथ ते कौतुकाविष्टाः सम्पातिं सर्ववानराः}
{पप्रच्छुर्भगवन् ब्रूहि स्वमुदन्तं त्वमादितः} %8-1

\twolineshloka
{सम्पातिः कथयामास स्ववृत्तान्तं पुरा कृतम्}
{अहं पुरा जटायुश्च भ्रातरौ रूढयौवनौ} %8-2

\twolineshloka
{बलेन दर्पितावावां बलजिज्ञासया खगौ}
{सूर्यमण्डलपर्यन्तं गन्तुमुत्पतितौ मदात्} %8-3

\twolineshloka
{बहुयोजनसाहस्रं गतौ तत्र प्रतापितः}
{जटायुस्तं परित्रातुं पक्षैराच्छाद्य मोहतः} %8-4

\twolineshloka
{स्थितोऽहं रश्मिभिर्दग्धपक्षोऽस्मिन् विन्ध्यमूर्धनि}
{पतितो दूरपतनान्मूर्च्छितोऽहं कपीश्वराः} %8-5

\twolineshloka
{दिनत्रयात्पुनः प्राणसहितो दग्धपक्षकः}
{देशं वा गिरिकूटान् वा न जाने भ्रान्तमानसः} %8-6

\twolineshloka
{शनैरुन्मील्य नयने दृष्ट्वा तत्राऽश्रमं शुभम्}
{शनैः शनैराश्रमस्य समीपं गतवानहम्} %8-7

\twolineshloka
{चन्द्रमा नाम मुनिराड्\mbox{}दृष्ट्वा मां विस्मितोऽवदत्}
{सम्पाते किमिदं तेऽद्य विरूपं केन वा कृतम्} %8-8

\twolineshloka
{जानामि त्वामहं पूर्वमत्यन्तं बलवानसि}
{दग्धौ किमर्थं ते पक्षौ कथ्यतां यदि मन्यसे} %8-9

\twolineshloka
{ततः स्वचेष्टितं सर्वं कथयित्वाऽतिदुःखितः}
{अब्रवं मुनिशार्दूल दह्येऽहं दाववह्निना} %8-10

\twolineshloka
{कथं धारयितुं शक्तो विपक्षो जीवितं प्रभो}
{इत्युक्तोऽथ मुनिर्वीक्ष्य मां दयार्द्रविलोचनः} %8-11

\twolineshloka
{शृणु वत्स वचो मेऽद्य श्रुत्वा कुरु यथेप्सितम्}
{देहमूलमिदं दुःखं देहः कर्मसमुद्भवः} %8-12

\twolineshloka
{कर्म प्रवर्तते देहेऽहम्बुद्ध्या पुरुषस्य हि}
{अहङ्कारस्त्वनादिः स्यादविद्यासम्भवो जडः} %8-13

\twolineshloka
{चिच्छायया सदा युक्तस्तप्तायःपिण्डवत् सदा}
{तेन देहस्य तादात्म्याद्देहश्चेतनवान् भवेत्} %8-14

\twolineshloka
{देहोऽहमिति बुद्धिः स्यादात्मनोऽहङ्कृतेर्बलात्}
{तन्मूल एष संसारः सुखदुःखादिसाधकः} %8-15

\twolineshloka
{आत्मनो निर्विकारस्य मिथ्या तादात्म्यतः सदा}
{देहोऽहं कर्मकर्ताऽहमिति सङ्कल्प्य सर्वदा} %8-16

\twolineshloka
{जीवः करोति कर्माणि तत्फलैर्बद्ध्यतेऽवशः}
{ऊर्ध्वाधो भ्रमते नित्यं पापपुण्यात्मकः स्वयम्} %8-17

\twolineshloka
{कृतं मयाऽधिकं पुण्यं यज्ञदानादि निश्चितम्}
{स्वर्गं गत्वा सुखं भोक्ष्य इति सङ्कल्पवान् भवेत्} %8-18

\twolineshloka
{तथैवाध्यासतस्तत्र चिरं भुक्त्वा सुखं महत्}
{क्षीणपुण्यः पतत्यर्वागनिच्छन् कर्मचोदितः} %8-19

\twolineshloka
{पतित्वा मण्डले चेन्दोस्ततो नीहारसंयुतः}
{भूमौ पतित्वा व्रीह्यादौ तत्र स्थित्वा चिरं पुनः} %8-20

\twolineshloka
{भूत्वा चतुर्विधं भोज्यं पुरुषैर्भुज्यते ततः}
{रेतो भूत्वा पुनस्तेन ऋतौ स्त्रीयोनिसिञ्चितः} %8-21

\twolineshloka
{योनिरक्तेन संयुक्तं जरायुपरिवेष्टितम्}
{दिनेनैकेन कललं भूत्वा रूढत्वमाप्नुयात्} %8-22

\twolineshloka
{तत्पुनः पञ्चरात्रेण बुद्बुदाकारतामियात्}
{सप्तरात्रेण तदपि मांसपेशित्वमाप्नुयात्} %8-23

\twolineshloka
{पक्षमात्रेण सा पेशी रुधिरेण परिप्लुता}
{तस्या एवाङ्कुरोत्पत्तिः पञ्चविंशतिरात्रिषु} %8-24

\twolineshloka
{ग्रीवा शिरश्च स्कन्धश्च पृष्ठवंशस्तथोदरम्}
{पञ्चधाङ्गानि चैकैकं जायन्ते मासतः क्रमात्} %8-25

\twolineshloka
{पाणिपादौ तथा पार्श्वः कटिर्जानु तथैव च}
{मासद्वयात् प्रजायन्ते क्रमेणैव न चान्यथा} %8-26

\twolineshloka
{त्रिभिर्मासैः प्रजायन्ते अङ्गानां सन्धयः क्रमात्}
{सर्वाङ्गुल्यः प्रजायन्ते क्रमान्मासचतुष्टये} %8-27

\twolineshloka
{नासा कर्णौ च नेत्रे च जायन्ते पञ्चमासतः}
{दन्तपङ्क्तिर्नखा गुह्यं पञ्चमे जायते तथा} %8-28

\twolineshloka
{अर्वाक् षण्मासतश्छिद्रं कर्णयोर्भवति स्फुटम्}
{पायुर्मेढ्रमुपस्थं च नाभिश्चापि भवेन्नृणाम्} %8-29

\twolineshloka
{सप्तमे मासि रोमाणि शिरः केशास्तथैव च}
{विभक्तावयवत्वं च सर्वं सम्पद्यतेऽष्टमे} %8-30

\twolineshloka
{जठरे वर्धते गर्भः स्त्रिया एवं विहङ्गम}
{पञ्चमे मासि चैतन्यं जीवः प्राप्नोति सर्वशः} %8-31

\twolineshloka
{नाभिसूत्राल्परन्ध्रेण मातृभुक्तान्नसारतः}
{वर्धते गर्भतः पिण्डो न म्रियेत स्वकर्मतः} %8-32

\twolineshloka
{स्मृत्वा सर्वाणि जन्मानि पूर्वकर्माणि सर्वशः}
{जठरानलतप्तोऽयमिदं वचनमब्रवीत्} %8-33

\twolineshloka
{नानायोनिसहस्रेषु जायमानोऽनुभूतवान्}
{पुत्रदारादिसम्बन्धं कोटिशः पशुबान्धवान्} %8-34

\twolineshloka
{कुटुम्बभरणासक्त्या न्यायान्यायैर्धनार्जनम्}
{कृतं नाकरवं विष्णुचिन्तां स्वप्नेऽपि दुर्भगः} %8-35

\twolineshloka
{इदानीं तत्फलं भुञ्जे गर्भदुःखं महत्तरम्}
{अशाश्वते शाश्वतवद्देहे तृष्णासमन्वितः} %8-36

\twolineshloka
{अकार्याण्येव कृतवान्न कृतं हितमात्मनः}
{इत्येवं बहुधा दुःखमनुभूय स्वकर्मतः} %8-37

\twolineshloka
{कदा निष्क्रमणं मे स्याद्गर्भान्निरयसन्निभात्}
{इत ऊर्ध्वं नित्यमहं विष्णुमेवानुपूजये} %8-38

\twolineshloka
{इत्यादि चिन्तयन् जीवो योनियन्त्रप्रपीडितः}
{जायमानोऽतिदुःखेन नरकात्पातकी यथा} %8-39

\twolineshloka
{पूतिव्रणान्निपतितः कृमिरेष इवापरः}
{ततो बाल्यादिदुःखानि सर्व एवं विभुञ्जते} %8-40

\twolineshloka
{त्वया चैवानुभूतानि सर्वत्र विदितानि च}
{न वर्णितानि मे गृध्र यौवनादिषु सर्वतः} %8-41

\twolineshloka
{एवं देहोऽहमित्यस्मादभ्यासान्निरयादिकम्}
{गर्भवासादिदुःखानि भवन्त्यभिनिवेशतः} %8-42

\twolineshloka
{तस्माद्देहद्वयादन्यमात्मानं प्रकृतेः परम्}
{ज्ञात्वा देहादिममतां त्यक्त्वाऽऽत्मज्ञानवान् भवेत्} %8-43

\twolineshloka
{जाग्रदादिविनिर्मुक्तं सत्यज्ञानादिलक्षणम्}
{शुद्धं बुद्धं सदा शान्तमात्मानमवधारयेत्} %8-44

\twolineshloka
{चिदात्मनि परिज्ञाते नष्टे मोहेऽज्ञसम्भवे}
{देहः पततु वाऽरब्धकर्मवेगेन तिष्ठतु} %8-45

\twolineshloka
{योगिनो न हि दुःखं वा सुखं वाऽज्ञानसम्भवम्}
{तस्माद्देहेन सहितो यावत्प्रारब्धसङ्क्षयः} %8-46

\twolineshloka
{तावत्तिष्ठ सुखेन त्वं धृतकञ्चुकसर्पवत्}
{अन्यद्वक्ष्यामि ते पक्षिन् शृणु मे परमं हितम्} %8-47

\twolineshloka
{त्रेतायुगे दाशरथिर्भूत्वा नारायणोऽव्ययः}
{रावणस्य वधार्थाय दण्डकानागमिष्यति} %8-48

\twolineshloka
{सीतया भार्यया सार्धं लक्ष्मणेन समन्वितः}
{तत्राऽश्रमे जनकजां भ्रातृभ्यां रहिते वने} %8-49

\twolineshloka
{रावणश्चोरवन्नीत्वा लङ्कायां स्थापयिष्यति}
{तस्याः सुग्रीवनिर्देशाद्वानराः परिमार्गणे} %8-50

\twolineshloka
{आगमिष्यन्ति जलधेस्तीरं तत्र समागमः}
{त्वया तैः कारणवशाद्भविष्यति न संशयः} %8-51

\twolineshloka
{तदा सीतास्थितिं तेभ्यः कथयस्व यथार्थतः}
{तदैव तव पक्षौ द्वावुत्पत्स्येते पुनर्नवौ} %8-52

\textbf{सम्पातिरुवाच}

\twolineshloka
{बोधयामास मां चन्द्रनामा मुनिकुलेश्वरः}
{पश्यन्तु पक्षौ मे जातौ नूतनावतिकोमलौ} %8-53

\twolineshloka
{स्वस्ति वोऽस्तु गमिष्यामि सीतां द्रक्ष्यथ निश्चयम्}
{यत्नं कुरुध्वं दुर्लङ्घ्यसमुद्रस्य विलङ्घने} %8-54

\fourlineindentedshloka
{यन्नामस्मृतिमात्रतोऽपरिमितं संसारवारान्निधिम्}
{तीर्त्वा गच्छति दुर्जनोऽपि परमं विष्णोः पदं शाश्वतम्}
{तस्यैव स्थितिकारिणस्त्रिजगतां रामस्य भक्ताः प्रिया}
{यूयं किं न समुद्रमात्रतरणे शक्ताः कथं वानराः} %8-55

{॥इति श्रीमदध्यात्मरामायणे उमामहेश्वरसंवादे किष्किन्धाकाण्डे
अष्टमः सर्गः॥८॥}
%%%%%%%%%%%%%%%%%%%%



\sect{नवमः सर्गः}

\textbf{श्रीमहादेव उवाच}

\twolineshloka
{गते विहायसा गृध्रराजे वानरपुङ्गवाः}
{हर्षेण महताऽऽविष्टाः सीतादर्शनलालसाः} %9-1

\twolineshloka
{ऊचुः समुद्रं पश्यन्तो नक्रचक्रभयङ्करम्}
{तरङ्गादिभिरुन्नद्धमाकाशमिव दुर्ग्रहम्} %9-2

\twolineshloka
{परस्परमवोचन् वै कथमेनं तरामहे}
{उवाच चाङ्गदस्तत्र शृणुध्वं वानरोत्तमाः} %9-3

\twolineshloka
{भवन्तोऽत्यन्तबलिनः शूराश्च कृतविक्रमाः}
{को वात्र वारिधिं तीर्त्वा राजकार्यं करिष्यति} %9-4

\twolineshloka
{एतेषां वानराणां स प्राणदाता न संशयः}
{तदुत्तिष्ठतु मे शीघ्रं पुरतो यो महाबलः} %9-5

\twolineshloka
{वानराणां च सर्वेषां रामसुग्रीवयोरपि}
{स एव पालको भूयान्नात्र कार्या विचारणा} %9-6

\twolineshloka
{इत्युक्ते युवराजेन तूष्णीं वानरसैनिकाः}
{आसन्नोचुः किञ्चिदपि परस्परविलोकिनः} %9-7

\textbf{अङ्गद उवाच}

\twolineshloka
{उच्यतां वै बलं सर्वैः प्रत्येकं कार्यसिद्धये}
{केन वा साध्यते कार्यं जानीमस्तदनन्तरम्} %9-8

\twolineshloka
{अङ्गदस्य वचः श्रुत्वा प्रोचुर्वीरा बलं पृथक्}
{योजनानां दशारभ्य दशोत्तरगुणं जगुः} %9-9

\twolineshloka
{शतादर्वाग्जाम्बवांस्तु प्राह मध्ये वनौकसाम्}
{पुरा त्रिविक्रमे देवे पादं भूमानलक्षणम्} %9-10

\twolineshloka
{त्रिःसप्तकृत्वोऽहमगां प्रदक्षिणविधानतः}
{इदानीं वार्धकग्रस्तो न शक्नोमि विलङ्घितुम्} %9-11

\twolineshloka
{अङ्गदोऽप्याह मे गन्तुं शक्यं पारं महोदधेः}
{पुनर्लङ्घनसामर्थ्यं न जानाम्यस्ति वा न वा} %9-12

\twolineshloka
{तमाह जाम्बवान् वीरस्त्वं राजा नो नियामकः}
{न युक्तं त्वां नियोक्तुं मे त्वं समर्थोऽसि यद्यपि} %9-13

\textbf{अङ्गद उवाच}

\twolineshloka
{एवं चेत्पूर्ववत्सर्वे स्वप्स्यामो दर्भविष्टरे}
{केनापि न कृतं कार्यं जीवितुं च न शक्यते} %9-14

\twolineshloka
{तमाह जाम्बवान् वीरो दर्शयिष्यामि ते सुत}
{येनास्माकं कार्यसिद्धिर्भविष्यत्यचिरेण च} %9-15

\twolineshloka
{इत्युक्त्वा जाम्बवान् प्राह हनूमन्तमवस्थितम्}
{हनूमन् किं रहस्तूष्णीं स्थीयते कार्यगौरवे} %9-16

\twolineshloka
{प्राप्तेऽज्ञेनेव सामर्थ्यं दर्शयाद्य महाबल}
{त्वं साक्षाद्वायुतनयो वायुतुल्यपराक्रमः} %9-17

\twolineshloka
{रामकार्यार्थमेव त्वं जनितोऽसि महात्मना}
{जातमात्रेण ते पूर्वं दृष्ट्वोद्यन्तं विभावसुम्} %9-18

\twolineshloka
{पक्वं फलं जिघृक्षामीत्युत्प्लुतं बालचेष्टया}
{योजनानां पञ्चशतं पतितोऽसि ततो भुवि} %9-19

\twolineshloka
{अतस्त्वद्बलमाहात्म्यं को वा शक्नोति वर्णितुम्}
{उत्तिष्ठ कुरु रामस्य कार्यं नः पाहि सुव्रत} %9-20

\twolineshloka
{श्रुत्वा जाम्बवतो वाक्यं हनूमानतिहर्षितः}
{चकार नादं सिंहस्य ब्रह्माण्डं स्फोटयन्निव} %9-21

\twolineshloka
{बभूव पर्वताकारस्त्रिविक्रम इवापरः}
{लङ्घयित्वा जलनिधिं कृत्वा लङ्कां च भस्मसात्} %9-22

\twolineshloka
{रावणं सकुलं हत्वाऽऽनेष्ये जनकनन्दिनीम्}
{यद्वा बद्ध्वा गले रज्ज्वा रावणं वामपाणिना} %9-23

\twolineshloka
{लङ्कां सपर्वतां धृत्वा रामस्याग्रे क्षिपाम्यहम्}
{यद्वा दृष्ट्वैव यास्यामि जानकीं शुभलक्षणाम्} %9-24

\twolineshloka
{श्रुत्वा हनुमतो वाक्यं जाम्बवानिदमब्रवीत्}
{दृष्ट्वैवाऽऽगच्छ भद्रं ते जीवन्तीं जानकीं शुभाम्} %9-25

\twolineshloka
{पश्चाद्रामेण सहितो दर्शयिष्यसि पौरुषम्}
{कल्याणं भवताद्भद्र गच्छतस्ते विहायसा} %9-26

\twolineshloka
{गच्छन्तं रामकार्यार्थं वायुस्त्वामनुगच्छतु}
{इत्याशीर्भिः समामन्त्र्य विसृष्टः प्लवगाधिपैः} %9-27

{महेन्द्राद्रिशिरो गत्वा बभूवाद्भुतदर्शनः॥२८॥} %9-28
\refstepcounter{shlokacount}


\twolineshloka
{महानगेन्द्रप्रतिमो महात्मा सुवर्णवर्णोऽरुणचारुवक्त्रः}
{महाफणीन्द्राभसुदीर्घबाहुर्वातात्मजोऽदृश्यत सर्वभूतैः} %9-29

{॥इति श्रीमदध्यात्मरामायणे उमामहेश्वरसंवादे किष्किन्धाकाण्डे
नवमः सर्गः॥९॥}
%%%%%%%%%%%%%%%%%%%%

इति श्रीमदध्यात्मरामायणे किष्किन्धाकाण्डः समाप्तः॥