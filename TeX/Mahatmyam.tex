

\chapt{अध्यात्मरामायणमाहात्म्यम्}

\twolineshloka*
{रामं विश्वमयं वन्दे रामं वन्दे रघूद्वहम्}
{रामं विप्रवरं वन्दे रामं श्यामाग्रजं भजे}

\twolineshloka*
{यस्य वागंशुतश्च्युतं रम्यं रामायणामृतम्}
{शैलजासेवितं वन्दे तं शिवं सोमरूपिणम्}

\twolineshloka*
{सच्चिदानन्दसन्दोहं भक्तिभूतिविभूषणम्}
{पूर्णानन्दमहं वन्दे सद्गुरुं शङ्करं स्वयम्}

\twolineshloka*
{अज्ञानध्वान्तसंहर्त्री ज्ञानलोकविलासिनी}
{चन्द्रचूडवचश्चन्द्रचन्द्रिकेयं विराजते}

\twolineshloka
{अप्रमेयत्रयातीतनिर्मलज्ञानमूर्तये}
{मनोगिरां विदूराय दक्षिणामूर्तये नमः} %0-1

\textbf{सूत उवाच}

\twolineshloka
{कदाचिन्नारदो योगी परानुग्रहवाञ्छया}
{पर्यटन् सकलान् लोकान् सत्यलोकमुपागमत्} %0-2

\twolineshloka
{तत्र दृष्ट्वा मूर्तिमद्भिश्छन्दोभिः परिवेष्टितम्}
{बालार्कप्रभया सम्यग्भासयन्तं सभागृहम्} %0-3

\twolineshloka
{मार्कण्डेयादिमुनिभिः स्तूयमानं मुहुर्मुहुः}
{सर्वार्थगोचरज्ञानं सरस्वत्या समन्वितम्} %0-4

\twolineshloka
{चतुर्मुखं जगन्नाथं भक्ताभीष्टफलप्रदम्}
{प्रणम्य दण्डवद्भक्त्या तुष्टाव मुनिपुङ्गवः} %0-5

\twolineshloka
{सन्तुष्टस्तं मुनिं प्राह स्वयम्भूर्वैष्णवोत्तमम्}
{किं प्रष्टुकामस्त्वमसि तद्वदिष्यामि ते मुने} %0-6

\twolineshloka
{इत्याकर्ण्य वचस्तस्य मुनिर्ब्रह्माणमब्रवीत्}
{त्वत्तः श्रुतं मया सर्वं पूर्वमेव शुभाशुभम्} %0-7

\twolineshloka
{इदानीमेकमेवास्ति श्रोतव्यं सुरसत्तम}
{तद्रहस्यमपि ब्रूहि यदि तेऽनुग्रहो मयि} %0-8

\twolineshloka
{प्राप्ते कलियुगे घोरे नराः पुण्यविवर्जिताः}
{दुराचाररताः सर्वे सत्यवार्तापराङ्मुखाः} %0-9

\twolineshloka
{परापवादनिरताः परद्रव्याभिलाषिणः}
{परस्त्रीसक्तमनसः परहिंसापरायणाः} %0-10

\twolineshloka
{देहात्मदृष्टयो मूढा नास्तिका पशुबुद्धयः}
{मातापितृकृतद्वेषाः स्त्रीदेवाः कामकिङ्कराः} %0-11

\twolineshloka
{विप्रा लोभग्रहग्रस्ता वेदविक्रयजीविनः}
{धनार्जनार्थमभ्यस्तविद्या मदविमोहिताः} %0-12

\twolineshloka
{त्यक्तस्वजातिकर्माणः प्रायशः परवञ्चकाः}
{क्षत्रियाश्च तथा वैश्याः स्वधर्मत्यागशीलिनः} %0-13

\twolineshloka
{तद्वच्छूद्राश्च ये केचिद्ब्राह्मणाचारतत्पराः}
{स्त्रियश्च प्रायशो भ्रष्टा भर्त्रवज्ञाननिर्भयाः} %0-14

\twolineshloka
{श्वशुरद्रोहकारिण्यो भविष्यन्ति न संशयः}
{एतेषां नष्टबुद्धीनां परलोकः कथं भवेत्} %0-15

\threelineshloka
{इति चिन्ताकुलं चित्तं जायते मम सन्ततम्}
{लघूपायेन येनैषां परलोकगतिर्भवेत्}
{तमुपायमुपाख्याहि सर्वं वेत्ति यतो भवान्} %0-16

\twolineshloka
{इत्यृषेर्वाक्यमाकर्ण्य प्रत्युवाचाम्बुजासनः}
{साधु पृष्टं त्वया साधो वक्ष्ये तच्छृणु सादरम्} %0-17

\twolineshloka
{पुरा त्रिपुरहन्तारं पार्वती भक्तवत्सला}
{श्रीरामतत्त्वं जिज्ञासुः पप्रच्छ विनयान्विता} %0-18

\twolineshloka
{प्रियायै गिरिशस्तस्यै गूढं व्याख्यातवान् स्वयम्}
{पुराणोत्तममध्यात्मरामायणमिति स्मृतम्} %0-19

\twolineshloka
{तत्पार्वती जगद्धात्री पूजयित्वा दिवानिशम्}
{आलोचयन्ती स्वानन्दमग्ना तिष्ठति साम्प्रतम्} %0-20

\twolineshloka
{प्रचरिष्यति तल्लोके प्राण्यदृष्टवशाद्यदा}
{तस्याध्ययनमात्रेण जना यास्यन्ति सद्गतिम्} %0-21

\twolineshloka
{तावद्विजृम्भते पापं ब्रह्महत्यापुरःसरम्}
{यावज्जगति नाध्यात्मरामायणमुदेष्यति} %0-22

\twolineshloka
{तावत्कलिमहोत्साहो निःशङ्कं सम्प्रवर्तते}
{यावज्जगति नाध्यात्मरामायणमुदेष्यति} %0-23

\twolineshloka
{तावद्यमभटाः शूराः सञ्चरिष्यन्ति निर्भयाः}
{यावज्जगति नाध्यात्मरामायणमुदेष्यति} %0-24

{तावत्सर्वाणि शास्त्राणि विवदन्ते परस्परम्॥२५॥} %0-25
\refstepcounter{shlokacount}


\twolineshloka
{तावत्स्वरूपं रामस्य दुर्बोधं महतामपि}
{यावज्जगति नाध्यात्मरामायणमुदेष्यति} %0-26

\twolineshloka
{अध्यात्मरामायणसङ्कीर्तनश्रवणादिजम्}
{फलं वक्तुं न शक्नोमि कार्त्स्न्येन मुनिसत्तम} %0-27

\twolineshloka
{तथाऽपि तस्य माहात्म्यं वक्ष्ये किञ्चित्तवानघ}
{शृणु चित्तं समाधाय शिवेनोक्तं पुरा मम} %0-28

\twolineshloka
{अध्यात्मरामायणतः श्लोकं श्लोकार्धमेव वा}
{यः पठेत् भक्तिसंयुक्तः स पापान्मुच्यते क्षणात्} %0-29

\twolineshloka
{यस्तु प्रत्यहमध्यात्मरामायणमनन्यधीः}
{यथाशक्ति वदेद्भक्त्या स जीवन्मुक्त उच्यते} %0-30

\twolineshloka
{यो भक्त्यार्चयतेऽध्यात्मरामायणमतन्द्रितः}
{दिने दिनेऽश्वमेधस्य फलं तस्य भवेन्मुने} %0-31

\twolineshloka
{यदृच्छयाऽपि योऽध्यात्मरामायणमनादरात्}
{अन्यतः शृणुयान्मर्त्यः सोऽपि मुच्येत पातकात्} %0-32

\twolineshloka
{नमस्करोति योऽध्यात्मरामायणमदूरतः}
{सर्वदेवार्चनफलं स प्राप्नोति न संशयः} %0-33

\twolineshloka
{लिखित्वा पुस्तकेऽध्यात्मरामायणमशेषतः}
{यो दद्याद्रामभक्तेभ्यस्तस्य पुण्यफलं शृणु} %0-34

\twolineshloka
{अधीतेषु च वेदेषु शास्त्रेषु व्याकृतेषु च}
{यत्फलं दुर्लभं लोके तत्फलं तस्य सम्भवेत्} %0-35

\twolineshloka
{एकादशीदिनेऽध्यात्मरामायणमुपोषितः}
{यो रामभक्तः सदसि व्याकरोति नरोत्तमः} %0-36

\twolineshloka
{तस्य पुण्यफलं वक्ष्ये शृणु वैष्णवसत्तम}
{प्रत्यक्षरं तु गायत्रीपुरश्चर्याफलं भवेत्} %0-37

\threelineshloka
{उपवासव्रतं कृत्वा श्रीरामनवमीदिने}
{रात्रौ जागरितोऽध्यात्मरामायणमनन्यधीः}
{यः पठेच्छृणुयाद्वाऽपि तस्य पुण्यं वदाम्यहम्} %0-38

\twolineshloka
{कुरुक्षेत्रादिनिखिलपुण्यतीर्थेष्वनेकशः}
{आत्मतुल्यं धनं सूर्यग्रहणे सर्वतोमुखे} %0-39

\twolineshloka
{विप्रेभ्यो व्यासतुल्येभ्यो दत्वा यत्फलमश्नुते}
{तत्फलं सम्भवेत्तस्य सत्यं सत्यं न संशयः} %0-40

\twolineshloka
{यो गायते मुदाऽध्यात्मरामायणमहर्निशम्}
{आज्ञां तस्य प्रतीक्षन्ते देवा इन्द्रपुरोगमाः} %0-41

\twolineshloka
{पठन् प्रत्यहमध्यात्मरामायणमनुव्रतः}
{यद्यत्करोति तत्कर्म ततः कोटिगुणं भवेत्} %0-42

\twolineshloka
{तत्र श्रीरामहृदयं यः पठेत् सुसमाहितः}
{स ब्रह्मघ्नोऽपि पूतात्मा त्रिभिरेव दिनैर्भवेत्} %0-43

\twolineshloka
{श्रीरामहृदयं यस्तु हनूमत्प्रतिमान्तिके}
{त्रिः पठेत् प्रत्यहं मौनी स सर्वेप्सितभाग्भवेत्} %0-44

\twolineshloka
{पठन् श्रीरामहृदयं तुलस्यश्वत्थयोर्यदि}
{प्रत्यक्षरं प्रकुर्वीत ब्रह्महत्यानिवर्तनम्} %0-45

\twolineshloka
{श्रीरामगीतामाहात्म्यं कृत्स्नं जानाति शङ्करः}
{तदर्धं गिरिजा वेत्ति तदर्धं वेद्म्यहं मुने} %0-46

\twolineshloka
{तत्ते किञ्चित्प्रवक्ष्यामि कृत्स्नं वक्तुं न शक्यते}
{यज्ज्ञात्वा तत्क्षणाल्लोकश्चित्तशुद्धिमवाप्नुयात्} %0-47

\threelineshloka
{श्रीरामगीता यत्पापं न नाशयति नारद}
{तन्न नश्यति तीर्थादौ लोके क्वापि कदाचन}
{तन्न पश्याम्यहं लोके मार्गमाणोऽपि सर्वदा} %0-48

\twolineshloka
{रामेणोपनिषत्सिन्धुमुन्मत्थ्योत्पादितं मुदा}
{लक्ष्मणायार्पितां गीतासुधां पीत्वाऽमरो भवेत्} %0-49

\twolineshloka
{जमदग्निसुतः पुर्वं कार्तवीर्यवधेच्छया}
{धनुर्विद्यामभ्यसितुं महेशस्यान्तिके वसन्} %0-50

\twolineshloka
{अधीयमानां पार्वत्या रामगीतां प्रयत्नतः}
{श्रूत्वा गृहीत्वाऽऽशु पठन्नारायणकलामगात्} %0-51

\twolineshloka
{ब्रह्महत्यादिपापानां निष्कृतिं यदि वाञ्छति}
{रामगीतां मासमात्रं पठित्वा मुच्यते नरः} %0-52

\twolineshloka
{दुष्प्;रतिग्रहदुर्भोज्यदुरालापादिसम्भवम्}
{पापं यत्तत्कीर्तनेन रामगीता विनाशयेत्} %0-53

\twolineshloka
{शालग्रामशिलाग्रे च तुलस्यश्वत्थसन्निधौ}
{यतीनां पुरतस्तद्वत् रामगीतां पठेत्तु यः} %0-54

{स तत्फलमवाप्नोति यद्वाचोऽपि न गोचरम्॥५५॥} %0-55
\refstepcounter{shlokacount}


\twolineshloka
{रामगीतां पठन् भक्त्या यः श्राद्धे भोजयेद्द्विजान्}
{तस्य ते पितरः सर्वे यान्ति विष्णोः परं पदम्} %0-56

\threelineshloka
{एकादश्यां निराहारो नियतो द्वादशीदिने}
{स्थित्वागस्त्यतरोर्मूले रामगीतां पठेत्तु यः}
{स एव राघवः साक्षात् सर्वदेवैश्च पूज्यते} %0-57

\threelineshloka
{विना दानां विना ध्यानं विना तीर्थावगाहनम्}
{बहुना किमिहोक्तेन शृणु नारद तत्त्वतः}
{रामगीतां नरोऽधीत्य तदनन्तफलं लभेत्} %0-58

\twolineshloka
{श्रुतिस्मृतिपुराणेतिहासागमशतानि च}
{अर्हन्ति नाल्पमध्यात्मरामायणकलामपि} %0-59

\fourlineindentedshloka
{अध्यात्मरामचरितस्य मुनीश्वराय}
{माहात्म्यमेतदुदितं कमलासनेन}
{यः श्रद्धया पठति वा शृणुयात् स मर्त्यः}
{प्राप्नोति विष्णुपदवीं सुरपूज्यमानः} %0-60

{॥इति श्रीब्रह्माण्डपुराणे उत्तरखण्डे अध्यात्मरामायणमाहात्म्यं सम्पूर्णम्॥
}
%%%%%%%%%%%%%%%%%%%%

