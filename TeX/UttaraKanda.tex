% !TeX program = XeLaTeX
% !TeX root = AdhyatmaRamayanaBook-kindle.tex
\chapt{उत्तरकाण्डः}


\sect{प्रथमः सर्गः}

\textbf{श्रीमहादेव उवाच}

\twolineshloka
{जयति रघुवंशतिलकः कौसल्याहृदयनन्दनो रामः}
{दशवदननिधनकारी दाशरथिः पुण्डरीकाक्षः} %1-1

\textbf{पार्वत्युवाच}

\twolineshloka
{अथ रामः किमकरोत्कौसल्यानन्दवर्धनः}
{हत्वा मृधे रावणादीन् राक्षसान् भीमविक्रमः} %1-2

\twolineshloka
{अभिषिक्तस्त्वयोध्यायां सीतया सह राघवः}
{मायामानुषतां प्राप्य कति वर्षाणि भूतले} %1-3

\twolineshloka
{स्थितवान् लीलया देवः परमात्मा सनातनः}
{अत्यजन्मानुषं लोकं कथमन्ते रघूद्वहः} %1-4

\threelineshloka
{एतदाख्याहि भगवन् श्रद्दधत्या मम प्रभो}
{कथापीयुषमास्वाद्य तृष्णा मेऽतीव वर्धते}
{रामचन्द्रस्य भगवन् ब्रूहि विस्तरशः कथाम्} %1-5

\textbf{श्रीमहादेव उवाच}

\twolineshloka
{राक्षसानां वधं कृत्वा राज्ये राम उपस्थिते}
{आययुर्मुनयः सर्वे श्रीराममभिवन्दितुम्} %1-6

\twolineshloka
{विश्वामित्रोऽसितः कण्वो दुर्वासा भृगुरङ्गिराः}
{कश्यपो वामदेवोऽत्रिस्तथा सप्तर्षयोऽमलाः} %1-7

\twolineshloka
{अगस्त्यः सह शिष्यैश्च मुनिभिः सहितोऽभ्यगात्}
{द्वारमासाद्य रामस्य द्वारपालमथाब्रवीत्} %1-8

\twolineshloka
{ब्रूहि रामाय मुनयः समागत्य बहिःस्थिताः}
{अगस्त्यप्रमुखाः सर्वे आशीर्भिरभिनन्दितुम्} %1-9

\twolineshloka
{प्रतीहारस्ततो राममगस्त्यवचनाद् द्रुतम्}
{नमस्कृत्याब्रवीद्वाक्यं विनयावनतः प्रभुम्} %1-10

\twolineshloka
{कृताञ्जलिरुवाचेदमगस्त्यो मुनिभिः सह}
{देव त्वद्दर्शनार्थाय प्राप्तो बहिरुपस्थितः} %1-11

\twolineshloka
{तमुवाच द्वारपालं प्रवेशय यथासुखम्}
{पूजिता विविशुर्वेश्म नानारत्नविभूषितम्} %1-12

\twolineshloka
{दृष्ट्वा रामो मुनीन् शीघ्रं प्रत्युत्थाय कृताञ्जलिः}
{पाद्यार्घ्यादिभिरापूज्य गां निवेद्य यथाविधि} %1-13

\twolineshloka
{नत्वा तेभ्यो ददौ दिव्यान्यासनानि यथार्हतः}
{उपविष्टा प्रहृष्टाश्च मुनयो रामपूजिताः} %1-14

\twolineshloka
{सम्पृष्टकुशलाः सर्वे रामं कुशलमब्रुवन्}
{कुशलं ते महाबाहो सर्वत्र रघुनन्दन} %1-15

\twolineshloka
{दिष्ट्येदानीं प्रपश्यामो हतशत्रुमरिन्दम}
{न हि भारः स ते राम रावणो राक्षसेश्वरः} %1-16

\twolineshloka
{सधनुस्त्वं हि लोकांस्त्रीन् विजेतुं शक्त एव हि}
{दिष्ट्या त्वया हताः सर्वे राक्षसा रावणादयः} %1-17

\twolineshloka
{सह्यमेतन्महाबाहो रावणस्य निबर्हणम्}
{असह्यमेतत्सम्प्राप्तं रावणेर्यन्निषूदनम्} %1-18

\twolineshloka
{अन्तकप्रतिमाः सर्वे कुम्भकर्णादयो मृधे}
{अन्तकप्रतिमैर्बाणैर्हतास्ते रघुसत्तम} %1-19

\twolineshloka
{दत्ता चेयं त्वयाऽस्माकं पुरा ह्यभयदक्षिणा}
{हत्वा रक्षोगणान् सङ्ख्ये कृतकृत्योऽद्य जीवसि} %1-20

\twolineshloka
{श्रुत्वा तु भाषितं तेषां मुनीनां भावितात्मनाम्}
{विस्मयं परमं गत्वा रामः प्राञ्जलिरब्रवीत्} %1-21

\twolineshloka
{रावणादीनतिक्रम्य कुम्भकर्णादिराक्षसान्}
{त्रिलोकजयिनो हित्वा किं प्रशंसथ रावणिम्} %1-22

\twolineshloka
{ततस्तद्वचनं श्रुत्वा राघवस्य महात्मनः}
{कुम्भयोनिर्महातेजा रामं प्रीत्या वचोऽब्रवीत्} %1-23

\twolineshloka
{शृणु राम यथा वृत्तं रावणे रावणस्य च}
{जन्म कर्म वरादानं सङ्क्षेपाद्वदतो मम} %1-24

\twolineshloka
{पुरा कृतयुगे राम पुलस्त्यो ब्रह्मणः सुतः}
{तपस्तप्तुं गतो विद्वान् मेरोः पार्श्वं महामतिः} %1-25

\twolineshloka
{तृणबिन्दोराश्रमेऽसौ न्यवसन्मुनिपुङ्गवः}
{तपस्तेपे महातेजाः स्वाध्यायनिरतः सदा} %1-26

\twolineshloka
{तत्राऽऽश्रमे महारम्ये देवगन्धर्वकन्यकाः}
{गायन्त्यो ननृतुस्तत्र हसन्त्यो वादयन्ति च} %1-27

\twolineshloka
{पुलस्त्यस्य तपोविघ्नं चक्रुः सर्वा अनिन्दिताः}
{ततः क्रुद्धो महातेजा व्याजहार वचो महत्} %1-28

\twolineshloka
{या मे दृष्टिपथं गच्छेत्सा गर्भं धारयिष्यति}
{ताः सर्वाः शापसंविग्ना न तं देशं प्रचक्रमुः} %1-29

\twolineshloka
{तृणबिन्दोस्तु राजर्षेः कन्या तन्नाशृणोद्वचः}
{विचचार मुनेरग्रे निर्भया तं प्रपश्यती} %1-30

\twolineshloka
{बभूव पाण्डुरतनुर्व्यञ्जितान्तःशरीरजा}
{दृष्ट्वा सा देहवैवर्ण्यं भीता पितरमन्वगात्} %1-31

\twolineshloka
{तृणबिन्दुश्च तां दृष्ट्वा राजर्षिरमितद्युतिः}
{ध्यात्वा मुनिकृतं सर्वमवैद्विज्ञानचक्षुषा} %1-32

\twolineshloka
{तां कन्यां मुनिवर्याय पुलस्त्याय ददौ पिता}
{तां प्रगृह्याब्रवीत्कन्यां बाढमित्येव स द्विजः} %1-33

\twolineshloka
{शुश्रूषणपरां दृष्ट्वा मुनिः प्रीतोऽब्रवीद्वचः}
{दास्यामि पुत्रमेकं ते उभयोर्वंशवर्धनम्} %1-34

\twolineshloka
{ततः प्रासूत सा पुत्रं पुलस्त्याल्लोकविश्रुतम्}
{विश्रवा इति विख्यातः पौलस्त्यो ब्रह्मविन्मुनिः} %1-35

\twolineshloka
{तस्य शीलादिकं दृष्ट्वा भरद्वाजो महामुनिः}
{भार्यार्थं स्वां दुहितरं ददौ विश्रवसे मुदा} %1-36

\twolineshloka
{तस्यां तु पुत्रः सञ्जज्ञे पौलस्त्याल्लोकसम्मतः}
{पितृतुल्यो वैश्रवणो ब्रह्मणा चानुमोदितः} %1-37

\twolineshloka
{ददौ तत्तपसा तुष्टो ब्रह्मा तस्मै वरं शुभम्}
{मनोऽभिलषितं तस्य धनेशत्वमखण्डितम्} %1-38

\twolineshloka
{ततो लब्धवरः सोऽपि पितरं द्रष्टुमागतः}
{पुष्पकेण धनाध्यक्षो ब्रह्मदत्तेन भास्वता} %1-39

\twolineshloka
{नमस्कृत्याथ पितरं निवेद्य तपसः फलम्}
{प्राह मे भगवान् ब्रह्मा दत्त्वा वरमनिन्दितम्} %1-40

\twolineshloka
{निवासाय न मे स्थानं दत्तवान् परमेश्वरः}
{ब्रूहि मे नियतं स्थानं हिंसा यत्र न कस्यचित्} %1-41

\twolineshloka
{विश्रवा अपि तं प्राह लङ्कानाम पुरी शुभा}
{राक्षसानां निवासाय निर्मिता विश्वकर्मणा} %1-42

\twolineshloka
{त्यक्त्वा विष्णुभयाद्दैत्या विविशुस्ते रसातलम्}
{सा पुरी दुष्प्रधर्षान्यैर्मध्येसागरमास्थिता} %1-43

\twolineshloka
{तत्र वासाय गच्छ त्वं नान्यैः साधिष्ठिता पुरा}
{पित्रादिष्टस्त्वसौ गत्वा तां पुरीं धनदोऽविशत्} %1-44

\twolineshloka
{स तत्र सुचिरं कालमुवास पितृसम्मतः}
{कस्यचित्त्वथ कालस्य सुमाली नाम राक्षसः} %1-45

\twolineshloka
{रसातलान्मर्त्यलोकं चचार पिशिताशनः}
{गृहीत्वा तनयां कन्यां साक्षाद्देवीमिव श्रियम्} %1-46

\twolineshloka
{अपश्यद्धनदं देवं चरन्तं पुष्पकेण सः}
{हिताय चिन्तयामास राक्षसानां महामनाः} %1-47

\twolineshloka
{उवाच तनयां तत्र कैकसीं नाम नामतः}
{वत्से विवाहकालस्ते यौवनं चातिवर्तते} %1-48

\twolineshloka
{प्रत्याख्यानाच्च भीतैस्त्वं न वरैर्गृह्यसे शुभे}
{सा त्वं वरय भद्रं ते मुनिं ब्रह्मकुलोद्भवम्} %1-49

\twolineshloka
{स्वयमेव ततः पुत्रा भविष्यन्ति महाबलाः}
{ईदृशाः सर्वशोभाढ्या धनदेन समाः शुभे} %1-50

\twolineshloka
{तथेति साऽऽश्रमं गत्वा मुनेरग्रे व्यवस्थिता}
{लिखन्ती भुवमग्रेण पादेनाधोमुखी स्थिता} %1-51

\twolineshloka
{तामपृच्छन्मुनिः का त्वं कन्याऽसि वरवर्णिनि}
{साऽब्रवीत्प्राञ्जलिर्ब्रह्मन् ध्यानेन ज्ञातुमर्हसि} %1-52

\twolineshloka
{ततो ध्यात्वा मुनिः सर्वं ज्ञात्वा तां प्रत्यभाषत}
{ज्ञातं तवाभिलषितं मत्तः पुत्रानभीप्स्यसि} %1-53

\twolineshloka
{दारुणायां तु वेलायामागताऽसि सुमध्यमे}
{अतस्ते दारुणौ पुत्रौ राक्षसौ सम्भविष्यतः} %1-54

\twolineshloka
{साऽब्रवीन्मुनिशार्दूल त्वत्तोऽप्येवंविधौ सुतौ}
{तामाह पश्चिमो यस्ते भविष्यति महामतिः} %1-55

\twolineshloka
{महाभागवतः श्रीमान् रामभक्त्येकतत्परः}
{इत्युक्ता सा तथा काले सुषुवे दशकन्धरम्} %1-56

\twolineshloka
{रावणं विंशतिभुजं दशशीर्षं सुदारुणम्}
{तद्रक्षोजातमात्रेण चचाल च वसुन्धरा} %1-57

\twolineshloka
{बभूवुर्नाशहेतूनि निमित्तान्यखिलान्यपि}
{कुम्भकर्णस्ततो जातो महापर्वतसन्निभः} %1-58

\twolineshloka
{ततः शूर्पणखा नाम जाता रावणसोदरी}
{ततो विभीषणो जातः शान्तात्मा सौम्यदर्शनः} %1-59

\twolineshloka
{स्वाध्यायी नियताहारो नित्यकर्मपरायणः}
{कुम्भकर्णस्तु दुष्टात्मा द्विजान् सन्तुष्टचेतसः} %1-60

\threelineshloka
{भक्षयन्नृषिसङ्घांश्च विचचारातिदारुणः}
{रावणोऽपि महासत्त्वो लोकानां भयदायकः}
{ववृधे लोकनाशाय ह्यामयो देहिनामिव} %1-61

\fourlineindentedshloka
{राम त्वं सकलान्तरस्थमभितो जानासि विज्ञानदृक्}
{साक्षी सर्वहृदि स्थितो हि परमो नित्योदितो निर्मलः}
{त्वं लीलामनुजाकृतिः स्वमहिमन् मायागुणैर्नाज्यसे}
{लीलार्थं प्रतिचोदितोऽद्य भवता वक्ष्यामि रक्षोद्भवम्} %1-62

\fourlineindentedshloka
{जानामि केवलमनन्तमचिन्त्यशक्तिम्}
{चिन्मात्रमक्षरमजं विदितात्मतत्त्वम्}
{त्वां राम गूढनिजरूपमनुप्रवृत्तो}
{मूढोऽप्यहं भवदनुग्रहतश्चरामि} %1-63

\fourlineindentedshloka
{एवं वदन्तमिनवंशपवित्रकीर्तिः}
{कुम्भोद्भवं रघुपतिः प्रहसन् बभाषे}
{मायाश्रितं सकलमेतदनन्यकत्वात्}
{मत्कीर्तनं जगति पापहरं निबोध} %1-64

{॥इति श्रीमदध्यात्मरामायणे उमामहेश्वरसंवादे उत्तरकाण्डे
प्रथमः सर्गः॥१॥
}
%%%%%%%%%%%%%%%%%%%%



\sect{द्वितीयः सर्गः}

\textbf{श्रीमहादेव उवाच}

\twolineshloka
{श्रीरामवचनं श्रुत्वा परमानन्दनिर्भरः}
{मुनिः प्रोवाच सदसि सर्वेषां तत्र शृण्वताम्} %2-1

\twolineshloka
{अथ वित्तेश्वरो देवस्तत्र कालेन केनचित्}
{आययौ पुष्पकारूढः पितरं द्रष्टुमञ्जसा} %2-2

\twolineshloka
{दृष्ट्वा तं कैकसी तत्र भ्राजमानं महौजसम्}
{राक्षसी पुत्रसामीप्यं गत्वा रावणमब्रवीत्} %2-3

\twolineshloka
{पुत्र पश्य धनाध्यक्षं ज्वलन्तं स्वेन तेजसा}
{त्वमप्येवं यथा भूयास्तथा यत्नं कुरु प्रभो} %2-4

\twolineshloka
{तच्छ्रुत्वा रावणो रोषात् प्रतिज्ञामकरोद्द्रुतम्}
{धनदेन समो वाऽपि ह्यधिको वाऽचिरेण तु} %2-5

\twolineshloka
{भविष्याम्यम्ब मां पश्य सन्तापं त्यज सुव्रते}
{इत्युक्त्वा दुष्करं कर्तुं तपः स दशकन्धरः} %2-6

\twolineshloka
{अगमत्फलसिद्ध्यर्थं गोकर्णं तु सहानुजः}
{स्वं स्वं नियममास्थाय भ्रातरस्ते तपो महत्} %2-7

\twolineshloka
{आस्थिता दुष्करं घोरं सर्वलोकैकतापनम्}
{दशवर्षसहस्राणि कुम्भकर्णोऽकरोत्तपः} %2-8

\twolineshloka
{विभीषणोऽपि धर्मात्मा सत्यधर्मपरायणः}
{पञ्चवर्षसहस्राणि पादेनैकेन तस्थिवान्} %2-9

\threelineshloka
{दिव्यवर्षसहस्रं तु निराहारो दशाननः}
{पूर्णे वर्षसहस्रे तु शीर्षमग्नौ जुहाव सः}
{एवं वर्षसहस्राणि नव तस्यातिचक्रमुः} %2-10

\threelineshloka
{अथ वर्षसहस्रं तु दशमे दशमं शिरः}
{छेत्तुकामस्य धर्मात्मा प्राप्तश्चाथ प्रजापतिः}
{वत्स वत्स दशग्रीव प्रीतोऽस्मीत्यभ्यभाषत} %2-11

\twolineshloka
{वरं वरय दास्यामि यत्ते मनसि काङ्क्षितम्}
{दशग्रीवोऽपि तच्छ्रुत्वा प्रहृष्टेनान्तरात्मना} %2-12

\threelineshloka
{अमरत्वं वृणोमीश वरदो यदि मे भवान्}
{सुपर्णनागयक्षाणां देवतानां तथाऽसुरैः}
{अवध्यत्वं तु मे देहि तृणभूता हि मानुषाः} %2-13

\twolineshloka
{तथाऽस्त्विति प्रजाध्यक्षः पुनराह दशाननम्}
{अग्नौ हुतानि शीर्षाणि यानि तेऽसुरपुङ्गव} %2-14

{भविष्यन्ति यथापूर्वमक्षयाणि च सत्तम॥१५॥} %2-15
\refstepcounter{shlokacount}


\twolineshloka
{एवमुक्त्वा ततो राम दशग्रीवं प्रजापतिः}
{विभीषणमुवाचेदं प्रणतं भक्तवत्सलः} %2-16

\twolineshloka
{विभीषण त्वया वत्स कृतं धर्मार्थमुत्तमम्}
{तपस्ततो वरं वत्स वृणीष्वाभिमतं हितम्} %2-17

\threelineshloka
{विभीषणोऽपि तं नत्वा प्राञ्जलिर्वाक्यमब्रवीत्}
{देव मे सर्वदा बुद्धिर्धर्मे तिष्ठतु शाश्वती}
{मा रोचयत्वधर्मं मे बुद्धिः सर्वत्र सर्वदा} %2-18

\twolineshloka
{ततः प्रजापतिः प्रीतो विभीषणमथाब्रवीत्}
{वत्स त्वं धर्मशीलोऽसि तथैव च भविष्यसि} %2-19

\twolineshloka
{अयाचितोऽपि ते दास्ये ह्यमरत्वं विभीषण}
{कुम्भकर्णमथोवाच वरं वरय सुव्रत} %2-20

\twolineshloka
{वाण्या व्याप्तोऽथ तं प्राह कुम्भकर्णः पितामहम्}
{स्वप्स्यामि देव षण्मासान् दिनमेकं तु भोजनम्} %2-21

\twolineshloka
{एवमस्त्विति तं प्राह ब्रह्मा दृष्ट्वा दिवौकसः}
{सरस्वती च तद्वक्त्रान्निर्गता प्रययौ दिवम्} %2-22

\twolineshloka
{कुम्भकर्णस्तु दुष्टात्मा चिन्तयामास दुःखितः}
{अनभिप्रेतमेवास्यात्किं निर्गतमहो विधिः} %2-23

\twolineshloka
{सुमाली वरलब्धांस्तान् ज्ञात्वा पौत्रान् निशाचरान्}
{पातालान्निर्भयः प्रायात् प्रहस्तादिभिरन्वितः} %2-24

\twolineshloka
{दशग्रीवं परिष्वज्य वचनं चेदमब्रवीत्}
{दिष्ट्या ते पुत्र संवृत्तो वाञ्छितो मे मनोरथः} %2-25

\twolineshloka
{यद्भयाच्च वयं लङ्कां त्यक्त्वा याता रसातलम्}
{तद्गतं नो महाबाहो महद्विष्णुकृतं भयम्} %2-26

\twolineshloka
{अस्माभिः पूर्वमुषिता लङ्केयं धनदेन ते}
{भ्रात्राक्रान्तामिदानीं त्वं प्रत्यानेतुमिहार्हसि} %2-27

\twolineshloka
{साम्ना वाऽथ बलेनापि राज्ञां बन्धुः कुतः सुहृत्}
{इत्युक्तो रावणः प्राह नार्हस्येवं प्रभाषितुम्} %2-28

\twolineshloka
{वित्तेशो गुरुरस्माकमेवं श्रुत्वा तमब्रवीत्}
{प्रहस्तः प्रश्रितं वाक्यं रावणं दशकन्धरम्} %2-29

\twolineshloka
{शृणु रावण यत्नेन नैवं त्वं वक्तुमर्हसि}
{नाधीता राजधर्मास्ते नीतिशास्त्रं तथैव च} %2-30

\twolineshloka
{शूराणां न हि सौभ्रात्रं शृणु मे वदतः प्रभो}
{कश्यपस्य सुता देवा राक्षसाश्च महाबलाः} %2-31

\twolineshloka
{परस्परमयुध्यन्त त्यक्त्वा सौहृदमायुधैः}
{नैवेदानीन्तनं राजन् वैरं देवैरनुष्ठितम्} %2-32

\twolineshloka
{प्रहस्तस्य वचः श्रुत्वा दशग्रीवो दुरात्मनः}
{तथेति क्रोधताम्राक्षस्त्रिकूटाचलमन्वगात्} %2-33

\twolineshloka
{दूतं प्रहस्तं सम्प्रेष्य निष्कास्य धनदेश्वरम्}
{लङ्कामाक्रम्य सचिवै राक्षसैः सुखमास्थितः} %2-34

\twolineshloka
{धनदः पितृवाक्येन त्यक्त्वा लङ्कां महायशाः}
{गत्वा कैलासशिखरं तपसाऽतोषयच्छिवम्} %2-35

\twolineshloka
{तेन सख्यमनुप्राप्य तेनैव परिपालितः}
{अलकां नगरीं तत्र निर्ममे विश्वकर्मणा} %2-36

\twolineshloka
{दिक्पालत्वं चकारात्र शिवेन परिपालितः}
{रावणो राक्षसैः सार्धमभिषिक्तः सहानुजैः} %2-37

\twolineshloka
{राज्यं चकारासुराणां त्रिलोकीं बाधयन् खलः}
{भगिनीं कालखञ्जाय ददौ विकटरूपिणीम्} %2-38

\twolineshloka
{विद्युज्जिह्वाय नाम्नासौ महामायी निशाचरः}
{ततो मयो विश्वकर्मा राक्षसानां दितेः सुतः} %2-39

\twolineshloka
{सुतां मन्दोदरीं नाम्ना ददौ लोकैकसुन्दरीम्}
{रावणाय पुनः शक्तिममोघां प्रीतमानसः} %2-40

\twolineshloka
{वैरोचनस्य दौहित्रीं वृत्रज्वालेति विश्रुताम्}
{स्वयं दत्तामुदवहत्कुम्भकर्णाय रावणः} %2-41

\twolineshloka
{गन्धर्वराजस्य सुतां शैलूषस्य महात्मनः}
{विभीषणस्य भार्यार्थे धर्मज्ञां समुदावहत्} %2-42

\twolineshloka
{सरमां नाम सुभगां सर्वलक्षणसंयुताम्}
{ततो मन्दोदरी पुत्रं मेघनादमजीजनत्} %2-43

\twolineshloka
{जातमात्रस्तु यो नादं मेघवत्प्रमुमोच ह}
{ततः सर्वेऽब्रुवन्मेघनादोऽयमिति चासकृत्} %2-44

\twolineshloka
{कुम्भकर्णस्ततः प्राह निद्रा मां बाधते प्रभो}
{ततश्च कारयामास गुहां दीर्घां सुविस्तराम्} %2-45

\twolineshloka
{तत्र सुष्वाप मूढात्मा कुम्भकर्णो विघूर्णितः}
{निद्रिते कुम्भकर्णे तु रावणो लोकरावणः} %2-46

\twolineshloka
{ब्राह्मणान् ऋषिमुख्यांश्च देवदानवकिन्नरान्}
{देवश्रियो मनुष्यांश्च निजघ्ने समहोरगान्} %2-47

\twolineshloka
{धनदोऽपि ततः श्रुत्वा रावणस्याक्रमं प्रभुः}
{अधर्मं मा कुरुष्वेति दूतवाक्यैर्न्यवारयत्} %2-48

\twolineshloka
{ततः क्रुद्धो दशग्रीवो जगाम धनदालयम्}
{विनिर्जित्य धनाध्यक्षं जहारोत्तमपुष्पकम्} %2-49

\twolineshloka
{ततो यमं च वरुणं निर्जित्य समरेऽसुरः}
{स्वर्गलोकमगात्तूर्णं देवराजजिघांसया} %2-50

\twolineshloka
{ततोऽभवन्महद्युद्धमिन्द्रेण सह दैवतैः}
{ततो रावणमभ्येत्य बबन्ध त्रिदशेश्वरः} %2-51

\twolineshloka
{तच्छ्रुत्वा सहसाऽऽगत्य मेघनादः प्रतापवन्}
{कृत्वा घोरं महद्युद्धं जित्वा त्रिदशपुङ्गवान्} %2-52

\twolineshloka
{इन्द्रं गृहीत्वा बध्वाऽसौ मेघनादो महाबलः}
{मोचयित्वा तु पितरं गृहीत्वेन्द्रं ययौ पुरम्} %2-53

\twolineshloka
{ब्रह्मा तु मोचयामास देवेन्द्रं मेघनादतः}
{दत्त्वा वरान् बहूंस्तस्मै ब्रह्मा स्वभवनं ययौ} %2-54

\twolineshloka
{रावणो विजयी लोकान् सर्वान् जित्वा क्रमेण तु}
{कैलासं तोलयामास बाहुभिः परिघोपमैः} %2-55

\twolineshloka
{तत्र नन्दीश्वरेणैवं शप्तोऽयं राक्षसेश्वरः}
{वानरैर्मानुषैश्चैव नाशं गच्छेति कोपिना} %2-56

\twolineshloka
{शप्तोऽप्यगणयन् वाक्यं ययौ हैहयपत्तनम्}
{तेन बद्धो दशग्रीवः पुलस्त्येन विमोचितः} %2-57

\twolineshloka
{ततोऽतिबलमासाद्य जिघांसुर्हरिपुङ्गवम्}
{धृतस्तेनैव कक्षेण वालिना दशकन्धरः} %2-58

\twolineshloka
{भ्रामयित्वा तु चतुरः समुद्रान् रावणं हरिः}
{विसर्जयामास ततस्तेन सख्यं चकार सः} %2-59

\twolineshloka
{रावणः परमप्रीत एवं लोकान् महाबलः}
{चकार स्ववशे राम बुभुजे स्वयमेव तान्} %2-60

\twolineshloka
{एवं प्रभावो राजेन्द्र दशग्रीवः सहेन्द्रजित्}
{त्वया विनिहतः सङ्ख्ये रावणो लोकरावणः} %2-61

\twolineshloka
{मेघनादश्च निहतो लक्ष्मणेन महात्मना}
{कुम्भकर्णश्च निहतस्त्वया पर्वतसन्निभः} %2-62

\twolineshloka
{भवान्नारायणः साक्षाज्जगतामादिकृद्विभुः}
{त्वत्स्वरूपमिदं सर्वं जगत्स्थावरजङ्गमम्} %2-63

\twolineshloka
{त्वन्नाभिकमलोत्पन्नो ब्रह्मा लोकपितामहः}
{अग्निस्ते मुखतो जातो वाचा सह रघूत्तम} %2-64

\twolineshloka
{बाहुभ्यां लोकपालौघाश्चक्षुर्भ्यां चन्द्रभास्करौ}
{दिशश्च विदिशश्चैव कर्णाभ्यां ते समुत्थिताः} %2-65

\twolineshloka
{घ्राणात्प्राणः समुत्पन्नश्चाश्विनौ देवसत्तमौ}
{जङ्घाजानूरुजघनाद्भुवर्लोकादयोऽभवन्} %2-66

\twolineshloka
{कुक्षिदेशात्समुत्पन्नाश्चत्वारः सागरा हरे}
{स्तनाभ्यामिन्द्रवरुणौ वालखिल्याश्च रेतसः} %2-67

\twolineshloka
{मेढ्राद्यमो गुदान्मृत्युर्मन्यो रुद्रस्त्रिलोचनः}
{अस्थिभ्यः पर्वता जाताः केशेभ्यो मेघसंहतिः} %2-68

\twolineshloka
{ओषध्यस्तव रोमेभ्यो नखेभ्यश्च खरादयः}
{त्वं विश्वरूपः पुरुषो मायाशक्तिसमन्वितः} %2-69

\twolineshloka
{नानारूप इवाऽऽभासि गुणव्यतिकरे सति}
{त्वामाश्रित्यैव विबुधाः पिबन्त्यमृतमध्वरे} %2-70

\twolineshloka
{त्वया सृष्टमिदं सर्वं विश्वं स्थावरजङ्गमम्}
{त्वामाश्रित्यैव जीवन्ति सर्वे स्थावरजङ्गमाः} %2-71

\twolineshloka
{त्वद्युक्तमखिलं वस्तु व्यवहारेऽपि राघव}
{क्षीरमध्यगतं सर्पिर्यथा व्याप्याखिलं पयः} %2-72

\twolineshloka
{त्वद्भासा भासतेऽर्कादि न त्वं तेनावभाससे}
{सर्वगं नित्यमेकं त्वां ज्ञानचक्षुर्विलोकयेत्} %2-73

\twolineshloka
{नाज्ञानचक्षुस्त्वां पश्येदन्धदृग् भास्करं यथा}
{योगिनस्त्वां विचिन्वन्ति स्वदेहे परमेश्वरम्} %2-74

\twolineshloka
{अतन्निरसनमुखैर्वेदशीर्षैरहर्निशम्}
{त्वत्पादभक्तिलेशेन गृहीता यदि योगिनः} %2-75

\threelineshloka
{विचिन्वन्तो हि पश्यन्ति चिन्मात्रं त्वां न चान्यथा}
{मया प्रलपितं किञ्चित्सर्वज्ञस्य तवाग्रतः}
{क्षन्तुमर्हसि देवेश तवानुग्रहभागहम्} %2-76

\fourlineindentedshloka
{दिग्देशकालपरिहीनमनन्यमेकम्}
{चिन्मात्रमक्षरमजं चलनादिहीनम्}
{सर्वज्ञमीश्वरमनन्तगुणं व्युदस्त-}
{मायं भजे रघुपतिं भजतामभिन्नम्} %2-77

{॥इति श्रीमदध्यात्मरामायणे उमामहेश्वरसंवादे उत्तरकाण्डे द्वितीयः
सर्गः॥२॥
}
%%%%%%%%%%%%%%%%%%%%



\sect{तृतीयः सर्गः}

\textbf{श्रीराम उवाच}

\twolineshloka
{वालिसुग्रीवयोर्जन्म श्रोतुमिच्छामि तत्त्वतः}
{रवीन्द्रौ वानराकारौ जज्ञाताविति नः श्रुतम्} %3-1

\textbf{अगस्त्य उवाच}

\twolineshloka
{मेरोः स्वर्णमयस्याद्रेर्मध्यशृङ्गे मणिप्रभे}
{तस्मिन् सभाऽऽस्ते विस्तीर्णा ब्रह्मणः शतयोजना} %3-2

\twolineshloka
{तस्यां चतुर्मुखः साक्षात्कदाचिद्योगमास्थितः}
{नेत्राभ्यां पतितं दिव्यमानन्दसलिलं बहु} %3-3

\twolineshloka
{तद्गृहीत्वा करे ब्रह्मा ध्यात्वा किञ्चित्तदत्यजत्}
{भूमौ पतितमात्रेण तस्माज्जातो महाकपिः} %3-4

\twolineshloka
{तमाह द्रुहिणो वत्स किञ्चित्कालं वसात्र मे}
{समीपे सर्वशोभाढ्ये ततः श्रेयो भविष्यति} %3-5

\twolineshloka
{इत्युक्तो न्यवसत्तत्र ब्रह्मणा वानरोत्तमः}
{एवं बहुतिथे काले गते ऋक्षाधिपः सुधीः} %3-6

\twolineshloka
{कदाचित्पर्यटन्नद्रौ फलमूलार्थमुद्यतः}
{अपश्यद्दिव्यसलिलां वापीं मणिशिलान्विताम्} %3-7

\twolineshloka
{पानीयं पातुमागच्छत्तत्र छायामयं कपिम्}
{दृष्ट्वा प्रतिकपिं मत्वा निपपात जलान्तरे} %3-8

\twolineshloka
{तत्रादृष्ट्वा हरिं शीघ्रं पुनरुत्प्लुत्य वानरः}
{अपश्यत्सुन्दरीं रामामात्मानं विस्मयं गतः} %3-9

\twolineshloka
{ततः सुरेशो देवेशं पूजयित्वा चतुर्मुखम्}
{गच्छन् मध्याह्नसमये दृष्ट्वा नारीं मनोरमाम्} %3-10

\twolineshloka
{कन्दर्पशरविद्धाङ्गस्त्यक्तवान् वीर्यमुत्तमम्}
{तामप्राप्यैव तद्बीजं वालदेशेऽपतद्भुवि} %3-11

\twolineshloka
{वाली समभवत्तत्र शक्रतुल्यपराक्रमः}
{तस्य दत्त्वा सुरेशानः स्वर्णमालां दिवं गतः} %3-12

\twolineshloka
{भानुरप्यागतस्तत्र तदानीमेव भामिनीम्}
{दृष्ट्वा कामवशो भूत्वा ग्रीवादेशेऽसृजन्महत्} %3-13

\twolineshloka
{बीजं तस्यास्ततः सद्यो महाकायोऽभवद्धरिः}
{तस्य दत्त्वा हनूमन्तं सहायार्थं गतो रविः} %3-14

\twolineshloka
{पुत्रद्वयं समादाय गत्वा सा निद्रिता क्वचित्}
{प्रभातेऽपश्यदात्मानं पूर्ववद्वानराकृतिम्} %3-15

\twolineshloka
{फलमूलादिभिः सार्धं पुत्राभ्यां सहितः कपिः}
{नत्वा चतुर्मुखस्याग्रे ऋक्षराजः स्थितः सुधीः} %3-16

\twolineshloka
{ततोऽब्रवीत्समाश्वास्य बहुशः कपिकुञ्जरम्}
{तत्रैकं देवतादूतमाहूयामरसन्निभम्} %3-17

\twolineshloka
{गच्छ दूत मयाऽऽदिष्टो गृहीत्वा वानरोत्तमम्}
{किष्किन्धां दिव्यनगरीं निर्मितां विश्वकर्मणा} %3-18

\twolineshloka
{सर्वसौभाग्यवलितां देवैरपि दुरासदाम्}
{तस्यां सिंहासने वीरं राजानमभिषेचय} %3-19

\twolineshloka
{सप्तद्वीपगता ये ये वानराः सन्ति दुर्जयाः}
{सर्वे ते ऋक्षराजस्य भविष्यन्ति वशेऽनुगाः} %3-20

\twolineshloka
{यदा नारायणः साक्षाद्रामो भूत्वा सनातनः}
{भूभारासुरनाशाय सम्भविष्यति भूतले} %3-21

\twolineshloka
{तदा सर्वे सहायार्थे तस्य गच्छन्तु वानराः}
{इत्युक्तो ब्रह्मणा दूतो देवानां स महामतिः} %3-22

\twolineshloka
{यथाऽऽज्ञप्तस्तथा चक्रे ब्रह्मणा तं हरीश्वरम्}
{देवदूतस्ततो गत्वा ब्रह्मणे तन्न्यवेदयत्} %3-23

{तदादि वानराणां सा किष्किन्धाऽभून्नृपाश्रयः॥२४॥} %3-24
\refstepcounter{shlokacount}


\threelineshloka
{सर्वेश्वरस्त्वमेवासीरिदानीं ब्रह्मणार्थितः}
{भूमेर्भारो हृतः कृत्स्नस्त्वया लीलानृदेहिना}
{सर्वभूतान्तरस्थस्य नित्यमुक्तचिदात्मनः} %3-25

\twolineshloka
{अखण्डानन्तरूपस्य कियानेष पराक्रमः}
{तथाऽपि वर्ण्यते सद्भिर्लीलामानुषरूपिणः} %3-26

\twolineshloka
{यशस्ते सर्वलोकानां पापहत्यै सुखाय च}
{य इदं कीर्तयेन्मर्त्यो वालिसुग्रीवयोर्महत्} %3-27

{जन्म त्वदाश्रयत्वात्स मुच्यते सर्वपातकैः॥२८॥} %3-28
\refstepcounter{shlokacount}


\twolineshloka
{अथान्यां सम्प्रवक्ष्यामि कथां राम त्वदाश्रयाम्}
{सीता हृता यदर्थं सा रावणेन दुरात्मना} %3-29

\threelineshloka
{पुरा कृतयुगे राम प्रजापतिसुतं विभुम्}
{सनत्कुमारमेकान्ते समासीनं दशाननः}
{विनयावनतो भूत्वा ह्यभिवाद्येदमब्रवीत्} %3-30

\twolineshloka
{को न्वस्मिन् प्रवरो लोके देवानां बलवत्तरः}
{देवाश्च यं समाश्रित्य युद्धे शत्रुं जयन्ति हि} %3-31

\twolineshloka
{कं यजन्ति द्विजा नित्यं कं ध्यायन्ति च योगिनः}
{एतन्मे शंस भगवन् प्रश्नं प्रश्नविदां वर} %3-32

\twolineshloka
{ज्ञात्वा तस्य हृदिस्थं यत्तदशेषेण योगदृक्}
{दशाननमुवाचेदं शृणु वक्ष्यामि पुत्रक} %3-33

\twolineshloka
{भर्ता यो जगतां नित्यं यस्य जन्मादिकं न हि}
{सुरासुरैर्नुतो नित्यं हरिर्नारायणोऽव्ययः} %3-34

\twolineshloka
{यन्नाभिपङ्कजाज्जातो ब्रह्मा विश्वसृजां पतिः}
{सृष्टं येनैव सकलं जगत्स्थावरजङ्गमम्} %3-35

\twolineshloka
{तं समाश्रित्य विबुधा जयन्ति समरे रिपून्}
{योगिनो ध्यानयोगेन तमेवानुजपन्ति हि} %3-36

\twolineshloka
{महर्षेर्वचनं श्रुत्वा प्रत्युवाच दशाननः}
{दैत्यदानवरक्षांसि विष्णुना निहतानि च} %3-37

\twolineshloka
{कां वा गतिं प्रपद्यन्ते प्रेत्य ते मुनिपुङ्गव}
{तमुवाच मुनिश्रेष्ठो रावणं राक्षसाधिपम्} %3-38

\twolineshloka
{दैवतैर्निहता नित्यं गत्वा स्वर्गमनुत्तमम्}
{भोगक्षये पुनस्तस्माद्भ्रष्टा भूमौ भवन्ति ते} %3-39

\twolineshloka
{पूर्वार्जितैः पुण्यपापैर्म्रियन्ते चोद्भवन्ति च}
{विष्णुना ये हतास्ते तु प्राप्नुवन्ति हरेर्गतिम्} %3-40

\twolineshloka
{श्रुत्वा मुनिमुखात्सर्वं रावणो हृष्टमानसः}
{योत्स्येऽहं हरिणा सार्धमिति चिन्तापरोऽभवत्} %3-41

\twolineshloka
{मनःस्थितं परिज्ञाय रावणस्य महामुनिः}
{उवाच वत्स तेऽभीष्टं भविष्यति न संशयः} %3-42

\twolineshloka
{कञ्चित्कालं प्रतीक्षस्व सुखी भव दशानन}
{एवमुक्त्वा महाबाहो मुनिः पुनरुवाच तम्} %3-43

\twolineshloka
{तस्य स्वरूपं वक्ष्यामि ह्यरूपस्यापि मायिनः}
{स्थावरेषु च सर्वेषु नदेषु च नदीषु च} %3-44

\twolineshloka
{ओङ्कारश्चैव सत्यं च सावित्री पृथिवी च सः}
{समस्तजगदाधारः शेषरूपधरो हि सः} %3-45

\twolineshloka
{सर्वे देवाः समुद्राश्च कालः सूर्यश्च चन्द्रमाः}
{सूर्योदयो दिवारात्री यमश्चैव तथाऽनिलः} %3-46

\twolineshloka
{अग्निरिन्द्रस्तथा मृत्युः पर्जन्यो वसवस्तथा}
{ब्रह्मा रुद्रादयश्चैव ये चान्ये देवदानवाः} %3-47

\twolineshloka
{विद्योतते ज्वलत्येष पाति चात्तीति विश्वकृत्}
{क्रीडां करोत्यव्ययात्मा सोऽयं विष्णुः सनातनः} %3-48

\twolineshloka
{तेन सर्वमिदं व्याप्तं त्रैलोक्यं सचराचरम्}
{नीलोत्पलदलश्यामो विद्युद्वर्णाम्बरावृतः} %3-49

\twolineshloka
{शुद्धजाम्बूनदप्रख्यां श्रियं वामाङ्कसंस्थिताम्}
{सदानपायिनीं देवीं पश्यन्नालिङ्ग्य तिष्ठति} %3-50

\twolineshloka
{द्रष्टुं न शक्यते कैश्चिद्देवदानवपन्नगैः}
{यस्य प्रसादं कुरुते स चैनं द्रष्टुमर्हति} %3-51

\twolineshloka
{न च यज्ञतपोभिर्वा न दानाध्ययनादिभिः}
{शक्यते भगवान् द्रष्टुमुपायैरितरैरपि} %3-52

\twolineshloka
{तद्भक्तैस्तद्गतप्राणैस्तच्चित्तैर्धूतकल्मषैः}
{शक्यते भगवान् विष्णुर्वेदान्तामलदृष्टिभिः} %3-53

\twolineshloka
{अथवा द्रष्टुमिच्छा ते शृणु त्वं परमेश्वरम्}
{त्रेतायुगे स देवेशो भविता नृपविग्रहः} %3-54

\twolineshloka
{हितार्थं देवमर्त्यानामिक्ष्वाकूणां कुले हरिः}
{रामो दाशरथिर्भूत्वा महासत्त्वपराक्रमः} %3-55

\twolineshloka
{पितुर्नियोगात्स भ्रात्रा भार्यया दण्डके वने}
{विचरिष्यति धर्मात्मा जगन्मात्रा स्वमायया} %3-56

\twolineshloka
{एवं ते सर्वमाख्यातं मया रावण विस्तरात्}
{भजस्व भक्तिभावेन सदा रामं श्रिया युतम्} %3-57

\textbf{अगस्त्य उवाच}

\twolineshloka
{एवं श्रुत्वाऽसुराध्यक्षो ध्यात्वा किञ्चिद्विचार्य च}
{त्वया सह विरोधेप्सुर्मुमुदे रावणो महान्} %3-58

\threelineshloka
{युद्धार्थी सर्वतो लोकान् पर्यटन् समवस्थितः}
{एतदर्थं महाराज रावणोऽतीव बुद्धिमान्}
{हृतवान् जानकीं देवीं त्वयाऽऽत्मवधकाङ्क्षया} %3-59

\fourlineindentedshloka
{इमां कथां यः शृणुयात्पठेद्वा}
{संश्रावयेद्वा श्रवणार्थिनां सदा}
{आयुष्यमारोग्यमनन्तसौख्यम्}
{प्राप्नोति लाभं धनमक्षयं च} %3-60

{॥इति श्रीमदध्यात्मरामायणे उमामहेश्वरसंवादे उत्तरकाण्डे
तृतीयः सर्गः॥३॥
}
%%%%%%%%%%%%%%%%%%%%



\sect{चतुर्थः सर्गः}

\textbf{श्रीमहादेव उवाच}

\twolineshloka
{एकदा ब्रह्मणो लोकादायान्तं नारदं मुनिम्}
{पर्यटन् रावणो लोकान् दृष्ट्वा नत्वाऽब्रवीद्वचः} %4-1

\twolineshloka
{भगवन् ब्रूहि मे योद्धुं कुत्र सन्ति महाबलाः}
{योद्धुमिच्छामि बलिभिस्त्वं ज्ञाताऽसि जगत्त्रयम्} %4-2

\twolineshloka
{मुनिर्ध्यात्वाऽऽह सुचिरं श्वेतद्वीपनिवासिनः}
{महाबला महाकायास्तत्र याहि महामते} %4-3

\twolineshloka
{विष्णुपूजारता ये वै विष्णुना निहताश्च ये}
{त एव तत्र सञ्जाता अजेयाश्च सुरासुरैः} %4-4

\twolineshloka
{श्रुत्वा तद्रावणो वेगान्मन्त्रिभिः पुष्पकेण तान्}
{योद्धुकामः समागत्य श्वेतद्वीपसमीपतः} %4-5

\twolineshloka
{तत्प्रभाहततेजस्कं पुष्पकं नाचलत्ततः}
{त्यक्त्वा विमानं प्रययौ मन्त्रिणश्च दशाननः} %4-6

\twolineshloka
{प्रविशन्नेव तद्द्वीपं धृतो हस्तेन योषिता}
{पृष्टश्च त्वं कुतः कोऽसि प्रेषितः केन वा वद} %4-7

\twolineshloka
{इत्युक्तो लीलया स्त्रीभिर्हसन्तीभिः पुनः पुनः}
{कृच्छ्राद्धस्ताद्विनिर्मुक्तस्तासां स्त्रीणां दशाननः} %4-8

\twolineshloka
{आश्चर्यमतुलं लब्ध्वा चिन्तयामास दुर्मतिः}
{विष्णुना निहतो यामि वैकुण्ठमिति निश्चितः} %4-9

\twolineshloka
{मयि विष्णुर्यथा कुप्येत्तथा कार्यं करोम्यहम्}
{इति निश्चित्य वैदेहीं जहार विपिनेऽसुरः} %4-10

\twolineshloka
{जानन्नेव परात्मानं स जहारावनीसुताम्}
{मातृवत्पालयामास त्वत्तः काङ्क्षन् वधं स्वकम्} %4-11

\fourlineindentedshloka
{राम त्वं परमेश्वरोऽसि सकलं जानासि विज्ञानदृग्}
{भूतं भव्यमिदं त्रिकालकलनासाक्षी विकल्पोज्झितः}
{भक्तानामनुवर्तनाय सकलां कुर्वन् क्रियासंहतिम्}
{त्वं शृण्वन्मनुजाकृतिर्मुनिवचो भासीश लोकार्चितः} %4-12

\twolineshloka
{स्तुत्वैवं राघवं तेन पूजितः कुम्भसम्भवः}
{स्वाश्रमं मुनिभिः सार्धं प्रययौ हृष्टमानसः} %4-13

\twolineshloka
{रामस्तु सीतया सार्धं भ्रातृभिः सह मन्त्रिभिः}
{संसारीव रमानाथो रममाणोऽवसद्गृहे} %4-14

\twolineshloka
{अनासक्तोऽपि विषयान् बुभुजे प्रियया सह}
{हनुमत्प्रमुखैः सद्भिर्वानरैः परिवेष्टितः} %4-15

\twolineshloka
{पुष्पकं चागमद्राममेकदा पूर्ववत्प्रभुम्}
{प्राह देव कुबेरेण प्रेषितं त्वामहं ततः} %4-16

\twolineshloka
{जितं त्वं रावणेनादौ पश्चाद्रामेण निर्जितम्}
{अतस्त्वं राघवं नित्यं वह यावद्वसेद्भुवि} %4-17

\twolineshloka
{यदा गच्छेद्रघुश्रेष्ठो वैकुण्ठं याहि मां तदा}
{तच्छ्रुत्वा राघवः प्राह पुष्पकं सूर्यसन्निभम्} %4-18

\twolineshloka
{यदा स्मरामि भद्रं ते तदाऽऽगच्छ ममान्तिकम्}
{तिष्ठान्तर्धाय सर्वत्र गच्छेदानीं ममाऽऽज्ञया} %4-19

\twolineshloka
{इत्युक्त्वा रामचन्द्रोऽपि पौरकार्याणि सर्वशः}
{भ्रातृभिर्मन्त्रिभिः सार्धं यथान्यायं चकार सः} %4-20

\twolineshloka
{राघवे शासति भुवं लोकनाथे रमापतौ}
{वसुधा सस्यसम्पन्ना फलवन्तश्च भूरुहाः} %4-21

\twolineshloka
{जना धर्मपराः सर्वे पतिभक्तिपराः स्त्रियः}
{नापश्यत्पुत्रमरणं कश्चिद्राजनि राघवे} %4-22

\twolineshloka
{समारुह्य विमानाग्र्यं राघवः सीतया सह}
{वानरैर्भ्रातृभिः सार्धं सञ्चचारावनिं प्रभुः} %4-23

\twolineshloka
{अमानुषाणि कार्याणि चकार बहुशो भुवि}
{ब्राह्मणस्य सुतं दृष्ट्वा बालं मृतमकालतः} %4-24

\twolineshloka
{शोचन्तं ब्राह्मणं चापि ज्ञात्वा रामो महामतिः}
{तपस्यन्तं वने शूद्रं हत्वा ब्राह्मणबालकम्} %4-25

\twolineshloka
{जीवयामास शूद्रस्य ददौ स्वर्गमनुत्तमम्}
{लोकानामुपदेशार्थं परमात्मा रघूत्तमः} %4-26

\twolineshloka
{कोटिशः स्थापयामास शिवलिङ्गानि सर्वशः}
{सीतां च रमयामास सर्वभोगैरमानुषैः} %4-27

\twolineshloka
{शशास रामो धर्मेण राज्यं परमधर्मवित्}
{कथां संस्थापयामास सर्वलोकमलापहाम्} %4-28

\twolineshloka
{दशवर्षसहस्राणि मायामानुषविग्रहः}
{चकार राज्यं विधिवल्लोकवन्द्यपदाम्बुजः} %4-29

\twolineshloka
{एकपत्नीव्रतो रामो राजर्षिः सर्वदा शुचिः}
{गृहमेधीयमखिलमाचरन् शिक्षयन् जनान्} %4-30

\twolineshloka
{सीता प्रेम्णाऽनुवृत्त्या च प्रश्रयेण दमेन च}
{भर्तुर्मनोहरा साध्वी भावज्ञा सा ह्रिया भिया} %4-31

\twolineshloka
{एकदा क्रीडविपिने सर्वभोगसमन्विते}
{एकान्ते दिव्यभवने सुखासीनं रघूत्तमम्} %4-32

\twolineshloka
{नीलमाणिक्यसङ्काशं दिव्याभरणभूषितम्}
{प्रसन्नवदनं शान्तं विद्युत्पुञ्जनिभाम्बरम्} %4-33

\twolineshloka
{सीता कमलपत्राक्षी सर्वाभरणभूषिता}
{राममाह कराभ्यां सा लालयन्ती पदाम्बुजे} %4-34

\twolineshloka
{देवदेव जगन्नाथ परमात्मन् सनातन}
{चिदानन्दादिमध्यान्तरहिताशेषकारण} %4-35

\twolineshloka
{देव देवाः समासाद्य मामेकान्तेऽब्रुवन्वचः}
{बहुशोऽर्थयमानास्ते वैकुण्ठागमनं प्रति} %4-36

\twolineshloka
{त्वया समेतश्चिच्छक्त्या रामस्तिष्ठति भूतले}
{विसृज्यास्मान् स्वकं धाम वैकुण्ठं च सनातनम्} %4-37

\twolineshloka
{आस्ते त्वया जगद्धात्रि रामः कमललोचनः}
{अग्रतो याहि वैकुण्ठं त्वं तथा चेद्रघूत्तमः} %4-38

\twolineshloka
{आगमिष्यति वैकुण्ठं सनाथान्नः करिष्यति}
{इति विज्ञापिताऽहं तैर्मया विज्ञापितो भवान्} %4-39

\twolineshloka
{यद्युक्तं तत्कुरुष्वाद्य नाहमाज्ञापये प्रभो}
{सीतायास्तद्वचः श्रुत्वा रामो ध्यात्वाऽब्रवीत्क्षणम्} %4-40

\twolineshloka
{देवि जानामि सकलं तत्रोपायं वदामि ते}
{कल्पयित्वा मिषं देवि लोकवादं त्वदाशयम्} %4-41

\twolineshloka
{त्यजामि त्वां वने लोकवादाद्भीत इवापरः}
{भविष्यतः कुमारौ द्वौ वाल्मीकेराश्रमान्तिके} %4-42

\twolineshloka
{इदानीं दृश्यते गर्भः पुनरागत्य मेऽन्तिकम्}
{लोकानां प्रत्ययार्थं त्वं कृत्वा शपथमादरात्} %4-43

\twolineshloka
{भूमेर्विवरमात्रेण वैकुण्ठं यास्यसि द्रुतम्}
{पश्चादहं गमिष्यामि एष एव सुनिश्चयः} %4-44

\twolineshloka
{इत्युक्त्वा तां विसृज्याथ रामो ज्ञानैकलक्षणः}
{मन्त्रिभिर्मन्त्रतत्त्वज्ञैर्बलमुख्यैश्च संवृतः} %4-45

\twolineshloka
{तत्रोपविष्टं श्रीरामं सुहृदः पर्युपासत}
{हास्यप्रौढकथासुज्ञा हासयन्तः स्थिता हरिम्} %4-46

\twolineshloka
{कथाप्रसङ्गात्पप्रच्छ रामो विजयनामकम्}
{पौरा जानपदा मे किं वदन्तीह शुभाशुभम्} %4-47

\twolineshloka
{सीतां वा मातरं वा मे भ्रातॄन्वा कैकयीमथ}
{न भेतव्यं त्वया ब्रूहि शापितोऽसि ममोपरि} %4-48

\twolineshloka
{इत्युक्तः प्राह विजयो देव सर्वे वदन्ति ते}
{कृतं सुदुष्करं सर्वं रामेण विदितात्मना} %4-49

\twolineshloka
{किन्तु हत्वा दशग्रीवं सीतामाहृत्य राघवः}
{अमर्षं पृष्ठतः कृत्वा स्वं वेश्म प्रत्यपादयत्} %4-50

\twolineshloka
{कीदृशं हृदये तस्य सीतासम्भोगजं सुखम्}
{या हृता विजनेऽरण्ये रावणेन दुरात्मना} %4-51

\twolineshloka
{अस्माकमपि दुष्कर्म योषितां मर्षणं भवेत्}
{यादृग्भवति वै राजा तादृश्यो नियतं प्रजाः} %4-52

\twolineshloka
{श्रुत्वा तद्वचनं रामः स्वजनान् पर्यपृच्छत}
{तेऽपि नत्वाऽब्रुवन् राममेवमेतन्न संशयः} %4-53

\twolineshloka
{ततो विसृज्य सचिवान् विजयं सुहृदस्तथा}
{आहूय लक्ष्मणं रामो वचनं चेदमब्रवीत्} %4-54

\twolineshloka
{लोकापवादस्तु महान् सीतामाश्रित्य मेऽभवत्}
{सीतां प्रातः समानीय वाल्मीकेराश्रमान्तिके} %4-55

\twolineshloka
{त्यक्त्वा शीघ्रं रथेन त्वं पुनरायाहि लक्ष्मण}
{वक्ष्यसे यदि वा किञ्चित्तदा मां हतवानसि} %4-56

\twolineshloka
{इत्युक्तो लक्ष्मणो भीत्या प्रातरुत्थाय जानकीम्}
{सुमन्त्रेण रथे कृत्वा जगाम सहसा वनम्} %4-57

\twolineshloka
{वाल्मीकेराश्रमस्यान्ते त्यक्त्वा सीतामुवाच सः}
{लोकापवादभीत्या त्वां त्यक्तवान् राघवो वने} %4-58

\twolineshloka
{दोषो न कश्चिन्मे मातर्गच्छाऽऽश्रमपदं मुनेः}
{इत्युक्त्वा लक्ष्मणः शीघ्रं गतवान् रामसन्निधिम्} %4-59

\twolineshloka
{सीताऽपि दुःखसन्तप्ता विललापातिमुग्धवत्}
{शिष्यैः श्रुत्वा च वाल्मीकिः सीतां ज्ञात्वा स दिव्यदृक्} %4-60

\twolineshloka
{अर्घ्यादिभिः पूजयित्वा समाश्वास्य च जानकीम्}
{ज्ञात्वा भविष्यं सकलमार्पयन् मुनियोषिताम्} %4-61

\threelineshloka
{तास्तां सम्पूजयन्ति स्म सीतां भक्त्या दिने दिने}
{ज्ञात्वा परात्मनो लक्ष्मीं मुनिवाक्येन योषितः}
{सेवां चक्रुः सदा तस्या विनयादिभिरादरात्} %4-62

\twolineshloka
{रामोऽपि सीतारहितः परात्मा विज्ञानदृक्केवल आदिदेवः}
{सन्त्यज्य भोगानखिलान् विरक्तो मुनिव्रतोऽभून्मुनिसेविताङ्घ्रिः} %4-63

{॥इति श्रीमदध्यात्मरामायणे उमामहेश्वरसंवादे उत्तरकाण्डे
चतुर्थः सर्गः॥४॥
}
%%%%%%%%%%%%%%%%%%%%



\sect{पञ्चमः सर्गः}

\textbf{श्रीमहादेव उवाच}

\fourlineindentedshloka
{ततो जगन्मङ्गलमङ्गलात्मना}
{विधाय रामायणकीर्तिमुत्तमाम्}
{चचार पूर्वाचरितं रघूत्तमो}
{राजर्षिवर्यैरभिसेवितं यथा} %5-1

\fourlineindentedshloka
{सौमित्रिणा पृष्ट उदारबुद्धिना}
{रामः कथाः प्राह पुरातनीः शुभाः}
{राज्ञः प्रमत्तस्य नृगस्य शापतो}
{द्विजस्य तिर्यक्त्वमथाह राघवः} %5-2

\fourlineindentedshloka
{कदाचिदेकान्त उपस्थितं प्रभुम्}
{रामं रमालालितपादपङ्कजम्}
{सौमित्रिरासादितशुद्धभावनः}
{प्रणम्य भक्त्या विनयान्वितोऽब्रवीत्} %5-3

\fourlineindentedshloka
{त्वं शुद्धबोधोऽसि हि सर्वदेहिनाम्}
{आत्मास्यधीशोऽसि निराकृतिः स्वयम्}
{प्रतीयसे ज्ञानदृशां महामते}
{पादाब्जभृङ्गाहितसङ्गसङ्गिनाम्} %5-4

\fourlineindentedshloka
{अहं प्रपन्नोऽस्मि पदाम्बुजं प्रभो}
{भवापवर्गं तव योगिभावितम्}
{यथाञ्जसाऽज्ञानमपारवारिधिम्}
{सुखं तरिष्यामि तथाऽनुशाधि माम्} %5-5

\fourlineindentedshloka
{श्रुत्वाऽथ सौमित्रिवचोऽखिलं तदा}
{प्राह प्रपन्नार्तिहरः प्रसन्नधीः}
{विज्ञानमज्ञानतमःप्रशान्तये}
{श्रुतिप्रपन्नं क्षितिपालभूषणः} %5-6

\fourlineindentedshloka
{आदौ स्ववर्णाश्रमवर्णिताः क्रियाः}
{कृत्वा समासादितशुद्धमानसः}
{समाप्य तत्पूर्वमुपात्तसाधनः}
{समाश्रयेत्सद्गुरुमात्मलब्धये} %5-7

\fourlineindentedshloka
{क्रिया शरीरोद्भवहेतुरादृता}
{प्रियाप्रियौ तौ भवतः सुरागिणः}
{धर्मेतरौ तत्र पुनः शरीरकम्}
{पुनः क्रिया चक्रवदीर्यते भवः} %5-8

\fourlineindentedshloka
{अज्ञानमेवास्य हि मूलकारणम्}
{तद्ध्यानमेवात्र विधौ विधीयते}
{विद्यैव तन्नाशविधौ पटीयसी}
{न कर्म तज्जं सविरोधमीरितम्} %5-9

\fourlineindentedshloka
{नाज्ञानहानिर्न च रागसङ्क्षयो}
{भवेत्ततः कर्म सदोषमुद्भवेत्}
{ततः पुनः संसृतिरप्यवारिता}
{तस्माद्बुधो ज्ञानविचारवान् भवेत्} %5-10

\fourlineindentedshloka
{ननु क्रिया वेदमुखेन चोदिता}
{तथैव विद्या पुरुषार्थसाधनम्}
{कर्तव्यता प्राणभृतः प्रचोदिता}
{विद्यासहायत्वमुपैति सा पुनः} %5-11

\fourlineindentedshloka
{कर्माकृतौ दोषमपि श्रुतिर्जगौ}
{तस्मात्सदा कार्यमिदं मुमुक्षुणा}
{ननु स्वतन्त्रा ध्रुवकार्यकारिणी}
{विद्या न किञ्चिन्मनसाऽप्यपेक्षते} %5-12

\fourlineindentedshloka
{न सत्यकार्योऽपि हि यद्वदध्वरः}
{प्रकाङ्क्षतेऽन्यानपि कारकादिकान्}
{तथैव विद्या विधितः प्रकाशितैः}
{विशिष्यते कर्मभिरेव मुक्तये} %5-13

\fourlineindentedshloka
{केचिद्वदन्तीति वितर्कवादिन-}
{स्तदप्यसद्दृष्टविरोधकारणात्}
{देहाभिमानादभिवर्धते क्रिया}
{विद्या गताहङ्कृतितः प्रसिद्ध्यति} %5-14

\fourlineindentedshloka
{विशुद्धविज्ञानविरोचनाञ्चिता}
{विद्यात्मवृत्तिश्चरमेति भण्यते}
{उदेति कर्माखिलकारकादिभिः}
{निहन्ति विद्याखिलकारकादिकम्} %5-15

\fourlineindentedshloka
{तस्मात्त्यजेत्कार्यमशेषतः सुधीः}
{विद्याविरोधान्न समुच्चयो भवेत्}
{आत्मानुसन्धानपरायणः सदा}
{निवृत्तसर्वेन्द्रियवृत्तिगोचरः} %5-16

\fourlineindentedshloka
{यावच्छरीरादिषु माययाऽऽत्मधी-}
{स्तावद्विधेयो विधिवादकर्मणाम्}
{नेतीति वाक्यैरखिलं निषिध्य तत्}
{ज्ञात्वा परात्मानमथ त्यजेत्क्रियाः} %5-17

\fourlineindentedshloka
{यदा परात्मात्मविभेदभेदकम्}
{विज्ञानमात्मन्यवभाति भास्वरम्}
{तदैव माया प्रविलीयतेऽञ्जसा}
{सकारका कारणमात्मसंसृतेः} %5-18

\fourlineindentedshloka
{श्रुतिप्रमाणाभिविनाशिता च सा}
{कथं भविष्यत्यपि कार्यकारिणी}
{विज्ञानमात्रादमलाद्वितीयत-}
{स्तस्मादविद्या न पुनर्भविष्यति} %5-19

\fourlineindentedshloka
{यदि स्म नष्टा न पुनः प्रसूयते}
{कर्ताहमस्येति मतिः कथं भवेत्}
{तस्मात्स्वतन्त्रा न किमप्यपेक्षते}
{विद्या विमोक्षाय विभाति केवला} %5-20

\fourlineindentedshloka
{सा तैत्तिरीयश्रुतिराह सादरम्}
{न्यासं प्रशस्ताखिलकर्मणां स्फुटम्}
{एतावदित्याह च वाजिनां श्रुतिः}
{ज्ञानं विमोक्षाय न कर्म साधनम्} %5-21

\fourlineindentedshloka
{विद्यासमत्वेन तु दर्शितस्त्वया}
{क्रतुर्न दृष्टान्त उदाहृतः समः}
{फलैः पृथक्त्वाद्बहुकारकैः क्रतुः}
{संसाध्यते ज्ञानमतो विपर्ययम्} %5-22

\fourlineindentedshloka
{सप्रत्यवायो ह्यहमित्यनात्मधी-}
{रज्ञप्रसिद्धा न तु तत्त्वदर्शिनः}
{तस्माद्बुधैस्त्याज्यमविक्रियात्मभिः}
{विधानतः कर्म विधिप्रकाशितम्} %5-23

\fourlineindentedshloka
{श्रद्धान्वितस्तत्त्वमसीति वाक्यतो}
{गुरोः प्रसादादपि शुद्धमानसः}
{विज्ञाय चैकात्म्यमथाऽऽत्मजीवयोः}
{सुखी भवेन्मेरुरिवाप्रकम्पनः} %5-24

\fourlineindentedshloka
{आदौ पदार्थावगतिर्हि कारणम्}
{वाक्यार्थविज्ञानविधौ विधानतः}
{तत्त्वम्पदार्थौ परमात्मजीवका-}
{वसीति चैकात्म्यमथानयोर्भवेत्} %5-25

\fourlineindentedshloka
{प्रत्यक्परोक्षादिविरोधमात्मनोः}
{विहाय सङ्गृह्य तयोश्चिदात्मताम्}
{संशोधितां लक्षणया च लक्षिताम्}
{ज्ञात्वा स्वमात्मानमथाद्वयो भवेत्} %5-26

\fourlineindentedshloka
{एकात्मकत्वाज्जहती न सम्भवेत्}
{तथाऽजहल्लक्षणता विरोधतः}
{सोऽयम्पदार्थाविव भागलक्षणा}
{युज्येत तत्त्वम्पदयोरदोषतः} %5-27

\fourlineindentedshloka
{रसादिपञ्चीकृतभूतसम्भवम्}
{भोगालयं दुःखसुखादिकर्मणाम्}
{शरीरमाद्यन्तवदादिकर्मजम्}
{मायामयं स्थूलमुपाधिमात्मनः} %5-28

\fourlineindentedshloka
{सूक्ष्मं मनोबुद्धिदशेन्द्रियैर्युतम्}
{प्राणैरपञ्चीकृतभूतसम्भवम्}
{भोक्तुः सुखादेरनुसाधनं भवेत्}
{शरीरमन्यद्विदुरात्मनो बुधाः} %5-29

\fourlineindentedshloka
{अनाद्यनिर्वाच्यमपीह कारणम्}
{मायाप्रधानं तु परं शरीरकम्}
{उपाधिभेदात्तु यतः पृथक् स्थितम्}
{स्वात्मानमात्मन्यवधारयेत्क्रमात्} %5-30

\fourlineindentedshloka
{कोशेष्वयं तेषु तु तत्तदाकृतिः}
{विभाति सङ्गात् स्फटिकोपलो यथा}
{असङ्गरूपोऽयमजो यतोऽद्वयो}
{विज्ञायतेऽस्मिन् परितो विचारिते} %5-31

\fourlineindentedshloka
{बुद्धेस्त्रिधा वृत्तिरपीह दृश्यते}
{स्वप्नादिभेदेन गुणत्रयात्मनः}
{अन्योन्यतोऽस्मिन् व्यभिचारतो मृषा}
{नित्ये परे ब्रह्मणि केवले शिवे} %5-32

\fourlineindentedshloka
{देहेन्द्रियप्राणमनश्चिदात्मनाम्}
{सङ्घादजस्रं परिवर्तते धियः}
{वृत्तिस्तमोमूलतयाज्ञलक्षणा}
{यावद्भवेत्तावदसौ भवोद्भवः} %5-33

\fourlineindentedshloka
{नेतिप्रमाणेन निराकृताखिलो}
{हृदा समास्वादितचिद्घनामृतः}
{त्यजेदशेषं जगदात्तसद्रसम्}
{पीत्वा यथाम्भः प्रजहाति तत्फलम्} %5-34

\fourlineindentedshloka
{कदाचिदात्मा न मृतो न जायते}
{न क्षीयते नापि विवर्धतेऽनवः}
{निरस्तसर्वातिशयः सुखात्मकः}
{स्वयम्प्रभः सर्वगतोऽयमद्वयः} %5-35

\fourlineindentedshloka
{एवंविधे ज्ञानमये सुखात्मके}
{कथं भवो दुःखमयः प्रतीयते}
{अज्ञानतोऽध्यासवशात्प्रकाशते}
{ज्ञाने विलीयेत विरोधतः क्षणात्} %5-36

\fourlineindentedshloka
{यदन्यदन्यत्र विभाव्यते भ्रमा-}
{दध्यासमित्याहुरमुं विपश्चितः}
{असर्पभूतेऽहिविभावनं यथा}
{रज्ज्वादिके तद्वदपीश्वरे जगत्} %5-37

\fourlineindentedshloka
{विकल्पमायारहिते चिदात्मके-}
{ऽहङ्कार एष प्रथमः प्रकल्पितः}
{अध्यास एवात्मनि सर्वकारणे}
{निरामये ब्रह्मणि केवले परे} %5-38

\fourlineindentedshloka
{इच्छादिरागादिसुखादिधर्मिकाः}
{सदा धियः संसृतिहेतवः परे}
{यस्मात्प्रसुप्तौ तदभावतः परः}
{सुखस्वरूपेण विभाव्यते हि नः} %5-39

\fourlineindentedshloka
{अनाद्यविद्योद्भवबुद्धिबिम्बितो}
{जीवः प्रकाशोऽयमितीर्यते चितः}
{आत्मा धियः साक्षितया पृथक् स्थितो}
{बुद्ध्यापरिच्छिन्नपरः स एव हि} %5-40

\fourlineindentedshloka
{चिद्बिम्बसाक्ष्यात्मधियां प्रसङ्गत-}
{स्त्वेकत्र वासादनलाक्तलोहवत्}
{अन्योन्यमध्यासवशात्प्रतीयते}
{जडाजडत्वं च चिदात्मचेतसोः} %5-41

\fourlineindentedshloka
{गुरोः सकाशादपि वेदवाक्यतः}
{सञ्जातविद्यानुभवो निरीक्ष्य तम्}
{स्वात्मानमात्मस्थमुपाधिवर्जितम्}
{त्यजेदशेषं जडमात्मगोचरम्} %5-42

\fourlineindentedshloka
{प्रकाशरूपोऽहमजोऽहमद्वयो-}
{ऽसकृद्विभातोऽहमतीव निर्मलः}
{विशुद्ध विज्ञानघनो निरामयः}
{सम्पूर्ण आनन्दमयोऽहमक्रियः} %5-43

\fourlineindentedshloka
{सदैव मुक्तोऽहमचिन्त्यशक्तिमान्}
{अतीन्द्रियज्ञानमविक्रियात्मकः}
{अनन्तपारोऽहमहर्निशं बुधैः}
{विभावितोऽहं हृदि वेदवादिभिः} %5-44

\fourlineindentedshloka
{एवं सदात्मानमखण्डितात्मना}
{विचारमाणस्य विशुद्धभावना}
{हन्यादविद्यामचिरेण कारकै}
{रसायनं यद्वदुपासितं रुजः} %5-45

\fourlineindentedshloka
{विविक्त आसीन उपारतेन्द्रियो}
{विनिर्जितात्मा विमलान्तराशयः}
{विभावयेदेकमनन्यसाधनो}
{विज्ञानदृक्केवल आत्मसंस्थितः} %5-46

\fourlineindentedshloka
{विश्वं यदेतत्परमात्मदर्शनम्}
{विलापयेदात्मनि सर्वकारणे}
{पूर्णश्चिदानन्दमयोऽवतिष्ठते}
{न वेद बाह्यं न च किञ्चिदान्तरम्} %5-47

\fourlineindentedshloka
{पूर्वं समाधेरखिलं विचिन्तये-}
{दोङ्कारमात्रं सचराचरं जगत्}
{तदेव वाच्यं प्रणवो हि वाचको}
{विभाव्यतेऽज्ञानवशान्न बोधतः} %5-48

\fourlineindentedshloka
{अकारसंज्ञः पुरुषो हि विश्वको}
{ह्युकारकस्तैजस ईर्यते क्रमात्}
{प्राज्ञो मकारः परिपठ्यतेऽखिलैः}
{समाधिपूर्वं न तु तत्त्वतो भवेत्} %5-49

\fourlineindentedshloka
{विश्वं त्वकारं पुरुषं विलापये-}
{दुकारमध्ये बहुधा व्यवस्थितम्}
{ततो मकारे प्रविलाप्य तैजसम्}
{द्वितीयवर्णं प्रणवस्य चान्तिमे} %5-50

\fourlineindentedshloka
{मकारमप्यात्मनि चिद्घने परे}
{विलापयेद्प्राज्ञमपीह कारणम्}
{सोऽहं परं ब्रह्म सदा विमुक्तिमद्-}
{विज्ञानदृङ्मुक्त उपाधितोऽमलः} %5-51

\fourlineindentedshloka
{एवं सदा जातपरात्मभावनः}
{स्वानन्दतुष्टः परिविस्मृताखिलः}
{आस्ते स नित्यात्मसुखप्रकाशकः}
{साक्षाद्विमुक्तोऽचलवारिसिन्धुवत्} %5-52

\fourlineindentedshloka
{एवं सदाभ्यस्तसमाधियोगिनो}
{निवृत्तसर्वेन्द्रियगोचरस्य हि}
{विनिर्जिताशेषरिपोरहं सदा}
{दृश्यो भवेयं जितषड्गुणात्मनः} %5-53

\fourlineindentedshloka
{ध्यात्वैवमात्मानमहर्निशं मुनि-}
{स्तिष्ठेत्सदा मुक्तसमस्तबन्धनः}
{प्रारब्धमश्नन्नभिमानवर्जितो}
{मय्येव साक्षात्प्रविलीयते ततः} %5-54

\fourlineindentedshloka
{आदौ च मध्ये च तथैव चान्ततो}
{भवं विदित्वा भयशोककारणम्}
{हित्वा समस्तं विधिवादचोदितम्}
{भजेत्स्वमात्मानमथाखिलात्मनाम्} %5-55

\fourlineindentedshloka
{आत्मन्यभेदेन विभावयन्निदम्}
{भवत्यभेदेन मयाऽऽत्मना तदा}
{यथा जलं वारिनिधौ यथा पयः}
{क्षीरे वियद्व्योम्न्यनिले यथाऽनिलः} %5-56

\fourlineindentedshloka
{इत्थं यदीक्षेत हि लोकसंस्थितो}
{जगन्मृषैवेति विभावयन्मुनिः}
{निराकृतत्वाच्छ्रुतियुक्तिमानतो}
{यथेन्दुभेदो दिशि दिग्भ्रमादयः} %5-57

\fourlineindentedshloka
{यावन्न पश्येदखिलं मदात्मकम्}
{तावन्मदाराधनतत्परो भवेत्}
{श्रद्धालुरत्यूर्जितभक्तिलक्षणो}
{यस्तस्य दृश्योऽहमहर्निशं हृदि} %5-58

\fourlineindentedshloka
{रहस्यमेतच्छ्रुतिसारसङ्ग्रहम्}
{मया विनिश्चित्य तवोदितं प्रिय}
{यस्त्वेतदालोचयतीह बुद्धिमान्}
{स मुच्यते पातकराशिभिः क्षणात्} %5-59

\fourlineindentedshloka
{भ्रातर्यदीदं परिदृश्यते जगन्-}
{मायैव सर्वं परिहृत्य चेतसा}
{मद्भावनाभावितशुद्धमानसः}
{सुखी भवानन्दमयो निरामयः} %5-60

\fourlineindentedshloka
{यः सेवते मामगुणं गुणात्परम्}
{हृदा कदा वा यदि वा गुणात्मकम्}
{सोऽहं स्वपादाञ्चितरेणुभिः स्पृशन्}
{पुनाति लोकत्रितयं यथा रविः} %5-61

\fourlineindentedshloka
{विज्ञानमेतदखिलं श्रुतिसारमेकम्}
{वेदान्तवेद्यचरणेन मयैव गीतम्}
{यः श्रद्धया परिपठेद्गुरुभक्तियुक्तो}
{मद्रूपमेति यदि मद्वचनेषु भक्तिः} %5-62

{॥इति श्रीमदध्यात्मरामायणे उमामहेश्वरसंवादे उत्तरकाण्डे
पञ्चमः सर्गः॥५॥}
%%%%%%%%%%%%%%%%%%%%



\sect{षष्ठः सर्गः}

\textbf{श्रीमहादेव उवाच}

\twolineshloka
{एकदा मुनयः सर्वे यमुनातीरवासिनः}
{आजग्मू राघवं द्रष्टुं भयाल्लवणरक्षसः} %6-1

\twolineshloka
{कृत्वाऽग्रे तु मुनिश्रेष्ठं भार्गवं च्यवनं द्विजाः}
{असङ्ख्याताः समायाता रामादभयकाङ्क्षिणः} %6-2

\twolineshloka
{तान् पूजयित्वा परया भक्त्या रघुकुलोत्तमः}
{उवाच मधुरं वाक्यं हर्षयन् मुनिमण्डलम्} %6-3

\twolineshloka
{करवाणि मुनिश्रेष्ठाः किमागमनकारणम्}
{धन्योऽस्मि यदि यूयं मां प्रीत्या द्रष्टुमिहागताः} %6-4

\twolineshloka
{दुष्करं चापि यत्कार्यं भवतां तत्करोम्यहम्}
{आज्ञापयन्तु मां भृत्यं ब्राह्मणा दैवतं हि मे} %6-5

\twolineshloka
{तच्छ्रुत्वा सहसा हृष्टश्च्यवनो वाक्यमब्रवीत्}
{मधुनामा महादैत्यः पुरा कृतयुगे प्रभो} %6-6

\twolineshloka
{आसीदतीव धर्मात्मा देवब्राह्मणपूजकः}
{तस्य तुष्टो महादेवो ददौ शूलमनुत्तमम्} %6-7

\twolineshloka
{प्राह चानेन यं हंसि स तु भस्मीभविष्यति}
{रावणस्यानुजा भार्या तस्य कुम्भीनसी श्रुता} %6-8

\twolineshloka
{तस्यां तु लवणो नाम राक्षसो भीमविक्रमः}
{आसीद्दुरात्मा दुर्धर्षो देवब्राह्मणहिंसकः} %6-9

\twolineshloka
{पीडितास्तेन राजेन्द्र वयं त्वां शरणं गताः}
{तच्छ्रुत्वा राघवोऽप्याह मा भीर्वो मुनिपुङ्गवाः} %6-10

\twolineshloka
{लवणं नाशयिष्यामि गच्छन्तु विगतज्वराः}
{इत्युक्त्वा प्राह रामोऽपि भ्रातॄन् को वा हनिष्यति} %6-11

\twolineshloka
{लवणं राक्षसं दद्यात् ब्राह्मणेभ्योऽभयं महत्}
{तच्छ्रुत्वा प्राञ्जलिः प्राह भरतो राघवाय वै} %6-12

\twolineshloka
{अहमेव हनिष्यामि देवाज्ञापय मां प्रभो}
{ततो रामं नमस्कृत्य शत्रुघ्नो वाक्यमब्रवीत्} %6-13

\twolineshloka
{लक्ष्मणेन महत्कार्यं कृतं राघव संयुगे}
{नन्दिग्रामे महाबुद्धिर्भरतो दुःखमन्वभूत्} %6-14

\twolineshloka
{अहमेव गमिष्यामि लवणस्य वधाय च}
{त्वत्प्रसादाद्रघुश्रेष्ठ हन्यां तं राक्षसं युधि} %6-15

\twolineshloka
{तच्छ्रुत्वा स्वाङ्कमारोप्य शत्रुघ्नं शत्रुसूदनः}
{प्राहाद्यैवाभिषेक्ष्यामि मथुराराज्यकारणात्} %6-16

\twolineshloka
{आनाय्य च सुसम्भारान् लक्षमणेनाभिषेचने}
{अनिच्छन्तमपि स्नेहादभिषेकमकारयत्} %6-17

\twolineshloka
{दत्त्वा तस्मै शरं दिव्यं रामः शत्रुघ्नमब्रवीत्}
{अनेन जहि बाणेन लवणं लोककण्टकम्} %6-18

\twolineshloka
{स तु सम्पूज्य तच्छूलं गेहे गच्छति काननम्}
{भक्षणार्थं तु जन्तूनां नानाप्राणिवधाय च} %6-19

\twolineshloka
{स तु नाऽऽयाति सदनं यावद्वनचरो भवेत्}
{तावदेव पुरद्वारि तिष्ठ त्वं धृतकार्मुकः} %6-20

\twolineshloka
{योत्स्यते स त्वया क्रुद्धस्तदा वध्यो भविष्यति}
{तं हत्वा लवणं क्रूरं तद्वनं मधुसंज्ञितम्} %6-21

\twolineshloka
{निवेश्य नगरं तत्र तिष्ठ त्वं मेऽनुशासनात्}
{अश्वानां पञ्चसाहस्रं रथानां च तदर्धकम्} %6-22

\twolineshloka
{गजानां षट् शतानीह पत्तीनामयुतत्रयम्}
{आगमिष्यति पश्चात्त्वमग्रे साधय राक्षसम्} %6-23

\twolineshloka
{इत्युक्त्वा मूर्ध्न्यवघ्राय प्रेषयामास राघवः}
{शत्रुघ्नं मुनिभिः सार्धमाशीर्भिरभिनन्द्य च} %6-24

\twolineshloka
{शत्रुघ्नोऽपि तथा चक्रे यथा रामेण चोदितः}
{हत्वा मधुसुतं युद्धे मथुरामकरोत्पुरीम्} %6-25

\twolineshloka
{स्फीतां जनपदां चक्रे मथुरां दानमानतः}
{सीताऽपि सुषुवे पुत्रौ द्वौ वाल्मीकेरथाऽऽश्रमे} %6-26

\twolineshloka
{मुनिस्तयोर्नाम चक्रे कुशो ज्येष्ठोऽनुजो लवः}
{क्रमेण विद्यासम्पन्नौ सीतापुत्रौ बभूवतुः} %6-27

\twolineshloka
{उपनीतौ च मुनिना वेदाध्ययनतत्परौ}
{कृत्स्नं रामायणं प्राह काव्यं बालकयोर्मुनिः} %6-28

\twolineshloka
{शङ्करेण पुरा प्रोक्तं पार्वत्यै पुरहारिणा}
{वेदोपबृंहनार्थाय तावग्राहयत प्रभुः} %6-29

\twolineshloka
{कुमारौ स्वरसम्पन्नौ सुन्दरावश्विनाविव}
{तन्त्रीतालसमायुक्तौ गायन्तौ चेरतुर्वने} %6-30

\twolineshloka
{तत्र तत्र मुनीनां तौ समाजे सुररूपिणौ}
{गायन्तावभितो दृष्ट्वा विस्मिता मुनयोऽब्रुवन्} %6-31

\fourlineindentedshloka
{गन्धर्वेष्विव किन्नरेषु भुवि वा देवेषु देवालये}
{पातालेष्वथवा चतुर्मुखगृहे लोकेषु सर्वेषु च}
{अस्माभिश्चिरजीविभिश्चिरतरं दृष्ट्वा दिशः सर्वतो}
{नाज्ञायीदृशगीतवाद्यगरिमा नादर्शि नाश्रावि च} %6-32

\twolineshloka
{एवं स्तुवद्भिरखिलैर्मुनिभिः प्रतिवासरम्}
{आसाते सुखमेकान्ते वाल्मीकेराश्रमे चिरम्} %6-33

\twolineshloka
{अथ रामोऽश्वमेधादींश्चकार बहुदक्षिणान्}
{यज्ञान् स्वर्णमयीं सीतां विधाय विपुलद्युतिः} %6-34

\twolineshloka
{तस्मिन् विताने ऋषयः सर्वे राजर्षयस्तथा}
{ब्राह्मणाः क्षत्रिया वैश्याः समाजग्मुर्दिदृक्षवः} %6-35

\twolineshloka
{वाल्मीकिरपि सङ्गृह्य गायन्तौ तौ कुशीलवौ}
{जगाम ऋषिवाटस्य समीपं मुनिपुङ्गवः} %6-36

\twolineshloka
{तत्रैकान्ते स्थितं शान्तं समाधिविरमे मुनिम्}
{कुशः पप्रच्छ वाल्मीकिं ज्ञानशास्त्रं कथान्तरे} %6-37

\twolineshloka
{भगवन् श्रोतुमिच्छामि सङ्क्षेपाद्भवतोऽखिलम्}
{देहिनः संसृतिर्बन्धः कथमुत्पद्यते दृढः} %6-38

\twolineshloka
{कथं विमुच्यते देही दृढबन्धाद्भवाभिधात्}
{वक्तुमर्हसि सर्वज्ञ मह्यं शिष्याय ते मुने} %6-39

\textbf{वाल्मीकिरुवाच}

\twolineshloka
{शृणु वक्ष्यामि ते सर्वं सङ्क्षेपाद्बन्धमोक्षयोः}
{स्वरूपं साधनं चापि मत्तः श्रुत्वा यथोदितम्} %6-40

\twolineshloka
{तथैवाऽऽचर भद्रं ते जीवन्मुक्तो भविष्यसि}
{देह एव महागेहमदेहस्य चिदात्मनः} %6-41

\twolineshloka
{तस्याहङ्कार एवास्मिन्मन्त्री तेनैव कल्पितः}
{देहगेहाभिमानं स्वं समारोप्य चिदात्मनि} %6-42

\twolineshloka
{तेन तादात्म्यमापन्नः स्वचेष्टितमशेषतः}
{विदधाति चिदानन्दे तद्वासितवपुः स्वयम्} %6-43

\twolineshloka
{तेन सङ्कल्पितो देही सङ्कल्पनिगडावृतः}
{पुत्रदारगृहादीनि सङ्कल्पयति चानिशम्} %6-44

\twolineshloka
{सङ्कल्पयन् स्वयं देही परिशोचति सर्वदा}
{त्रयस्तस्याहमो देहा अधमोत्तममध्यमाः} %6-45

\twolineshloka
{तमः सत्त्वरजः संज्ञा जगतः कारणं स्थितेः}
{तमोरूपाद्धि सङ्कल्पान्नित्यं तामसचेष्टया} %6-46

\twolineshloka
{अत्यन्तं तामसो भूत्वा कृमिकीटत्वमाप्नुयात्}
{सत्त्वरूपो हि सङ्कल्पो धर्मज्ञानपरायणः} %6-47

\twolineshloka
{अदूरमोक्षसाम्राज्यः सुखरूपो हि तिष्ठति}
{रजोरूपो हि सङ्कल्पो लोके स व्यवहारवान्} %6-48

\twolineshloka
{परितिष्ठति संसारे पुत्रदारानुरञ्जितः}
{त्रिविधं तु परित्यज्य रूपमेतन्महामते} %6-49

\twolineshloka
{सङ्कल्पं परमाप्नोति पदमात्मपरिक्षये}
{दृष्टीः सर्वाः परित्यज्य नियम्य मनसा मनः} %6-50

\twolineshloka
{सबाह्याभ्यन्तरार्थस्य सङ्कल्पस्य क्षयं कुरु}
{यदि वर्षसहस्राणि तपश्चरसि दारुणम्} %6-51

\twolineshloka
{पातालस्थस्य भूस्थस्य स्वर्गस्थस्यापि तेऽनघ}
{नान्यः कश्चिदुपायोऽस्ति सङ्कल्पोपशमादृते} %6-52

\twolineshloka
{अनाबाधेऽविकारे स्वे सुखे परमपावने}
{सङ्कल्पोपशमे यत्नं पौरुषेण परं कुरु} %6-53

\twolineshloka
{सङ्कल्पतन्तौ निखिला भावाः प्रोताः किलानघ}
{छिन्ने तन्तौ न जानीमः क्व यान्ति विभवाः पराः} %6-54

\twolineshloka
{निःसङ्कल्पो यथाप्राप्तव्यवहारपरो भव}
{क्षये सङ्कल्पजालस्य जीवो ब्रह्मत्वमाप्नुयात्} %6-55

\twolineshloka
{अधिगतपरमार्थतामुपेत्य प्रसभमपास्य विकल्पजालमुच्चैः}
{अधिगमय पदं तदद्वितीयं विततसुखाय सुषुप्तचित्तवृत्तिः} %6-56

{॥इति श्रीमदध्यात्मरामायणे उमामहेश्वरसंवादे उत्तरकाण्डे षष्ठः
सर्गः॥६॥
}
%%%%%%%%%%%%%%%%%%%%



\sect{सप्तमः सर्गः}

\textbf{श्रीमहादेव उवाच}

\twolineshloka
{वाल्मीकिना बोधितोऽसौ कुशः सद्योगतभ्रमः}
{अन्तर्मुक्तो बहिः सर्वमनुकुर्वंश्चकार सः} %7-1

\twolineshloka
{वाल्मीकिरपि तौ प्राह सीतापुत्रौ महाधियौ}
{तत्र तत्र च गायन्तौ पुरे वीथिषु सर्वतः} %7-2

\twolineshloka
{रामस्याग्रे प्रगायेतं शुश्रूषुर्यदि राघवः}
{न ग्राह्यं वै युवाभ्यां तद्यदि किञ्चित्प्रदास्यति} %7-3

\twolineshloka
{इति तौ चोदितौ तत्र गायमानौ विचेरतुः}
{यथोक्तमृषिणा पूर्वं तत्र तत्राभ्यगायताम्} %7-4

\twolineshloka
{तां स शुश्राव काकुत्स्थः पूर्वचर्यां ततस्ततः}
{अपूर्वपाठजातिं च गेयेन समभिप्लुताम्} %7-5

\twolineshloka
{बालयो राघवः श्रुत्वा कौतूहलमुपेयिवान्}
{अथ कर्मान्तरे राजा समाहूय महामुनीन्} %7-6

\twolineshloka
{राज्ञश्चैव नरव्याघ्रः पण्डितांश्चैव नैगमान्}
{पौराणिकान् शब्दविदो ये च वृद्धा द्विजातयः} %7-7

\twolineshloka
{एतान् सर्वान् समाहूय गायकौ समवेशयत्}
{ते सर्वे हृष्टमनसो राजानो ब्राह्मणादयः} %7-8

\twolineshloka
{रामं तौ दारकौ दृष्ट्वा विस्मिताः ह्यनिमेषणाः}
{अवोचन् सर्व एवैते परस्परमथागताः} %7-9

\twolineshloka
{इमौ रामस्य सदृशौ बिम्बाद्बिम्बमिवोदितौ}
{जटिलौ यदि न स्यातां न च वल्कलधारिणौ} %7-10

\twolineshloka
{विशेषं नाधिगच्छामो राघवस्यानयोस्तदा}
{एवं संवदतां तेषां विस्मितानां परस्परम्} %7-11

\twolineshloka
{उपचक्रमतुर्गातुं तावुभौ मुनिदारकौ}
{ततः प्रवृत्तं मधुरं गान्धर्वमतिमानुषम्} %7-12

\twolineshloka
{श्रुत्वा तन्मधुरं गीतमपराह्णे रघूत्तमः}
{उवाच भरतं चाभ्यां दीयतामयुतं वसु} %7-13

\twolineshloka
{दीयमानं सुवर्णं तु न तज्जगृहतुस्तदा}
{किमनेन सुवर्णेन राजन्नौ वन्यभोजनौ} %7-14

\twolineshloka
{इति सन्त्यज्य सन्दत्तं जग्मतुर्मुनिसन्निधिम्}
{एवं श्रुत्वा तु चरितं रामः स्वस्यैव विस्मितः} %7-15

\twolineshloka
{ज्ञात्वा सीताकुमारौ तौ शत्रुघ्नं चेदमब्रवीत्}
{हनूमन्तं सुषेणं च विभीषणमथाङ्गदम्} %7-16

\twolineshloka
{भगवन्तं महात्मानं वाल्मीकिं मुनिसत्तमम्}
{आनयध्वं मुनिवरं ससीतं देवसम्मितम्} %7-17

\twolineshloka
{अस्यास्तु पर्षदो मध्ये प्रत्ययं जनकात्मजा}
{करोतु शपथं सर्वे जानन्तु गतकल्मषाम्} %7-18

\twolineshloka
{सीतां तद्वचनं श्रुत्वा गताः सर्वेऽतिविस्मिताः}
{ऊचुर्यथोक्तं रामेण वाल्मीकिं रामपार्षदाः} %7-19

\twolineshloka
{रामस्य हृद्गतं सर्वं ज्ञात्वा वाल्मीकिरब्रवीत्}
{श्वः करिष्यति वै सीता शपथं जनसंसदि} %7-20

\twolineshloka
{योषितां परमं दैवं पतिरेव न संशयः}
{तच्छ्रुत्वा सहसा गत्वा सर्वे प्रोचुर्मुनेर्वचः} %7-21

\twolineshloka
{राघवस्यापि रामोऽपि श्रुत्वा मुनिवचस्तथा}
{राजानो मुनयः सर्वे शृणुध्वमिति चाब्रवीत्} %7-22

\twolineshloka
{सीतायाः शपथं लोका विजानन्तु शुभाशुभम्}
{इत्युक्ता राघवेणाथ लोकाः सर्वे दिदृक्षवः} %7-23

\twolineshloka
{ब्राह्मणाः क्षत्रिया वैश्याः शूद्राश्चैव महर्षयः}
{वानराश्च समाजग्मुः कौतूहलसमन्विताः} %7-24

\twolineshloka
{ततो मुनिवरस्तूर्णं ससीतः समुपागमत्}
{अग्रतस्तमृषिं कृत्वाऽऽयान्ती किञ्चिदवाङ्मुखी} %7-25

\twolineshloka
{कृताञ्जलिर्बाष्पकण्ठा सीता यज्ञं विवेश तम्}
{दृष्ट्वा लक्ष्मीमिवायान्तीं ब्रह्माणमनुयायिनीम्} %7-26

\twolineshloka
{वाल्मीकेः पृष्ठतः सीतां साधुवादो महानभूत्}
{तदा मध्ये जनौघस्य प्रविश्य मुनिपुङ्गवः} %7-27

\twolineshloka
{सीतासहायो वाल्मीकिरिति प्राह च राघवम्}
{इयं दाशरथे सीता सुव्रता धर्मचारिणी} %7-28

\twolineshloka
{अपापा ते पुरा त्यक्ता ममाश्रमसमीपतः}
{लोकापवादभीतेन त्वया राम महावने} %7-29

\twolineshloka
{प्रत्ययं दास्यते सीता तदनुज्ञातुमर्हसि}
{इमौ तु सीतातनयाविमौ यमलजातकौ} %7-30

\twolineshloka
{सुतौ तु तव दुर्धर्षौ तथ्यमेतद्ब्रवीमि ते}
{प्रचेतसोऽहं दशमः पुत्रो रघुकुलोद्वह} %7-31

\twolineshloka
{अनृतं न स्मराम्युक्तं तथेमौ तव पुत्रकौ}
{बहून् वर्षगणान् सम्यक् तपश्चर्या मया कृता} %7-32

\twolineshloka
{नोपाश्नीयां फलं तस्या दुष्टेयं यदि मैथिली}
{वाल्मीकिनैवमुक्तस्तु राघवः प्रत्यभाषत} %7-33

\twolineshloka
{एवमेतन्महाप्राज्ञ यथा वदसि सुव्रत}
{प्रत्ययो जनितो मह्यं तव वाक्यैरकिल्बिषैः} %7-34

\twolineshloka
{लङ्कायामपि दत्तो मे वैदेह्या प्रत्ययो महान्}
{देवानां पुरतस्तेन मन्दिरे सम्प्रवेशिता} %7-35

\twolineshloka
{सेयं लोकभयाद्ब्रह्मन्नपापाऽपि सती पुरा}
{सीता मया परित्यक्ता भवांस्तत्क्षन्तुमर्हति} %7-36

\twolineshloka
{ममैव जातौ जानामि पुत्रावेतौ कुशीलवौ}
{शुद्धायां जगतीमध्ये सीतायां प्रीतिरस्तु मे} %7-37

\twolineshloka
{देवाः सर्वे परिज्ञाय रामाभिप्रायमुत्सुकाः}
{ब्रह्माणमग्रतः कृत्वा समाजग्मुः सहस्रशः} %7-38

\twolineshloka
{प्रजाः समागमन् हृष्टाः सीता कौशेयवासिनी}
{उदङ्मुखी ह्यधोदृष्टिः प्राञ्जलिर्वाक्यमब्रवीत्} %7-39

\twolineshloka
{रामादन्यं यथाऽहं वै मनसाऽपि न चिन्तये}
{तथा मे धरणी देवी विवरं दातुमर्हति} %7-40

\twolineshloka
{तथा शपन्त्याः सीतायाः प्रादुरासीन्महाद्भुतम्}
{भूतलाद्दिव्यमत्यर्थं सिंहासनमनुत्तमम्} %7-41

\twolineshloka
{नागेन्द्रैर्ध्रियमाणं च दिव्यदेहै रविप्रभम्}
{भूदेवी जानकीं दोर्भ्यां गृहीत्वा स्नेहसंयुता} %7-42

\twolineshloka
{स्वागतं तामुवाचैनामासने सन्न्यवेशयत्}
{सिंहासनस्थां वैदेहीं प्रविशन्तीं रसातलम्} %7-43

\twolineshloka
{निरन्तरा पुष्पवृष्टिर्दिव्या सीतामवाकिरत्}
{साधुवादश्च सुमहान् देवानां परमाद्भुतः} %7-44

\twolineshloka
{ऊचुश्च बहुधा वाचो ह्यन्तरिक्षगताः सुराः}
{अन्तरिक्षे च भूमौ च सर्वे स्थावरजङ्गमाः} %7-45

\twolineshloka
{वानराश्च महाकायाः सीताशपथकारणात्}
{केचिच्चिन्तापरास्तस्य केचिद्ध्यानपरायणाः} %7-46

\twolineshloka
{केचिद्रामं निरीक्षन्तः केचित्सीतामचेतसः}
{मुहूर्तमात्रं तत्सर्वं तूष्णीम्भूतमचेतनम्} %7-47

\twolineshloka
{सीताप्रवेशनं दृष्ट्वा सर्वं सम्मोहितं जगत्}
{रामस्तु सर्वं ज्ञात्वैव भविष्यत्कार्यगौरवम्} %7-48

\twolineshloka
{अजानन्निव दुःखेन शुशोच जनकात्मजाम्}
{ब्रह्मणा ऋषिभिः सार्धं बोधितो रघुनन्दनः} %7-49

\twolineshloka
{प्रतिबुद्ध इव स्वप्नाच्चकारानन्तराः क्रियाः}
{विससर्ज ऋषीन् सर्वानृत्विजो ये समागताः} %7-50

\twolineshloka
{तान् सर्वान् धनरत्नाद्यैस्तोषयामास भूरिशः}
{उपादाय कुमारौ तावयोध्यामगमत्प्रभुः} %7-51

\twolineshloka
{तदादि निःस्पृहो रामः सर्वभोगेषु सर्वदा}
{आत्मचिन्तापरो नित्यमेकान्ते समुपस्थितः} %7-52

\twolineshloka
{एकान्ते ध्याननिरते एकदा राघवे सति}
{ज्ञात्वा नारायणं साक्षात्कौसल्या प्रियवादिनी} %7-53

\twolineshloka
{भक्त्याऽऽगत्य प्रसन्नं तं प्रणता प्राह हृष्टधीः}
{राम त्वं जगतामादिरादिमध्यान्तवर्जितः} %7-54

\twolineshloka
{परमात्मा परानन्दः पूर्णः पुरुष ईश्वरः}
{जातोऽसि मे गर्भगृहे मम पुण्यातिरेकतः} %7-55

\twolineshloka
{अवसाने ममाप्यद्य समयोऽभूद्रघूत्तम}
{नाद्याप्यबोधजः कृत्स्नो भवबन्धो निवर्तते} %7-56

\twolineshloka
{इदानीमपि मे ज्ञानं भवबन्धनिवर्तकम्}
{यथा सङ्क्षेपतो भूयात्तथा बोधय मां विभो} %7-57

\twolineshloka
{निर्वेदवादिनीमेवं मातरं मातृवत्सलः}
{दयालुः प्राह धर्मात्मा जराजर्जरितां शुभाम्} %7-58

\twolineshloka
{मार्गास्त्रयो मया प्रोक्ताः पुरा मोक्षाप्तिसाधकाः}
{कर्मयोगो ज्ञानयोगो भक्तियोगश्च शाश्वतः} %7-59

\twolineshloka
{भक्तिर्विभिद्यते मातस्त्रिविधा गुणभेदतः}
{स्वभावो यस्य यस्तेन तस्य भक्तिर्विभिद्यते} %7-60

\twolineshloka
{यस्तु हिंसां समुद्दिश्य दम्भं मात्सर्यमेव वा}
{भेददृष्टिश्च संरम्भी भक्तो मे तामसः स्मृतः} %7-61

\twolineshloka
{फलाभिसन्धिर्भोगार्थी धनकामो यशस्तथा}
{अर्चादौ भेदबुद्ध्या मां पूजयेत्स तु राजसः} %7-62

\twolineshloka
{परस्मिन्नर्पितं यस्तु कर्मनिर्हरणाय वा}
{कर्तव्यमिति वा कुर्याद्भेदबुद्ध्या स सात्त्विकः} %7-63

\twolineshloka
{मद्गुणाश्रयणादेव मय्यनन्तगुणालये}
{अविच्छिन्ना मनोवृत्तिर्यथा गङ्गाम्बुनोऽम्बुधौ} %7-64

\twolineshloka
{तदेव भक्तियोगस्य लक्षणं निर्गुणस्य हि}
{अहैतुक्यव्यवहिता या भक्तिर्मयि जायते} %7-65

\twolineshloka
{सा मे सालोक्यसामीप्यसार्ष्टिसायुज्यमेव वा}
{ददात्यपि न गृह्णन्ति भक्ता मत्सेवनं विना} %7-66

\twolineshloka
{स एवात्यन्तिको योगो भक्तिमार्गस्य भामिनि}
{मद्भावं प्राप्नुयात्तेन अतिक्रम्य गुणत्रयम्} %7-67

\twolineshloka
{महता कामहीनेन स्वधर्माचरणेन च}
{कर्मयोगेन शस्तेन वर्जितेन विहिंसनात्} %7-68

\twolineshloka
{मद्दर्शनस्तुतिमहापूजाभिः स्मृतिवन्दनैः}
{भूतेषु मद्भावनया सङ्गेनासत्यवर्जनैः} %7-69

\twolineshloka
{बहुमानेन महतां दुःखिनामनुकम्पया}
{स्वसमानेषु मैत्र्या च यमादीनां निषेवया} %7-70

\twolineshloka
{वेदान्तवाक्यश्रवणान्मम नामानुकीर्तनात्}
{सत्सङ्गेनार्जवेनैव ह्यहमः परिवर्जनात्} %7-71

\twolineshloka
{काङ्क्षया मम धर्मस्य परिशुद्धान्तरो जनः}
{मद्गुणश्रवणादेव याति मामञ्जसा जनः} %7-72

\twolineshloka
{यथा वायुवशाद्गन्धः स्वाश्रयाद्\mbox{}घ्राणमाविशेत्}
{योगाभ्यासरतं चित्तमेवमात्मानमाविशेत्} %7-73

\twolineshloka
{सर्वेषु प्राणिजातेषु ह्यहमात्मा व्यवस्थितः}
{तमज्ञात्वा विमूढात्मा कुरुते केवलं बहिः} %7-74

\twolineshloka
{क्रियोत्पन्नैर्नैकभेदैर्द्रव्यैर्मे नाम्ब तोषणम्}
{भूतावमानिनार्चायामर्चितोऽहं न पूजितः} %7-75

\twolineshloka
{तावन्मामर्चयेद्देवं प्रतिमादौ स्वकर्मभिः}
{यावत्सर्वेषु भूतेषु स्थितं चात्मनि न स्मरेत्} %7-76

\twolineshloka
{यस्तु भेदं प्रकुरुते स्वात्मनश्च परस्य च}
{भिन्नदृष्टेर्भयं मृत्युस्तस्य कुर्यान्न संशयः} %7-77

\twolineshloka
{मामतः सर्वभूतेषु परिच्छिन्नेषु संस्थितम्}
{एकं ज्ञानेन मानेन मैत्र्या चार्चेदभिन्नधीः} %7-78

\twolineshloka
{चेतसैवानिशं सर्वभूतानि प्रणमेत्सुधीः}
{ज्ञात्वा मां चेतनं शुद्धं जीवरूपेण संस्थितम्} %7-79

\twolineshloka
{तस्मात्कदाचिन्नेक्षेत भेदमीश्वरजीवयोः}
{भक्तियोगो ज्ञानयोगो मया मातरुदीरितः} %7-80

\twolineshloka
{आलम्ब्यैकतरं वाऽपि पुरुषः शुभमृच्छति}
{ततो मां भक्तियोगेन मातः सर्वहृदि स्थितम्} %7-81

\twolineshloka
{पुत्ररूपेण वा नित्यं स्मृत्वा शान्तिमवाप्स्यसि}
{श्रुत्वा रामस्य वचनं कौसल्याऽऽनन्दसंयुता} %7-82

\twolineshloka
{रामं सदा हृदि ध्यात्वा छित्त्वा संसारबन्धनम्}
{अतिक्रम्य गतीस्तिस्रोऽप्यवाप परमां गतिम्} %7-83

\fourlineindentedshloka
{कैकेयी चापि योगं रघुपतिगदितं पूर्वमेवाधिगम्य}
{श्रद्धाभक्तिप्रशान्ता हृदि रघुतिलकं भावयन्ती गतासुः}
{गत्वा स्वर्गं स्फुरन्ती दशरथसहिता मोदमानावतस्थे}
{माता श्रीलक्ष्मणस्याप्यतिविमलमतिः प्राप भर्तुः समीपम्} %7-84

{॥इति श्रीमदध्यात्मरामायणे उमामहेश्वरसंवादे उत्तरकाण्डे सप्तमः
सर्गः॥७॥
}
%%%%%%%%%%%%%%%%%%%%



\sect{अष्टमः सर्गः}

\textbf{श्रीमहादेव उवाच}

\twolineshloka
{अथ काले गते कस्मिन् भरतो भीमविक्रमः}
{युधाजिता मातुलेन ह्याहूतोऽगात्ससैनिकः} %8-1

\twolineshloka
{रामाज्ञया गतस्तत्र हत्वा गन्धर्वनायकान्}
{तिस्रः कोटीः पुरे द्वे तु निवेश्य रघुनन्दनः} %8-2

\twolineshloka
{पुष्करं पुष्करावत्यां तक्षं तक्षशिलाह्वये}
{अभिषिच्य सुतौ तत्र धनधान्यसुहृद्वृतौ} %8-3

\twolineshloka
{पुनरागत्य भरतो रामसेवापरोऽभवत्}
{ततः प्रीतो रघुश्रेष्ठो लक्ष्मणं प्राह सादरम्} %8-4

\twolineshloka
{उभौ कुमारौ सौमित्रे गृहीत्वा पश्चिमां दिशम्}
{तत्र भिल्लान् विनिर्जित्य दुष्टान् सर्वापकारिणः} %8-5

\twolineshloka
{अङ्गदश्चित्रकेतुश्च महासत्त्वपराक्रमौ}
{द्वयोर्द्वे नगरे कृत्वा गजाश्वधनरत्नकैः} %8-6

\twolineshloka
{अभिषिच्य सुतौ तत्र शीघ्रमागच्छ मां पुनः}
{रामस्याज्ञां पुरस्कृत्य गजाश्वबलवाहनः} %8-7

\twolineshloka
{गत्वा हत्वा रिपून् सर्वान् स्थापयित्वा कुमारकौ}
{सौमित्रिः पुनरागत्य रामसेवापरोऽभवत्} %8-8

\fourlineindentedshloka
{ततस्तु काले महति प्रयाते}
{रामं सदा धर्मपथे स्थितं हरिम्}
{द्रष्टुं समागादृषिवेषधारी}
{कालस्ततो लक्ष्मणमित्युवाच} %8-9

\fourlineindentedshloka
{निवेदयस्वातिबलस्य दूतम्}
{मां द्रष्टुकामं पुरुषोत्तमाय}
{रामाय विज्ञापनमस्ति तस्य}
{महर्षिमुख्यस्य चिराय धीमन्} %8-10

\twolineshloka
{तस्य तद्वचनं श्रुत्वा सौमित्रिस्त्वरयान्वितः}
{आचचक्षेऽथ रामाय स सम्प्राप्तं तपोधनम्} %8-11

\twolineshloka
{एवं ब्रुवन्तं प्रोवाच लक्ष्मणं राघवो वचः}
{शीघ्रं प्रवेश्यतां तात मुनिः सत्कारपूर्वकम्} %8-12

\twolineshloka
{लक्ष्मणस्तु तथेत्युक्त्वा प्रावेशयत तापसम्}
{स्वतेजसा ज्वलन्तं तं घृतसिक्तं यथाऽनलम्} %8-13

\twolineshloka
{सोऽभिगम्य रघुश्रेष्ठं दीप्यमानः स्वतेजसा}
{मुनिर्मधुरवाक्येन वर्धस्वेत्याह राघवम्} %8-14

\twolineshloka
{तस्मै स मुनये रामः पूजां कृत्वा यथाविधि}
{पृष्ट्वाऽनामयमव्यग्रो रामः पृष्टोऽथ तेन सः} %8-15

\twolineshloka
{दिव्यासने समासीनो रामः प्रोवाच तापसम्}
{यदर्थमागतोऽसि त्वमिह तत्प्रापयस्व मे} %8-16

\twolineshloka
{वाक्येन चोदितस्तेन रामेणाह मुनिर्वचः}
{द्वन्द्वमेव प्रयोक्तव्यमनालक्ष्यं तु तद्वचः} %8-17

\twolineshloka
{नान्येन चैतच्छ्रोतव्यं नाख्यातव्यं च कस्यचित्}
{शृणुयाद्वा निरीक्षेद्वा यः स वध्यस्त्वया प्रभो} %8-18

\twolineshloka
{तथेति च प्रतिज्ञाय रामो लक्ष्मणमब्रवीत्}
{तिष्ठ त्वं द्वारि सौमित्रे नायात्वत्र जनो रहः} %8-19

\twolineshloka
{यद्यागच्छति को वाऽपि स वध्यो मे न संशयः}
{ततः प्राह मुनिं रामो येन वा त्वं विसर्जितः} %8-20

\twolineshloka
{यत्ते मनीषितं वाक्यं तद्वदस्व ममाग्रतः}
{ततः प्राह मुनिर्वाक्यं शृणु राम यथातथम्} %8-21

\twolineshloka
{ब्रह्मणा प्रेषितोऽस्मीश कार्यार्थे तेऽन्तिकं प्रभो}
{अहं हि पूर्वजो देव तव पुत्रः परन्तप} %8-22

\twolineshloka
{मायासङ्गमजो वीर कालः सर्वहरः स्मृतः}
{ब्रह्मा त्वामाह भगवान् सर्वदेवर्षिपूजितः} %8-23

\twolineshloka
{रक्षितुं स्वर्गलोकस्य समयस्ते महामते}
{पुरा त्वमेक एवासीर्लोकान् संहृत्य मायया} %8-24

\twolineshloka
{भार्यया सहितस्त्वं मामादौ पुत्रमजीजनः}
{तथा भोगवतं नागमनन्तमुदकेशयम्} %8-25

\twolineshloka
{मायया जनयित्वा त्वं द्वौ ससत्त्वौ महाबलौ}
{मधुकैटभकौ दैत्यौ हत्वा मेदोऽस्थिसञ्चयम्} %8-26

\twolineshloka
{इमां पर्वतसम्बद्धां मेदिनीं पुरुषर्षभ}
{पद्मे दिव्यार्कसङ्काशे नाभ्यामुत्पाद्य मामपि} %8-27

\twolineshloka
{मां विधाय प्रजाध्यक्षं मयि सर्वं न्यवेदयत्}
{सोऽहं संयुक्तसम्भारस्त्वामवोचं जगत्पते} %8-28

\twolineshloka
{रक्षां विधत्स्व भूतेभ्यो ये मे वीर्यापहारिणः}
{ततस्त्वं कश्यपाज्जातो विष्णुर्वामनरूपधृक्} %8-29

\twolineshloka
{हृतवानसि भूभारं वधाद्रक्षोगणस्य च}
{सर्वासूत्सार्यमाणासु प्रजासु धरणीधर} %8-30

\twolineshloka
{रावणस्य वधाकाङ्क्षी मर्त्यलोकमुपागतः}
{दशवर्षसहस्राणि दशवर्षशतानि च} %8-31

\twolineshloka
{कृत्वा वासस्य समयं त्रिदशेष्वात्मनः पुरा}
{स ते मनोरथः पूर्णः पूर्णे चायुषि ते नृषु} %8-32

\twolineshloka
{कालस्तापसरूपेण त्वत्समीपमुपागमत्}
{ततो भूयश्च ते बुद्धिर्यदि राज्यमुपासितुम्} %8-33

\twolineshloka
{तत्तथा भव भद्रं ते एवमाह पितामहः}
{यदि ते गमने बुद्धिर्देवलोकं जितेन्द्रिय} %8-34

\twolineshloka
{सनाथा विष्णुना देवा भजन्तु विगतज्वराः}
{चतुर्मुखस्य तद्वाक्यं श्रुत्वा कालेन भाषितम्} %8-35

\twolineshloka
{हसन् रामस्तदा वाक्यं कृत्स्नस्यान्तकमब्रवीत्}
{श्रुतं तव वचो मेऽद्य ममापीष्टतरं तु तत्} %8-36

\twolineshloka
{सन्तोषः परमो ज्ञेयस्त्वदागमनकारणात्}
{त्रयाणामपि लोकानां कार्यार्थं मम सम्भवः} %8-37

\twolineshloka
{भद्रं तेऽस्त्वागमिष्यामि यत एवाहमागतः}
{मनोरथस्तु सम्प्राप्तो न मेऽत्रास्ति विचारणा} %8-38

\twolineshloka
{मत्सेवकानां देवानां सर्वकार्येषु वै मया}
{स्थातव्यं मायया पुत्र यथा चाह प्रजापतिः} %8-39

\twolineshloka
{एवं तयोः कथयतोर्दुर्वासा मुनिरभ्यगात्}
{राजद्वारं राघवस्य दर्शनापेक्षया द्रुतम्} %8-40

\twolineshloka
{मुनिर्लक्ष्मणमासाद्य दुर्वासा वाक्यमब्रवीत्}
{शीघ्रं दर्शय रामं मे कार्यं मेऽत्यन्तमाहितम्} %8-41

\twolineshloka
{तच्छ्रुत्वा प्राह सौमित्रिर्मुनिं ज्वलनतेजसम्}
{रामेण कार्यं किं तेऽद्य किं तेऽभीष्टं करोम्यहम्} %8-42

\twolineshloka
{राजा कार्यान्तरे व्यग्रो मुहूर्तं सम्प्रतीक्ष्यताम्}
{तच्छ्रुत्वा क्रोधसन्तप्तो मुनिः सौमित्रिमब्रवीत्} %8-43

\twolineshloka
{अस्मिन् क्षणे तु सौमित्रे न दर्शयसि चेद्विभुम्}
{रामं सविषयं वंशं भस्मीकुर्यां न संशयः} %8-44

\twolineshloka
{श्रुत्वा तद्वचनं घोरमृषेर्दुर्वाससो भृशम्}
{स्वरूपं तस्य वाक्यस्य चिन्तयित्वा स लक्ष्मणः} %8-45

\twolineshloka
{सर्वनाशाद्वरं मेऽद्य नाशो ह्येकस्य कारणात्}
{निश्चित्यैवं ततो गत्वा रामाय प्राह लक्ष्मणः} %8-46

\twolineshloka
{सौमित्रेर्वचनं श्रुत्वा रामः कालं व्यसर्जयत्}
{शीघ्रं निर्गम्य रामोऽपि ददर्शात्रेः सुतं मुनिम्} %8-47

\twolineshloka
{रामोऽभिवाद्य सम्प्रीतो मुनिं पप्रच्छ सादरम्}
{किं कार्यं ते करोमीति मुनिमाह रघूत्तमः} %8-48

\twolineshloka
{तच्छ्रुत्वा रामवचनं दुर्वासा राममब्रवीत्}
{अद्य वर्षसहस्राणामुपवाससमापनम्} %8-49

\twolineshloka
{अतो भोजनमिच्छामि सिद्धं यत्ते रघूत्तम}
{रामो मुनिवचः श्रुत्वा सन्तोषेण समन्वितः} %8-50

\twolineshloka
{स सिद्धमन्नं मुनये यथावत्समुपाहरत्}
{मुनिर्भुक्त्वाऽन्नममृतं सन्तुष्टः पुनरभ्यगात्} %8-51

\twolineshloka
{स्वमाश्रमं गते तस्मिन् रामः सस्मार भाषितम्}
{कालेन शोकदुःखार्तो विमनाश्चातिविह्वलः} %8-52

\twolineshloka
{अवाङ्मुखो दीनमना न शशाकाभिभाषितुम्}
{मनसा लक्ष्मणं ज्ञात्वा हतप्रायं रघूद्वहः} %8-53

\twolineshloka
{अवाङ्मुखो बभूवाथ तूष्णीमेवाखिलेश्वरः}
{ततो रामं विलोक्याऽऽह सौमित्रिर्दुःखसम्प्लुतम्} %8-54

\twolineshloka
{तूष्णीम्भूतं चिन्तयन्तं गर्हन्तं स्नेहबन्धनम्}
{मत्कृते त्यज सन्तापं जहि मां रघुनन्दन} %8-55

\twolineshloka
{गतिः कालस्य कलिता पूर्वमेवेदृशी प्रभो}
{त्वयि हीनप्रतिज्ञे तु नरको मे ध्रुवं भवेत्} %8-56

\twolineshloka
{मयि प्रीतिर्यदि भवेद्यद्यनुग्राह्यता तव}
{त्यक्त्वा शङ्कां जहि प्राज्ञ मा मा धर्मं त्यज प्रभो} %8-57

\twolineshloka
{सौमित्रिणोक्तं तच्छ्रुत्वा रामश्चलितमानसः}
{आहूय मन्त्रिणः सर्वान् वसिष्ठं चेदमब्रवीत्} %8-58

\twolineshloka
{मुनेरागमनं यत्तु कालस्यापि हि भाषितम्}
{प्रतिज्ञामात्मनश्चैव सर्वमावेदयत्प्रभुः} %8-59

\twolineshloka
{श्रुत्वा रामस्य वचनं मन्त्रिणः सपुरोहिताः}
{ऊचुः प्राञ्जलयः सर्वे राममक्लिष्टकारिणम्} %8-60

\twolineshloka
{पूर्वमेव हि निर्दिष्टं तव भूभारहारिणः}
{लक्ष्मणेन वियोगस्ते ज्ञातो विज्ञानचक्षुषा} %8-61

\twolineshloka
{त्यजाऽऽशु लक्ष्मणं राम मा प्रतिज्ञां त्यज प्रभो}
{प्रतिज्ञाते परित्यक्ते धर्मो भवति निष्फलः} %8-62

\twolineshloka
{धर्मे नष्टेऽखिले राम त्रैलोक्यं नश्यति ध्रुवम्}
{त्वं तु सर्वस्य लोकस्य पालकोऽसि रघूत्तम} %8-63

\twolineshloka
{त्यक्त्वा लक्ष्मणमेवैकं त्रैलोक्यं त्रातुमर्हसि}
{रामो धर्मार्थसहितं वाक्यं तेषामनिन्दितम्} %8-64

\twolineshloka
{सभामध्ये समाश्रुत्य प्राह सौमित्रिमञ्जसा}
{यथेष्टं गच्छ सौमित्रे मा भूद्धर्मस्य संशयः} %8-65

\twolineshloka
{परित्यागो वधो वाऽपि सतामेवोभयं समम्}
{एवमुक्ते रघुश्रेष्ठे दुःखव्याकुलितेक्षणः} %8-66

\twolineshloka
{रामं प्रणम्य सौमित्रिः शीघ्रं गृहमगात्स्वकम्}
{ततोऽगात्सरयूतीरमाचम्य स कृताञ्जलिः} %8-67

\twolineshloka
{नव द्वाराणि संयम्य मूर्ध्नि प्राणमधारयत्}
{यदक्षरं परं ब्रह्म वासुदेवाख्यमव्ययम्} %8-68

\twolineshloka
{पदं तत्परमं धाम चेतसा सोऽभ्यचिन्तयत्}
{वायुरोधेन संयुक्तं सर्वे देवाः सहर्षयः} %8-69

\twolineshloka
{साग्नयो लक्ष्मणं पुष्पैस्तुष्टुवुश्च समाकिरन्}
{अदृश्यं विबुधैः कैश्चित्सशरीरं च वासवः} %8-70

\threelineshloka
{गृहीत्वा लक्ष्मणं शक्रः स्वर्गलोकमथागमत्}
{ततो विष्णोश्चतुर्भागं तं देवं सुरसत्तमाः}
{सर्वे देवर्षयो दृष्ट्वा लक्ष्मणं समपूजयन्} %8-71

\twolineshloka
{लक्ष्मणे हि दिवमागते हरौ सिद्धलोकगतयोगिनस्तदा}
{ब्रह्मणा सह समागमन्मुदा द्रष्टुमाहितमहाहिरूपकम्} %8-72

{॥इति श्रीमदध्यात्मरामायणे उमामहेश्वरसंवादे उत्तरकाण्डे अष्टमः
सर्गः॥८॥}
%%%%%%%%%%%%%%%%%%%%



\sect{नवमः सर्गः}

\textbf{श्री महादेव उवाच}

\twolineshloka
{लक्ष्मणं तु परित्यज्य रामो दुःखसमन्वितः}
{मन्त्रिणो नैगमांश्चैव वसिष्ठं चेदमब्रवीत्} %9-1

\twolineshloka
{अभिषेक्ष्यामि भरतमधिराज्ये महामतिम्}
{अद्य चाहं गमिष्यामि लक्ष्मणस्य पदानुगः} %9-2

\twolineshloka
{एवमुक्ते रघुश्रेष्ठे पौरजानपदास्तदा}
{द्रुमा इवच्छिन्नमूला दुःखार्ताः पतिता भुवि} %9-3

\twolineshloka
{मूर्च्छितो भरतो वाऽपि श्रुत्वा रामाभिभाषितम्}
{गर्हयामास राज्यं स प्राहेदं रामसन्निधौ} %9-4

\twolineshloka
{सत्येन च शपे नाहं त्वां विना दिवि वा भुवि}
{काङ्क्षे राज्यं रघुश्रेष्ठ शपे त्वत्पादयोः प्रभो} %9-5

\twolineshloka
{इमौ कुशलवौ राजन्नभिषिञ्चस्व राघव}
{कोसलेषु कुशं वीरमुत्तरेषु लवं तथा} %9-6

\twolineshloka
{गच्छन्तु दूतास्त्वरितं शत्रुघ्नानयनाय हि}
{अस्माकमेतद्गमनं स्वर्वासाय शृणोतु सः} %9-7

\twolineshloka
{भरतेनोदितं श्रुत्वा पतितास्ताः समीक्ष्य तम्}
{प्रजाश्च भयसंविग्ना रामविश्लेषकातराः} %9-8

\twolineshloka
{वसिष्ठो भगवान् राममुवाच सदयं वचः}
{पश्य ताताऽऽदरात्सर्वाः पतिता भूतले प्रजाः} %9-9

\twolineshloka
{तासां भावानुगं राम प्रसादं कर्तुमर्हसि}
{श्रुत्वा वसिष्ठवचनं ताः समुत्थाप्य पूज्य च} %9-10

\twolineshloka
{सस्नेहो रघुनाथस्ताः किं करोमीति चाब्रवीत्}
{ततः प्राञ्जलयः प्रोचुः प्रजा भक्त्या रघूद्वहम्} %9-11

\twolineshloka
{गन्तुमिच्छसि यत्र त्वमनुगच्छामहे वयम्}
{अस्माकमेषा परमा प्रीतिर्धर्मोऽयमक्षयः} %9-12

\twolineshloka
{तवानुगमने राम हृद्गता नो दृढा मतिः}
{पुत्रदारादिभिः सार्धमनुयामोऽद्य सर्वथा} %9-13

\twolineshloka
{तपोवनं वा स्वर्गं वा पुरं वा रघुनन्दन}
{ज्ञात्वा तेषां मनोदार्ढ्यं कालस्य वचनं तथा} %9-14

\twolineshloka
{भक्तं पौरजनं चैव बाढमित्याह राघवः}
{कृत्वैव निश्चयं रामस्तस्मिन्नेवाहनि प्रभुः} %9-15

\twolineshloka
{प्रस्थापयामास च तौ रामभद्रः कुशीलवौ}
{अष्टौ रथसहस्राणि सहस्रं चैव दन्तिनाम्} %9-16

\twolineshloka
{षष्टिं चाश्वसहस्राणामेकैकस्मै ददौ बलम्}
{बहुरत्नौ बहुधनौ हृष्टपुष्टजनावृतौ} %9-17

\twolineshloka
{अभिवाद्य गतौ रामं कृच्छ्रेण तु कुशीलवौ}
{शत्रुघ्नानयने दूतान् प्रेषयामास राघवः} %9-18

\twolineshloka
{ते दूतास्त्वरितं गत्वा शत्रुघ्नाय न्यवेदयन्}
{कालस्याऽऽगमनं पश्चादत्रिपुत्रस्य चेष्टितम्} %9-19

\twolineshloka
{लक्ष्मणस्य च निर्याणं प्रतिज्ञां राघवस्य च}
{पुत्राभिषेचनं चैव सर्वं रामचिकीर्षितम्} %9-20

\threelineshloka
{श्रुत्वा तद्दूतवचनं शत्रुघ्नः कुलनाशनम्}
{व्यथितोऽपि धृतिं लब्ध्वा पुत्रावाहूय सत्वरः}
{अभिषिच्य सुबाहुं वै मथुरायां महाबलः} %9-21

\twolineshloka
{यूपकेतुं च विदिशानगरे शत्रुसूदनः}
{अयोध्यां त्वरितं प्रागात्स्वयं रामदिदृक्षया} %9-22

\twolineshloka
{ददर्श च महात्मानं तेजसा ज्वलनप्रभम्}
{दुकूलयुगसंवीतं ऋषिभिश्चाक्षयैर्वृतम्} %9-23

\twolineshloka
{अभिवाद्य रमानाथं शत्रुघ्नो रघुपुङ्गवम्}
{प्राञ्जलिर्धर्मसहितं वाक्यं प्राह महामतिः} %9-24

\twolineshloka
{अभिषिच्य सुतौ तत्र राज्ये राजीवलोचन}
{तवानुगमने राजन् विद्धि मां कृतनिश्चयम्} %9-25

\twolineshloka
{त्यक्तुं नार्हसि मां वीर भक्तं तव विशेषतः}
{शत्रुघ्नस्य दृढां बुद्धिं विज्ञाय रघुनन्दनः} %9-26

\twolineshloka
{सज्जीभवतु मध्याह्ने भवानित्यब्रवीद्वचः}
{अथ क्षणात्समुत्पेतुर्वानराः कामरूपिणः} %9-27

\twolineshloka
{ऋक्षाश्च राक्षसाश्चैव गोपुच्छाश्च सहस्रशः}
{ऋषीणां देवतानां च पुत्रा रामस्य निर्गमम्} %9-28

\twolineshloka
{श्रुत्वा प्रोचू रघुश्रेष्ठं सर्वे वानरराक्षसाः}
{तवानुगमने विद्धि निश्चितार्थान् हि नः प्रभो} %9-29

\twolineshloka
{एतस्मिन्नन्तरे रामं सुग्रीवोऽपि महाबलः}
{यथावदभिवाद्याह राघवं भक्तवत्सलम्} %9-30

\twolineshloka
{अभिषिच्याङ्गदं राज्ये आगतोऽस्मि महाबलम्}
{तवानुगमने राम विद्धि मां कृतनिश्चयम्} %9-31

\twolineshloka
{श्रुत्वा तेषां दृढं वाक्यं ऋक्षवानररक्षसाम्}
{विभीषणमुवाचेदं वचनं मृदु सादरम्} %9-32

\twolineshloka
{धरिष्यति धरा यावत्प्रजास्तावत्प्रशाधि मे}
{वचनाद्राक्षसं राज्यं शापितोऽसि ममोपरि} %9-33

\twolineshloka
{न किञ्चिदुत्तरं वाच्यं त्वया मत्कृतकारणात्}
{एवं विभीषणं तूक्त्वा हनूमन्तमथाब्रवीत्} %9-34

\twolineshloka
{मारुते त्वं चिरञ्जीव ममाज्ञां मा मृषा कृथाः}
{जाम्बवन्तमथ प्राह तिष्ठ त्वं द्वापरान्तरे} %9-35

\threelineshloka
{मया सार्धं भवेद्युद्धं यत्किञ्चित्कारणान्तरे}
{ततस्तान् राघवः प्राह ऋक्षराक्षसवानरान्}
{सर्वानेव मया सार्धं प्रयातेति दयान्वितः} %9-36

\fourlineindentedshloka
{ततः प्रभाते रघुवंशनाथो}
{विशालवक्षाः सितकञ्जनेत्रः}
{पुरोधसं प्राह वसिष्ठमार्यम्}
{यान्त्वग्निहोत्राणि पुरो गुरो मे} %9-37

\fourlineindentedshloka
{ततो वसिष्ठोऽपि चकार सर्वम्}
{प्रास्थानिकं कर्म महद्विधानात्}
{क्षौमाम्बरो दर्भपवित्रपाणिः}
{महाप्रयाणाय गृहीतबुद्धिः} %9-38

\fourlineindentedshloka
{निष्क्रम्य रामो नगरात्सिताभ्रा-}
{च्छशीव यातः शशिकोटिकान्तिः}
{रामस्य सव्ये सितपद्महस्ता}
{पद्मा गता पद्मविशालनेत्रा} %9-39

\fourlineindentedshloka
{पार्श्वेऽथ दक्षेऽरुणकञ्जहस्ता}
{श्यामा ययौ भूरपि दीप्यमाना}
{शास्त्राणि शस्त्राणि धनुश्च बाणा}
{जग्मुः पुरस्ताद्धृतविग्रहास्ते} %9-40

\fourlineindentedshloka
{वेदाश्च सर्वे धृतविग्रहाश्च}
{ययुश्च सर्वे मुनयश्च दिव्याः}
{माता श्रुतीनां प्रणवेन साध्वी}
{ययौ हरिं व्याहृतिभिः समेता} %9-41

\fourlineindentedshloka
{गच्छन्तमेवानुगता जनास्ते}
{सपुत्रदाराः सह बन्धुवर्गैः}
{अनावृतद्वारमिवापवर्गम्}
{रामं व्रजन्तं ययुराप्तकामाः}%9-42

\begin{minipage}{\linewidth}
\centering
{\hspace{-9ex}सान्तःपुरः\hspace{0.5ex} सानुचरः \hspace{0.5ex}सभार्यः}\\
{\hspace{-1ex}शत्रुघ्नयुक्तो\hspace{3ex}भरतोऽनुयातः।} 
\fourlineindentedshloka
{गच्छन्तमालोक्य रमासमेतम्}
{श्रीराघवं पौरजनाः समस्ताः}
{सबालवृद्धाश्च ययुर्द्विजाग्र्याः}
{सामात्यवर्गाश्च समन्त्रिणो ययुः} %9-43
\end{minipage}

\fourlineindentedshloka
{सर्वे गताः क्षत्रमुखाः प्रहृष्टा}
{वैश्याश्च शूद्राश्च तथा परे च}
{सुग्रीवमुख्या हरिपुङ्गवाश्च}
{स्नाता विशुद्धाः शुभशब्दयुक्ताः} %9-44

\fourlineindentedshloka
{न कश्चिदासीद्भवदुःखयुक्तो}
{दीनोऽथवा बाह्यसुखेषु सक्तः}
{आनन्दरूपानुगता विरक्ता}
{ययुश्च रामं पशुभृत्यवर्गैः} %9-45

\fourlineindentedshloka
{भूतान्यदृश्यानि च यानि तत्र}
{ये प्राणिनः स्थावरजङ्गमाश्च}
{साक्षात्परात्मानमनन्तशक्तिम्}
{जग्मुर्विमुक्ताः परमेकमीशम्} %9-46

\fourlineindentedshloka
{नासीदयोध्यानगरे तु जन्तुः}
{कश्चित्तदा राममना न यातः}
{शून्यं बभूवाखिलमेव तत्र}
{पुरं गते राजनि रामचन्द्रे} %9-47

\fourlineindentedshloka
{ततोऽतिदूरं नगरात्स गत्वा}
{दृष्ट्वा नदीं तां हरिनेत्रजाताम्}
{ननन्द रामः स्मृतपावनोऽतो}
{ददर्श चाशेषमिदं हृदिस्थम्} %9-48

\fourlineindentedshloka
{अथाऽऽगतस्तत्र पितामहो महान्}
{देवाश्च सर्वे ऋषयश्च सिद्धाः}
{विमानकोटीभिरपारपारम्}
{समावृतं खं सुरसेविताभिः} %9-49

\fourlineindentedshloka
{रविप्रकाशाभिरभिस्फुरत्स्वम्}
{ज्योतिर्मयं तत्र नभो बभूव}
{स्वयम्प्रकाशैर्महतां महद्भिः}
{समावृतं पुण्यकृतां वरिष्ठैः} %9-50

\fourlineindentedshloka
{ववुश्च वाताश्च सुगन्धवन्तो}
{ववर्ष वृष्टिः कुसुमावलीनाम्}
{उपस्थिते देवमृदङ्गनादे}
{गायत्सु विद्याधरकिन्नरेषु} %9-51

\fourlineindentedshloka
{रामस्तु पद्भ्यां सरयूजलं सकृत्}
{स्पृष्ट्वा परिक्रामदनन्तशक्तिः}
{ब्रह्मा तदा प्राह कृताञ्जलिस्तम्}
{रामं परात्मन् परमेश्वरस्त्वम्} %9-52

\fourlineindentedshloka
{विष्णुः सदानन्दमयोऽसि पूर्णो}
{जानासि तत्त्वं निजमैशमेकम्}
{तथाऽपि दासस्य ममाखिलेश}
{कृतं वचो भक्तपरोऽसि विद्वन्} %9-53

\fourlineindentedshloka
{त्वं भ्रातृभिर्वैष्णवमेवमाद्यम्}
{प्रविश्य देहं परिपाहि देवान्}
{यद्वा परो वा यदि रोचते तम्}
{प्रविश्य देहं परिपाहि नस्त्वम्} %9-54

\fourlineindentedshloka
{त्वमेव देवाधिपतिश्च विष्णुः}
{जानन्ति न त्वां पुरुषा विना माम्}
{सहस्रकृत्वस्तु नमो नमस्ते}
{प्रसीद देवेश पुनर्नमस्ते} %9-55

\fourlineindentedshloka
{पितामहप्रार्थनया स रामः}
{पश्यत्सु देवेषु महाप्रकाशः}
{मुष्णंश्च चक्षूंषि दिवौकसां तदा}
{बभूव चक्रादियुतश्चतुर्भुजः} %9-56

\fourlineindentedshloka
{शेषो बभूवेश्वरतल्पभूतः}
{सौमित्रिरत्यद्भुतभोगधारी}
{बभूवतुश्चक्रदरौ च दिव्यौ}
{कैकेयिसूनुर्लवणान्तकश्च} %9-57

\fourlineindentedshloka
{सीता च लक्ष्मीरभवत्पुरेव}
{रामो हि विष्णुः पुरुषः पुराणः}
{सहानुजः पूर्वशरीरकेण}
{बभूव तेजोमयदिव्यमूर्तिः} %9-58

\fourlineindentedshloka
{विष्णुं समासाद्य सुरेन्द्रमुख्या}
{देवाश्च सिद्धा मुनयश्च दक्षाः}
{पितामहाद्याः परितः परेशम्}
{स्तवैर्गृणन्तः परिपूजयन्तः} %9-59

\fourlineindentedshloka
{आनन्दसम्प्लावितपूर्णचित्ता}
{बभूविरे प्राप्तमनोरथास्ते}
{तदाऽऽह विष्णुर्द्रुहिणं महात्मा}
{एते हि भक्ता मयि चानुरक्ताः} %9-60

\fourlineindentedshloka
{यान्तं दिवं मामनुयान्ति सर्वे}
{तिर्यक्षरीरा अपि पुण्ययुक्ताः}
{वैकुण्ठसाम्यं परमं प्रयान्तु}
{समाविशस्वाऽऽशु ममऽऽज्ञया त्वम्} %9-61

\fourlineindentedshloka
{श्रुत्वा हरेर्वाक्यमथाब्रवीत्कः}
{सान्तानिकान् यान्तु विचित्रभोगान्}
{लोकान्मदीयोपरि दीप्यमानान्}
{त्वद्भावयुक्ताः कृतपुण्यपुञ्जाः} %9-62

\fourlineindentedshloka
{ये चापि ते राम पवित्रनाम}
{गृणन्ति मर्त्या लयकाल एव}
{अज्ञानतो वाऽपि भजन्तु लोकान्}
{तानेव योगैरपि चाधिगम्यान्} %9-63

\fourlineindentedshloka
{ततोऽतिहृष्टा हरिराक्षसाद्याः}
{स्पृष्ट्वा जलं त्यक्तकलेवरास्ते}
{प्रपेदिरे प्राक्तनमेव रूपम्}
{यदंशजा ऋक्षहरीश्वरास्ते} %9-64

\fourlineindentedshloka
{प्रभाकरं प्राप हरिप्रवीरः}
{सुग्रीव आदित्यजवीर्यवत्त्वात्}
{ततो विमग्नाः सरयूजलेषु}
{नराः परित्यज्य मनुष्यदेहम्} %9-65

\fourlineindentedshloka
{आरुह्य दिव्याभरणा विमानम्}
{प्रापुश्च ते सान्तनिकाख्यलोकान्}
{तिर्यक्प्रजाता अपि रामदृष्टा}
{जलं प्रविष्टा दिवमेव याताः} %9-66

\fourlineindentedshloka
{दिदृक्षवो जानपदाश्च लोका}
{रामं समालोक्य विमुक्तसङ्गाः}
{स्मृत्वा हरिं लोकगुरुं परेशम्}
{स्पृष्ट्वा जलं स्वर्गमवापुरञ्जः} %9-67

\fourlineindentedshloka
{एतावदेवोत्तरमाह शम्भुः}
{श्रीरामचन्द्रस्य कथावशेषम्}
{यः पादमप्यत्र पठेत्स पापाद्-}
{विमुच्यते जन्मसहस्रजातात्} %9-68

\fourlineindentedshloka
{दिने दिने पापचयं प्रकुर्वन्}
{पठेन्नरः श्लोकमपीह भक्त्या}
{विमुक्तसर्वाघचयः प्रयाति}
{रामस्य सालोक्यमनन्यलभ्यम्} %9-69

\fourlineindentedshloka
{आख्यानमेतद्रघुनायकस्य}
{कृतं पुरा राघवचोदितेन}
{महेश्वरेणाप्तभविष्यदर्थम्}
{श्रुत्वा तु रामः परितोषमेति} %9-70

\fourlineindentedshloka
{रामायणं काव्यमनन्तपुण्यम्}
{श्रीशङ्करेणाभिहितं भवान्यै}
{भक्त्या पठेद्यः शृणुयात् स पापैर्-}
{विमुच्यते जन्मशतोद्भवैश्च} %9-71

\fourlineindentedshloka
{अध्यात्मरामं पठतश्च नित्यम्}
{श्रोतुश्च भक्त्या लिखितुश्च रामः}
{अतिप्रसन्नश्च सदा समीपे}
{सीतासमेतः श्रियमातनोति} %9-72

\fourlineindentedshloka
{रामायणं जनमनोहरमादिकाव्यम्}
{ब्रह्मादिभिः सुरवरैरपि संस्तुतं च}
{श्रद्धान्वितः पठति यः शृणुयात्तु नित्यम्}
{विष्णोः प्रयाति सदनं स विशुद्धदेहः} %9-73

{॥इति श्रीमदध्यात्मरामायणे उमामहेश्वरसंवादे
उत्तरकाण्डे नवमः सर्गः॥९॥}
%%%%%%%%%%%%%%%%%%%%

॥इति श्रीमदध्यात्मरामायणे उत्तरकाण्डः समाप्तः॥\\
॥इति श्रीमदध्यात्मरामायणं सम्पूर्णम्॥