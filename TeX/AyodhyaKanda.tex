

\chapt{अयोध्याकाण्डः}


\sect{प्रथमः सर्गः}

\twolineshloka
{एकदा सुखमासीनं रामं स्वान्तःपुराजिरे}
{सर्वाभरणसम्पन्नं रत्नसिंहासने स्थितम्} %1-1

\twolineshloka
{नीलोत्पलदलश्यामं कौस्तुभामुक्तकन्धरम्}
{सीतया रत्नदण्डेन चामरेणाथ वीजितम्} %1-2

\twolineshloka
{विनोदयन्तं ताम्बूलचर्वणादिभिरादरात्}
{नारदोऽवतरद्द्रष्टुमम्बराद्यत्र राघवः} %1-3

\twolineshloka
{शुद्धस्फटिकसङ्काशः शरच्चन्द्र इवामलः}
{अतर्कितमुपायातो नारदो दिव्यदर्शनः} %1-4

\twolineshloka
{तं दृष्ट्वा सहसोत्थाय रामः प्रीत्या कृताञ्जलिः}
{ननाम शिरसा भूमौ सीतया सह भक्तिमान्} %1-5

\threelineshloka
{उवाच नारदं रामः प्रीत्या परमया युतः}
{संसारिणां मुनिश्रेष्ठ दुर्लभं तव दर्शनम्}
{अस्माकं विषयासक्तचेतसां नितरां मुने} %1-6

\twolineshloka
{अवाप्तं मे पूर्वजन्मकृतपुण्यमहोदयैः}
{संसारिणाऽपि हि मुने लभ्यते सत्समागमः} %1-7

\twolineshloka
{अतस्त्वद्दर्शनादेव कृतार्थोऽस्मि मुनीश्वर}
{किं कार्यं ते मया कार्यं ब्रूहि तत्करवाणि भोः} %1-8

\twolineshloka
{अथ तं नारदोऽप्याह राघवं भक्तवत्सलम्}
{किं मोहयसि मां राम वाक्यैर्लोकानुसारिभिः} %1-9

\twolineshloka
{संसार्यहमिति प्रोक्तं सत्यमेतत्त्वया विभो}
{जगतामादिभूता या सा माया गृहिणी तव} %1-10

\twolineshloka
{त्वत्सन्निकर्षाज्जायन्ते तस्यां ब्रह्मादयः प्रजाः}
{त्वदाश्रया सदा भाति माया या त्रिगुणात्मिका} %1-11

\twolineshloka
{सूतेऽजस्रं शुक्लकृष्णलोहिताः सर्वदा प्रजाः}
{लोकत्रयमहागेहे गृहस्थस्त्वमुदाहृतः} %1-12

\twolineshloka
{त्वं विष्णुर्जानकी लक्ष्मीः शिवस्त्वं जानकी शिवा}
{ब्रह्मा त्वं जानकी वाणी सूर्यस्त्वं जानकी प्रभा} %1-13

\twolineshloka
{भवान् शशाङ्कः सीता तु रोहिणी शुभलक्षणा}
{शक्रस्त्वमेव पौलोमी सीता स्वाहानलो भवान्} %1-14

\twolineshloka
{यमस्त्वं कालरूपश्च सीता संयमिनी प्रभो}
{निरृतिस्त्वं जगन्नाथ तामसी जानकी शुभा} %1-15

\twolineshloka
{राम त्वमेव वरुणो भार्गवी जानकी शुभा}
{वायुस्त्वं राम सीता तु सदागतिरितीरिता} %1-16

\twolineshloka
{कुबेरस्त्वं राम सीता सर्वसम्पत्प्रकीर्तिता}
{रुद्राणी जानकी प्रोक्ता रुद्रस्त्वं लोकनाशकृत्} %1-17

\twolineshloka
{लोके स्त्रीवाचकं यावत्तत्सर्वं जानकी शुभा}
{पुन्नामवाचकं यावत्तत्सर्वं त्वं हि राघव} %1-18

{तस्माल्लोकत्रये देव युवाभ्यां नास्ति किञ्चन॥१९॥} %1-19
\refstepcounter{shlokacount}


\twolineshloka
{त्वदाभासोदिताज्ञानमव्याकृतमितीर्यते}
{तस्मान्महान्स्ततः सूत्रं लिङ्गं सर्वात्मकं ततः} %1-20

\twolineshloka
{अहङ्कारश्च बुद्धिश्च पञ्चप्राणेन्द्रियाणि च}
{लिङ्गमित्युच्यते प्राज्ञैर्जन्ममृत्युसुखादिमत्} %1-21

\twolineshloka
{स एव जीवसंज्ञश्च लोके भाति जगन्मयः}
{अवाच्यानाद्यविद्यैव कारणोपाधिरुच्यते} %1-22

\twolineshloka
{स्थूलं सूक्ष्मं कारणाख्यमुपाधित्रितयं चितेः}
{एतैर्विशिष्टो जीवः स्याद्वियुक्तः परमेश्वरः} %1-23

\twolineshloka
{जाग्रत्स्वप्नसुषुप्त्याख्या संसृतिर्या प्रवर्तते}
{तस्या विलक्षणः साक्षी चिन्मात्रस्त्वं रघूत्तम} %1-24

\twolineshloka
{त्वत्त एव जगज्जातं त्वयि सर्वं प्रतिष्ठितम्}
{त्वय्येव लीयते कृत्स्नं तस्मात्त्वं सर्वकारणम्} %1-25

\twolineshloka
{रज्जावहिमिवात्मानं जीवं ज्ञात्वा भयं भवेत्}
{परात्माहमिति ज्ञात्वा भयदुःखैर्विमुच्यते} %1-26

\twolineshloka
{चिन्मात्रज्योतिषा सर्वाः सर्वदेहेषु बुद्धयः}
{त्वया यस्मात्प्रकाश्यन्ते सर्वस्यात्मा ततो भवान्} %1-27

{अज्ञानान्न्यस्यते सर्वं त्वयि रज्जौ भुजङ्गवत्॥२८॥} %1-28
\refstepcounter{shlokacount}


\twolineshloka
{त्वत्पादभक्तियुक्तानां विज्ञानं भवति क्रमात्}
{तस्मात्त्वद्भक्तियुक्ता ये मुक्तिभाजस्त एव हि} %1-29

\twolineshloka
{अहं त्वद्भक्तभक्तानां तद्भक्तानां च किङ्करः}
{अतो मामनुगृह्णीष्व मोहयस्व न मां प्रभो} %1-30

\twolineshloka
{त्वन्नाभिकमलोत्पन्नो ब्रह्मा मे जनकः प्रभो}
{अतस्तवाहं पौत्रोऽस्मि भक्तं मां पाहि राघव} %1-31

\twolineshloka
{इत्युक्त्वा बहुशो नत्वा स्वानन्दाश्रुपरिप्लुतः}
{उवाच वचनं राम ब्रह्मणा नोदितोऽस्म्यहम्} %1-32

\twolineshloka
{रावणस्य वधार्थाय जातोऽसि रघुसत्तम}
{इदानीं राज्यरक्षार्थं पिता त्वामभिषेक्ष्यति} %1-33

\twolineshloka
{यदि राज्याभिसंसक्तो रावणं न हनिष्यसि}
{प्रतिज्ञा ते कृता राम भूभारहरणाय वै} %1-34

\twolineshloka
{तत्सत्यं कुरु राजेन्द्र सत्यसन्धस्त्वमेव हि}
{श्रुत्वैतद्गदितं रामो नारदं प्राह सस्मितम्} %1-35

\twolineshloka
{शृणु नारद मे किञ्चिद्विद्यतेऽविदितं क्वचित्}
{प्रतिज्ञातं च यत्पूर्वं करिष्ये तन्न संशयः} %1-36

\twolineshloka
{किन्तु कालानुरोधेन तत्तत्प्रारब्धसङ्क्षयात्}
{हरिष्ये सर्वभूभारं क्रमेणासुरमण्डलम्} %1-37

\twolineshloka
{रावणस्य विनाशार्थं श्वो गन्ता दण्डकाननम्}
{चतुर्दश समास्तत्र ह्युषित्वा मुनिवेषधृक्} %1-38

\twolineshloka
{सीतामिषेण तं दुष्टं सकुलं नाशयाम्यहम्}
{एवं रामे प्रतिज्ञाते नारदः प्रमुमोद ह} %1-39

\twolineshloka
{प्रदक्षिणत्रयं कृत्वा दण्डवत्प्रणिपत्य तम्}
{अनुज्ञातश्च रामेण ययौ देवगतिं मुनिः} %1-40

\fourlineindentedshloka
{संवादं पठति शृणोति संस्मरेद्वा}
{यो नित्यं मुनिवररामयोः सभक्त्या}
{सम्प्राप्नोत्यमरसुदुर्लभं विमोक्षम्}
{कैवल्यं विरतिपुरःसरं क्रमेण} %1-41

{॥इति श्रीमदध्यात्मरामायणे उमामहेश्वरसंवादे
अयोध्याकाण्डे प्रथमः सर्गः॥१॥
}
%%%%%%%%%%%%%%%%%%%%



\sect{द्वितीयः सर्गः}

\twolineshloka
{अथ राजा दशरथः कदाचिद्रहसि स्थितः}
{वसिष्ठं स्वकुलाचार्यमाहूयेदमभाषत} %2-1

\twolineshloka
{भगवन् राममखिलाः प्रशंसन्ति मुहुर्मुहुः}
{पौराश्च निगमा वृद्धा मन्त्रिणश्च विशेषतः} %2-2

\twolineshloka
{ततः सर्वगुणोपेतं रामं राजीवलोचनम्}
{ज्येष्ठं राज्येऽभिषेक्ष्यामि वृद्धोऽहं मुनिपुङ्गव} %2-3

\twolineshloka
{भरतो मातुलं द्रष्टुं गतः शत्रुघ्नसंयुतः}
{अभिषेक्ष्ये श्व एवाशु भवान्स्तच्चानुमोदताम्} %2-4

\twolineshloka
{सम्भाराः सम्भ्रियन्तां च गच्छ मन्त्रय राघवम्}
{उच्छ्रीयन्तां पताकाश्च नानावर्णाः समन्ततः} %2-5

\twolineshloka
{तोरणानि विचित्राणि स्वर्णमुक्तामयानि वै}
{आहूय मन्त्रिणं राजा सुमन्त्रम्} %2-6

\twolineshloka
{आज्ञापयति यद्यत्त्वां मुनिस्तत्तत्समानय}
{यौवराज्येऽभिषेक्ष्यामि श्वोभूते रघुनन्दनम्} %2-7

\twolineshloka
{तथेति हर्षात्स मुनिं किं करोमीत्यभाषत}
{तमुवाच महातेजा वसिष्ठो ज्ञानिनां वरः} %2-8

\twolineshloka
{श्वः प्रभाते मध्यकक्षे कन्यकाः स्वर्णभूषिताः}
{तिष्ठन्तु षोडश गजाः स्वर्णरत्नादि भूषिताः} %2-9

\twolineshloka
{चतुर्दन्तः समायातु ऐरावतकुलोद्भवः}
{नानातीर्थोदकैः पूर्णाः स्वर्णकुम्भाः सहस्रशः} %2-10

\twolineshloka
{स्थाप्यन्तां नववैयाघ्रचर्माणि त्रीणि चानय}
{श्वेतच्छत्रं रत्नदण्डं मुक्तामणिविराजितम्} %2-11

\twolineshloka
{दिव्यमाल्यानि वस्त्राणि दिव्यान्याभरणानि च}
{मुनयः सत्कृतास्तत्र तिष्ठन्तु कुशपाणयः} %2-12

\twolineshloka
{नर्तक्यो वारमुख्याश्च गायका वेणुकास्तथा}
{नानावादित्रकुशला वादयन्तु नृपाङ्गणे} %2-13

\twolineshloka
{हस्त्यश्वरथपादाता बहिस्तिष्ठन्तु सायुधाः}
{नगरे यानि तिष्ठन्ति देवतायतनानि च} %2-14

\twolineshloka
{तेषु प्रवर्ततां पूजा नानाबलिभिरावृता}
{राजानः शीघ्रमायान्तु नानोपायनपाणयः} %2-15

\twolineshloka
{इत्यादिश्य मुनिः श्रीमान् सुमन्त्रं नृपमन्त्रिणम्}
{स्वयं जगाम भवनं राघवस्यातिशोभनम्} %2-16

\twolineshloka
{रथमारुह्य भगवान् वसिष्ठो मुनिसत्तमः}
{त्रीणि कक्षाण्यतिक्रम्य रथात्क्षितिमवातरत्} %2-17

\twolineshloka
{अन्तः प्रविश्य भवनं स्वाचार्यत्वादवारितः}
{गुरुमागतमाज्ञाय रामस्तूर्णः कृताञ्जलिः} %2-18

\twolineshloka
{प्रत्युद्गम्य नमस्कृत्य दण्डवद्भक्तिसंयुतः}
{स्वर्णपात्रेण पानीयमानिनायाऽऽशु जानकी} %2-19

\twolineshloka
{रत्नासने समावेश्य पादौ प्रक्षाल्य भक्तितः}
{तदपः शिरसा धृत्वा सीताया सह राघवः} %2-20

\twolineshloka
{धन्योऽस्मीत्यब्रवीद्रामस्तव पादाम्बुधारणात्}
{श्रीरामेणैवमुक्तस्तु प्रहसन् मुनिरब्रवीत्} %2-21

\twolineshloka
{त्वत्पादसलिलं धृत्वा धन्योऽभूद्गिरिजापतिः}
{ब्रह्माऽपि मत्पिता ते हि पादतीर्थहताशुभः} %2-22

\twolineshloka
{इदानीं भाषसे यत्त्वं लोकानामुपदेशकृत्}
{जानामि त्वां परात्मानं लक्ष्म्या सञ्जातमीश्वरम्} %2-23

\twolineshloka
{देवकार्यार्थसिद्ध्यर्थं भक्तानां भक्तिसिद्धये}
{रावणस्य वधार्थाय जातं जानामि राघव} %2-24

\twolineshloka
{तथाऽपि देवकार्यार्थं गुह्यं नोद्घाटयाम्यहम्}
{तथा त्वं मायया सर्वं करोषि रघुनन्दन} %2-25

\twolineshloka
{तथैवानुविधास्येऽहं शिष्यस्त्वं गुरुरप्यहम्}
{गुरुर्गुरूणां त्वं देव पितॄणां त्वं पितामहः} %2-26

\twolineshloka
{अन्तर्यामी जगद्यात्रावाहकस्त्वमगोचरः}
{शुद्धसत्त्वमयं देहं धृत्वा स्वाधीनसम्भवम्} %2-27

\twolineshloka
{मनुष्य इव लोकेऽस्मिन् भासि त्वं योगमायया}
{पौरोहित्यमहं जाने विगर्ह्यं दूष्यजीवनम्} %2-28

\twolineshloka
{इक्ष्वाकूणां कुले रामः परमात्मा जनिष्यते}
{इति ज्ञातं मया पूर्वं ब्रह्मणा कथितं पुरा} %2-29

\twolineshloka
{ततोऽहमाशया राम तव सम्बन्धकाङ्क्षया}
{अकार्षं गर्हितमपि तवाचार्यत्वसिद्धये} %2-30

\twolineshloka
{ततो मनोरथो मेऽद्य फलितो रघुनन्दन}
{त्वदधीना महामाया सर्वलोकैकमोहिनी} %2-31

\twolineshloka
{मां यथा मोहयेन्नैव तथा कुरु रघूद्वह}
{गुरुनिष्कृतिकामस्त्वं यदि देह्येतदेव मे} %2-32

\twolineshloka
{प्रसङ्गात्सर्वमप्युक्तं न वाच्यं कुत्रचिन्मया}
{राज्ञा दशरथेनाहं प्रेषितोऽस्मि रघूद्वह} %2-33

\twolineshloka
{त्वामामन्त्रयितुं राज्ये श्वोऽभिषेक्ष्यति राघव}
{अद्य त्वं सीतया सार्धमुपवासं यथाविधि} %2-34

\twolineshloka
{कृत्वा शुचिर्भूमिशायी भव राम जितेन्द्रियः}
{गच्छामि राजसान्निध्यं त्वं तु प्रातर्गमिष्यसि} %2-35

\twolineshloka
{इत्युक्त्वा रथमारुह्य ययौ राजगुरुर्द्रुतम्}
{रामोऽपि लक्ष्मणं दृष्ट्वा प्रहसन्निदमब्रवीत्} %2-36

\twolineshloka
{सौमित्रे यौवराज्ये मे श्वोऽभिषेको भविष्यति}
{निमित्तमात्रमेवाहं कर्ता भोक्ता त्वमेव हि} %2-37

\twolineshloka
{मम त्वं हि बहिःप्राणो नात्र कार्या विचारणा}
{ततो वसिष्ठेन यथा भाषितं तत्तथाऽकरोत्} %2-38

\twolineshloka
{वसिष्ठोऽपि नृपं गत्वा कृतं सर्वं न्यवेदयत्}
{वसिष्ठस्य पुरो राज्ञा ह्युक्तं रामाभिषेचनम्} %2-39

\twolineshloka
{यदा तदैव नगरे श्रुत्वा कश्चित्पुमान् जगौ}
{कौसल्यायै राममात्रे सुमित्रायै तथैव च} %2-40

\twolineshloka
{श्रुत्वा ते हर्षसम्पूर्णे ददतुर्हारमुत्तमम्}
{तस्मै ततः प्रीतमनाः कौसल्या पुत्रवत्सला} %2-41

\twolineshloka
{लक्ष्मीं पर्यचरद्देवीं रामस्यार्थप्रसिद्धये}
{सत्यवादी दशरथः करोत्येव प्रतिश्रुतम्} %2-42

\twolineshloka
{कैकेयीवशगः किन्तु कामुकः किं करिष्यति}
{इति व्याकुलचित्ता सा दुर्गां देवीमपूजयत्} %2-43

\twolineshloka
{एतस्मिन्नन्तरे देवा देवीं वाणीमचोदयन्}
{गच्छ देवि भुवो लोकमयोध्यायां प्रयत्नतः} %2-44

\twolineshloka
{रामाभिषेकविघ्नार्थं यतस्व ब्रह्मवाक्यतः}
{मन्थरां प्रविशस्वादौ कैकेयीं च ततः परम्} %2-45

\twolineshloka
{ततो विघ्ने समुत्पन्ने पुनरेहि दिवं शुभे}
{तथेत्युक्त्वा तथा चक्रे प्रविवेशाथ मन्थराम्} %2-46

\twolineshloka
{साऽपि कुब्जा त्रिवक्रा तु प्रासादाग्रमथारुहत्}
{नगरं परितो दृष्ट्वा सर्वतः समलङ्कृतम्} %2-47

\twolineshloka
{नानातोरणसम्बाधं पताकाभिरलङ्कृतम्}
{दानोत्सवसमायुक्ता कौसल्या चातिहर्षिता} %2-48

\twolineshloka
{धात्रीं पप्रच्छ मातः किं नगरं समलङ्कृतम्}
{दानोत्सवसमायुक्ता कौसल्या चातिहर्षिता} %2-49

\twolineshloka
{ददाति विप्रमुख्येभ्यो वस्त्राणि विविधानि च}
{तामुवाच तदा धात्री रामचन्द्राभिषेचनम्} %2-50

\twolineshloka
{श्वो भविष्यति तेनाद्य सर्वतोऽलङ्कृतं पुरम्}
{तच्छ्रुत्वा त्वरितं गत्वा कैकेयीं वाक्यमब्रवीत्} %2-51

\twolineshloka
{पर्यङ्कस्थां विशालाक्षीमेकान्ते पर्यवस्थिताम्}
{किं शेषे दुर्भगे मूढे महद्भयमुपस्थितम्} %2-52

{न जानीषेऽतिसौन्दर्यमानिनी मत्तगामिनी॥५३॥} %2-53
\refstepcounter{shlokacount}


\twolineshloka
{रामस्यानुग्रहाद्राज्ञः श्वोऽभिषेको भविष्यति}
{तच्छ्रुत्वा सहसोत्थाय कैकेयी प्रियवादिनी} %2-54

\twolineshloka
{तस्यै दिव्यं ददौ स्वर्णनूपुरं रत्नभूषितम्}
{हर्षस्थाने किमिति मे कथ्यते भयमागतम्} %2-55

\twolineshloka
{भरतादधिको रामः प्रियकृन्मे प्रियंवदः}
{कौसल्यां मां समं पश्यन् सदा शुश्रूषते हि माम्} %2-56

\twolineshloka
{रामाद्भयं किमापन्नं तव मूढे वदस्व मे}
{तच्छ्रुत्वा विषसादाथ कुब्जाऽकारणवैरिणी} %2-57

\twolineshloka
{शृणु मद्वचनं देवि यथार्थं ते महद्भयम्}
{त्वां तोषयन् सदा राजा प्रियवाक्यानि भाषते} %2-58

\twolineshloka
{कामुकोऽतथ्यवादी च त्वां वाचा परितोषयन्}
{कार्यं करोति तस्या वै राममातुः सुपुष्कलम्} %2-59

\twolineshloka
{मनस्येतन्निधायैव प्रेषयामास ते सुतम्}
{भरतं मातुलकुले प्रेषयामास सानुजम्} %2-60

\twolineshloka
{सुमित्रायाः समीचीनं भविष्यति न संशयः}
{लक्ष्मणो राममन्वेति राज्यं सोऽनुभविष्यति} %2-61

\twolineshloka
{भरतो राघवस्याग्रे किङ्करो वा भविष्यति}
{विवास्यते वा नगरात्प्राणैर्वा हायतेऽचिरात्} %2-62

\twolineshloka
{त्वं तु दासीव कौसल्यां नित्यं परिचरिष्यसि}
{ततोऽपि मरणं श्रेयो यत्सपत्न्याः पराभवः} %2-63

\twolineshloka
{अतः शीघ्रं यतस्वाद्य भरतस्याभिषेचने}
{रामस्य वनवासार्थं वर्षाणि नव पञ्च च} %2-64

\twolineshloka
{ततो रूढोऽभये पुत्रस्तव राज्ञि भविष्यति}
{उपायं ते प्रवक्ष्यामि पूर्वमेव सुनिश्चितम्} %2-65

\twolineshloka
{पुरा देवासुरे युद्धे राजा दशरथः स्वयम्}
{इन्द्रेण याचितो धन्वी सहायार्थं महारथः} %2-66

\twolineshloka
{जगाम सेनया सार्धं त्वया सह शुभानने}
{युद्धं प्रकुर्वतस्तस्य राक्षसैः सह धन्विनः} %2-67

\twolineshloka
{तदाऽक्षकीलो न्यपतच्छिन्नस्तस्य न वेद सः}
{त्वं तु हस्तं समावेश्य कीलरन्ध्रेऽतिधैर्यतः} %2-68

\twolineshloka
{स्थितवत्यसितापाङ्गि पतिप्राणपरीप्सया}
{ततो हत्वाऽसुरान् सर्वान् ददर्श त्वामरिन्दमः} %2-69

\twolineshloka
{आश्चर्यं परमं लेभे त्वामालिङ्ग्य मुदान्वितः}
{वृणीष्व यत्ते मनसि वाञ्छितं वरदोऽस्म्यहम्} %2-70

\twolineshloka
{वरद्वयं वृणीष्व त्वमेवं राजावदत्स्वयम्}
{त्वयोक्तो वरदो राजन् यदि दत्तं वरद्वयम्} %2-71

\twolineshloka
{त्वय्येव तिष्ठतु चिरं न्यासभूतं ममानघ}
{यदा मेऽवसरो भूयात्तदा देहि वरद्वयम्} %2-72

\twolineshloka
{तथेत्युक्त्वा स्वयं राजा मन्दिरं व्रज सुव्रते}
{त्वत्तः श्रुतं मया पूर्वमिदानीं स्मृतिमागतम्} %2-73

\twolineshloka
{अतः शीघ्रं प्रविश्याद्य क्रोधागारं रुषान्विता}
{विमुच्य सर्वाभरणं सर्वतो विनिकीर्य च} %2-74

\twolineshloka
{भूमावेव शयाना त्वं तूष्णीमातिष्ठ भामिनि}
{यावत्सत्यं प्रतिज्ञाय राजाभीष्टं करोति ते} %2-75

\twolineshloka
{श्रुत्वा त्रिवक्रयोक्तं तत्तदा केकयनन्दिनी}
{तथ्यमेवाखिलं मेने दुःसङ्गाहितविभ्रमा} %2-76

\twolineshloka
{तामाह कैकेयी दुष्टा कुतस्ते बुद्धिरीदृशी}
{एवं त्वां बुद्धिसम्पन्नां न जाने वक्रसुन्दरि} %2-77

\twolineshloka
{भरतो यदि राजा मे भविष्यति सुतः प्रियः}
{ग्रामान् शतं प्रदास्यामि मम त्वं प्राणवल्लभा} %2-78

\twolineshloka
{इत्युक्त्वा कोपभवनं प्रविश्य सहसा रुषा}
{विमुच्य सर्वाभरणं परिकीर्य समन्ततः} %2-79

\twolineshloka
{भूमौ शयाना मलिना मलिनाम्बरधारिणी}
{प्रोवाच शृणु मे कुब्जे यावद्रामो वनं व्रजेत्} %2-80

\twolineshloka
{प्राणान्स्त्यक्ष्येऽथ वा वक्रे शयिष्ये तावदेव हि}
{निश्चयं कुरु कल्याणि कल्याणं ते भविष्यति} %2-81

{इत्युक्त्वा प्रययौ कुब्जा गृहं साऽपि तथाऽकरोत्॥८२॥} %2-82
\refstepcounter{shlokacount}


\fourlineindentedshloka
{धीरोऽत्यन्तदयान्वितोऽपि सगुणाचारान्वितो वाथवा}
{नीतिज्ञो विधिवाददेशिकपरो विद्याविवेकोऽथवा}
{दुष्टानामतिपापभावितधियां सङ्गं सदा चेद्भजेत्}
{तद्बुद्ध्या परिभावितो व्रजति तत्साम्यं क्रमेण} %2-83

\twolineshloka
{अतः सङ्गः परित्याज्यो दुष्टानां सर्वदैव हि}
{दुःसङ्गी च्यवते स्वार्थाद्यथेयं राजकन्यका} %2-84

{॥इति श्रीमदध्यात्मरामायणे उमामहेश्वरसंवादे
अयोध्याकाण्डे द्वितीयः सर्गः॥२॥
}
%%%%%%%%%%%%%%%%%%%%



\sect{तृतीयः सर्गः}

\twolineshloka
{ततो दशरथो राजा रामाभ्युदयकारणात्}
{आदिश्य मन्त्रिप्रकृतीः सानन्दो गृहमाविशत्} %3-1

\twolineshloka
{तत्रादृष्ट्वा प्रियां राजा किमेतदिति विह्वलः}
{या पुरा मन्दिरं तस्याः प्रविष्टे मयि शोभना} %3-2

\twolineshloka
{हसन्ती मामुपायाति सा किं नैवाद्य दृश्यते}
{इत्यात्मन्येव सञ्चिन्त्य मनसातिविदूयता} %3-3

\twolineshloka
{पप्रच्छ दासीनिकरं कुतो वः स्वामिनी शुभा}
{नायाति मां यथापूर्वं मत्प्रिया प्रियदर्शना} %3-4

\twolineshloka
{ता ऊचुः क्रोधभवनं प्रविष्टा नैव विद्महे}
{कारणं तत्र देव त्वं गच्छ निश्चेतुमर्हसि} %3-5

\twolineshloka
{इत्युक्तो भयसन्त्रस्तो राजा तस्याः समीपगः}
{उपविश्य शनैर्देहं स्पृशन्वै पाणिनाब्रवीत्} %3-6

\twolineshloka
{किं शेषे वसुधापृष्ठे पर्यङ्कादीन् विहाय च}
{मां त्वं खेदयसे भीरु यतो मां नावभाषसे} %3-7

\twolineshloka
{अलङ्कारं परित्यज्य भूमौ मलिनवाससा}
{किमर्थं ब्रूहि सकलं विधास्ये तव वाञ्छितम्} %3-8

\twolineshloka
{को वा तवाहितं कर्ता नारी वा पुरुषोऽपि वा}
{स मे दण्ड्यश्च वध्यश्च भविष्यति न संशयः} %3-9

\twolineshloka
{ब्रूहि देवि यथा प्रीतिस्तदवश्यं ममाग्रतः}
{तदिदानीं साधयिष्ये सुदुर्लभमपि क्षणात्} %3-10

\twolineshloka
{जानासि त्वं मम स्वान्तं प्रियं मां स्ववशे स्थितम्}
{तथाऽपि मां खेदयसे वृथा तव परिश्रमः} %3-11

\twolineshloka
{ब्रूहि किं धनिनं कुर्यां दरिद्रं ते प्रियङ्करम्}
{धनिनं क्षणमात्रेण निर्धनं च तवाहितम्} %3-12

\twolineshloka
{ब्रूहि कं वा वधिष्यामि वधार्हो वा विमोक्ष्यते}
{किमत्र बहुनोक्तेन प्राणान् दास्यामि ते प्रिये} %3-13

\twolineshloka
{मम प्राणात्प्रियतरो रामो राजीवलोचनः}
{तस्योपरि शपे ब्रूहि त्वद्धितं तत्करोम्यहम्} %3-14

\twolineshloka
{इति ब्रुवाणं राजानं शपन्तं राघवोपरि}
{शनैर्विमृज्य नेत्रे सा राजानं प्रत्यभाषत} %3-15

\twolineshloka
{यदि सत्यप्रतिज्ञोऽसि शपथं कुरुषे यदि}
{याच्ञां मे सफलां कर्तुं शीघ्रमेव त्वमर्हसि} %3-16

\twolineshloka
{पूर्वं देवासुरे युद्धे मया त्वं परिरक्षितः}
{तदा वरद्वयं दत्तं त्वया मे तुष्टचेतसा} %3-17

\twolineshloka
{तद्द्वयं न्यासभूतं मे स्थापितं त्वयि सुव्रत}
{तत्रैकेन वरेणाशु भरतं मे प्रियं सुतम्} %3-18

\twolineshloka
{एभिः सम्भृतसम्भारैर्यौवराज्येऽभिषेचय}
{अपरेण वरेणाशु रामो गच्छतु दण्डकान्} %3-19

\twolineshloka
{मुनिवेषधरः श्रीमान् जटावल्कलभूषणः}
{चतुर्दश समास्तत्र कन्दमूलफलाशनः} %3-20

\twolineshloka
{पुनरायातु तस्यान्ते वने वा तिष्ठतु स्वयम्}
{प्रभाते गच्छतु वनं रामो राजीवलोचनः} %3-21

\twolineshloka
{यदि किञ्चिद्विलम्बेत प्राणान्स्त्यक्ष्ये तवाग्रतः}
{भव सत्यप्रतिज्ञस्त्वमेतदेव मम प्रियम्} %3-22

\twolineshloka
{श्रुत्वैतद्दारुणं वाक्यं कैकेय्या रोमहर्षणम्}
{निपपात महीपालो वज्राहत इवाचलः} %3-23

\twolineshloka
{शनैरुन्मील्य नयने विमृज्य परया भिया}
{दुःस्वप्नो वा मया दृष्टो ह्यथवा चित्तविभ्रमः} %3-24

\twolineshloka
{इत्यालोक्य पुरः पत्नीं व्याघ्रीमिव पुरः स्थिताम्}
{किमिदं भाषसे भद्रे मम प्राणहरं वचः} %3-25

\twolineshloka
{रामः कमपराधं ते कृतवान् कमलेक्षणः}
{ममाग्रे राघवगुणान् वर्णयस्यनिशं शुभान्} %3-26

\twolineshloka
{कौसल्यां मां समं पश्यन् शुश्रूषा कुरुते सदा}
{इति ब्रुवन्ती त्वं पूर्वमिदानीं भाषसेऽन्यथा} %3-27

\twolineshloka
{राज्यं गृहाण पुत्राय रामस्तिष्ठतु मन्दिरे}
{अनुगृह्णीष्व मां वामे रामान्नास्ति भयं तव} %3-28

\twolineshloka
{इत्युक्त्वाऽश्रुपरीताक्षः पादयोर्निपपात ह}
{कैकेयी प्रत्युवाचेदं साऽपि रक्तान्तलोचना} %3-29

\twolineshloka
{राजेन्द्र किं त्वं भ्रान्तोऽसि उक्तं तद्भाषसेऽन्यथा}
{मिथ्या करोषि चेत्स्वीयं भाषितं नरको भवेत्} %3-30

\fourlineindentedshloka
{वनं न गच्छेद्यदि रामचन्द्रः}
{प्रभातकालेऽजिनचीरयुक्तः}
{उद्बन्धनं वा विषभक्षणं वा}
{कृत्वा मरिष्ये पुरतस्तवाहम्} %3-31

\fourlineindentedshloka
{सत्यप्रतिज्ञोऽहमितीह लोके}
{विडम्बसे सर्वसभान्तरेषु}
{रामोपरि त्वं शपथं च कृत्वा}
{मिथ्याप्रतिज्ञो नरकं प्रयाहि} %3-32

\twolineshloka
{इत्युक्तः प्रियया दीनो मग्नो दुःखार्णवे नृपः}
{मूर्च्छितः पतितो भूमौ विसंज्ञो मृतको यथा} %3-33

\twolineshloka
{एवं रात्रिर्गता तस्य दुःखात्संवत्सरोपमा}
{अरुणोदयकाले तु वन्दिनो गायका जगुः} %3-34

\twolineshloka
{निवारयित्वा तान् सर्वान् कैकेयी रोषमास्थिता}
{ततः प्रभातसमये मध्यकक्षमुपस्थिताः} %3-35

\twolineshloka
{ब्राह्मणाः क्षत्रिया वैश्या ऋषयः कन्यकास्तथा}
{छत्रं च चामरं दिव्यं गजो वाजी तथैव च} %3-36

\twolineshloka
{अन्याश्च वारमुख्या याः पौरजानपदास्तथा}
{वसिष्ठेन यथाज्ञप्तं तत्सर्वं तत्र संस्थितम्} %3-37

\twolineshloka
{स्त्रियो बालाश्च वृद्धाश्च रात्रौ निद्रां न लेभिरे}
{कदा द्रक्ष्यामहे रामं पीतकौशेयवाससम्} %3-38

\twolineshloka
{सर्वाभरणसम्पन्नं किरीटकटकोज्ज्वलम्}
{कौस्तुभाभरणं श्यामं कन्दर्पशतसुन्दरम्} %3-39

\twolineshloka
{अभिषिक्तं समायातं गजारूढं स्मिताननम्}
{श्वेतच्छत्रधरं तत्र लक्ष्मणं लक्षणान्वितम्} %3-40

\twolineshloka
{रामं कदा वा द्रक्ष्यामः प्रभातं वा कदा भवेत्}
{इत्युत्सुकधियः सर्वे बभूवुः पुरवासिनः} %3-41

\twolineshloka
{नेदानीमुत्थितो राजा किमर्थं चेति चिन्तयन्}
{सुमन्त्रः शनकैः प्रायाद्यत्र राजाऽवतिष्ठते} %3-42

\twolineshloka
{वर्धयन् जयशब्देन प्रणमन् शिरसा नृपम्}
{अतिखिन्नं नृपं दृष्ट्वा कैकेयीं समपृच्छत} %3-43

\twolineshloka
{देवि कैकेयि वर्धस्व किं राजा दृश्यतेऽन्यथा}
{तमाह कैकेयी राजा रात्रौ निद्रां न लब्धवान्} %3-44

\threelineshloka
{राम रामेति रामेति राममेवानुचिन्तयन्}
{प्रजागरेण वै राजा ह्यस्वस्थ इव लक्ष्यते}
{राममानय शीघ्रं त्वं राजा द्रष्टुमिहेच्छति} %3-45

\twolineshloka
{अश्रुत्वा राजवचनं कथं गच्छामि भामिनि}
{तच्छ्रुत्वा मन्त्रिणो वाक्यं राजा मन्त्रिणमब्रवीत्} %3-46

\twolineshloka
{सुमन्त्र रामं द्रक्ष्यामि शीघ्रमानय सुन्दरम्}
{इत्युक्तस्त्वरितं गत्वा सुमन्त्रो राममन्दिरम्} %3-47

\twolineshloka
{अवारितः प्रविष्टोऽयं त्वरितं राममब्रवीत्}
{शीघ्रमागच्छ भद्रं ते राम राजीवलोचन} %3-48

\twolineshloka
{पितुर्गेहं मया सार्धं राजा त्वां द्रष्टुमिच्छति}
{इत्युक्तो रथमारुह्य सम्भ्रमात्त्वरितो ययौ} %3-49

\twolineshloka
{रामः सारथिना सार्धं लक्ष्मणेन समन्वितः}
{मध्यकक्षे वसिष्ठादीन् पश्यन्नेव त्वरान्वितः} %3-50

\twolineshloka
{पितुः समीपं सङ्गम्य ननाम चरणौ पितुः}
{राममालिङ्गितुं राजा समुत्थाय ससम्भ्रमः} %3-51

\twolineshloka
{बाहू प्रसार्य रामेति दुःखान्मध्ये पपात ह}
{हा हेति रामस्तं शीघ्रमालिङ्ग्याङ्के न्यवेशयत्} %3-52

\twolineshloka
{राजानं मूर्च्छितं दृष्ट्वा चुक्रुशुः सर्वयोषितः}
{किमर्थं रोदनमिति वसिष्ठोऽपि समाविशत्} %3-53

\twolineshloka
{रामः पप्रच्छ किमिदं राज्ञो दुःखस्य कारणम्}
{एवं पृच्छति रामे सा कैकेयी राममब्रवीत्} %3-54

\twolineshloka
{त्वमेव कारणं ह्यत्र राज्ञो दुःखोपशान्तये}
{किञ्चित्कार्यं त्वया राम कर्तव्यं नृपतेर्हितम्} %3-55

\twolineshloka
{कुरु सत्यप्रतिज्ञस्त्वं राजानं सत्यवादिनम्}
{राज्ञा वरद्वयं दत्तं मम सन्तुष्टचेतसा} %3-56

\twolineshloka
{त्वदधीनं तु तत्सर्वं वक्तुं त्वां लज्जते नृपः}
{सत्यपाशेन सम्बद्धं पितरं त्रातुमर्हसि} %3-57

\twolineshloka
{पुत्रशब्देन चैतद्धि नरकात्त्रायते पिता}
{रामस्तयोदितं श्रुत्वा शूलेनाभिहतो यथा} %3-58

\twolineshloka
{व्यथितः कैकेयीं प्राह किं मामेवं प्रभाषसे}
{पित्रर्थे जीवितं दास्ये पिबेयं विषमुल्बणम्} %3-59

\twolineshloka
{सीतां त्यक्ष्येऽथ कौसल्यां राज्यं चापि त्यजाम्यहम्}
{अनाज्ञप्तोऽपि कुरुते पितुः कार्यं स उत्तमः} %3-60

\twolineshloka
{उक्तः करोति यः पुत्रः स मध्यम उदाहृतः}
{उक्तोऽपि कुरुते नैव स पुत्रो मल उच्यते} %3-61

\twolineshloka
{अतः करोमि तत्सर्वं यन्मामाह पिता मम}
{सत्यं सत्यं करोम्येव रामो द्विर्नाभिभाषते} %3-62

\twolineshloka
{इति रामप्रतिज्ञां सा श्रुत्वा वक्तुं प्रचक्रमे}
{राम त्वदभिषेकार्थं सम्भाराः सम्भृताश्च ये} %3-63

\twolineshloka
{तैरेव भरतोऽवश्यमभिषेच्यः प्रियो मम}
{अपरेण वरेणाशु चीरवासा जटाधरः} %3-64

\twolineshloka
{वनं प्रयाहि शीघ्रं त्वमद्यैव पितुराज्ञया}
{चतुर्दश समास्तत्र वस मुन्यन्नभोजनः} %3-65

\twolineshloka
{एतदेव पितुस्तेऽद्य कार्यं त्वं कर्तुमर्हसि}
{राजा तु लज्जते वक्तुं त्वामेवं रघुनन्दन} %3-66

\textbf{श्रीराम उवाच}

\twolineshloka
{भरतस्यैव राज्यं स्यादहं गच्छामि दण्डकान्}
{किन्तु राजा न वक्तीह मां न जानेऽत्र कारणम्} %3-67

\twolineshloka
{श्रुत्वैतद्रामवचनं दृष्ट्वा रामं पुरः स्थितम्}
{प्राह राजा दशरथो दुःखितो दुःखितं वचः} %3-68

\twolineshloka
{स्त्रीजितं भ्रान्तहृदयमुन्मार्गपरिवर्तिनम्}
{निगृह्य मां गृहाणेदं राज्यं पापं न तद्भवेत्} %3-69

\twolineshloka
{एवं चेदनृतं नैव मां स्पृशेद्रघुनन्दन}
{इत्युक्त्वा दुःखसन्तप्तो विललाप नृपस्तदा} %3-70

\twolineshloka
{हा राम हा जगन्नाथ हा मम प्राणवल्लभ}
{मां विसृज्य कथं घोरं विपिनं गन्तुमर्हसि} %3-71

\twolineshloka
{इति रामं समालिङ्ग्य मुक्तकण्ठो रुरोद ह}
{विमृज्य नयने रामः पितुः सजलपाणिना} %3-72

\twolineshloka
{आश्वासयामास नृपं शनैः स नयकोविदः}
{किमत्र दुःखेन विभो राज्यं शासतु मेऽनुजः} %3-73

\twolineshloka
{अहं प्रतिज्ञां निस्तीर्य पुनर्यास्यामि ते पुरम्}
{राज्यात्कोटिगुणं सौख्यं मम राजन् वने सतः} %3-74

\twolineshloka
{त्वत्सत्यपालनं देव कार्यं चापि भविष्यति}
{कैकेय्याश्च प्रियो राजन् वनवासो महागुणः} %3-75

\twolineshloka
{इदानीं गन्तुमिच्छामि व्येतु मातुश्च हृज्ज्वरः}
{सम्भारश्चोपह्रीयन्तामभिषेकार्थमाहृताः} %3-76

\twolineshloka
{मातरं समनुश्वास्य अनुनीय च जानकीम्}
{आगत्य पादौ वन्दित्वा तव यास्ये सुखं वनम्} %3-77

\twolineshloka
{इत्युक्त्वा तु परिक्रम्य मातरं द्रष्टुमाययौ}
{कौसल्याऽपि हरेः पूजां कुरुते रामकारणात्} %3-78

\twolineshloka
{होमं च कारयामास ब्राह्मणेभ्यो ददौ धनम्}
{ध्यायते विष्णुमेकाग्रमनसा मौनमास्थिता} %3-79

\fourlineindentedshloka
{अन्तःस्थमेकं घनचित्प्रकाशम्}
{निरस्तसर्वातिशयस्वरूपम्}
{विष्णुं सदानन्दमयं हृदब्जे}
{सा भावयन्ती न ददर्श रामम्} %3-80

{॥इति श्रीमदध्यात्मरामायणे उमामहेश्वरसंवादे
अयोध्याकाण्डे तृतीयः सर्गः॥३॥
}
%%%%%%%%%%%%%%%%%%%%



\sect{चतुर्थः सर्गः}

\twolineshloka
{ततः सुमित्रा दृष्ट्वैनं रामं राज्ञीं ससम्भ्रमा}
{कौसल्यां बोधयामास रामोऽयं समुपस्थितः} %4-1

\twolineshloka
{श्रुत्वैव रामनामैषा बहिर्दृष्टिप्रवाहिता}
{रामं दृष्ट्वा विशालाक्षमालिङ्ग्याङ्के न्यवेशयत्} %4-2

\twolineshloka
{मूर्ध्न्यवघ्राय पस्पर्श गात्रं नीलोत्पलच्छवि}
{भुङ्क्ष्व पुत्रेति च प्राह मिष्टमन्नं क्षुधार्दितः} %4-3

\twolineshloka
{रामः प्राह न मे मातर्भोजनावसरः कुतः}
{दण्डकागमने शीघ्रं मम कालोऽद्य निश्चितः} %4-4

\twolineshloka
{कैकेयीवरदानेन सत्यसन्धः पिता मम}
{भरताय ददौ राज्यं ममाप्यारण्यमुत्तमम्} %4-5

\twolineshloka
{चतुर्दश समास्तत्र ह्युषित्वा मुनिवेषधृक्}
{आगमिष्ये पुनः शीघ्रं न चिन्तां कर्तुमर्हसि} %4-6

\twolineshloka
{तच्छ्रुत्वा सहसोद्विग्ना मूर्च्छिता पुनरुत्थिता}
{आह रामं सुदुःखार्ता दुःखसागरसम्प्लुता} %4-7

\twolineshloka
{यदि राम वनं सत्यं यासि चेन्नय मामपि}
{त्वद्विहीना क्षणार्धं वा जीवितं धारये कथम्} %4-8

\twolineshloka
{यथा गौर्बालकं वत्सं त्यक्त्वा तिष्ठेन्न कुत्रचित्}
{तथैव त्वां न शक्नोमि त्यक्तुं प्राणात्प्रियं सुतम्} %4-9

\twolineshloka
{भरताय प्रसन्नश्चेद्राज्यं राजा प्रयच्छतु}
{किमर्थं वनवासाय त्वामाज्ञापयति प्रियम्} %4-10

\twolineshloka
{कैकेय्या वरदो राजा सर्वस्वं वा प्रयच्छतु}
{त्वया किमपराद्धं हि कैकेय्या वा नृपस्य वा} %4-11

\twolineshloka
{पिता गुरुर्यथा राम तवाहमधिका ततः}
{पित्राऽऽज्ञप्तो वनं गन्तुं वारयेयमहं सुतम्} %4-12

\twolineshloka
{यदि गच्छसि मद्वाक्यमुल्लङ्घ्य नृपवाक्यतः}
{तदा प्राणान् परित्यज्य गच्छामि यमसादनम्} %4-13

\twolineshloka
{लक्ष्मणोऽपि ततः श्रुत्वा कौसल्यावचनं रुषा}
{उवाच राघवं वीक्ष्य दहन्निव जगत्त्रयम्} %4-14

\twolineshloka
{उन्मत्तं भ्रान्तमनसं कैकेयीवशवर्तिनम्}
{बद्ध्वा निहन्मि भरतं तद्बन्धून्मातुलानपि} %4-15

\twolineshloka
{अद्य पश्यन्तु मे शौर्यं लोकान् प्रदहतः पुरा}
{राम त्वमभिषेकाय कुरु यत्नमरिन्दम} %4-16

\twolineshloka
{धनुष्पाणिरहं तत्र निहन्यां विघ्नकारिणः}
{इति ब्रुवन्तं सौमित्रिमालिङ्ग्य रघुनन्दनः} %4-17

\twolineshloka
{शूरोऽसि रघुशार्दूल ममात्यन्तहिते रतः}
{जानामि सर्वं ते सत्यं किन्तु तत्समयो न हि} %4-18

\twolineshloka
{यदिदं दृश्यते सर्वं राज्यं देहादिकं च यत्}
{यदि सत्यं भवेत्तत्र आयासः सफलश्च ते} %4-19

\twolineshloka
{भोगा मेघवितानस्थविद्युल्लेखेव चञ्चलाः}
{आयुरप्यग्निसन्तप्तलोहस्थजलबिन्दुवत्} %4-20

\twolineshloka
{यथा व्यालगलस्थोऽपि भेको दंशानपेक्षते}
{तथा कालाहिना ग्रस्तो लोको भोगानशाश्वतान्} %4-21

\fourlineindentedshloka
{करोति दुःखेन हि कर्मतन्त्रम्}
{शरीरभोगार्थमहर्निशं नरः}
{देहस्तु भिन्नः पुरुषात्समीक्ष्यते}
{को वात्र भोगः पुरुषेण भुज्यते} %4-22

\twolineshloka
{पितृमातृसुतभ्रातृदारबन्ध्वादिसङ्गमः}
{प्रपायामिव जन्तूनां नद्यां काष्ठौघवच्चलः} %4-23

\fourlineindentedshloka
{छायेव लक्ष्मीश्चपला प्रतीता}
{तारुण्यमम्बूर्मिवदध्रुवं च}
{स्वप्नोपमं स्त्रीसुखमायुरल्पम्}
{तथाऽपि जन्तोरभिमान एषः} %4-24

\twolineshloka
{संसृतिः स्वप्नसदृशी सदा रोगादिसङ्कुला}
{गन्धर्वनगरप्रख्या मूढस्तामनुवर्तते} %4-25

\twolineshloka
{आयुष्यं क्षीयते यस्मादादित्यस्य गतागतैः}
{दृष्ट्वाऽन्येषां जरामृत्यू कथञ्चिन्नैव बुध्यते} %4-26

\twolineshloka
{स एव दिवसः सैव रात्रिरित्येव मूढधीः}
{भोगाननुपतत्येव कालवेगं न पश्यति} %4-27

\twolineshloka
{प्रतिक्षणं क्षरत्येतदायुरामघटाम्बुवत्}
{सपत्ना इव रोगौघाः शरीरं प्रहरन्त्यहो} %4-28

\twolineshloka
{जरा व्याघ्रीव पुरतस्तर्जयन्त्यवतिष्ठते}
{मृत्युः सहैव यात्येष समयं सम्प्रतीक्षते} %4-29

\twolineshloka
{देहेऽहम्भावमापन्नो राजाहं लोकविश्रुतः}
{इत्यस्मिन्मनुते जन्तुः कृमिविड्भस्मसंज्ञिते} %4-30

\twolineshloka
{त्वगस्थिमान्सविण्मूत्ररेतोरक्तादिसंयुतः}
{विकारी परिणामी च देह आत्मा कथं वद} %4-31

\twolineshloka
{यमास्थाय भवान्ल्लोकं दग्धुमिच्छति लक्ष्मण}
{देहाभिमानिनः सर्वे दोषाः प्रादुर्भवन्ति हि} %4-32

\twolineshloka
{देहोऽहमिति या बुद्धिरविद्या सा प्रकीर्तिता}
{नाहं देहश्चिदात्मेति बुद्धिर्विद्येति भण्यते} %4-33

\threelineshloka
{अविद्या संसृतेर्हेतुर्विद्या तस्या निवर्तिका}
{तस्माद्यत्नः सदा कार्यो विद्याभ्यासे मुमुक्षुभिः}
{कामक्रोधादयस्तत्र शत्रवः शत्रुसूदन} %4-34

\twolineshloka
{तत्रापि क्रोध एवालं मोक्षविघ्नाय सर्वदा}
{येनाविष्टः पुमान् हन्ति पितृभ्रातृसुहृत्सखीन्} %4-35

\twolineshloka
{क्रोधमूलो मनस्तापः क्रोधः संसारबन्धनम्}
{धर्मक्षयकरः क्रोधस्तस्मात्क्रोधं परित्यज} %4-36

\twolineshloka
{क्रोध एष महान् शत्रुस्तृष्णा वैतरणी नदी}
{सन्तोषो नन्दनवनं शान्तिरेव हि कामधुक्} %4-37

\twolineshloka
{तस्माच्छान्तिं भजस्वाद्य शत्रुरेवं भवेन्न ते}
{देहेन्द्रियमनःप्राणबुद्ध्यादिभ्यो विलक्षणः} %4-38

\twolineshloka
{आत्मा शुद्धः स्वयञ्ज्योतिरविकारी निराकृतिः}
{यावद्देहेन्द्रियप्राणैर्भिन्नत्वं नात्मनो विदुः} %4-39

\twolineshloka
{तावत्संसारदुःखौघैः पीड्यन्ते मृत्युसंयुताः}
{तस्मात्त्वं सर्वदा भिन्नमात्मानं हृदि भावय} %4-40

\twolineshloka
{बुद्ध्यादिभ्यो बहिः सर्वमनुवर्तस्व मा खिदः}
{भुञ्जन् प्रारब्धमखिलं सुखं वा दुःखमेव वा} %4-41

\twolineshloka
{प्रवाहपतितं कार्यं कुर्वन्नपि न लिप्यसे}
{बाह्ये सर्वत्र कर्तृत्वमावहन्नपि राघव} %4-42

\twolineshloka
{अन्तःशुद्धस्वभावस्त्वं लिप्यसे न च कर्मभिः}
{एतन्मयोदितं कृत्स्नं हृदि भावय सर्वदा} %4-43

\twolineshloka
{संसारदुःखैरखिलैर्बाध्यसे न कदाचन}
{त्वमप्यम्ब ममाऽऽदिष्टं हृदि भावय नित्यदा} %4-44

\twolineshloka
{समागमं प्रतीक्षस्व न दुःखैः पीड्यसे चिरम्}
{न सदैकत्र संवासः कर्ममार्गानुवर्तिनाम्} %4-45

\twolineshloka
{यथा प्रवाहपतितप्लवानां सरितां तथा}
{चतुर्दशसमा सङ्ख्या क्षणार्द्धमिव जायते} %4-46

\twolineshloka
{अनुमन्यस्व मामम्ब दुःखं सन्त्यज्य दूरतः}
{एवं चेत्सुखसंवासो भविष्यति वने मम} %4-47

\twolineshloka
{इत्युक्त्वा दण्डवन्मातुः पादयोरपतच्चिरम्}
{उत्थाप्याङ्के समावेश्य आशीर्भिरभ्यनन्दयत्} %4-48

\twolineshloka
{सर्वे देवाः सगन्धर्वा ब्रह्मविष्णुशिवादयः}
{रक्षन्तु त्वां सदा यान्तं तिष्ठन्तं निद्रया युतम्} %4-49

\twolineshloka
{इति प्रस्थापयामास समालिङ्ग्य पुनः पुनः}
{लक्ष्मणोऽपि तदा रामं नत्वा हर्षाश्रुगद्गदः} %4-50

\twolineshloka
{आह राम ममान्तःस्थः संशयोऽयं त्वया हृतः}
{यास्यामि पृष्ठतो राम सेवां कर्तुं तदादिश} %4-51

\twolineshloka
{अनुगृह्णीष्व मां राम नोचेत्प्राणान्स्त्यजाम्यहम्}
{तथेति राघवोऽप्याह लक्ष्मणं याहि मा चिरम्} %4-52

\twolineshloka
{प्रतस्थे तां समाधातुं गतः सीतापतिर्विभुः}
{आगतं पतिमालोक्य सीता सुस्मितभाषिणी} %4-53

\twolineshloka
{स्वर्णपात्रस्थसलिलैः पादौ प्रक्षाल्य भक्तितः}
{पप्रच्छ पतिमालोक्य देव किं सेनया विना} %4-54

\twolineshloka
{आगतोऽसि गतः कुत्र श्वेतच्छत्रं च ते कुतः}
{वादित्राणि न वाद्यन्ते किरीटादिविवर्जितः} %4-55

\twolineshloka
{सामन्तराजसहितः सम्भ्रमान्नागतोऽसि किम्}
{इति स्म सीतया पृष्टो रामः सस्मितमब्रवीत्} %4-56

\twolineshloka
{राज्ञा मे दण्डकारण्ये राज्यं दत्तं शुभेऽखिलम्}
{अतस्तत्पालनार्थाय शीघ्रं यास्यामि भामिनि} %4-57

\twolineshloka
{अद्यैव यास्यामि वनं त्वं तु श्वश्रूसमीपगा}
{शुश्रूषां कुरु मे मातुर्न मिथ्यावादिनो वयम्} %4-58

\twolineshloka
{इति ब्रुवन्तं श्रीरामं सीता भीताब्रवीद्वचः}
{किमर्थं वनराज्यं ते पित्रा दत्तं महात्मना} %4-59

\twolineshloka
{तामाह रामः कैकेय्यै राजा प्रीतो वरं ददौ}
{भरताय ददौ राज्यं वनवासं ममानघे} %4-60

\twolineshloka
{चतुर्दश समास्तत्र वासो मे किल याचितः}
{तया देव्या ददौ राजा सत्यवादी दयापरः} %4-61

\twolineshloka
{अतः शीघ्रं गमिष्यामि मा विघ्नं कुरु भामिनि}
{श्रुत्वा तद्रामवचनं जानकी प्रीतिसंयुता} %4-62

\twolineshloka
{अहमग्रे गमिष्यामि वनं पश्चात्त्वमेष्यसि}
{इत्याह मां विना गन्तुं तव राघव नोचितम्} %4-63

\twolineshloka
{तामाह राघवः प्रीतः स्वप्रियां प्रियवादिनीम्}
{कथं वनं त्वां नेष्येऽहं बहुव्याघ्रमृगाकुलम्} %4-64

\twolineshloka
{राक्षसा घोररूपाश्च सन्ति मानुषभोजिनः}
{सिंहव्याघ्रवराहाश्च सञ्चरन्ति समन्ततः} %4-65

\twolineshloka
{कट्वम्लफलमूलानि भोजनार्थं सुमध्यमे}
{अपूपान्नव्यञ्जनानि विद्यन्ते न कदाचन} %4-66

\twolineshloka
{काले काले फलं वाऽपि विद्यते कुत्र सुन्दरि}
{मार्गो न दृश्यते क्वापि शर्कराकण्टकान्वितः} %4-67

\twolineshloka
{गुहागह्वरसम्बाधं झिल्लीदंशादिभिर्युतम्}
{एवं बहुविधं दोषं वनं दण्डकसंज्ञितम्} %4-68

\twolineshloka
{पादचारेण गन्तव्यं शीतवातातपादिमत्}
{राक्षसादीन् वने दृष्ट्वा जीवितं हास्यसेऽचिरात्} %4-69

\twolineshloka
{तस्माद्भद्रे गृहे तिष्ठ शीघ्रं द्रक्ष्यसि मां पुनः}
{रामस्य वचनं श्रुत्वा सीता दुःखसमन्विता} %4-70

\twolineshloka
{प्रत्युवाच स्फुरद्वक्त्रा किञ्चित्कोपसमन्विता}
{कथं मामिच्छसे त्यक्तुं धर्मपत्नीं पतिव्रताम्} %4-71

\twolineshloka
{त्वदनन्यामदोषां मां धर्मज्ञोऽसि दयापरः}
{त्वत्समीपे स्थितां राम को वा मां धर्षयेद्वने} %4-72

\twolineshloka
{फलमूलादिकं यद्यत्तव भुक्तावशेषितम्}
{तदेवामृततुल्यं मे तेन तुष्टा रमाम्यहम्} %4-73

\twolineshloka
{त्वया सह चरन्त्या मे कुशाः काशाश्च कण्टकाः}
{पुष्पास्तरणतुल्या मे भविष्यन्ति न संशयः} %4-74

\twolineshloka
{अहं त्वां क्लेशये नैव भवेयं कार्यसाधिनी}
{बाल्ये मां वीक्ष्य कश्चिन्मां ज्योतिःशास्त्रविशारदः} %4-75

\twolineshloka
{प्राह ते विपिने वासः पत्या सह भविष्यति}
{सत्यवादी द्विजो भूयाद्गमिष्यामि त्वया सह} %4-76

\twolineshloka
{अन्यत्किञ्चित्प्रवक्ष्यामि श्रुत्वा मां नय काननम्}
{रामायणानि बहुशः श्रुतानि बहुभिर्द्विजैः} %4-77

\twolineshloka
{सीतां विना वनं रामो गतः किं कुत्रचिद्वद}
{अतस्त्वया गमिष्यामि सर्वथा त्वत्सहायिनी} %4-78

\twolineshloka
{यदि गच्छसि मां त्यक्त्वा प्राणान्स्त्यक्ष्यामि तेऽग्रतः}
{इति तं निश्चयं ज्ञात्वा सीताया रघुनन्दनः} %4-79

\twolineshloka
{अब्रवीद्देवि गच्छ त्वं वनं शीघ्रं मया सह}
{अरुन्धत्यै प्रयच्छाशु हारानाभरणानि च} %4-80

\twolineshloka
{ब्राह्मणेभ्यो धनं सर्वं दत्त्वा गच्छामहे वनम्}
{इत्युक्त्वा लक्ष्मणेनाशु द्विजानाहूय भक्तितः} %4-81

\fourlineindentedshloka
{ददौ गवां वृन्दशतं धनानि}
{वस्त्राणि दिव्यानि विभूषणानि}
{कुटुम्बवद्भ्यः श्रुतशीलवद्भ्यो}
{मुदा द्विजेभ्यो रघुवंशकेतुः} %4-82

\twolineshloka
{अरुन्धत्यै ददौ सीता मुख्यान्याभरणानि च}
{रामो मातुः सेवकेभ्यो ददौ धनमनेकधा} %4-83

\twolineshloka
{स्वकान्तःपुरवासिभ्यः सेवकेभ्यस्तथैव च}
{पौरजानपदेभ्यश्च ब्राह्मणेभ्यः सहस्रशः} %4-84

\twolineshloka
{लक्ष्मणोऽपि सुमित्रां तु कौसल्यायै समर्पयत्}
{धनुष्पाणिः समागत्य रामस्याग्रे व्यवस्थितः} %4-85

{रामः सीता लक्ष्मणश्च जग्मुः सर्वे नृपालयम्॥८६॥} %4-86
\refstepcounter{shlokacount}

{श्रीरामः सह सीतया नृपपथे गच्छन् शनैः सानुजः}
{पौरान् जानपदान् कुतूहलदृशः सानन्दमुद्वीक्षयन्}
{श्यामः कामसहस्रसुन्दरवपुः कान्त्या दिशो भासयन्}
{पादन्यासपवित्रिताऽखिलजगत् प्रापालयं तत्पितुः} %4-87

{॥इति श्रीमदध्यात्मरामायणे उमामहेश्वरसंवादे
अयोध्याकाण्डे चतुर्थः सर्गः॥४॥
}
%%%%%%%%%%%%%%%%%%%%



\sect{पञ्चमः सर्गः}

\twolineshloka
{आयान्तं नागरा दृष्ट्वा मार्गे रामं सजानकीम्}
{लक्ष्मणेन समं वीक्ष्य ऊचुः सर्वे परस्परम्} %5-1

\twolineshloka
{कैकेय्या वरदानादि श्रुत्वा दुःखसमावृताः}
{बत राजा दशरथः सत्यसन्धं प्रियं सुतम्} %5-2

\twolineshloka
{स्त्रीहेतोरत्यजत्कामी तस्य सत्यवता कुतः}
{कैकेयी वा कथं दुष्टा रामं सत्यं प्रियङ्करम्} %5-3

\twolineshloka
{विवासयामास कथं क्रूरकर्माऽतिमूढधीः}
{हे जना नात्र वस्तव्यं गच्छामोऽद्यैव काननम्} %5-4

\twolineshloka
{यत्र रामः सभार्यश्च सानुजो गन्तुमिच्छति}
{पश्यन्तु जानकीं सर्वे पादचारेण गच्छतीम्} %5-5

\twolineshloka
{पुम्भिः कदाचिद्दृष्ट्वा वा जानकी लोकसुन्दरी}
{साऽपि पादेन गच्छन्ती जनसङ्घेष्वनावृता} %5-6

\twolineshloka
{रामोऽपि पादचारेण गजाश्वादिविवर्जितः}
{गच्छति द्रक्ष्यथ विभुं सर्वलोकैकसुन्दरम्} %5-7

\twolineshloka
{राक्षसी कैकेयीनाम्नी जाता सर्वविनाशिनी}
{रामस्यापि भवेद्दुःखं सीतायाः पादयानतः} %5-8

\twolineshloka
{बलवान् विधिरेवात्र पुम्प्रयत्नो हि दुर्बलः}
{इति दुःखाकुले वृन्दे साधूनां मुनिपुङ्गवः} %5-9

\twolineshloka
{अब्रवीद्वामदेवोऽथ साधूनां सङ्घमध्यगः}
{मानुशोचथ रामं वा सीतां वा वच्मि तत्त्वतः} %5-10

\twolineshloka
{एष रमः परो विष्णुरादिनारायणः स्मृतः}
{एषा सा जानकी लक्ष्मीर्योगमायेति विश्रुता} %5-11

\twolineshloka
{असौ शेषस्तमन्वेति लक्ष्मणाख्यश्च साम्प्रतम्}
{एष मायागुणैर्युक्तस्तत्तदाकारवानिव} %5-12

\twolineshloka
{एष एव रजोयुक्तो ब्रह्माभूद्विश्वभावनः}
{सत्त्वाविष्टस्तथा विष्णुस्त्रिजगत्प्रतिपालकः} %5-13

\twolineshloka
{एष रुद्रस्तामसोऽन्ते जगत्प्रलयकारणम्}
{एष मत्स्यः पुरा भूत्वा भक्तं वैवस्वतं मनुम्} %5-14

\twolineshloka
{नाव्यारोप्य लयस्यान्ते पालयामास राघवः}
{समुद्रमथने पूर्वं मन्दरे सुतलं गते} %5-15

\twolineshloka
{अधारयत्स्वपृष्ठेऽद्रिं कूर्मरूपी रघूत्तमः}
{मही रसातलं याता प्रलये सूकरोऽभवत्} %5-16

\twolineshloka
{तोलयामास दंष्ट्राग्रे तां क्षोणीं रघुनन्दनः}
{नारसिंहं वपुः कृत्वा प्रह्लादवरदः पुरा} %5-17

\twolineshloka
{त्रैलोक्यकण्टकं रक्षः पाटयामास तन्नखैः}
{पुत्रराज्यं हृतं दृष्ट्वा ह्यदित्या याचितः पुरा} %5-18

\twolineshloka
{वामनत्वमुपागम्य याच्ञया चाहरत्पुनः}
{दुष्टक्षत्रियभूभारनिवृत्त्यै भार्गवोऽभवत्} %5-19

\twolineshloka
{स एव जगतां नाथ इदानीं रामतां गतः}
{रावणादीनि रक्षांसि कोटिशो निहनिष्यति} %5-20

\twolineshloka
{मानुषेणैव मरणं तस्य दृष्टं दुरात्मनः}
{राज्ञा दशरथेनापि तपसाराधितो हरिः} %5-21

\twolineshloka
{पुत्रत्वाकाङ्क्षया विष्णोस्तदा पुत्रोऽभवद्धरिः}
{स एव विष्णुः श्रीरामो रावणादिवधाय हि} %5-22

\twolineshloka
{गन्ताद्यैव वनं रामो लक्ष्मणेन सहायवान्}
{एषा सीता हरेर्माया सृष्टिस्थित्यन्तकारिणी} %5-23

\twolineshloka
{राजा वा कैकेयी वाऽपि नात्र कारणमण्वपि}
{पूर्वेद्युर्नारदः प्राह भूभारहरणाय च} %5-24

\twolineshloka
{रामोऽप्याह स्वयं साक्षाच्छ्वो गमिष्याम्यहं वनम्}
{अतो रामं समुद्दिश्य चिन्तां त्यजत बालिशाः} %5-25

\twolineshloka
{रामरामेति ये नित्यं जपन्ति मनुजा भुवि}
{तेषां मृत्युभयादीनि न भवन्ति कदाचन} %5-26

\twolineshloka
{का पुनस्तस्य रामस्य दुःखशङ्का महात्मनः}
{रामनाम्नैव मुक्तिः स्यात्कलौ नान्येन केनचित्} %5-27

\twolineshloka
{मायामानुषरूपेण विडम्बयति लोककृत्}
{भक्तानां भजनार्थाय रावणस्य वधाय च} %5-28

\twolineshloka
{राज्ञश्चाभीष्टसिद्ध्यर्थं मानुषं वपुराश्रितः}
{इत्युक्त्वा विररामाथ वामदेवो माहामुनिः} %5-29

\twolineshloka
{श्रुत्वा तेऽपि द्विजाः सर्वे रामं ज्ञात्वा हरिं विभुम्}
{जहुर्हृत्संशयग्रन्थिं राममेवान्वचिन्तयन्} %5-30

\twolineshloka
{य इदं चिन्तयेन्नित्यं रहस्यं रामसीतयोः}
{तस्य रामे दृढा भक्तिर्भवेद्विज्ञानपूर्विका} %5-31

\twolineshloka
{रहस्यं गोपनीयं वो यूयं वै राघवप्रियाः}
{इत्युक्त्वा प्रययौ विप्रस्तेऽपि रामं परं विदुः} %5-32

\twolineshloka
{ततो रामः समाविश्य पितृगेहमवारितः}
{सानुजः सीतया गत्वा कैकेयीमिदमब्रवीत्} %5-33

\twolineshloka
{आगताः स्मो वयं मातस्त्रयस्ते सम्मतं वनम्}
{गन्तुं कृतधियः शीघ्रमाज्ञापयतु नः पिता} %5-34

\twolineshloka
{इत्युक्ता सहसोत्थाय चीराणि प्रददौ स्वयम्}
{रामाय लक्ष्मणायाथ सीतायै च पृथक् पृथक्} %5-35

\twolineshloka
{रामस्तु वस्त्राण्युत्सृज्य वन्यचीराणि पर्यधात्}
{लक्ष्मणोऽपि तथा चक्रे सीता तन्न विजानती} %5-36

\twolineshloka
{हस्ते गृहीत्वा रामस्य लज्जया मुखमैक्षत}
{रामो गृहीत्वा तच्चीरमंशुके पर्यवेष्टयत्} %5-37

\twolineshloka
{तद् दृष्ट्वा रुरुदुः सर्वे राजदाराः समन्ततः}
{वसिष्ठस्तु तदाकर्ण्य रुदितं भर्त्सयन् रुषा} %5-38

\twolineshloka
{कैकेयीं प्राह दुर्वृत्ते राम एव त्वया वृतः}
{वनवासाय दुष्टे त्वं सीतायै किं प्रयच्छसि} %5-39

\twolineshloka
{यदि रामं समन्वेति सीता भक्त्या पतिव्रता}
{दिव्याम्बरधरा नित्यं सर्वाभरणभूषिता} %5-40

\twolineshloka
{रमयत्वनिशं रामं वनदुःखनिवारिणी}
{राजा दशरथोऽप्याह सुमन्त्रं रथमानय} %5-41

\twolineshloka
{रथमारुह्य गच्छन्तु वनं वनचरप्रियाः}
{इत्युक्त्वा राममालोक्य सीतां चैव सलक्ष्मणम्} %5-42

\twolineshloka
{दुःखान्निपतितो भूमौ रुरोदाश्रुपरिप्लुतः}
{आरुरोह रथं सीता शीघ्रं रामस्य पश्यतः} %5-43

\twolineshloka
{रामः प्रदक्षिणं कृत्वा पितरं रथमारुहत्}
{लक्ष्मणः खड्गयुगलं धनुस्तूणीयुगं तथा} %5-44

\twolineshloka
{गृहीत्वा रथमारुह्य नोदयामास सारथिम्}
{तिष्ठ तिष्ठ सुमन्त्रेति राजा दशरथोऽब्रवीत्} %5-45

\twolineshloka
{गच्छ गच्छेति रामेण नोदितोऽचोदयद्रथम्}
{रामे दूरं गते राजा मूर्च्छितः प्रापतद्भुवि} %5-46

\twolineshloka
{पौरास्तु बालवृद्धाश्च वृद्धा ब्राह्मणसत्तमाः}
{तिष्ठ तिष्ठेति रामेति क्रोशन्तो रथमन्वयुः} %5-47

\twolineshloka
{राजा रुदित्वा सुचिरं मां नयन्तु गृहं प्रति}
{कौसल्याया राममातुरित्याह परिचारकान्} %5-48

\twolineshloka
{किञ्चित्कालं भवेत्तत्र जीवनं दुःखितस्य मे}
{अत ऊर्ध्वं न जीवामि चिरं रामं विना कृतः} %5-49

\twolineshloka
{ततो गृहं प्रविश्यैव कौसल्यायाः पपात ह}
{मूर्च्छितश्च चिराद्बुद्ध्वा तूष्णीमेवावतस्थिवान्} %5-50

\twolineshloka
{रामस्तु तमसातीरं गत्वा तत्रावसत्सुखी}
{जलं प्राश्य निराहारो वृक्षमूलेऽस्वपद्विभुः} %5-51

\twolineshloka
{सीतया सह धर्मात्मा धनुष्पाणिस्तु लक्ष्मणः}
{पालयामास धर्मज्ञः सुमन्त्रेण समन्वितः} %5-52

\twolineshloka
{पौराः सर्वे समागत्य स्थितास्तस्याविदूरतः}
{शक्ता रामं पुरं नेतुं नो चेद्गच्छामहे वनम्} %5-53

\twolineshloka
{इति निश्चयमाज्ञाय तेषां रामोऽतिविस्मितः}
{नाहं गच्छामि नगरमेते वै क्लेशभागिनः} %5-54

\twolineshloka
{भविष्यन्तीति निश्चित्य सुमन्त्रमिदमब्रवीत्}
{इदानीमेव गच्छामः सुमन्त्र रथमानय} %5-55

\twolineshloka
{इत्याज्ञप्तः सुमन्त्रोऽपि रथं वाहैरयोजयत्}
{आरुह्य रामः सीता च लक्ष्मणोऽपि ययुर्द्रुतम्} %5-56

\twolineshloka
{अयोध्याभिमुखं गत्वा किञ्चिद्दूरं ततो ययुः}
{तेऽपि राममदृष्ट्वैव प्रातरुत्थाय दुःखिताः} %5-57

\twolineshloka
{रथनेमिगतं मार्गं पश्यन्तस्ते पुरं ययुः}
{हृदि रामं ससीतं ते ध्यायन्तस्तस्थुरन्वहम्} %5-58

\twolineshloka
{सुमन्त्रोऽपि रथं शीघ्रं नोदयामास सादरम्}
{स्फीतान् जनपदान् पश्यन् रामः सीतासमन्वितः} %5-59

\twolineshloka
{गङ्गातीरं समागच्छच्छृङ्गवेराविदूरतः}
{गङ्गां दृष्ट्वा नमस्कृत्य स्नात्वा सानन्दमानसः} %5-60

\twolineshloka
{शिंशपावृक्षमूले स निषसाद रघूत्तमः}
{ततो गुहो जनैः श्रुत्वा रामागममहोत्सवम्} %5-61

\twolineshloka
{सखायं स्वामिनं द्रष्टुं हर्षात्तूर्णं समापतत्}
{फलानि मधुपुष्पादि गृहीत्वा भक्तिसंयुतः} %5-62

\twolineshloka
{रामस्याग्रे विनिक्षिप्य दण्डवत्प्रापतद्भुवि}
{गुहमुत्थाप्य तं तूर्णं राघवः परिषस्वजे} %5-63

\twolineshloka
{सम्पृष्टकुशलो रामं गुहः प्राञ्जलिरब्रवीत्}
{धन्योऽहमद्य मे जन्म नैषादं लोकपावन} %5-64

\twolineshloka
{बभूव परमानन्दः स्पृष्ट्वा तेऽङ्गं रघूत्तम}
{नैषादराज्यमेतत्ते किङ्करस्य रघूत्तम} %5-65

\twolineshloka
{त्वदधीनं वसन्नत्र पालयास्मान् रघूद्वह}
{आगच्छ यामो नगरं पावनं कुरु मे गृहम्} %5-66

\twolineshloka
{गृहाण फलमूलानि त्वदर्थं सञ्चितानि मे}
{अनुगृह्णीष्व भगवन् दासस्तेऽहं सुरोत्तम} %5-67

\twolineshloka
{रामस्तमाह सुप्रीतो वचनं शृणु मे सखे}
{न वेक्ष्यामि गृहं ग्रामं नव वर्षाणि पञ्च च} %5-68

\twolineshloka
{दत्तमन्येन नो भुञ्जे फलमूलादि किञ्चन}
{राज्यं ममैतत्ते सर्वं त्वं सखा मेऽतिवल्लभः} %5-69

\twolineshloka
{वटक्षीरं समानाय्य जटामुकुटमादरात्}
{बबन्ध लक्ष्मणेनाथ सहितो रघुनन्दनः} %5-70

\twolineshloka
{जलमात्रं तु सम्प्राश्य सीतया सह राघवः}
{आस्तृतं कुशपर्णाद्यैः शयनं लक्ष्मणेन हि} %5-71

\twolineshloka
{उवास तत्र नगरप्रासादाग्रे यथा पुरा}
{सुष्वाप तत्र वैदेह्या पर्यङ्क इव संस्कृते} %5-72

\fourlineindentedshloka
{ततोऽविदूरे परिगृह्य चापम्}
{सबाणतूणीरधनुः स लक्ष्मणः}
{ररक्ष रामं परितो विपश्यनः}
{गुहेन सार्धं सशरासनेन} %5-73

{॥इति श्रीमदध्यात्मरामायणे उमामहेश्वरसंवादे
अयोध्याकाण्डे पञ्चमः सर्गः॥५॥
}
%%%%%%%%%%%%%%%%%%%%



\sect{षष्ठः सर्गः}

\twolineshloka
{सुप्तं रामं समालोक्य गुहः सोऽश्रुपरिप्लुतः}
{लक्ष्मणं प्राह विनयाद् भ्रातः पश्यसि राघवम्} %6-1

\twolineshloka
{शयानं कुशपत्रौघसंस्तरे सीतया सह}
{यः शेते स्वर्णपर्यङ्के स्वास्तीर्णे भवनोत्तमे} %6-2

\twolineshloka
{कैकेयी रामदुःखस्य कारणं विधिना कृता}
{मन्थराबुद्धिमास्थाय कैकेयी पापमाचरत्} %6-3

\twolineshloka
{तच्छ्रुत्वा लक्ष्मणः प्राह सखे शृणु वचो मम}
{कः कस्य हेतुर्दुःखस्य कश्च हेतुः सुखस्य च} %6-4

{स्वपूर्वार्जितकर्मैव कारणं सुखदुःखयोः॥५॥} %6-5
\refstepcounter{shlokacount}


\fourlineindentedshloka
{सुखस्य दुःखस्य न कोऽपि दाता}
{परो ददातीति कुबुद्धिरेषा}
{अहं करोमीति वृथाभिमानः}
{स्वकर्मसूत्रग्रथितो हि लोकः} %6-6

\twolineshloka
{सुहृन्मित्रार्युदासीनद्वेष्यमध्यस्थबान्धवाः}
{स्वयमेवाचरन् कर्म तथा तत्र विभाव्यते} %6-7

\twolineshloka
{सुखं वा यदि वा दुःखं स्वकर्मवशगो नरः}
{यद्यद्यथागतं तत्तद् भुक्त्वा स्वस्थमना भवेत्} %6-8

\twolineshloka
{न मे भोगागमे वाञ्छा न मे भोगविवर्जने}
{आगच्छत्वथ मागच्छत्वभोगवशगो भवेत्} %6-9

\twolineshloka
{स्वस्मिन् देशे च काले च यस्माद्वा येन केन वा}
{कृतं शुभाशुभं कर्म भोज्यं तत्तत्र नान्यथा} %6-10

\twolineshloka
{अलं हर्षविषादाभ्यां शुभाशुभफलोदये}
{विधात्रा विहितं यद्यत्तदलङ्घ्यं सुरासुरैः} %6-11

\twolineshloka
{सर्वदा सुखदुःखाभ्यां नरः प्रत्यवरुध्यते}
{शरीरं पुण्यपापाभ्यामुत्पन्नं सुखदुःखवत्} %6-12

\twolineshloka
{सुखस्यानन्तरं दुःखं दुःखस्यानन्तरं सुखम्}
{द्वयमेतद्धि जन्तूनामलङ्घ्यं दिनरात्रिवत्} %6-13

\twolineshloka
{सुखमध्ये स्थितं दुःखं दुःखमध्ये स्थितं सुखम्}
{द्वयमन्योन्यसंयुक्तं प्रोच्यते जलपङ्कवत्} %6-14

\twolineshloka
{तस्माद्धैर्येण विद्वांस इष्टानिष्टोपपत्तिषु}
{न हृष्यन्ति न मुह्यन्ति समं मायेति भावनात्} %6-15

\twolineshloka
{गुहलक्ष्मणयोरेवं भाषतोर्विमलं नभः}
{बभूव रामः सलिलं स्पृष्ट्वा प्रातः समाहितः} %6-16

\twolineshloka
{उवाच शीघ्रं सुदृढं नावमानय मे सखे}
{श्रुत्वा रामस्य वचनं निषादाधिपतिर्गुहः} %6-17

\twolineshloka
{स्वयमेव दृढं नावमानिनाय सुलक्षणाम्}
{स्वामिन्नारुह्यतां नौकां सीतया लक्ष्मणेन च} %6-18

\twolineshloka
{वाहये ज्ञातिभिः सार्धमहमेव समाहितः}
{तथेति राघवः सीतामारोप्य शुभलक्षणाम्} %6-19

\twolineshloka
{गुहस्य हस्तावालम्ब्य स्वयं चारोहदच्युतः}
{आयुधादीन् समारोप्य लक्ष्मणोऽप्यारुरोह च} %6-20

\twolineshloka
{गुहस्तान् वाहयामास ज्ञातिभिः सहितः स्वयम्}
{गङ्गामध्ये गतां गङ्गां प्रार्थयामास जानकी} %6-21

\twolineshloka
{देवि गङ्गे नमस्तुभ्यं निवृत्ता वनवासतः}
{रामेण सहिताहं त्वां लक्ष्मणेन च पूजये} %6-22

\twolineshloka
{सुरामान्सोपहारैश्च नानाबलिभिरादृता}
{इत्युक्त्वा परकूलं तौ शनैरुत्तीर्य जग्मतुः} %6-23

\twolineshloka
{गुहोऽपि राघवं प्राह गमिष्यामि त्वया सह}
{अनुज्ञां देहि राजेन्द्र नो चेत्प्राणान्स्त्यजाम्यहम्} %6-24

\twolineshloka
{श्रुत्वा नैषादिवचनं श्रीरामस्तमथाब्रवीत्}
{चतुर्दश समाः स्थित्वा दण्डके पुनरप्यहम्} %6-25

\twolineshloka
{आयास्याम्युदितं सत्यं नासत्यं रामभाषितम्}
{इत्युक्त्वाऽऽलिङ्ग्य तं भक्तं समाश्वास्य पुनः पुनः} %6-26

{निवर्तयामास गुहं सोऽपि कृच्छ्राद्ययौ गृहम्॥२७॥} %6-27
\refstepcounter{shlokacount}


\twolineshloka
{तत्र मेध्यं मृगं हत्वा पक्त्वा हुत्वा च ते त्रयः}
{भुक्त्वा वृक्षतले सुप्त्वा सुखमासत तां निशाम्} %6-28

\threelineshloka
{ततो रामस्तु वैदेह्या लक्ष्मणेन समन्वितः}
{भरद्वाजाश्रमपदं गत्वा बहिरुपस्थितः}
{तत्रैकं वटुकं दृष्ट्वा रामः प्राह च हे वटो} %6-29

\twolineshloka
{रामो दाशरथिः सीतालक्ष्मणाभ्यां समन्वितः}
{आस्ते बहिर्वनस्येति ह्युच्यतां मुनिसन्निधौ} %6-30

\twolineshloka
{तच्छ्रुत्वा सहसा गत्वा पादयोः पतितो मुनेः}
{स्वामिन् रामः समागत्य वनाद्बहिरवस्थितः} %6-31

\twolineshloka
{सभार्यः सानुजः श्रीमानाह मां देवसन्निभः}
{भरद्वाजाय मुनये ज्ञापयस्व यथोचितम्} %6-32

\twolineshloka
{तच्छ्रुत्वा सहसोत्थाय भरद्वाजो मुनीश्वरः}
{गृहीत्वाऽर्घ्यं च पाद्यं च रामसामीप्यमाययौ} %6-33

\twolineshloka
{दृष्ट्वा रामं यथान्यायं पूजयित्वा सलक्ष्मणम्}
{आह मे पर्णशालां भो राम राजीवलोचन} %6-34

\twolineshloka
{आगच्छ पादरजसा पुनीहि रघुनन्दन}
{इत्युक्त्वोटजमानीय सीतया सह रघावौ} %6-35

\twolineshloka
{भक्त्या पुनः पूजयित्वा चकारातिथ्यमुत्तमम्}
{अद्याहं तपसः पारं गतोऽस्मि तव सङ्गमात्} %6-36

\twolineshloka
{ज्ञातं राम तवोदन्तं भूतं चागामिकं च यत्}
{जानामि त्वां परात्मानं मायया कार्यमानुषम्} %6-37

\twolineshloka
{यदर्थमवतीर्णोऽसि प्रार्थितो ब्रह्मणा पुरा}
{यदर्थं वनवासस्ते यत्करिष्यसि वै पुरः} %6-38

\twolineshloka
{जानामि ज्ञानदृष्ट्याहं जातया त्वदुपासनात्}
{इतः परं त्वां किं वक्ष्ये कृतार्थोऽहं रघूत्तम} %6-39

\twolineshloka
{यस्त्वां पश्यामि काकुत्स्थं पुरुषं प्रकृतेः परम्}
{रामस्तमभिवाद्याह सीतालक्ष्मणसंयुतः} %6-40

\twolineshloka
{अनुग्राह्यास्त्वया ब्रह्मन्वयं क्षत्रियबान्धवाः}
{इति सम्भाष्य तेऽन्योन्यमुषित्वा मुनिसन्निधौ} %6-41

\twolineshloka
{प्रातरुत्थाय यमुनामुत्तीर्य मुनिवारकैः}
{कृताप्लवेन मुनिना दृष्टमार्गेण राघवः} %6-42

\twolineshloka
{प्रययौ चित्रकूटाद्रिं वाल्मीकेर्यत्र चाश्रमः}
{गत्वा रामोऽथ वाल्मीकेराश्रमं ऋषिसङ्कुलम्} %6-43

\twolineshloka
{नानामृगद्विजाकीर्णं नित्यपुष्पफलाकुलम्}
{तत्र दृष्ट्वा समासीनं वाल्मीकिं मुनिसत्तमम्} %6-44

\twolineshloka
{ननाम शिरसा रामो लक्ष्मणेन च सीतया}
{दृष्ट्वा रामं रमानाथं वाल्मीकिर्लोकसुन्दरम्} %6-45

\twolineshloka
{जानकीलक्ष्मणोपेतं जटामुकुटमण्डितम्}
{कन्दर्पसदृशाकारं कमनीयाम्बुजेक्षणम्} %6-46

\twolineshloka
{दृष्ट्वैव सहसोत्तस्थौ विस्मयानिमिषेक्षणः}
{आलिङ्ग्य परमानन्दं रामं हर्षाश्रुलोचनः} %6-47

\twolineshloka
{पूजयित्वा जगत्पूज्यं भक्त्यार्घ्यादिभिरादृतः}
{फलमूलैः स मधुरैर्भोजयित्वा च लालितः} %6-48

\twolineshloka
{राघवः प्राञ्जलिः प्राह वाल्मीकिं विनयान्वितः}
{पितुराज्ञां पुरस्कृत्य दण्डकानागता वयम्} %6-49

\twolineshloka
{भवन्तो यदि जानन्ति किं वक्ष्यामोऽत्र कारणम्}
{यत्र मे सुखवासाय भवेत्स्थानं वदस्व तत्} %6-50

\twolineshloka
{सीतया सहितः कालं किञ्चित्तत्र नयाम्यहम्}
{इत्युक्तो राघवेणासौ मुनिः सस्मितमब्रवीत्} %6-51

\twolineshloka
{त्वमेव सर्वलोकानां निवासस्थानमुत्तमम्}
{तवापि सर्वभूतानि निवाससदनानि हि} %6-52

\threelineshloka
{एवं साधारणं स्थानमुक्तं ते रघुनन्दन}
{सीतया सहितस्येति विशेषं पृच्छतस्तव}
{तद्वक्ष्यामि रघुश्रेष्ठ यत्ते नियतमन्दिरम्} %6-53

\twolineshloka
{शान्तानां समदृष्टीनामद्वेष्टॄणां च जन्तुषु}
{त्वामेव भजतां नित्यं हृदयं तेऽधिमन्दिरम्} %6-54

\twolineshloka
{धर्माधर्मान् परित्यज्य त्वामेव भजतोऽनिशम्}
{सीतया सह ते राम तस्य हृत्सुखमन्दिरम्} %6-55

\twolineshloka
{त्वन्मन्त्रजापको यस्तु त्वामेव शरणं गतः}
{निर्द्वन्द्वो निःस्पृहस्तस्य हृदयं ते सुमन्दिरम्} %6-56

\twolineshloka
{निरहङ्कारिणः शान्ता ये रागद्वेषवर्जिताः}
{समलोष्टाश्मकनकास्तेषां ते हृदयं गृहम्} %6-57

\twolineshloka
{त्वयि दत्तमनोबुद्धिर्यः सन्तुष्टः सदा भवेत्}
{त्वयि सन्त्यक्तकर्मा यस्तन्मनस्ते शुभं गृहम्} %6-58

\twolineshloka
{यो न द्वेष्ट्यप्रियं प्राप्य प्रियं प्राप्य न हृष्यति}
{सर्वं मायेति निश्चित्य त्वां भजेत्तन्मनो गृहम्} %6-59

\twolineshloka
{षड्भावादिविकारान् यो देहे पश्यति नात्मनि}
{क्षुत्तृट् सुखं भयं दुःखं प्राणबुद्ध्योर्निरीक्षते} %6-60

{संसारधर्मैर्निर्मुक्तस्तस्य ते मानसं गृहम्॥६१॥} %6-61
\refstepcounter{shlokacount}


\fourlineindentedshloka
{पश्यन्ति ये सर्वगुहाशयस्थम्}
{त्वां चिद्घनं सत्यमनन्तमेकम्}
{अलेपकं सर्वगतं वरेण्यम्}
{तेषां हृदब्जे सह सीतया वस} %6-62

\fourlineindentedshloka
{निरन्तराभ्यासदृढीकृतात्मनाम्}
{त्वत्पादसेवापरिनिष्ठितानाम्}
{त्वन्नामकीर्त्या हतकल्मषाणाम्}
{सीतासमेतस्य गृहं हृदब्जे} %6-63

\twolineshloka
{राम त्वन्नाममहिमा वर्ण्यते केन वा कथम्}
{यत्प्रभावादहं राम ब्रह्मर्षित्वमवाप्तवान्} %6-64

\twolineshloka
{अहं पुरा किरातेषु किरातैः सह वर्धितः}
{जन्ममात्रद्विजत्वं मे शूद्राचाररतः सदा} %6-65

\twolineshloka
{शूद्रायां बहवः पुत्रा उत्पन्ना मेऽजितात्मनः}
{ततश्चोरैश्च सङ्गम्य चौरोऽहमभवं पुरा} %6-66

\twolineshloka
{धनुर्बाणधरो नित्यं जीवानामन्तकोपमः}
{एकदा मुनयः सप्त दृष्टा महति कानने} %6-67

\twolineshloka
{साक्षान्मया प्रकाशन्तो ज्वलनार्कसमप्रभाः}
{तानन्वधावं लोभेन तेषां सर्वपरिच्छदान्} %6-68

\twolineshloka
{ग्रहीतुकामस्तत्राहं तिष्ठ तिष्ठेति चाब्रवम्}
{दृष्ट्वा मां मुनयोऽपृच्छन् किमायासि द्विजाधम} %6-69

\twolineshloka
{अहं तानब्रवं किञ्चिदादातुं मुनिसत्तमाः}
{पुत्रदारादयः सन्ति बहवो मे बुभुक्षिताः} %6-70

\twolineshloka
{तेषां संरक्षणार्थाय चरामि गिरिकानने}
{ततो मामूचुरव्यग्राः पृच्छ गत्वा कुटुम्बकम्} %6-71

\twolineshloka
{यो यो मया प्रतिदिनं क्रियते पापसञ्चयः}
{यूयं तद्भागिनः किं वा नेति वेतिपृथक्पृथक्} %6-72

\twolineshloka
{वयं स्थास्यामहे तावदागमिष्यसि निश्चयः}
{तथेत्युक्त्वा गृहं गत्वा मुनिभिर्यदुदीरितम्} %6-73

\twolineshloka
{अपृच्छं पुत्रदारादीन्स्तैरुक्तोऽहं रघूत्तम}
{पापं तवैव तत्सर्वं वयं तु फलभागिनः} %6-74

\twolineshloka
{तच्छ्रुत्वा जातनिर्वेदो विचार्य पुनरागमम्}
{मुनयो यत्र तिष्ठन्ति करुणापूर्णमानसाः} %6-75

\twolineshloka
{मुनीनां दर्शनादेव शुद्धान्तःकरणोऽभवम्}
{धनुरादीन् परित्यज्य दण्डवत्पतितोऽस्म्यहम्} %6-76

\twolineshloka
{रक्षध्वं मां मुनिश्रेष्ठा गच्छन्तं निरयार्णवम्}
{इत्यग्रे पतितं दृष्ट्वा मामूचुर्मुनिसत्तमाः} %6-77

\twolineshloka
{उत्तिष्ठोत्तिष्ठ भद्रं ते सफलः सत्समागमः}
{उपदेक्ष्यामहे तुभ्यं किञ्चित्तेनैव मोक्ष्यसे} %6-78

\threelineshloka
{परस्परं समालोच्य दुर्वृत्तोयं द्विजाधमः}
{उपेक्ष्य एव सद्वृत्तैस्तथाऽपि शरणं गतः}
{रक्षणीयः प्रयत्नेन मोक्षमार्गोपदेशतः} %6-79

\twolineshloka
{इत्युक्त्वा राम ते नाम व्यत्यस्ताक्षरपूर्वकम्}
{एकाग्रमनसात्रैव मरेति जप सर्वदा} %6-80

\twolineshloka
{आगच्छामः पुनर्यावत्तावदुक्तं सदा जप}
{इत्युक्त्वा प्रययुः सर्वे मुनयो दिव्यदर्शनाः} %6-81

\twolineshloka
{अहं यथोपदिष्टं तैस्तथाऽकरवमञ्जसा}
{जपन्नेकाग्रमनसा बाह्यं विस्मृतवानहम्} %6-82

\twolineshloka
{एवं बहुतिथे काले गते निश्चलरूपिणः}
{सर्वसङ्गविहीनस्य वल्मीकोऽभून्ममोपरि} %6-83

\twolineshloka
{ततो युगसहस्रान्ते ऋषयः पुनरागमन्}
{मामूचुर्निष्क्रमस्वेति तच्छ्रुत्वा तूर्णमुत्थितः} %6-84

\twolineshloka
{वल्मीकान्निर्गतश्चाहं नीहारादिव भास्करः}
{मामप्याहुर्मुनिगणा वाल्मीकिस्त्वं मुनीश्वर} %6-85

\twolineshloka
{वल्मीकात्सम्भवो यस्माद् द्वितीयं जन्म तेऽभवत्}
{इत्युक्त्वा ते ययुर्दिव्यगतिं रघुकुलोत्तम} %6-86

\twolineshloka
{अहं ते राम नाम्नश्च प्रभावादीदृशोऽभवम्}
{अद्य साक्षात्प्रपश्यामि ससीतं लक्ष्मणेन च} %6-87

\twolineshloka
{रामं राजीवपत्राक्षं त्वां मुक्तो नात्र संशयः}
{आगच्छ राम भद्रं ते स्थलं वै दर्शयाम्यहम्} %6-88

\twolineshloka
{एवमुक्त्वा मुनिः श्रीमान्ल्लक्ष्मणेन समन्वितः}
{शिष्यैः परिवृतो गत्वा मध्ये पर्वतगङ्गयोः} %6-89

\twolineshloka
{तत्र शालां सुविस्तीर्णां कारयामास वासभूः}
{प्राक्पश्चिमं दक्षिणोदक् शोभनं मन्दिरद्वयम्} %6-90

\twolineshloka
{जानक्या सहितो रामो लक्ष्मणेन समन्वितः}
{तत्र ते देवसदृशा ह्यवसन् भवनोत्तमे} %6-91

\fourlineindentedshloka
{वाल्मीकिना तत्र सुपूजितोऽयम्}
{रामः ससीतः सह लक्ष्मणेन}
{देवैर्मुनीद्रैः सहितो मुदास्ते}
{स्वर्गे यथा देवपतिः सशच्या} %6-92

{॥इति श्रीमदध्यात्मरामायणे उमामहेश्वरसंवादे
अयोध्याकाण्डे षष्ठः सर्गः॥६॥
}
%%%%%%%%%%%%%%%%%%%%



\sect{सप्तमः सर्गः}

\twolineshloka
{सुमन्त्रोऽपि तदाऽयोध्यां दिनान्ते प्रविवेश ह}
{वस्त्रेण मुखमाच्छाद्य बाष्पाकुलितलोचनः} %7-1

\twolineshloka
{बहिरेव रथं स्थाप्य राजानं द्रष्टुमाययौ}
{जयशब्देन राजानं स्तुत्वा तं प्रणनाम ह} %7-2

\twolineshloka
{ततो राजा नमन्तं तं सुमन्त्रं विह्वलोऽब्रवीत्}
{सुमन्त्र रामः कुत्रास्ते सीतया लक्ष्मणेन च} %7-3

\twolineshloka
{कुत्र त्यक्तस्त्वया रामः किं मां पापिनमब्रवीत्}
{सीता वा लक्ष्मणो वाऽपि निर्दयं मां किमब्रवीत्} %7-4

\twolineshloka
{हा राम हा गुणनिधे हा सीते प्रियवादिनि}
{दुःखार्णवे निमग्नं मां म्रियमाणं न पश्यसि} %7-5

\twolineshloka
{विलप्यैवं चिरं राजा निमग्नो दुःखसागरे}
{एवं मन्त्री रुदन्तं तं प्राञ्जलिर्वाक्यमब्रवीत्} %7-6

\twolineshloka
{रामः सीता च सौमित्रिर्मया नीता रथेन ते}
{शृङ्गवेरपुराभ्याशे गङ्गाकूले व्यवस्थिताः} %7-7

\twolineshloka
{गुहेन किञ्चिदानीतं फलमूलादिकं च यत्}
{स्पृष्ट्वा हस्तेन सम्प्रीत्या नाग्रहीद्विससर्ज तत्} %7-8

\twolineshloka
{वटक्षीरं समानाय्य गुहेन रघुनन्दनः}
{जटामुकुटमाबद्ध्य मामाह नृपते स्वयम्} %7-9

\twolineshloka
{सुमन्त्र ब्रूहि राजानं शोकस्तेऽस्तु न मत्कृते}
{साकेतादधिकं सौख्यं विपिने नो भविष्यति} %7-10

\twolineshloka
{मातुर्मे वन्दनं ब्रूहि शोकं त्यजतु मत्कृते}
{आश्वासयतु राजानं वृद्धं शोकपरिप्लुतम्} %7-11

\twolineshloka
{सीता चाश्रुपरीताक्षी मामाह नृपसत्तम}
{दुःखगद्गदया वाचा रामं किञ्चिदवेक्षती} %7-12

\twolineshloka
{साष्टाङ्गं प्रणिपातं मे ब्रूहि श्वश्र्वोः पदाम्बुजे}
{इति प्ररुदती सीता गता किञ्चिदवाङ्मुखी} %7-13

\twolineshloka
{ततस्तेऽश्रुपरीताक्षा नावमारुरुहुस्तदा}
{यावद्गङ्गां समुत्तीर्य गतास्तावदहं स्थितः} %7-14

\twolineshloka
{ततो दुःखेन महता पुनरेवाहमागतः}
{ततो रुदन्ती कौसल्या राजानमिदमब्रवीत्} %7-15

\twolineshloka
{कैकेय्यै प्रियभार्यायै प्रसन्नो दत्तवान् वरम्}
{त्वं राज्यं देहि तस्यैव मत्पुत्रः किं विवासितः} %7-16

\twolineshloka
{कृत्वा त्वमेव तत्सर्वमिदानीं किं नु रोदिषि}
{कौसल्यावचनं श्रुत्वा क्षते स्पृष्ट इवाग्निना} %7-17

\twolineshloka
{पुनः शोकाश्रुपूर्णाक्षः कौसल्यामिदमब्रवीत्}
{दुःखेन म्रियमाणं मां किं पुनर्दुःखयस्यलम्} %7-18

\twolineshloka
{इदानीमेव मे प्राणा उत्क्रमिष्यन्ति निश्चयः}
{शप्तोऽहं बाल्यभावेन केनचिन्मुनिना पुरा} %7-19

\twolineshloka
{पुराहं यौवने दृप्तश्चापबाणधरो निशि}
{अचरं मृगयासक्तो नद्यास्तीरे महावने} %7-20

\threelineshloka
{तत्रार्धरात्रसमये मुनिः कश्चित्तृषार्दितः}
{पिपासार्दितयोः पित्रोर्जलमानेतुमुद्यतः}
{अपूरयज्जले कुम्भं तदा शब्दोऽभवन्महान्} %7-21

\twolineshloka
{गजः पिबति पानीयमिति मत्वा महानिशि}
{बाणं धनुषि सन्धाय शब्दवेधिनमक्षिपम्} %7-22

\twolineshloka
{हा हतोऽस्मीति तत्राभूच्छब्दो मानुषसूचकः}
{कस्यापि न कृतो दोषो मया केन हतो विधे} %7-23

\twolineshloka
{प्रतीक्षते मां माता च पिता च जलकाङ्क्षया}
{तच्छ्रुत्वा भयसन्त्रस्तस्ततोऽहं पौरुषं वचः} %7-24

\twolineshloka
{शनैर्गत्वाऽथ तत्पार्श्वं स्वामिन् दशरथोऽस्म्यहम्}
{अजानता मया विद्धस्त्रातुमर्हसि मां मुने} %7-25

\twolineshloka
{इत्युक्त्वा पादयोस्तस्य पतितो गद्गदाक्षरः}
{तदा मामाह स मुनिर्मा भैषीर्नृपसत्तम} %7-26

\twolineshloka
{ब्रह्महत्या स्पृशेन्न त्वां वैश्योऽहं तपसि स्थितः}
{पितरौ मां प्रतीक्षेते क्षुत्तृड्भ्यां परिपीडितौ} %7-27

\twolineshloka
{तयोस्त्वमुदकं देहि शीघ्रमेवाविचारयन्}
{न चेत्त्वां भस्मसात्कुर्यात्पिता मे यदि कुप्यति} %7-28

\twolineshloka
{जलं दत्वा तु तौ नत्वा कृतं सर्वं निवेदय}
{शल्यमुद्धर मे देहात्प्राणान्स्त्यक्ष्यामि पीडितः} %7-29

\twolineshloka
{इत्युक्तो मुनिना शीघ्रं बाणमुत्पाट्य देहतः}
{सजलं कलशं धृत्वा गतोऽहं यत्र दम्पती} %7-30

\twolineshloka
{अतिवृद्धावन्धदृशौ क्षुत्पिपासार्दितौ निशि}
{नायाति सलिलं गृह्य पुत्रः किं वात्र कारणम्} %7-31

\twolineshloka
{अनन्यगतिकौ वृद्धौ शोच्यौ तृट्परिपीडितौ}
{आवामुपेक्षते किं वा भक्तिमानावयोः सुतः} %7-32

\twolineshloka
{इति चिन्ताव्याकुलौ तौ मत्पादन्यासजं ध्वनिम्}
{श्रुत्वा प्राह पिता पुत्र किं विलम्बः कृतस्त्वया} %7-33

\twolineshloka
{देह्यावयोः सुपानीयं पिब त्वमपि पुत्रक}
{इत्येवं लपतोर्भीत्या सकाशमगमं शनैः} %7-34

\twolineshloka
{पादयोः प्रणिपत्याहमब्रवं विनयान्वितः}
{नाहं पुत्रस्त्वयोध्याया राजा दशरथोऽस्म्यहम्} %7-35

\twolineshloka
{पापोऽहं मृगयासक्तो रात्रौ मृगविहिंसकः}
{जलावताराद्दूरेऽहं स्थित्वा जलगतं ध्वनिम्} %7-36

\twolineshloka
{श्रुत्वाऽहं शब्दवेधित्वादेकं बाणमथात्यजम्}
{हतोऽस्मीति ध्वनिं श्रुत्वा भयात्तत्राहमागतः} %7-37

\twolineshloka
{जटां विकीर्य पतितं दृष्ट्वाऽहं मुनिदारकम्}
{भीतो गृहीत्वा तत्पादौ रक्ष रक्षेति चाब्रवम्} %7-38

\twolineshloka
{मा भैषीरिति मां प्राह ब्रह्महत्याभयं न ते}
{मत्पित्रोः सलिलं दत्त्वा नत्वा प्रार्थय जीवितम्} %7-39

\twolineshloka
{इत्युक्तो मुनिना तेन ह्यागतो मुनिहिंसकः}
{रक्षेतां मां दयायुक्तौ युवां हि शरणागतम्} %7-40

\twolineshloka
{इति श्रुत्वा तु दुःखार्तौ विलप्य बहु शोच्य तम्}
{पतितो नौ सुतो यत्र नय तत्राविलम्बयन्} %7-41

\twolineshloka
{ततो नीतौ सुतो यत्र मया तौ वृद्धदम्पती}
{स्पृष्ट्वा सुतं तौ हस्ताभ्यां बहुशोऽथ विलेपतुः} %7-42

\twolineshloka
{हाहेति क्रन्दमानौ तौ पुत्र पुत्रेत्यवोचताम्}
{जलं देहीति पुत्रेति किमर्थं न ददास्यलम्} %7-43

\threelineshloka
{ततो मामूचतुः शीघ्रं चितिं रचय भूपते}
{मया तदैव रचिता चितिस्तत्र निवेशिताः}
{त्रयस्तत्राग्निरुत्सृष्टो दग्धास्ते त्रिदिवं ययुः} %7-44

\twolineshloka
{तत्र वृद्धः पिता प्राह त्वमप्येवं भविष्यसि}
{पुत्रशोकेन मरणं प्राप्स्यसे वचनान्मम} %7-45

\twolineshloka
{स इदानीं मम प्राप्तः शापकालोऽनिवारितः}
{इत्युक्त्वा विललापाथ राजा शोकसमाकुलः} %7-46

\twolineshloka
{हा राम पुत्र हा सीते हा लक्ष्मण गुणाकर}
{त्वद्वियोगादहं प्राप्तो मृत्युं कैकेयिसम्भवम्} %7-47

\twolineshloka
{वदन्नेवं दशरथः प्राणान्स्त्यक्त्वा दिवं गतः}
{कौसल्या च सुमित्रा च तथाऽन्या राजयोषितः} %7-48

\twolineshloka
{चुक्रुशुश्च विलेपुश्च उरस्ताडनपूर्वकम्}
{वसिष्ठः प्रययौ तत्र प्रातर्मन्त्रिभिरावृतः} %7-49

\twolineshloka
{तैलद्रोण्यां दशरथं क्षिप्त्वा दूतानथाब्रवीत्}
{गच्छत त्वरितं साश्वा युधाजिन्नगरं प्रति} %7-50

\twolineshloka
{तत्रास्ते भरतः श्रीमाञ्छत्रुघ्नसहितः प्रभुः}
{उच्यतां भरतः शीघ्रमागच्छेति ममाऽऽज्ञया} %7-51

\twolineshloka
{अयोध्यां प्रति राजानं कैकेयीं चापि पश्यतु}
{इत्युक्तास्त्वरितं दूता गत्वा भरतमातुलम्} %7-52

\twolineshloka
{युधाजितं प्रणम्योचुर्भरतं सानुजं प्रति}
{वसिष्ठस्त्वब्रवीद्राजन् भरतः सानुजः प्रभुः} %7-53

\twolineshloka
{शीघ्रमागच्छतु पुरीमयोध्यामविचारयन्}
{इत्याज्ञप्तोऽथ भरतस्त्वरितं भयविह्वलः} %7-54

\twolineshloka
{आययौ गुरुणादिष्टः सह दूतैस्तु सानुजः}
{राज्ञो वा राघवस्यापि दुःखं किञ्चिदुपस्थितम्} %7-55

\twolineshloka
{इति चिन्तापरो मार्गे चिन्तयन्नगरं ययौ}
{नगरं भ्रष्टलक्ष्मीकं जनसम्बाधवर्जितम्} %7-56

\twolineshloka
{उत्सवैश्च परित्यक्तं दृष्ट्वा चिन्तापरोऽभवत्}
{प्रविश्य राजभवनं राजलक्ष्मीविवर्जितम्} %7-57

\twolineshloka
{अपश्यत्कैकेयीं तत्र एकामेवासने स्थिताम्}
{ननाम शिरसा पादौ मातुर्भक्तिसमन्वितः} %7-58

\twolineshloka
{आगतं भरतं दृष्ट्वा कैकेयी प्रेमसम्भ्रमात्}
{उत्थायालिङ्ग्य रभसा स्वाङ्कमारोप्य संस्थिता} %7-59

\twolineshloka
{मूर्ध्न्यवघ्राय पप्रच्छ कुशलं स्वकुलस्य सा}
{पिता मे कुशलो भ्राता माता च शुभलक्षणा} %7-60

\twolineshloka
{दिष्ट्या त्वमद्य कुशली मया दृष्टोऽसि पुत्रक}
{इति पृष्टः स भरतो मात्रा चिन्ताकुलेन्द्रियः} %7-61

\twolineshloka
{दूयमानेन मनसा मातरं समपृच्छत}
{मातः पिता मे कुत्रास्ते एका त्वमिह संस्थिता} %7-62

\twolineshloka
{त्वया विना न मे तातः कदाचिद्रहसि स्थितः}
{इदानीं दृश्यते नैव कुत्र तिष्ठति मे वद} %7-63

\twolineshloka
{अदृष्ट्वा पितरं मेऽद्य भयं दुःखं च जायते}
{अथाह कैकेयी पुत्र किं दुःखेन तवानघ} %7-64

\twolineshloka
{या गतिर्धर्मशीलानामश्वमेधादियाजिनाम्}
{तां गतिं गतवानद्य पिता ते पितृवत्सल} %7-65

\twolineshloka
{तच्छ्रुत्वा निपपातोर्व्यां भरतः शोकविह्वलः}
{हा तात क्व गतोऽसि त्वं त्यक्त्वा मां वृजिनार्णवे} %7-66

\twolineshloka
{असमर्प्यैव रामाय राज्ञे मां क्व गतोऽसि भोः}
{इति विलपितं पुत्रं पतितं मुक्तमूर्धजम्} %7-67

\twolineshloka
{उत्थाप्यामृज्य नयने कैकेयी पुत्रमब्रवीत्}
{समाश्वसिहि भद्रं ते सर्वं सम्पादितं मया} %7-68

\twolineshloka
{तामाह भरतस्तातो म्रियमाणः किमब्रवीत्}
{तमाह कैकेयी देवी भरतं भयवर्जिता} %7-69

\twolineshloka
{हा राम राम सीतेति लक्ष्मणेति पुनः पुनः}
{विलपन्नेव सुचिरं देहं त्यक्त्वा दिवं ययौ} %7-70

\twolineshloka
{तामाह भरतो हेऽम्ब रामः सन्निहितो न किम्}
{तदानीं लक्ष्मणो वाऽपि सीता वा कुत्र ते गताः} %7-71

\twolineshloka
{रामस्य यौवराज्यार्थं पित्रा ते सम्भ्रमः कृतः}
{तव राज्यप्रदानाय तदाऽहं विघ्नमाचरम्} %7-72

\twolineshloka
{राज्ञा दत्तं हि मे पूर्वं वरदेन वरद्वयम्}
{याचितं तदिदानीं मे तयोरेकेन तेऽखिलम्} %7-73

\twolineshloka
{राज्यं रामस्य चैकेन वनवासो मुनिव्रतम्}
{ततः सत्यपरो राजा राज्यं दत्त्वा तवैव हि} %7-74

\twolineshloka
{रामं सम्प्रेषयामास वनमेव पिता तव}
{सीताप्यनुगता रामं पातिव्रत्यमुपाश्रिता} %7-75

\twolineshloka
{सौभ्रात्रं दर्शयन् राममनुयातोऽपि लक्ष्मणः}
{वनं गतेषु सर्वेषु राजा तानेव चिन्तयन्} %7-76

\twolineshloka
{प्रलपन् रामरामेति ममार नृपसत्तमः}
{इति मातुर्वचः श्रुत्वा वज्राहत इव द्रुमः} %7-77

\twolineshloka
{पपात भूमौ निःसंज्ञस्तं दृष्ट्वा दुःखिता तदा}
{कैकेयी पुनरप्याह वत्स शोकेन किं तव} %7-78

\twolineshloka
{राज्ये महति सम्प्राप्ते दुःखस्यावसरः कुतः}
{इति ब्रुवन्तीमालोक्य मातरं प्रदहन्निव} %7-79

\threelineshloka
{असम्भाष्यासि पापे मे घोरे त्वं भर्तृघातिनी}
{पापे त्वद्गर्भजातोऽहं पापवानस्मि साम्प्रतम्}
{अहमग्निं प्रवेक्ष्यामि विषं वा भक्षयाम्यहम्} %7-80

\twolineshloka
{खड्गेन वाथ चात्मानं हत्वा यामि यमक्षयम्}
{भर्तृघातिनि दुष्टे त्वं कुम्भीपाकं गमिष्यसि} %7-81

\twolineshloka
{इति निर्भर्त्स्य कैकेयीं कौसल्याभवनं ययौ}
{साऽपि तं भरतं दृष्ट्वा मुक्तकण्ठा रुरोद ह} %7-82

\threelineshloka
{पादयोः पतितस्तस्या भरतोऽपि तदाऽरुदत्}
{आलिङ्ग्य भरतं साध्वी राममाता यशस्विनी}
{कृशाऽतिदीनवदना साश्रुनेत्रेदमब्रवीत्} %7-83

\twolineshloka
{पुत्र त्वयि गते दूरमेवं सर्वमभूदिदम्}
{उक्तं मात्रा श्रुतं सर्वं त्वया ते मातृचेष्टितम्} %7-84

\fourlineindentedshloka
{पुत्रः सभार्यो वनमेव यातः}
{सलक्ष्मणो मे रघुरामचन्द्रः}
{चीराम्बरो बद्धजटाकलापः}
{सन्त्यज्य मां दुःखसमुद्रमग्नाम्} %7-85

\fourlineindentedshloka
{हा राम हा मे रघुवंशनाथ}
{जातोऽसि मे त्वं परतः परात्मा}
{तथाऽपि दुःखं न जहाति मां वै}
{विधिर्बलीयानिति मे मनीषा} %7-86

\twolineshloka
{स एवं भरतो वीक्ष्य विलपन्तीं भृशं शुचा}
{पादौ गृहीत्वा प्राहेदं शृणु मातर्वचो मम} %7-87

\twolineshloka
{कैकेय्या यत्कृतं कर्म रामराज्याभिषेचने}
{अन्यद्वा यदि जानामि सा मया नोदिता यदि} %7-88

\twolineshloka
{पापं मेऽस्तु तदा मातर्ब्रह्महत्याशतोद्भवम्}
{हत्वा वसिष्ठं खड्गेन अरुन्धत्या समन्वितम्} %7-89

\twolineshloka
{भूयात्तत्पापमखिलं मम जानामि यद्यहम्}
{इत्येवं शपथं कृत्वा रुरोद भरतस्तदा} %7-90

\twolineshloka
{कौसल्या तमथालिङ्ग्य पुत्र जानामि मा शुचः}
{एतस्मिन्नन्तरे श्रुत्वा भरतस्य समागमम्} %7-91

\twolineshloka
{वसिष्ठो मन्त्रिभिः सार्धं प्रययौ राजमन्दिरम्}
{रुदन्तं भरतं दृष्ट्वा वसिष्ठः प्राह सादरम्} %7-92

\twolineshloka
{वृद्धो राजा दशरथो ज्ञानी सत्यपराक्रमः}
{भुक्त्वा मर्त्यसुखं सर्वमिष्ट्वा विपुलदक्षिणैः} %7-93

\twolineshloka
{अश्वमेधादिभिर्यज्ञैर्लब्ध्वा रामं सुतं हरिम्}
{अन्ते जगाम त्रिदिवं देवेन्द्रार्द्धासनं प्रभुः} %7-94

\twolineshloka
{तं शोचसि वृथैव त्वमशोच्यं मोक्षभाजनम्}
{आत्मा नित्योऽव्ययः शुद्धो जन्मनाशादिवर्जितः} %7-95

\twolineshloka
{शरीरं जडमत्यर्थमपवित्रं विनश्वरम्}
{विचार्यमाणे शोकस्य नावकाशः कथञ्चन} %7-96

\twolineshloka
{पिता वा तनयो वाऽपि यदि मृत्युवशं गतः}
{मूढास्तमनुशोचन्ति स्वात्मताडनपूर्वकम्} %7-97

\twolineshloka
{निःसारे खलु संसारे वियोगो ज्ञानिनां यदा}
{भवेद्वैराग्यहेतुः स शान्तिसौख्यं तनोति च} %7-98

\twolineshloka
{जन्मवान् यदि लोकेऽस्मिन्स्तर्हि तं मृत्युरन्वगात्}
{तस्मादपरिहार्योऽयं मृत्युर्जन्मवतां सदा} %7-99

\twolineshloka
{स्वकर्मवशतः सर्वजन्तूनां प्रभवाप्ययौ}
{विजानन्नप्यविद्वान् यः कथं शोचति बान्धवान्} %7-100

\twolineshloka
{ब्रह्माण्डकोटयो नष्टाः सृष्टयो बहुशो गताः}
{शुष्यन्ति सागराः सर्वे कैवास्था क्षणजीविते} %7-101

\twolineshloka
{चलपत्रान्तलग्नाम्बुबिन्दुवत्क्षणभङ्गुरम्}
{आयुस्त्यजत्यवेलायां कस्तत्र प्रत्ययस्तव} %7-102

\twolineshloka
{देही प्राक्तनदेहोत्थकर्मणा देहवान् पुनः}
{तद्देहोत्थेन च पुनरेवं देहः सदात्मनः} %7-103

\twolineshloka
{यथा त्यजति वै जीर्णं वासो गृह्णाति नूतनम्}
{तथा जीर्णं परित्यज्य देही देहं पुनर्नवम्} %7-104

\twolineshloka
{भजत्येव सदा तत्र शोकस्यावसरः कुतः}
{आत्मा न म्रियते जातु जायते न च वर्धते} %7-105

\twolineshloka
{षड्भावरहितोऽनन्तः सत्यप्रज्ञानविग्रहः}
{आनन्दरूपो बुद्ध्यादिसाक्षी लयविवर्जितः} %7-106

\twolineshloka
{एक एव परो ह्यात्मा ह्यद्वितीयः समः स्थितः}
{इत्यात्मानं दृढं ज्ञात्वा त्यक्त्वा शोकं कुरु क्रियाम्} %7-107

\twolineshloka
{तैलद्रोण्याः पितुर्देहमुद्धृत्य सचिवैः सह}
{कृत्यं कुरु यथान्यायमस्माभिः कुलनन्दन} %7-108

\twolineshloka
{इति सम्बोधितः साक्षाद्गुरुणा भरतस्तदा}
{विसृज्याज्ञानजं शोकं चक्रे सविधिवत्क्रियाम्} %7-109

\twolineshloka
{गुरुणोक्तप्रकारेण आहिताग्नेर्यथाविधि}
{संस्कृत्य स पितुर्देहं विधिदृष्टेन कर्मणा} %7-110

\twolineshloka
{एकादशेऽहनि प्राप्ते ब्राह्मणान् वेदपारगान्}
{भोजयामास विधिवच्छतशोऽथ सहस्रशः} %7-111

\twolineshloka
{उद्दिश्य पितरं तत्र ब्राह्मणेभ्यो धनं बहु}
{ददौ गवां सहस्राणि ग्रामान् रत्नाम्बराणि च} %7-112

\twolineshloka
{अवसत्स्वगृहे यत्र राममेवानुचिन्तयन्}
{वसिष्ठेन सह भ्रात्रा मन्त्रिभिः परिवारितः} %7-113

\fourlineindentedshloka
{रामेऽरण्यं प्रयाते सह जनकसुतालक्ष्मणाभ्यां सुघोरम्}
{माता मे राक्षसीव प्रदहति हृदयं दर्शनादेव सद्यः}
{गच्छाम्यारण्यमद्य स्थिरमतिरखिलं दूरतोऽपास्य राज्यम्}
{रामं सीतासमेतं स्मितरुचिरमुखं नित्यमेवानुसेवे} %7-114

{॥इति श्रीमदध्यात्मरामायणे उमामहेश्वरसंवादे
अयोध्याकाण्डे सप्तमः सर्गः॥७॥
}
%%%%%%%%%%%%%%%%%%%%



\sect{अष्टमः सर्गः}

\twolineshloka
{वसिष्ठो मुनिभिः सार्धं मन्त्रिभिः परिवारितः}
{राज्ञः सभां देवसभासन्निभामविशद्विभुः} %8-1

\twolineshloka
{तत्रासने समासीनश्चतुर्मुख इवापरः}
{आनीय भरतं तत्र उपवेश्य सहानुजम्} %8-2

\twolineshloka
{अब्रवीद्वचनं देशकालोचितमरिन्दमम्}
{वत्स राज्येऽभिषेक्ष्यामस्त्वामद्य पितृशासनात्} %8-3

\twolineshloka
{कैकेय्या याचितं राज्यं त्वदर्थे पुरुषर्षभ}
{सत्यसन्धो दशरथः प्रतिज्ञाय ददौ किल} %8-4

\twolineshloka
{अभिषेको भवत्वद्य मुनिभिर्मन्त्रपूर्वकम्}
{तच्छ्रुत्वा भरतोऽप्याह मम राज्येन किं मुने} %8-5

\twolineshloka
{रामो राजाधिराजश्च वयं तस्यैव किङ्कराः}
{श्वः प्रभाते गमिष्यामो राममानेतुमञ्जसा} %8-6

\twolineshloka
{अहं यूयं मातरश्च कैकेयीं राक्षसीं विना}
{हनिष्याम्यधुनैवाहं कैकेयीं मातृगन्धिनीम्} %8-7

\twolineshloka
{किन्तु मां नो रघुश्रेष्ठः स्त्रीहन्तारं सहिष्यते}
{तच्छ्वोभूते गमिष्यामि पादचारेण दण्डकान्} %8-8

\twolineshloka
{शत्रुघ्नसहितस्तूर्णं यूयमायात वा न वा}
{रामो यथा वने यातस्तथाऽहं वल्कलाम्बरः} %8-9

\twolineshloka
{फलमूलकृताहारः शत्रुघ्नसहितो मुने}
{भूमिशायी जटाधारी यावद्रामो निवर्तते} %8-10

\twolineshloka
{इति निश्चित्य भरतस्तूष्णीमेवावतस्थिवान्}
{साधुसाध्विति तं सर्वे प्रशशंसुर्मुदान्विताः} %8-11

\twolineshloka
{ततः प्रभाते भरतं गच्छन्तं सर्वसैनिकाः}
{अनुजग्मुः सुमन्त्रेण नोदिताः साश्वकुञ्जराः} %8-12

\twolineshloka
{कौसल्याद्या राजदारा वसिष्ठप्रमुखा द्विजाः}
{छादयन्तो भुवं सर्वे पृष्ठतः पार्श्वतोऽग्रतः} %8-13

\twolineshloka
{शृङ्गवेरपुरं गत्वा गङ्गाकूले समन्ततः}
{उवास महती सेना शत्रुघ्नपरिचोदिता} %8-14

\twolineshloka
{आगतं भरतं श्रुत्वा गुहः शङ्कितमानसः}
{महत्या सेनया सार्धमागतो भरतः किल} %8-15

\twolineshloka
{पापं कर्तुं न वा याति रामस्याविदितात्मनः}
{गत्वा तद्धृदयं ज्ञेयं यदि शुद्धस्तरिष्यति} %8-16

\twolineshloka
{गङ्गा नो चेत्समाकृष्य नावस्तिष्ठन्तु सायुधाः}
{ज्ञातयो मे समायत्ताः पश्यन्तः सर्वतोदिशम्} %8-17

\twolineshloka
{इति सर्वान् समादिश्य गुहो भरतमागतः}
{उपायनानि सङ्गृह्य विविधानि बहून्यपि} %8-18

\twolineshloka
{प्रययौ ज्ञातिभिः सार्धं बहुभिर्विविधायुधैः}
{निवेद्योपायनान्यग्रे भरतस्य समन्ततः} %8-19

\twolineshloka
{दृष्ट्वा भरतमासीनं सानुजं सह मन्त्रिभिः}
{चीराम्बरं घनश्यामं जटामुकुटधारिणम्} %8-20

\twolineshloka
{राममेवानुशोचन्तं रामरामेति वादिनम्}
{ननाम शिरसा भूमौ गुहोऽहमिति चाब्रवीत्} %8-21

\twolineshloka
{शीघ्रमुत्थाप्य भरतो गाढमालिङ्ग्य सादरम्}
{पृष्ट्वाऽनामयमव्यग्रः सखायमिदमब्रवीत्} %8-22

\twolineshloka
{भ्रातस्त्वं राघवेणात्र समेतः समवस्थितः}
{रामेणालिङ्गितः सार्द्रनयनेनामलात्मना} %8-23

\twolineshloka
{धन्योऽसि कृतकृत्योऽसि यत्त्वया परिभाषितः}
{रामो राजीवपत्राक्षो लक्ष्मणेन च सीतया} %8-24

\twolineshloka
{यत्र रामस्त्वया दृष्टस्तत्र मां नय सुव्रत}
{सीतया सहितो यत्र सुप्तस्तद्दर्शयस्व मे} %8-25

\twolineshloka
{त्वं रामस्य प्रियतमो भक्तिमानसि भाग्यवान्}
{इति संस्मृत्य संस्मृत्य रामं साश्रुविलोचनः} %8-26

\twolineshloka
{गुहेन सहितस्तत्र यत्र रामः स्थितो निशि}
{ययौ ददर्श शयनस्थलं कुशसमास्तृतम्} %8-27

\twolineshloka
{सीताऽऽभरणसंलग्नस्वर्णबिन्दुभिरर्चितम्}
{दुःखसन्तप्तहृदयो भरतः पर्यदेवयत्} %8-28

\twolineshloka
{अहोऽतिसुकुमारी या सीता जनकनन्दिनी}
{प्रासादे रत्नपर्यङ्के कोमलास्तरणे शुभे} %8-29

\twolineshloka
{रामेण सहिता शेते सा कथं कुशविष्टरे}
{सीता रामेण सहिता दुःखेन मम दोषतः} %8-30

\twolineshloka
{धिङ्मां जातोऽस्मि कैकेय्या पापराशिसमानतः}
{मन्निमित्तमिदं क्लेशं रामस्य परमात्मनः} %8-31

\twolineshloka
{अहोऽतिसफलं जन्म लक्ष्मणस्य महात्मनः}
{राममेव सदान्वेति वनस्थमपि हृष्टधीः} %8-32

\twolineshloka
{अहं रामस्य दासा ये तेषां दासस्य किङ्करः}
{यदि स्यां सफलं जन्म मम भूयान्न संशयः} %8-33

\twolineshloka
{भ्रातर्जानासि यदि तत्कथयस्व ममाखिलम्}
{यत्र तिष्ठति तत्राहं गच्छाम्यानेतुमञ्जसा} %8-34

\twolineshloka
{गुहस्तं शुद्धहृदयं ज्ञात्वा सस्नेहमब्रवीत्}
{देव त्वमेव धन्योऽसि यस्य ते भक्तिरीदृशी} %8-35

\twolineshloka
{रामे राजीवपत्राक्षे सीतायां लक्ष्मणे तथा}
{चित्रकूटाद्रिनिकटे मन्दाकिन्यविदूरतः} %8-36

\twolineshloka
{मुनीनामाश्रमपदे रामस्तिष्ठति सानुजः}
{जानक्या सहितो नन्दात्सुखमास्ते किल प्रभुः} %8-37

\twolineshloka
{तत्र गच्छामहे शीघ्रं गङ्गां तर्तुमिहार्हसि}
{इत्युक्त्वा त्वरितं गत्वा नावः पञ्चशतानि ह} %8-38

\twolineshloka
{समानयत्ससैन्यस्य तर्तुं गङ्गां महानदीम्}
{स्वयमेवानिनायैकां राजनावं गुहस्तदा} %8-39

\twolineshloka
{आरोप्य भरतं तत्र शत्रुघ्नं राममातरम्}
{वसिष्ठं च तथाऽन्यत्र कैकेयीं चान्ययोषितः} %8-40

\twolineshloka
{तीर्त्वा गङ्गां ययौ शीघ्रं भरद्वाजाश्रमं प्रति}
{दूरे स्थाप्य महासैन्यं भरतः सानुजो ययौ} %8-41

\twolineshloka
{आश्रमे मुनिमासीनं ज्वलन्तमिव पावकम्}
{दृष्ट्वा ननाम भरतः साष्टाङ्गमतिभक्तितः} %8-42

\twolineshloka
{ज्ञात्वा दाशरथिं प्रीत्या पूजयामास मौनिराट्}
{पप्रच्छ कुशलं दृष्ट्वा जटावल्कलधारिणम्} %8-43

\twolineshloka
{राज्यं प्रशासतस्तेऽद्य किमेतद्वल्कलादिकम्}
{आगतोऽसि किमर्थं त्वं विपिनं मुनिसेवितम्} %8-44

\twolineshloka
{भरद्वाजवचः श्रुत्वा भरतः साश्रुलोचनः}
{सर्वं जानासि भगवन् सर्वभूताशयस्थितः} %8-45

\twolineshloka
{तथाऽपि पृच्छसे किञ्चित्तदनुग्रह एव मे}
{कैकेय्या यत्कृतं कर्म रामराज्यविघातनम्} %8-46

\twolineshloka
{वनवासादिकं वाऽपि न हि जानामि किञ्चन}
{भवत्पादयुगं मेऽद्य प्रमाणं मुनिसत्तम} %8-47

\twolineshloka
{इत्युक्त्वा पादयुगलं मुनेः स्पृष्ट्वाऽर्त्तमानसः}
{ज्ञातुमर्हसि मां देव शुद्धो वाऽशुद्ध एव वा} %8-48

\twolineshloka
{मम राज्येन किं स्वामिन् रामे तिष्ठति राजनि}
{किङ्करोऽहं मुनिश्रेष्ठ रामचन्द्रस्य शाश्वतः} %8-49

\twolineshloka
{अतो गत्वा मुनिश्रेष्ठ रामस्य चरणान्तिके}
{पतित्वा राज्यसम्भारान् समर्प्यात्रैव राघवम्} %8-50

\twolineshloka
{अभिषेक्ष्ये वसिष्ठाद्यैः पौरजानपदैः सह}
{नेष्येऽयोध्यां रमानाथं दासः सेवेऽतिनीचवत्} %8-51

\twolineshloka
{इत्युदीरितमाकर्ण्य भरतस्य वचो मुनिः}
{आलिङ्ग्य मूर्ध्न्यवघ्राय प्रशशंस सविस्मयः} %8-52

\twolineshloka
{वत्स ज्ञातं पुरैवैतद्भविष्यं ज्ञानचक्षुषा}
{मा शुचस्त्वं परो भक्तः श्रीरामे लक्ष्मणादपि} %8-53

\twolineshloka
{आतिथ्यं कर्तुमिच्छामि ससैन्यस्य तवानघ}
{अद्य भुक्त्वा ससैन्यस्त्वं श्वो गन्ता रामसन्निधिम्} %8-54

\twolineshloka
{यथाऽऽज्ञापयति भवान्स्तथेति भरतोऽब्रवीत्}
{भरद्वाजस्त्वपः स्पृष्ट्वा मौनी होमगृहे स्थितः} %8-55

\twolineshloka
{दध्यौ कामदुघां कामवर्षिणीं कामदो मुनिः}
{असृजत्कामधुक् सर्वं यथाकाममलौकिकम्} %8-56

\twolineshloka
{भरतस्य ससैन्यस्य यथेष्टं च मनोरथम्}
{यथा ववर्ष सकलं तृप्तास्ते सर्वसैनिकाः} %8-57

\twolineshloka
{वसिष्ठं पूजयित्वाऽग्रे शास्त्रदृष्टेन कर्मणा}
{पश्चात्ससैन्यं भरतं तर्पयामास योगिराट्} %8-58

\twolineshloka
{उषित्वा दिनमेकं तु आश्रमे स्वर्गसन्निभे}
{अभिवाद्य पुनः प्रातर्भरद्वाजं सहानुजः} %8-59

\threelineshloka
{भरतस्तु कृतानुज्ञः प्रययौ रामसन्निधिम्}
{चित्रकूटमनुप्राप्य दूरे संस्थाप्य सैनिकान्}
{रामसन्दर्शनाकाङ्क्षी प्रययौ भरतः स्वयम्} %8-60

\twolineshloka
{शत्रुघ्नेन सुमन्त्रेण गुहेन च परन्तपः}
{तपस्विमण्डलं सर्वं विचिन्वानो न्यवर्तत} %8-61

\twolineshloka
{अदृष्ट्वा रामभवनमपृच्छदृषिमण्डलम्}
{कुत्रास्ते सीतया सार्धं लक्ष्मणेन रघूत्तमः} %8-62

\twolineshloka
{ऊचुरग्रे गिरेः पश्चाद्गङ्गाया उत्तरे तटे}
{विविक्तं रामसदनं रम्यं काननमण्डितम्} %8-63

\twolineshloka
{सफलैराम्रपनसैः कदलीखण्डसंवृतम्}
{चम्पकैः कोविदारैश्च पुन्नागैर्विपुलैस्तथा} %8-64

\twolineshloka
{एवं दर्शितमालोक्य मुनिभिर्भरतोऽग्रतः}
{हर्षाद्ययौ रघुश्रेष्ठभवनं मन्त्रिणा सह} %8-65

\fourlineindentedshloka
{ददर्श दूरादतिभासुरं शुभम्}
{रामस्य गेहं मुनिवृन्दसेवितम्}
{वृक्षाग्रसंलग्नसुवल्कलाजिनम्}
{रामाभिरामं भरतः सहानुजः} %8-66

{॥इति श्रीमदध्यात्मरामायणे उमामहेश्वरसंवादे
अयोध्याकाण्डे अष्टमः सर्गः॥८॥
}
%%%%%%%%%%%%%%%%%%%%



\sect{नवमः सर्गः}

\twolineshloka
{अथ गत्वाऽऽश्रमपदसमीपं भरतो मुदा}
{सीतारामपदैर्युक्तं पवित्रमतिशोभनम्} %9-1

\fourlineindentedshloka
{स तत्र वज्राङ्कुशवारिजाञ्चित-}
{ध्वजादिचिह्नानि पदानि सर्वतः}
{ददर्श रामस्य भुवोऽतिमङ्गलानि}
{अचेष्टयत्पादरजःसु सानुजः} %9-2

\fourlineindentedshloka
{अहो सुधन्योऽहममूनि}
{रामपादारविन्दाङ्कितभूतलानि}
{पश्यामि यत्पादरजो विमृग्यम्}
{ब्रह्मादिदेवैः श्रुतिभिश्च नित्यम्} %9-3

\fourlineindentedshloka
{इत्यद्भुतप्रेमरसाप्लुताशयो}
{विगाढचेता रघुनाथभावने}
{आनन्दजाश्रुस्नपितस्तनान्तरः}
{शनैरवापाश्रमसन्निधिं हरेः} %9-4

\fourlineindentedshloka
{स तत्र दृष्ट्वा रघुनाथमास्थितम्}
{दूर्वादलश्यामलमायतेक्षणम्}
{जटाकिरीटं नववल्कलाम्बरम्}
{प्रसन्नवक्त्रं तरुणारुणद्युतिम्} %9-5

\fourlineindentedshloka
{विलोकयन्तं जनकात्मजां शुभाम्}
{सौमित्रिणा सेवितपादपङ्कजम्}
{तदाऽभिदुद्राव रघूत्तमं शुचा}
{हर्षाच्च तत्पादयुगं त्वराग्रहीत्} %9-6

\fourlineindentedshloka
{रामस्तमाकृष्य सुदीर्घबाहुर्दोर्भ्याम्}
{परिष्वज्य सिषिञ्च नेत्रजैः}
{जलैरथाङ्कोपरि सन्न्यवेशयत्}
{पुनः पुनः सम्परिषस्वजे विभुः} %9-7

\twolineshloka
{अथ ता मातरः सर्वाः समाजग्मुस्त्वरान्विताः}
{राघवं द्रष्टुकामास्तास्तृषार्ता गौर्यथा जलम्} %9-8

\twolineshloka
{रामः स्वमातरं वीक्ष्य द्रुतमुत्थाय पादयोः}
{ववन्दे साश्रु सा पुत्रमालिङ्ग्यातीव दुःखिता} %9-9

\twolineshloka
{इतराश्च तथा नत्वा जननी रघुनन्दनः}
{ततः समागतं दृष्ट्वा वसिष्ठं मुनिपुङ्गवम्} %9-10

\twolineshloka
{साष्टाङ्गं प्रणिपत्याह धन्योऽस्मीति पुनः पुनः}
{यथार्हमुपवेश्याह सर्वानेव रघूद्वहः} %9-11

\twolineshloka
{पिता मे कुशली किं वा मां किमाहातिदुःखितः}
{वसिष्ठस्तमुवाचेदं पिता ते रघुनन्दन} %9-12

\twolineshloka
{त्वद्वियोगाभितप्तात्मा त्वामेव परिचिन्तयन्}
{रामरामेति सीतेति लक्ष्मणेति ममार ह} %9-13

\twolineshloka
{श्रुत्वा तत्कर्णशूलाभं गुरोर्वचनमञ्जसा}
{हा हतोऽस्मीति पतितो रुदन् रामः सलक्ष्मणः} %9-14

\twolineshloka
{ततोऽनुरुरुदुः सर्वा मातरश्च तथाऽपरे}
{हा तात मां परित्यज्य क्व गतोऽसि घृणाकर} %9-15

\twolineshloka
{अनाथोऽस्मि महाबाहो मां को वा लालयेदितः}
{सीता च लक्ष्मणश्चैव विलेपतुरतो भृशम्} %9-16

\twolineshloka
{वसिष्ठः शान्तवचनैः शमयामास तां शुचम्}
{ततो मन्दाकिनीं गत्वा स्नात्वा ते वीतकल्मषाः} %9-17

\twolineshloka
{राज्ञे ददुर्जलं तत्र सर्वे ते जलकाङ्क्षिणे}
{पिण्डान्निर्वापयामास रामो लक्ष्मणसंयुतः} %9-18

\twolineshloka
{इङ्गुदीफलपिण्याकरचितान्मधुसम्प्लुतान्}
{वयं यदन्नाः पितरस्तदन्नाः स्मृतिनोदिताः} %9-19

\twolineshloka
{इति दुखाश्रुपूर्णाक्षः पुनः स्नात्वा गृहं ययौ}
{सर्वे रुदित्वा सुचिरं स्नात्वा जग्मुस्तदाश्रमम्} %9-20

\twolineshloka
{तस्मिन्स्तु दिवसे सर्वे उपवासं प्रचक्रिरे}
{ततः परेद्युर्विमले स्नात्वा मन्दाकिनीजले} %9-21

\twolineshloka
{उपविष्टं समागम्य भरतो राममब्रवीत्}
{राम राम महाभाग स्वात्मानमभिषेचय} %9-22

\twolineshloka
{राज्यं पालय पित्र्यं ते ज्येष्ठस्त्वं मे पिता तथा}
{क्षत्रियाणामयं धर्मो यत्प्रजापरिपालनम्} %9-23

\twolineshloka
{इष्ट्वा यज्ञैर्बहुविधैः पुत्रानुत्पाद्य तन्तवे}
{राज्ये पुत्रं समारोप्य गमिष्यसि ततो वनम्} %9-24

\twolineshloka
{इदानीं वनवासस्य कालो नैव प्रसीद मे}
{मातुर्मे दुष्कृतं किञ्चित्स्मर्तुं नार्हसि पाहि नः} %9-25

\twolineshloka
{इत्युक्त्वा चरणौ भ्रातुः शिरस्याधाय भक्तितः}
{रामस्य पुरतः साक्षाद्दण्डवत्पतितो भुवि} %9-26

\twolineshloka
{उत्थाप्य राघवः शीघ्रमारोप्याङ्केऽतिभक्तितः}
{उवाच भरतं रामः स्नेहार्द्रनयनः शनैः} %9-27

\twolineshloka
{शृणु वत्स प्रवक्ष्यामि त्वयोक्तं यत्तथैव तत्}
{किन्तु मामब्रवीत्तातो नव वर्षाणि पञ्च च} %9-28

\twolineshloka
{उषित्वा दण्डकारण्ये पुरं पश्चात्समाविश}
{इदानीं भरतायेदं राज्यं दत्तं मयाऽखिलम्} %9-29

\twolineshloka
{ततः पित्रैव सुव्यक्तं राज्यं दत्तं तवैव हि}
{दण्डकारण्यराज्यं मे दत्तं पित्रा तथैव च} %9-30

\twolineshloka
{अतः पितुर्वचः कार्यमावाभ्यामतियत्नतः}
{पितुर्वचनमुल्लङ्घ्य स्वतन्त्रो यस्तु वर्तते} %9-31

\twolineshloka
{स जीवन्नेव मृतको देहान्ते निरयं व्रजेत्}
{तस्माद्राज्यं प्रशाधि त्वं वयं दण्डकपालकाः} %9-32

\threelineshloka
{भरतस्त्वब्रवीद्रामं कामुको मूढधीः पिता}
{स्त्रीजितो भ्रान्तहृदय उन्मत्तो यदि वक्ष्यति}
{तत्सत्यमिति न ग्राह्यं भ्रान्तवाक्यं यथा सुधीः} %9-33

\textbf{श्रीराम उवाच}

\twolineshloka
{न स्त्रीजितः पिता ब्रूयान्न कामी नैव मूढधीः}
{पूर्वं प्रतिश्रुतं तस्य सत्यवादी ददौ भयात्} %9-34

\twolineshloka
{असत्याद्भीतिरधिका महतां नरकादपि}
{करोमीत्यहमप्येतत्सत्यं तस्यै प्रतिश्रुतम्} %9-35

\twolineshloka
{कथं वाक्यमहं कुर्यामसत्यं राघवो हि सन्}
{इत्युदीरितमाकर्ण्य रामस्य भरतोऽब्रवीत्} %9-36

\textbf{श्रीभरत उवाच}

\twolineshloka
{तथैव चीरवसनो वने वत्स्यामि सुव्रत}
{चतुर्दश समास्त्वं तु राज्यं कुरु यथासुखम्} %9-37

\textbf{श्रीराम उवाच}

\twolineshloka
{पित्रा दत्तं तवैवैतद्राज्यं मह्यं वनं ददौ}
{व्यत्ययं यद्यहं कुर्यामसत्यं पूर्ववत् स्थितम्} %9-38

\twolineshloka
{अहमप्यागमिष्यामि सेवे त्वां लक्ष्मणो यथा}
{नोचेत्प्रायोपवेशेन त्यजाम्येतत्कलेवरम्} %9-39

\twolineshloka
{इत्येवं निश्चयं कृत्वा दर्भानास्तीर्य चातपे}
{मनसाऽपि विनिश्चित्य प्राङ्मुखोपविवेश सः} %9-40

\twolineshloka
{भरतस्यापि निर्बन्धं दृष्ट्वा रामोऽतिविस्मितः}
{नेत्रान्तसंज्ञां गुरवे चकार रघुनन्दनः} %9-41

\twolineshloka
{एकान्ते भरतं प्राह वसिष्ठो ज्ञानिनां वरः}
{वत्स गुह्यं शृणुष्वेदं मम वाक्यात्सुनिश्चितम्} %9-42

\twolineshloka
{रामो नारायणः साक्षाद्ब्रह्मणा याचितः पुरा}
{रावणस्य वधार्थाय जातो दशरथात्मजः} %9-43

\twolineshloka
{योगमायाऽपि सीतेति जाता जनकनन्दिनी}
{शेषोऽपि लक्ष्मणो जातो राममन्वेति सर्वदा} %9-44

\twolineshloka
{रावणं हन्तुकामास्ते गमिष्यन्ति न संशयः}
{कैकेय्या वरदानादि यद्यन्निष्ठुरभाषणम्} %9-45

\twolineshloka
{सर्वं देवकृतं नो चेदेवं सा भाषयेत्कथम्}
{तस्मात्त्यजाऽऽग्रहं तात रामस्य विनिवर्तने} %9-46

\twolineshloka
{निवर्तस्व महासैन्यैर्भ्रातृभिः सहितः पुरम्}
{रावणं सकुलं हत्वा शीघ्रमेवागमिष्यति} %9-47

\twolineshloka
{इति श्रुत्वा गुरोर्वाक्यं भरतो विस्मयान्वितः}
{गत्वा समीपं रामस्य विस्मयोत्फुल्ललोचनः} %9-48

\twolineshloka
{पादुके देहि राजेन्द्र राज्याय तव पूजिते}
{तयोः सेवां करोम्येव यावदागमनं तव} %9-49

\twolineshloka
{इत्युक्त्वा पादुके दिव्ये योजयामास पादयोः}
{रामस्य ते ददौ रामो भरतायातिभक्तितः} %9-50

\twolineshloka
{गृहीत्वा पादुके दिव्ये भरतो रत्नभूषिते}
{रामं पुनः परिक्रम्य प्रणनाम पुनः पुनः} %9-51

\twolineshloka
{भरतः पुनराहेदं भक्त्या गद्गदया गिरा}
{नवपञ्चसमान्ते तु प्रथमे दिवसे यदि} %9-52

\twolineshloka
{नागमिष्यसि चेद्राम प्रविशामि महानलम्}
{बाढमित्येव तं रामो भरतं सन्न्यवर्तयत्} %9-53

\twolineshloka
{ससैन्यः सवसिष्ठश्च शत्रुघ्नसहितः सुधीः}
{मातृभिर्मन्त्रिभिः सार्धं गमनायोपचक्रमे} %9-54

\twolineshloka
{कैकेयी राममेकान्ते स्रवन्नेत्रजलाकुला}
{प्राञ्जलिः प्राह हे राम तव राज्यविघातनम्} %9-55

\twolineshloka
{कृतं मया दुष्टधिया मायामोहितचेतसा}
{क्षमस्व मम दौरात्म्यं क्षमासारा हि साधवः} %9-56

\threelineshloka
{त्वं साक्षाद्विष्णुरव्यक्तः परमात्मा सनातनः}
{मायामानुषरूपेण मोहयस्यखिलं जगत्}
{त्वयैव प्रेरितो लोकः कुरुते साध्वसाधु वा} %9-57

\twolineshloka
{त्वदधीनमिदं विश्वमस्वतन्त्रं करोति किम्}
{यथा कृत्रिमनर्तक्यो नृत्यन्ति कुहकेच्छया} %9-58

\twolineshloka
{त्वदधीना तथा माया नर्तकी बहुरूपिणी}
{त्वयैव प्रेरिताहं च देवकार्यं करिष्यता} %9-59

\twolineshloka
{पापिष्ठं पापमनसा कर्माचरमरिन्दम}
{अद्य प्रतीतोऽसि मम देवानामप्यगोचरः} %9-60

\twolineshloka
{पाहि विश्वेश्वरानन्त जगन्नाथ नमोऽस्तु ते}
{छिन्धि स्नेहमयं पाशं पुत्रवित्तादिगोचरम्} %9-61

\twolineshloka
{त्वज्ज्ञानानलखड्गेन त्वामहं शरणं गता}
{कैकेय्या वचनं श्रुत्वा रामः सस्मितमब्रवीत्} %9-62

\twolineshloka
{यदाह मां महाभागे नानृतं सत्यमेव तत्}
{मयैव प्रेरिता वाणी तव वक्त्राद्विनिर्गता} %9-63

\twolineshloka
{देवकार्यार्थसिद्ध्यर्थमत्र दोषः कुतस्तव}
{गच्छ त्वं हृदि मां नित्यं भावयन्ती दिवानिशम्} %9-64

\twolineshloka
{सर्वत्र विगतस्नेहा मद्भक्त्या मोक्ष्यसेऽचिरात्}
{अहं सर्वत्र समदृग् द्वेष्यो वा प्रिय एव वा} %9-65

\twolineshloka
{नास्ति मे कल्पकस्येव भजतोऽनुभजाम्यहम्}
{मन्मायामोहितधियो मामम्ब मनुजाकृतिम्} %9-66

\twolineshloka
{सुखदुःखाद्यनुगतं जानन्ति न तु तत्त्वतः}
{दिष्ट्या मद्गोचरं ज्ञानमुत्पन्नं ते भवापहम्} %9-67

\twolineshloka
{स्मरन्ती तिष्ठ भवने लिप्यसे न च कर्मभिः}
{इत्युक्ता सा परिक्रम्य रामं सानन्दविस्मया} %9-68

\twolineshloka
{प्रणम्य शतशो भूमौ ययौ गेहं मुदान्विता}
{भरतस्तु सहामात्यैर्मातृभिर्गुरुणा सह} %9-69

\twolineshloka
{अयोध्यामगमच्छ्रीघ्रं राममेवानुचिन्तयन्}
{पौरजानपदान् सर्वानयोध्यायामुदारधीः} %9-70

\twolineshloka
{स्थापयित्वा यथान्यायं नन्दिग्रामं ययौ स्वयम्}
{तत्र सिंहासने नित्यं पादुके स्थाप्य भक्तितः} %9-71

\twolineshloka
{पूजयित्वा यथा रामं गन्धपुष्पाक्षतादिभिः}
{राजोपचारैरखिलैः प्रत्यहं नियतव्रतः} %9-72

\twolineshloka
{फलमूलाशनो दान्तो जटावल्कलधारकः}
{अधःशायी ब्रह्मचारी शत्रुघ्नसहितस्तदा} %9-73

\twolineshloka
{राजकार्याणि सर्वाणि यावन्ति पृथिवीतले}
{तानि पादुकयोः सम्यङ्निवेदयति राघवः} %9-74

\twolineshloka
{गणयन् दिवसान्येव रामागमनकाङ्क्षया}
{स्थितो रामार्पितमनाःसाक्षाद्ब्रह्ममुनिर्यथा} %9-75

\twolineshloka
{रामस्तु चित्रकूटाद्रौ वसन्मुनिभिरावृतः}
{सीतया लक्ष्मणेनापि किञ्चित्कालमुपावसत्} %9-76

\twolineshloka
{नागराश्च सदा यान्ति रामदर्शनलालसाः}
{चित्रकूटस्थितं ज्ञात्वा सीतया लक्ष्मणेन च} %9-77

\twolineshloka
{दृष्ट्वा तज्जनसम्बाधं रामस्तत्याज तं गिरिम्}
{दण्डकारण्यगमने कार्यमप्यनुचिन्तयन्} %9-78

\twolineshloka
{अन्वगात्सीतया भ्रात्रा ह्यत्रेराश्रममुत्तमम्}
{सर्वत्र सुखसंवासं जनसम्बाधवर्जितम्} %9-79

\twolineshloka
{गत्वा मुनिमुपासीनं भासयन्तं तपोवनम्}
{दण्डवत्प्रणिपत्याह रामोऽहमभिवादये} %9-80

\twolineshloka
{पितुराज्ञां पुरस्कृत्य दण्डकाननमागतः}
{वनवासमिषेणापि धन्योऽहं दर्शनात्तव} %9-81

\twolineshloka
{श्रुत्वा रामस्य वचनं रामं ज्ञात्वा हरिं परम्}
{पूजयामास विधिवद्भक्त्या परमया मुनिः} %9-82

\twolineshloka
{वन्यैः फलैः कृतातिथ्यमुपविष्टं रघूत्तमम्}
{सीतां च लक्ष्मणं चैव सन्तुष्टो वाक्यमब्रवीत्} %9-83

\twolineshloka
{भार्या मेऽतीव संवृद्धा ह्यनसूयेति विश्रुता}
{तपश्चरन्ती सुचिरं धर्मज्ञा धर्मवत्सला} %9-84

\twolineshloka
{अन्तस्तिष्ठति तां सीता पश्यत्वरिनिषूदन}
{तथेति जानकीं प्राह रामो राजीवलोचनः} %9-85

\twolineshloka
{गच्छ देवीं नमस्कृत्य शीघ्रमेहि पुनः शुभे}
{तथेति रामवचनं सीता चापि तथाऽकरोत्} %9-86

\twolineshloka
{दण्डवत्पतितामग्रे सीतां दृष्ट्वाऽतिहृष्टधीः}
{अनसूया समालिङ्ग्य वत्से सीतेति सादरम्} %9-87

\twolineshloka
{दिव्ये ददौ कुण्डले द्वे निर्मिते विश्वकर्मणा}
{दुकूले द्वे ददौ तस्यै निर्मले भक्तिसंयुता} %9-88

\twolineshloka
{अङ्गरागं च सीतायै ददौ दिव्यं शुभानना}
{न त्यक्ष्यतेऽङ्गरागेण शोभा त्वां कमलानने} %9-89

\twolineshloka
{पातिव्रत्यं पुरस्कृत्य राममन्वेहि जानकि}
{कुशली राघवो यातु त्वया सह पुनर्गृहम्} %9-90

\twolineshloka
{भोजयित्वा यथान्यायं रामं सीतासमन्वितम्}
{लक्ष्मणं च तदा रामं पुनः प्राह कृताञ्जलिः} %9-91

\fourlineindentedshloka
{राम त्वमेव भुवनानि विधाय तेषाम्}
{संरक्षणाय सुरमानुषतिर्यगादीन्}
{देहान् बिभर्षि न च देहगुणैर्विलिप्तस्-}
{त्वत्तो बिभेत्यखिलमोहकरी च माया} %9-93

{॥इति श्रीमदध्यात्मरामायणे उमामहेश्वरसंवादे
अयोध्याकाण्डे नवमः सर्गः॥९॥
}
%%%%%%%%%%%%%%%%%%%%

इति श्रीमदध्यात्मरामायणे अयोध्याकाण्डः समाप्तः॥