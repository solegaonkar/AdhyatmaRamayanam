

\chapt{बालकाण्डः}


\sect{प्रथमः सर्गः}
॥राम हृदयम्॥

\fourlineindentedshloka
{यः पृथिवीभरवारणाय दिविजैः सम्प्रार्थितश्चिन्मयः}
{सञ्जातः पृथिवीतले रविकुले मायामनुष्योऽव्ययः}
{निश्चक्रं हतराक्षसः पुनरगाद् ब्रह्मत्वमाद्यं स्थिराम्}
{कीर्तिं पापहरां विधाय जगतां तं जानकीशं भजे} %1-1

\fourlineindentedshloka
{विश्वोद्भवस्थितिलयादिषु हेतुमेकम्}
{मायाश्रयं विगतमायमचिन्त्यमूर्तिम्}
{आनन्दसान्द्रममलं निजबोधरूपम्}
{सीतापतिं विदिततत्त्वमहं नमामि} %1-2

\fourlineindentedshloka
{पठन्ति ये नित्यमनन्यचेतसः}
{शृण्वन्ति चाध्यात्मिकसंज्ञितं शुभम्}
{रामायणं सर्वपुराणसम्मतम्}
{निर्धूतपापा हरिमेव यान्ति ते} %1-3

\fourlineindentedshloka
{अध्यात्मरामायणमेव नित्यम्}
{पठेद्यदीच्छेद्भवबन्धमुक्तिम्}
{गवां सहस्रायुतकोटिदानात्}
{फलं लभेद्यः शृणुयात्स नित्यम्} %1-4

\twolineshloka
{पुरारिगिरिसम्भूता श्रीरामार्णवसङ्गता}
{अध्यात्मरामगङ्गेयं पुनाति भुवनत्रयम्} %1-5

\fourlineindentedshloka
{कैलासाग्रे कदाचिद्रविशतविमले मन्दिरे रत्नपीठे}
{संविष्टं ध्याननिष्ठं त्रिनयनमभयं सेवितं सिद्धसन्घैः}
{देवी वामाङ्कसंस्था गिरिवरतनया पार्वती भक्तिनम्रा}
{प्राहेदं देवमीशं सकलमलहरं वाक्यमानन्दकन्दम्} %1-6

\uvacha{पार्वत्युवाच}

\fourlineindentedshloka
{नमोऽस्तु ते देव जगन्निवास}
{सर्वात्मदृक् त्वं परमेश्वरोऽसि}
{पृच्छामि तत्त्वं पुरुषोत्तमस्य}
{सनातनं त्वं च सनातनोऽसि} %1-7

\fourlineindentedshloka
{गोप्यं यदत्यन्तमनन्यवाच्यम्}
{वदन्ति भक्तेषु महानुभावाः}
{तदप्यहोऽहं तव देव भक्ता}
{प्रियोऽसि मे त्वं वद यत्तु पृष्टम्} %1-8

\fourlineindentedshloka
{ज्ञानं सविज्ञानमथानुभक्तिवैराग्ययुक्तम्}
{च मितं विभास्वत्}
{जानाम्यहं योषिदपि त्वदुक्तम्}
{यथा तथा ब्रूहि तरन्ति येन} %1-9

\fourlineindentedshloka
{पृच्छामि चान्यच्च परं रहस्यम्}
{तदेव चाग्रे वद वारिजाक्ष}
{श्रीरामचन्द्रेऽखिललोकसारे}
{भक्तिर्दृढा नौर्भवति प्रसिद्धा} %1-10

\fourlineindentedshloka
{भक्तिः प्रसिद्धा भवमोक्षणाय}
{नान्यत्ततः साधनमस्ति किञ्चित्}
{तथाऽपि हृत्संशयबन्धनं मे}
{विभेत्तुमर्हस्यमलोक्तिभिस्त्वम्} %1-11

\fourlineindentedshloka
{वदन्ति रामं परमेकमाद्यम्}
{निरस्तमायागुणसम्प्रवाहम्}
{भजन्ति चाहर्निशमप्रमत्ताः}
{परं पदं यान्ति तथैव सिद्धाः} %1-12

\fourlineindentedshloka
{वदन्ति केचित्परमोऽपि रामः}
{स्वाविद्यया संवृतमात्मसंज्ञम्}
{जानाति नात्मानमतः परेण}
{सम्बोधितो वेद परात्मतत्त्वम्} %1-13

\fourlineindentedshloka
{यदि स्म जानाति कुतो विलापः}
{सीताकृतेऽनेन कृतः परेण}
{जानाति नैवं यदि केन सेव्यः}
{समो हि सर्वैरपि जीवजातैः} %1-14

\twolineshloka
{अत्रोत्तरं किं विदितं भवद्भिः}
{तद्ब्रूत मे संशयभेदि वाक्यम्} %1-15

\uvacha{श्रीमहादेव उवाच}

\fourlineindentedshloka
{धन्यासि भक्तासि परात्मनस्त्वम्}
{यज्ज्ञातुमिच्छा तव रामतत्त्वम्}
{पुरा न केनाप्यभिचोदितोऽहम्}
{वक्तुं रहस्यं परमं निगूढम्} %1-16

\fourlineindentedshloka
{त्वयाऽद्य भक्त्या परिनोदितोऽहम्}
{वक्ष्ये नमस्कृत्य रघूत्तमं ते}
{रामः परात्मा प्रकृतेरनादि-}
{रानन्द एकः पुरुषोत्तमो हि} %1-17

\fourlineindentedshloka
{स्वमायया कृत्स्नमिदं हि सृष्ट्वा}
{नभोवदन्तर्बहिरास्थितो यः}
{सर्वान्तरस्थोऽपि निगूढ आत्मा}
{स्वमायया सृष्टमिदं विचष्टे} %1-18

\fourlineindentedshloka
{जगन्ति नित्यं परितो भ्रमन्ति}
{यत्सन्निधौ चुम्बकलोहवद्धि}
{एतन्न जानन्ति विमूढचित्ताः}
{स्वाविद्यया संवृतमानसा ये} %1-19

\fourlineindentedshloka
{स्वाज्ञानमप्यात्मनि शुद्धबुद्धे}
{स्वारोपयन्तीह निरस्तमाये}
{संसारमेवानुसरन्ति ते वै}
{पुत्रादिसक्ताः पुरुकर्मयुक्ताः} %1-20

\fourlineindentedshloka
{यथाऽप्रकाशो न तु विद्यते रवौ}
{ज्योतिःस्वभावे परमेश्वरे तथा}
{विशुद्धविज्ञानघने रघूत्तमेऽविद्या}
{कथं स्यात्परतः परात्मनि} %1-21

\fourlineindentedshloka
{यथा हि चाक्ष्णा भ्रमता गृहादिकम्}
{विनष्टदृष्टेर्भ्रमतीव दृश्यते}
{तथैव देहेन्द्रियकर्तुरात्मनः}
{कृते परेऽध्यस्य जनो विमुह्यति} %1-22

\fourlineindentedshloka
{नाहो न रात्रिः सवितुर्यथा भवेत्}
{प्रकाशरूपाव्यभिचारतः क्वचित्}
{ज्ञानं तथाऽज्ञानमिदं द्वयं हरौ}
{रामे कथं स्थास्यति शुद्धचिद्घने} %1-23

\fourlineindentedshloka
{तस्मात्परानन्दमये रघूत्तमे}
{विज्ञानरूपे हि न विद्यते तमः}
{अज्ञानसाक्षिण्यरविन्दलोचने}
{मायाश्रयत्वान्न हि मोहकारणम्} %1-24

\twolineshloka
{अत्र ते कथयिष्यामि रहस्यमपि दुर्लभम्}
{सीताराममरुत्सूनुसंवादं मोक्षसाधनम्} %1-25

\twolineshloka
{पुरा रामायणे रामे रावणं देवकण्टकम्}
{हत्वा रणे रणश्लाघी सपुत्रबलवाहनम्} %1-26

\twolineshloka
{सीतया सह सुग्रीवलक्ष्मणाभ्यां समन्वितः}
{अयोध्यामगमद्रामो हनूमत्प्रमुखैर्वृतः} %1-27

\twolineshloka
{अभिषिक्तः परिवृतो वसिष्ठाद्यैर्महात्मभिः}
{सिंहासने समासीनः कोटिसूर्यसमप्रभः} %1-28

\twolineshloka
{दृष्ट्वा तदा हनूमन्तं प्राञ्जलिं पुरतः स्थितम्}
{कृतकार्यं निराकाङ्क्षं ज्ञानापेक्षं महामतिम्} %1-29

\twolineshloka
{रामः सीतामुवाचेदं ब्रूहि तत्त्वं हनूमते}
{निष्कल्मषोऽयं ज्ञानस्य पात्रं नो नित्यभक्तिमान्} %1-30

\twolineshloka
{तथेति जानकी प्राह तत्त्वं रामस्य निश्चितम्}
{हनूमते प्रपन्नाय सीता लोकविमोहिनी} %1-31

\uvacha{सीतोवाच}

\twolineshloka
{रामं विद्धि परं ब्रह्म सच्चिदानन्दमद्वयम्}
{सर्वोपाधिविनिर्मुक्तं सत्तामात्रमगोचरम्} %1-32

\twolineshloka
{आनन्दं निर्मलं शान्तं निर्विकारं निरञ्जनम्}
{सर्वव्यापिनमात्मानं स्वप्रकाशमकल्मषम्} %1-33

\twolineshloka
{मां विद्धि मूलप्रकृतिं सर्गस्थित्यन्तकारिणीम्}
{तस्य सन्निधिमात्रेण सृजामीदमतन्द्रिता} %1-34

\twolineshloka
{तत्सान्निध्यान्मया सृष्टं तस्मिन्नारोप्यतेऽबुधैः}
{अयोध्यानगरे जन्म रघुवंशेऽतिनिर्मले} %1-35

\twolineshloka
{विश्वामित्रसहायत्वं मखसंरक्षणं ततः}
{अहल्याशापशमनं चापभङ्गो महेशितुः} %1-36

\twolineshloka
{मत्पाणिग्रहणं पश्चाद्भार्गवस्य मदक्षयः}
{अयोध्यानगरे वासो मया द्वादशवार्षिकः} %1-37

\twolineshloka
{दण्डकारण्यगमनं विराधवध एव च}
{मायामारीचमरणं मायासीताहृतिस्तथा} %1-38

\twolineshloka
{जटायुषो मोक्षलाभः कबन्धस्य तथैव च}
{शबर्याः पूजनं पश्चात्सुग्रीवेण समागमः} %1-39

\twolineshloka
{वालिनश्च वधः पश्चात्सीतान्वेषणमेव च}
{सेतुबन्धश्च जलधौ लङ्कायाश्च निरोधनम्} %1-40

\twolineshloka
{रावणस्य वधो युद्धे सपुत्रस्य दुरात्मनः}
{विभीषणे राज्यदानं पुष्पकेण मया सह} %1-41

\threelineshloka
{अयोध्यागमनं पश्चाद्राज्ये रामाभिषेचनम्}
{एवमादीनि कर्माणि मयैवाचरितान्यपि}
{आरोपयन्ति रामेऽस्मिन्निर्विकारेऽखिलात्मनि} %1-42

\fourlineindentedshloka
{रामो न गच्छति न तिष्ठति नानुशोचत्याकाङ्क्षते}
{त्यजति नो न करोति किञ्चित्}
{आनन्दमूर्तिरचलः परिणामहीनो}
{मायागुणाननुगतो हि तथा विभाति} %1-43

\twolineshloka
{ततो रामः स्वयं प्राह हनूमन्तमुपस्थितम्}
{शृणु तत्त्वं प्रवक्ष्यामि ह्यात्मानात्मपरात्मनाम्} %1-44

\threelineshloka
{आकाशस्य यथा भेदस्त्रिविधो दृश्यते महान्}
{जलाशये महाकाशस्तदवच्छिन्न एव हि}
{प्रतिबिम्बाख्यमपरं दृश्यते त्रिविधं नभः} %1-45

\twolineshloka
{बुद्ध्यवच्छिन्नचैतन्यमेकं पूर्णमथापरम्}
{आभासस्त्वपरं बिम्बभूतमेवं त्रिधा चितिः} %1-46

\twolineshloka
{साभासबुद्धेः कर्तृत्वमविच्छिन्नेऽविकारिणि}
{साक्षिण्यारोप्यते भ्रान्त्या जीवत्वं च तथा बुधैः} %1-47

\twolineshloka
{आभासस्तु मृषा बुद्धिरविद्याकार्यमुच्यते}
{अविच्छिन्नं तु तद्ब्रह्म विच्छेदस्तु विकल्पतः} %1-48

\twolineshloka
{अविच्छिन्नस्य पूर्णेन एकत्वं प्रतिपाद्यते}
{तत्त्वमस्यादिवाक्यैश्च साभासस्याहमस्तथा} %1-49

\twolineshloka
{ऐक्यज्ञानं यदोत्पन्नं महावाक्येन चात्मनोः}
{तदाऽविद्या स्वकार्यैश्च नश्यत्येव न संशयः} %1-50

\threelineshloka
{एतद्विज्ञाय मद्भक्तो मद्भावायोपपद्यते}
{मद्भक्तिविमुखानां हि शास्त्रगर्तेषु मुह्यताम्}
{न ज्ञानं न च मोक्षः स्यात्तेषां जन्मशतैरपि} %1-51

\fourlineindentedshloka
{इदं रहस्यं हृदयं ममात्मनो}
{मयैव साक्षात्कथितं तवानघ}
{मद्भक्तिहीनाय शठाय न त्वया}
{दातव्यमैन्द्रादपि राज्यतोऽधिकम्} %1-52

\uvacha{श्रीमहादेव उवाच}

\twolineshloka
{एतत्तेऽभिहितं देवि श्रीरामहृदयं मया}
{अतिगुह्यतमं हृद्यं पवित्रं पापशोधनम्} %1-53

\twolineshloka
{साक्षाद्रामेण कथितं सर्ववेदान्तसङ्ग्रहम्}
{यः पठेत्सततं भक्त्या स मुक्तो नात्र संशयः} %1-54

\twolineshloka
{ब्रह्महत्यादि पापानि बहुजन्मार्जितान्यपि}
{नश्यन्त्येव न सन्देहो रामस्य वचनं यथा} %1-55

\fourlineindentedshloka
{योऽतिभ्रष्टोऽतिपापी परधनपरदारेषु नित्योद्यतो वा}
{स्तेयी ब्रह्मघ्नमातापितृवधनिरतो योगिवृन्दापकारी}
{यः सम्पूज्याभिरामं पठति च हृदयं रामचन्द्रस्य भक्त्या}
{योगीन्द्रैरप्यलभ्यं पदमिह लभते सर्वदेवैः स पूज्यम्} %1-56

{॥इति श्रीमदध्यात्मरामायणे उमामहेश्वरसंवादे बालकाण्डे
श्रीरामहृदयं नाम प्रथमः सर्गः॥१॥
}
%%%%%%%%%%%%%%%%%%%%



\sect{द्वितीयः सर्गः}

\uvacha{पार्वत्युवाच}

\twolineshloka
{धन्यास्म्यनुगृहीतास्मि कृतार्थास्मि जगत्प्रभो}
{विच्छिन्नो मेऽतिसन्देहग्रन्थिर्भवदनुग्रहात्} %2-1

\twolineshloka
{त्वन्मुखाद्गलितं रामतत्त्वामृतरसायनम्}
{पिबन्त्या मे मनो देव न तृप्यति भवापहम्} %2-2

\twolineshloka
{श्रीरामस्य कथा त्वत्तः श्रुता सङ्क्षेपतो मया}
{इदानीं श्रोतुमिच्छामि विस्तरेण स्फुटाक्षरम्} %2-3

\uvacha{श्रीमहादेव उवाच}

\twolineshloka
{शृणु देवि प्रवक्ष्यामि गुह्याद्गुह्यतरं महत्}
{अध्यात्मरामचरितं रामेणोक्तं पुरा मम} %2-4

\threelineshloka
{तदद्य कथयिष्यामि शृणु तापत्रयापहम्}
{यच्छ्रुत्वा मुच्यते जन्तुरज्ञानोत्थमहाभयात्}
{प्राप्नोति परमामृद्धिम् दीर्घायुः पुत्रसन्ततिम्} %2-5

\fourlineindentedshloka
{भूमिर्भारेण मग्ना दशवदनमुखाशेषरक्षोगणानाम्}
{धृत्वा गोरूपमादौ दिविजमुनिजनैः साकमब्जासनस्य}
{गत्वा लोकं रुदन्ती व्यसनमुपगतं ब्रह्मणे प्राह सर्वम्}
{ब्रह्मा ध्यात्वा मुहूर्तं सकलमपि हृदावेदशेषात्मकत्वात्} %2-6

\fourlineindentedshloka
{तस्मात्क्षीरसमुद्रतीरमगमद् ब्रह्माथ देवैर्वृतो}
{देव्या चाखिललोकहृत्स्थमजरं सर्वज्ञमीशं हरिम्}
{अस्तौषीच्छ्रुतिसिद्धनिर्मलपदैः स्तोत्रैः पुराणोद्भवैः}
{भक्त्या गद्गदया गिरातिविमलैरानन्दबाष्पैर्वृतः} %2-7

\twolineshloka
{ततः स्फुरत्सहस्रांशुसहस्रसदृशप्रभः}
{आविरासीद्धरिः प्राच्यां दिशां व्यपनयन्स्तमः} %2-8

\twolineshloka
{कथञ्चिद्दृष्टवान् ब्रह्मा दुर्दर्शमकृतात्मनाम्}
{इन्द्रनीलप्रतीकाशं स्मितास्यं पद्मलोचनम्} %2-9

\twolineshloka
{किरीटहारकेयूरकुण्डलैः कटकादिभिः}
{विभ्राजमानं श्रीवत्सकौस्तुभप्रभयान्वितम्} %2-10

\twolineshloka
{स्तुवद्भिः सनकाद्यैश्च पार्षदैः परिवेष्टितम्}
{शङ्खचक्रगदापद्मवनमालाविराजितम्} %2-11

\twolineshloka
{स्वर्णयज्ञोपवीतेन स्वर्णवर्णाम्बरेण च}
{श्रिया भूम्या च सहितं गरुडोपरि संस्थितम्} %2-12

\onelineshloka
{हर्षगद्गदया वाचा स्तोतुं समुपचक्रमे} %2-13

\uvacha{ब्रह्मोवाच}

\twolineshloka
{नतोऽस्मि ते पदं देव प्राणबुद्धीन्द्रियात्मभिः}
{यच्चिन्त्यते कर्मपाशाद्धृदि नित्यं मुमुक्षुभिः} %2-14

\twolineshloka
{मायया गुणमय्या त्वं सृजस्यवसि लुम्पसि}
{जगत्तेन न ते लेप आनन्दानुभवात्मनः} %2-15

\twolineshloka
{तथा शुद्धिर्न दुष्टानां दानाध्ययनकर्मभिः}
{शुद्धात्मता ते यशसि सदा भक्तिमतां यथा} %2-16

\twolineshloka
{अतस्तवाङ्घ्रिर्मे दृष्टश्चित्तदोषापनुत्तये}
{सद्योऽन्तर्हृदये नित्यं मुनिभिः सात्वतैर्वृतः} %2-17

\twolineshloka
{ब्रह्माद्यैः स्वार्थसिद्ध्यर्थमस्माभिः पूर्वसेवितः}
{अपरोक्षानुभूत्यर्थं ज्ञानिभिर्हृदि भावितः} %2-18

\twolineshloka
{तवाङ्घ्रिपूजानिर्माल्यतुलसीमालया विभो}
{स्पर्धते वक्षसि पदं लब्ध्वाऽपि श्रीः सपत्निवत्} %2-19

\twolineshloka
{अतस्त्वत्पादभक्तेषु तव भक्तिः श्रियोऽधिका}
{भक्तिमेवाभिवाञ्छन्ति त्वद्भक्ताः सारवेदिनः} %2-20

\twolineshloka
{अतस्त्वत्पादकमले भक्तिरेव सदास्तु मे}
{संसाराऽऽमयतप्तानां भेषजं भक्तिरेव ते} %2-21

\twolineshloka
{इति ब्रुवन्तं ब्रह्माणं बभाषे भगवान् हरिः}
{किं करोमीति तं वेधाः प्रत्युवाचातिहर्षितः} %2-22

\twolineshloka
{भगवन् रावणो नाम पौलस्त्यतनयो महान्}
{राक्षसानामधिपतिर्मद्दत्तवरदर्पितः} %2-23

\twolineshloka
{त्रिलोकीं लोकपालान्श्च बाधते विश्वबाधकः}
{मानुषेण मृतिस्तस्य मया कल्याण कल्पिता} %2-24

\onelineshloka
{अतस्त्वं मानुषो भूत्वा जहि देवरिपुं प्रभो} %2-25


\uvacha{श्रीभगवानुवाच}

\threelineshloka
{कश्यपस्य वरो दत्तस्तपसा तोषितेन मे}
{याचितः पुत्रभावाय तथेत्यङ्गीकृतं मया}
{स इदानीं दशरथो भूत्वा तिष्ठति भूतले} %2-26

\twolineshloka
{तस्याहं पुत्रतामेत्य कौसल्यायां शुभे दिने}
{चतुर्धाऽऽत्मानमेवाहं सृजामीतरयोः पृथक्} %2-27

\threelineshloka
{योगमायाऽपि सीतेति जनकस्य गृहे तदा}
{उत्पत्स्यते तया सार्धं सर्वं सम्पादयाम्यहम्}
{इत्युक्त्वाऽन्तर्दधे विष्णुर्ब्रह्मा देवानथाब्रवीत्} %2-28

\uvacha{ब्रह्मोवाच}

\onelineshloka
{विष्णुर्मानुषरूपेण भविष्यति रघोः कुले} %2-29

\twolineshloka
{यूयं सृजध्वं सर्वेऽपि वानरेष्वंशसम्भवान्}
{विष्णोः सहायं कुरुत यावत्स्थास्यति भूतले} %2-30

\twolineshloka
{इति देवान् समादिश्य समाश्वास्य च मेदिनीम्}
{ययौ ब्रह्मा स्वभवनं विज्वरः सुखमास्थितः} %2-31

\fourlineindentedshloka
{देवाश्च सर्वे हरिरूपधारिणः}
{स्थिताः सहायार्थमितस्ततो हरेः}
{महाबलाः पर्वतवृक्षयोधिनः}
{प्रतीक्षमाणा भगवन्तमीश्वरम्} %2-32

{॥इति श्रीमदध्यात्मरामायणे उमामहेश्वरसंवादे
बालकाण्डे द्वितीयः सर्गः॥२॥
}
%%%%%%%%%%%%%%%%%%%%



\sect{तृतीयः सर्गः}

\uvacha{श्रीमहादेव उवाच}

\twolineshloka
{अथ राजा दशरथः श्रीमान् सत्यपरायणः}
{अयोध्याधिपतिर्वीरः सर्वलोकेषु विश्रुतः} %3-1

\twolineshloka
{सोऽनपत्यत्वदुःखेन पीडितो गुरुमेकदा}
{वसिष्ठं स्वकुलाचार्यमभिवाद्येदमब्रवीत्} %3-2

\twolineshloka
{स्वामिन् पुत्राः कथं मे स्युः सर्वलक्षणलक्षिताः}
{पुत्रहीनस्य मे राज्यं सर्वं दुःखाय कल्पते} %3-3

\twolineshloka
{ततोऽब्रवीद्वसिष्ठस्तं भविष्यन्ति सुतास्तव}
{चत्वारः सत्त्वसम्पन्ना लोकपाला इवापराः} %3-4

\twolineshloka
{शान्ताभर्तारमानीय ऋष्यशृङ्गं तपोधनम्}
{अस्माभिः सहितः पुत्रकामेष्टिं शीघ्रमाचर} %3-5

\twolineshloka
{तथेति मुनिमानीय मन्त्रिभिः सहितः शुचिः}
{यज्ञकर्म समारेभे मुनिभिर्वीतकल्मषैः} %3-6

\twolineshloka
{श्रद्धया हूयमानेऽग्नौ तप्तजाम्बूनदप्रभः}
{पायसं स्वर्णपात्रस्थं गृहीत्वोवाच हव्यवाट्} %3-7

\twolineshloka
{गृहाण पायसं दिव्यं पुत्रीयं देवनिर्मितम्}
{लप्स्यसे परमात्मानं पुत्रत्वेन न संशयः} %3-8

\twolineshloka
{इत्युक्त्वा पायसं दत्त्वा राज्ञे सोऽन्तर्दधेऽनलः}
{ववन्दे मुनिशार्दूलौ राजा लब्धमनोरथः} %3-9

\twolineshloka
{वसिष्ठऋष्यशृङ्गाभ्यामनुज्ञातो ददौ हविः}
{कौसल्यायै सकैकेय्यै अर्धमर्धं प्रयत्नतः} %3-10

\twolineshloka
{ततः सुमित्रा सम्प्राप्ता जगृध्नुः पौत्रिकं चरुम्}
{कौसल्या तु स्वभागार्धं ददौ तस्यै मुदान्विता} %3-11

\twolineshloka
{कैकेयी च स्वभागार्धं ददौ प्रीतिसमन्विता}
{उपभुज्य चरुं सर्वाः स्त्रियो गर्भसमन्विताः} %3-12

\twolineshloka
{देवता इव रेजुस्ताः स्वभासा राजमन्दिरे}
{दशमे मासि कौसल्या सुषुवे पुत्रमद्भुतम्} %3-13

\twolineshloka
{मधुमासे सिते पक्षे नवम्यां कर्कटे शुभे}
{पुनर्वस्वृक्षसहिते उच्चस्थे ग्रहपञ्चके} %3-14

\twolineshloka
{मेषं पूषणि सम्प्राप्ते पुष्पवृष्टिसमाकुले}
{आविरासीज्जगन्नाथः परमात्मा सनातनः} %3-15

\twolineshloka
{नीलोत्पलदलश्यामः पीतवासाश्चतुर्भुजः}
{जलजारुणनेत्रान्तः स्फुरत्कुण्डलमण्डितः} %3-16

\twolineshloka
{सहस्रार्कप्रतीकाशः किरीटी कुञ्चितालकः}
{शङ्खचक्रगदापद्मवनमालाविराजितः} %3-17

\threelineshloka
{अनुग्रहाख्यहृत्स्थेन्दुसूचकस्मितचन्द्रिकः}
{करुणारससम्पूर्णविशालोत्पललोचनः}
{श्रीवत्सहारकेयूरनूपुरादिविभूषणः} %3-18

\twolineshloka
{दृष्ट्वा तं परमात्मानं कौसल्या विस्मयाकुला}
{हर्षाश्रुपूर्णनयना नत्वा प्राञ्जलिरब्रवीत्} %3-19

\uvacha{कौसल्योवाच}

\twolineshloka
{देवदेव नमस्तेऽस्तु शङ्खचक्रगदाधर}
{परमात्माऽच्युतोऽनन्तः पूर्णस्त्वं पुरुषोत्तमः} %3-20

\twolineshloka
{वदन्त्यगोचरं वाचां बुद्ध्यादीनामतीन्द्रियम्}
{त्वां वेदवादिनः सत्तामात्रं ज्ञानैकविग्रहम्} %3-21

\twolineshloka
{त्वमेव मायया विश्वं सृजस्यवसि हंसि च}
{सत्त्वादिगुणसंयुक्तस्तुर्य एवामलः सदा} %3-22

\twolineshloka
{करोषीव न कर्ता त्वं गच्छसीव न गच्छसि}
{शृणोषि न शृणोषीव पश्यसीव न पश्यसि} %3-23

\twolineshloka
{अप्राणो ह्यमनाः शुद्ध इत्यादि श्रुतिरब्रवीत्}
{समः सर्वेषु भूतेषु तिष्ठन्नपि न लक्ष्यसे} %3-24

\twolineshloka
{अज्ञानध्वान्तचित्तानां व्यक्त एव सुमेधसाम्}
{जठरे तव दृश्यन्ते ब्रह्माण्डाः परमाणवः} %3-25

\twolineshloka
{त्वं ममोदरसम्भूत इति लोकान् विडम्बसे}
{भक्तेषु पारवश्यं ते दृष्टं मेऽद्य रघूत्तम} %3-26

\twolineshloka
{संसारसागरे मग्ना पतिपुत्रधनादिषु}
{भ्रमामि मायया तेऽद्य पादमूलमुपागता} %3-27

\twolineshloka
{देव त्वद्रूपमेतन्मे सदा तिष्ठतु मानसे}
{आवृणोतु न मां माया तव विश्वविमोहिनी} %3-28

\threelineshloka
{उपसंहर विश्वात्मन्नदो रूपमलौकिकम्}
{दर्शयस्व महानन्दबालभावं सुकोमलम्}
{ललितालिङ्गनालापैस्तरिष्याम्युत्कटं तमः} %3-29

\uvacha{श्रीभगवानुवाच}

\onelineshloka
{यद्यदिष्टं तवास्त्यम्ब तत्तद्भवतु नान्यथा} %3-30

\twolineshloka
{अहं तु ब्रह्मणा पूर्वं भूमेर्भारापनुत्तये}
{प्रार्थितो रावणं हन्तुं मानुषत्वमुपागतः} %3-31

\twolineshloka
{त्वया दशरथेनाहं तपसाराधितः पुरा}
{मत्पुत्रत्वाभिकाङ्क्षिण्या तथा कृतमनिन्दिते} %3-32

\twolineshloka
{रूपमेतत्त्वया दृष्टं प्राक्तनं तपसः फलम्}
{मद्दर्शनं विमोक्षाय कल्पते ह्यन्यदुर्लभम्} %3-33

\twolineshloka
{संवादमावयोर्यस्तु पठेद्वा शृणुयादपि}
{स याति मम सारूप्यं मरणे मत्स्मृतिं लभेत्} %3-34

\twolineshloka
{इत्युक्त्वा मातरं रामो बालो भूत्वा रुरोद ह}
{बालत्वेऽपीन्द्रनीलाभो विशालाक्षोऽतिसुन्दरः} %3-35

\threelineshloka
{बालारुणप्रतीकाशो लालिताखिललोकपः}
{अथ राजा दशरथः श्रुत्वा पुत्रोद्भवोत्सवम्}
{आनन्दार्णवमग्नोऽसावाययौ गुरुणा सह} %3-36

\twolineshloka
{रामं राजीवपत्राक्षं दृष्ट्वा हर्षाश्रुसम्प्लुतः}
{गुरुणा जातकर्माणि कर्तव्यानि चकार सः} %3-37

\twolineshloka
{कैकेयी चाथ भरतमसूत कमलेक्षणा}
{सुमित्रायां यमौ जातौ पूर्णेन्दुसदृशाननौ} %3-38

\twolineshloka
{तदा ग्रामसहस्राणि ब्राह्मणेभ्यो मुदा ददौ}
{सुवर्णानि च रत्नानि वासांसि सुरभीः शुभाः} %3-39

\twolineshloka
{यस्मिन् रमन्ते मुनयो विद्यया ज्ञानविप्लवे}
{तं गुरुः प्राह रामेति रमणाद्राम इत्यपि} %3-40

\twolineshloka
{भरणाद्भरतो नाम लक्ष्मणं लक्षणान्वितम्}
{शत्रुघ्नं शत्रुहन्तारमेवं गुरुरभाषत} %3-41

\twolineshloka
{लक्ष्मणो रामचन्द्रेण शत्रुघ्नो भरतेन च}
{द्वन्द्वीभूय चरन्तौ तौ पायसांशानुसारतः} %3-42

\twolineshloka
{रामस्तु लक्ष्मणेनाथ विचरन् बाललीलया}
{रमयामास पितरौ चेष्टितैर्मुग्धभाषितैः} %3-43

\twolineshloka
{भाले स्वर्णमयाश्वत्थपर्णमुक्ताफलप्रभम्}
{कण्ठे रत्नमणिव्रातमध्यद्वीपिनखाञ्चितम्} %3-44

\twolineshloka
{कर्णयोः स्वर्णसम्पन्नरत्नार्जुनसटालुकम्}
{शिञ्जानमणिमञ्जीरकटिसूत्राङ्गदैर्वृतम्} %3-45

\threelineshloka
{स्मितवक्त्राल्पदशनमिन्द्रनीलमणिप्रभम्}
{अङ्गणे रिङ्गमाणं तं तर्णकाननु सर्वतः}
{दृष्ट्वा दशरथो राजा कौसल्या मुमुदे तदा} %3-46

\twolineshloka
{भोक्ष्यमाणो दशरथो राममेहीति चासकृत्}
{आह्वयत्यतिहर्षेण प्रेम्णा नायाति लीलया} %3-47

\twolineshloka
{आनयेति च कौसल्यामाह सा सस्मिता सुतम्}
{धावत्यपि न शक्नोति स्प्रष्टुं योगिमनोगतिम्} %3-48

\twolineshloka
{प्रहसन् स्वयमायाति कर्दमाङ्कितपाणिना}
{किञ्चिद्गृहीत्वा कवलं पुनरेव पलायते} %3-49

\twolineshloka
{कौसल्या जननी तस्य मासि मासि प्रकुर्वती}
{वायनानि विचित्राणि समलङ्कृत्य राघवम्} %3-50

\twolineshloka
{अपूपान् मोदकान् कृत्वा कर्णशष्कुलिकास्तथा}
{कर्णपूरान्श्च विविधान् वर्षवृद्धौ च वायनम्} %3-51

\twolineshloka
{गृहकृत्यं तया त्यक्तं तस्य चापल्यकारणात्}
{एकदा रघुनाथोऽसौ गतो मातरमन्तिके} %3-52

\twolineshloka
{भोजनं देहि मे मातर्न श्रुतं कार्यसक्तया}
{ततः क्रोधेन भाण्डानि लगुडेनाहनत्तदा} %3-53

\twolineshloka
{शिक्यस्थं पातयामास गव्यं च नवनीतकम्}
{लक्ष्मणाय ददौ रामो भरताय यथाक्रमम्} %3-54

\twolineshloka
{शत्रुघ्नाय ददौ पश्चाद्दधि दुग्धं तथैव च}
{सूदेन कथिते मात्रे हास्यं कृत्वा प्रधावति} %3-55

\twolineshloka
{आगतां तां विलोक्याथ ततः सर्वैः पलायितम्}
{कौसल्या धावमानाऽपि प्रस्खलन्ती पदे पदे} %3-56

\twolineshloka
{रघुनाथं करे धृत्वा किञ्चिन्नोवाच भामिनी}
{बालभावं समाश्रित्य मन्दं मन्दं रुरोद ह} %3-57

\twolineshloka
{ते सर्वे लालिता मात्रा गाढमालिङ्ग्य यत्नतः}
{एवमानन्दसन्दोहजगदानन्दकारकः} %3-58

\twolineshloka
{मायाबालवपुर्धृत्वा रमयामास दम्पती}
{अथ कालेन ते सर्वे कौमारं प्रतिपेदिरे} %3-59

\twolineshloka
{उपनीता वसिष्ठेन सर्वविद्याविशारदाः}
{धनुर्वेदे च निरताः सर्वशास्त्रार्थवेदिनः} %3-60

\twolineshloka
{बभूवुर्जगतां नाथा लीलया नररूपिणः}
{लक्ष्मणस्तु सदा राममनुगच्छति सादरम्} %3-61

\twolineshloka
{सेव्यसेवकभावेन शत्रुघ्नो भरतं तथा}
{रामश्चापधरो नित्यं तूणीबाणान्वितः प्रभुः} %3-62

\twolineshloka
{अश्वारूढो वनं याति मृगयायै सलक्ष्मणः}
{हत्वा दुष्टमृगान् सर्वान् पित्रे सर्वं न्यवेदयत्} %3-63

\twolineshloka
{प्रातरुत्थाय सुस्नातः पितरावभिवाद्य च}
{पौरकार्याणि सर्वाणि करोति विनयान्वितः} %3-64

\twolineshloka
{बन्धुभिः सहितो नित्यं भुक्त्वा मुनिभिरन्वहम्}
{धर्मशास्त्ररहस्यानि शृणोति व्याकरोति च} %3-65

\twolineshloka
{एवं परात्मा मनुजावतारो मनुष्यलोकाननुसृत्य सर्वम्}
{चक्रेऽविकारी परिणामहीनो विचार्यमाणे न करोति किञ्चित्} %3-66

{॥इति श्रीमदध्यात्मरामायणे उमामहेश्वरसंवादे
बालकाण्डे तृतीयः सर्गः॥३॥
}
%%%%%%%%%%%%%%%%%%%%



\sect{चतुर्थः सर्गः}

\uvacha{श्रीमहादेव उवाच}

\twolineshloka
{कदाचित्कौशिकोऽभ्यागादयोध्यां ज्वलनप्रभः}
{द्रष्टुं रामं परात्मानं जातं ज्ञात्वा स्वमायया} %4-1

\twolineshloka
{दृष्ट्वा दशरथो राजा प्रत्युत्थायाचिरेण तु}
{वसिष्ठेन समागम्य पूजयित्वा यथाविधि} %4-2

\twolineshloka
{अभिवाद्य मुनिं राजा प्राञ्जलिर्भक्तिनम्रधीः}
{कृतार्थोऽस्मि मुनीन्द्राहं त्वदागमनकारणात्} %4-3

\twolineshloka
{त्वद्विधा यद्गृहम् यान्ति तत्रैवायान्ति सम्पदः}
{यदर्थमागतोऽसि त्वं ब्रूहि सत्यं करोमि तत्} %4-4

\twolineshloka
{विश्वामित्रोऽपि तं प्रीतः प्रत्युवाच महीपतिम्}
{अहं पर्वणि सम्प्राप्ते दृष्ट्वा यष्टुं सुरान् पितॄन्} %4-5

\twolineshloka
{यदारभे तदा दैत्या विघ्नं कुर्वन्ति नित्यशः}
{मारीचश्च सुबाहुश्चापरे चानुचरास्तयोः} %4-6

\twolineshloka
{अतस्तयोर्वधार्थाय ज्येष्ठं रामं प्रयच्छ मे}
{लक्ष्मणेन सह भ्रात्रा तव श्रेयो भविष्यति} %4-7

\twolineshloka
{वसिष्ठेन सहामन्त्र्य दीयतां यदि रोचते}
{पप्रच्छ गुरुमेकान्ते राजा चिन्तापरायणः} %4-8

\twolineshloka
{किं करोमि गुरो रामं त्यक्तुं नोत्सहते मनः}
{बहुवर्षसहस्रान्ते कष्टेनोत्पादिताः सुताः} %4-9

\twolineshloka
{चत्वारोऽमरतुल्यास्ते तेषां रामोऽतिवल्लभः}
{रामस्त्वितो गच्छति चेन्न जीवामि कथञ्चन} %4-10

\twolineshloka
{प्रत्याख्यातो यदि मुनिः शापं दास्यत्यसंशयः}
{कथं श्रेयो भवेन्मह्यमसत्यं चापि न स्पृशेत्} %4-11

\uvacha{वसिष्ठ उवाच}

\twolineshloka
{शृणु राजन् देवगुह्यं गोपनीयं प्रयत्नतः}
{रामो न मानुषो जातः परमात्मा सनातनः} %4-12

\twolineshloka
{भूमेर्भारावताराय ब्रह्मणा प्रार्थितः पुरा}
{स एव जातो भवने कौसल्यायां तवानघ} %4-13

\twolineshloka
{त्वं तु प्रजापतिः पूर्वं कश्यपो ब्रह्मणः सुतः}
{कौसल्या चादितिर्देवमाता पूर्वं यशस्विनी} %4-14

\threelineshloka
{भवन्तौ तप उग्रं वै तेपाथे बहुवत्सरम्}
{अग्राम्यविषयौ विष्णुपूजाध्यानैकतत्परौ}
{तदा प्रसन्नो भगवान् वरदो भक्तवत्सलः} %4-15

\twolineshloka
{वृणीष्व वरमित्युक्ते त्वं मे पुत्रो भवामल}
{इति त्वया याचितोऽसौ भगवान् भूतभावनः} %4-16

\twolineshloka
{तथेत्युक्त्वाऽद्य पुत्रस्ते जातो रामः स एव हि}
{शेषस्तु लक्ष्मणो राजन् राममेवान्वपद्यत} %4-17

\twolineshloka
{जातौ भरतशत्रुघ्नौ शङ्खचक्रे गदाभृतः}
{योगमायाऽपि सीतेति जाता जनकनन्दिनी} %4-18

\twolineshloka
{विश्वामित्रोऽपि रामाय तां योजयितुमागतः}
{एतद्गुह्यतमं राजन्न वक्तव्यं कदाचन} %4-19

\twolineshloka
{अतः प्रीतेन मनसा पूजयित्वाऽथ कौशिकम्}
{प्रेषयस्व रमानाथं राघवं सहलक्ष्मणम्} %4-20

\twolineshloka
{वसिष्ठेनैवमुक्तस्तु राजा दशरथस्तदा}
{कृतकृत्यमिवात्मानं मेने प्रमुदितान्तरः} %4-21

\twolineshloka
{आहूय रामरामेति लक्ष्मणेति च सादरम्}
{आलिङ्ग्य मूर्ध्न्यवघ्राय कौशिकाय समर्पयत्} %4-22

\threelineshloka
{ततोऽतिहृष्टो भगवान् विश्वामित्रः प्रतापवान्}
{आशीर्भिरभिनन्द्याथ आगतौ रामलक्ष्मणौ}
{गृहीत्वा चापतूणीरबाणखड्गधरौ ययौ} %4-23

\twolineshloka
{किञ्चिद्देशमतिक्रम्य राममाहूय भक्तितः}
{ददौ बलां चातिबलां विद्ये द्वे देवनिर्मिते} %4-24

\onelineshloka
{ययोर्ग्रहणमात्रेण क्षुत्क्षामादि न जायते} %4-25

\twolineshloka
{तत उत्तीर्य गङ्गां ते ताटकावनमागमन्}
{विश्वामित्रस्तदा प्राह रामं सत्यपराक्रमम्} %4-26

\twolineshloka
{अत्रास्ति ताटका नाम राक्षसी कामरूपिणी}
{बाधते लोकमखिलं जहि तामविचारयन्} %4-27

\twolineshloka
{तथेति धनुरादाय सगुणं रघुनन्दनः}
{टङ्कारमकरोत्तेन शब्देनापूरयद्वनम्} %4-28

\twolineshloka
{तच्छ्रुत्वाऽसहमाना सा ताटका घोररूपिणी}
{क्रोधसम्मूर्च्छिता राममभिदुद्राव मेघवत्} %4-29

\twolineshloka
{तामेकेन शरेणाशु ताडयामास वक्षसि}
{पपात विपिने घोरा वमन्ती रुधिरं बहु} %4-30

\twolineshloka
{ततोऽतिसुन्दरी यक्षी सर्वाभरणभूषिता}
{शापात्पिशाचतां प्राप्ता मुक्ता रामप्रसादतः} %4-31

\onelineshloka
{नत्वा रामं परिक्रम्य गता रामाज्ञया दिवम्} %4-32

\fourlineindentedshloka
{ततोऽतिहृष्टः परिरभ्य रामम्}
{मूर्धन्यवघ्राय विचिन्त्य किञ्चित्}
{सर्वास्त्रजालं सरहस्यमन्त्रम्}
{प्रीत्याभिरामाय ददौ मुनीन्द्रः} %4-33

{॥इति श्रीमदध्यात्मरामायणे उमामहेश्वरसंवादे
बालकाण्डे चतुर्थः सर्गः॥४॥
}
%%%%%%%%%%%%%%%%%%%%



\sect{पञ्चमः सर्गः}

\uvacha{श्रीमहादेव उवाच}

\twolineshloka
{तत्र कामाश्रमे रम्ये कानने मुनिसङ्कुले}
{उषित्वा रजनीमेकां प्रभाते प्रस्थिताः शनैः} %5-1

\twolineshloka
{सिद्धाश्रमं गताः सर्वे सिद्धचारणसेवितम्}
{विश्वामित्रेण सन्दिष्टा मुनयस्तन्निवासिनः} %5-2

\twolineshloka
{पूजां च महतीं चक्रू रामलक्ष्मणयोर्द्रुतम्}
{श्रीरामः कौशिकं प्राह मुने दीक्षां प्रविश्यताम्} %5-3

\twolineshloka
{दर्शयस्व महाभाग कुतस्तौ राक्षसाधमौ}
{तथेत्युक्त्वा मुनिर्यष्टुमारेभे मुनिभिः सह} %5-4

\twolineshloka
{मध्याह्ने ददृशाते तौ राक्षसौ कामरूपिणौ}
{मारीचश्च सुबाहुश्च वर्षन्तौ रुधिरास्थिनी} %5-5

\twolineshloka
{रामोऽपि धनुरादाय द्वौ बाणौ सन्दधे सुधीः}
{आकर्णान्तं समाकृष्य विससर्ज तयोः पृथक्} %5-6

\twolineshloka
{तयोरेकस्तु मारीचं भ्रामयञ्छतयोजनम्}
{पातयामास जलधौ तदद्भुतमिवाभवत्} %5-7

\twolineshloka
{द्वितीयोऽग्निमयो बाणः सुबाहुमजयत्क्षणात्}
{अपरे लक्षमणेनाशु हतास्तदनुयायिनः} %5-8

\twolineshloka
{पुष्पौघैराकिरन् देवा राघवं सहलक्ष्मणम्}
{देवदुन्दुभयो नेदुस्तुष्टुवुः सिद्धचारणाः} %5-9

\twolineshloka
{विश्वामित्रस्तु सम्पूज्य पूजार्हं रघुनन्दनम्}
{अङ्के निवेश्य चालिङ्ग्य भक्त्या बाष्पाकुलेक्षणः} %5-10

\twolineshloka
{भोजयित्वा सह भ्रात्रा रामं पक्वफलादिभिः}
{पुराणवाक्यैर्मधुरैर्निनाय दिवसत्रयम्} %5-11

\twolineshloka
{चतुर्थेऽहनि सम्प्राप्ते कौशिको राममब्रवीत्}
{राम राम महायज्ञं द्रष्टुं गच्छामहे वयम्} %5-12

\twolineshloka
{विदेहराजनगरे जनकस्य महात्मनः}
{तत्र माहेश्वरं चापमस्ति न्यस्तं पिनाकिना} %5-13

\twolineshloka
{द्रक्ष्यसि त्वं महासत्त्वं पूज्यसे जनकेन च}
{इत्युक्त्वा मुनिभिस्ताभ्यां ययौ गङ्गासमीपगम्} %5-14

\twolineshloka
{गौतमस्याश्रमं पुण्यं यत्राहल्याऽऽस्थिता तपः}
{दिव्यपुष्पफलोपेतपादपैः परिवेष्टितम्} %5-15

\twolineshloka
{मृगपक्षिगणैर्हीनं नानाजन्तुविवर्जितम्}
{दृष्ट्वोवाच मुनिं श्रीमान् रामो राजीवलोचनः} %5-16

\twolineshloka
{कस्यैतदाश्रमपदं भाति भास्वच्छुभं महत्}
{पत्रपुष्पफलैर्युक्तं जन्तुभिः परिवर्जितम्} %5-17

\onelineshloka
{आह्लादयति मे चेतो भगवन् ब्रूहि तत्त्वतः} %5-18

\uvacha{विश्वामित्र उवाच}

\twolineshloka
{शृणु राम पुरा वृत्तं गौतमो लोकविश्रुतः}
{सर्वधर्मभृतां श्रेष्ठस्तपसाराधयन् हरिम्} %5-19

\twolineshloka
{तस्मै ब्रह्मा ददौ कन्यामहल्यां लोकसुन्दरीम्}
{ब्रह्मचर्येण सन्तुष्टः शुश्रूषणपरायणाम्} %5-20

\twolineshloka
{तया सार्धमिहावात्सीद्गौतमस्तपतां वरः}
{शक्रस्तु तां धर्षयितुमन्तरं प्रेप्सुरन्वहम्} %5-21

\twolineshloka
{कदाचिन्मुनिवेषेण गौतमे निर्गते गृहात्}
{धर्षयित्वाऽथ निरगात्त्वरितं मुनिरप्यगात्} %5-22

\twolineshloka
{दृष्ट्वा यान्तं स्वरूपेण मुनिः परमकोपनः}
{पप्रच्छ कस्त्वं दुष्टात्मन् मम रूपधरोऽधमः} %5-23

\twolineshloka
{सत्यं ब्रूहि न चेद्भस्म करिष्यामि न संशयः}
{सोऽब्रवीद्देवराजोऽहं पाहि मां कामकिङ्करम्} %5-24

\twolineshloka
{कृतं जुगुप्सितं कर्म मया कुत्सितचेतसा}
{गौतमः क्रोधताम्राक्षः शशाप दिविजाधिपम्} %5-25

\twolineshloka
{योनिलम्पट दुष्टात्मन् सहस्रभगवान् भव}
{शप्त्वा तं देवराजानं प्रविश्य स्वाश्रमं द्रुतम्} %5-26

\twolineshloka
{दृष्ट्वाऽहल्यां वेपमानां प्राञ्जलिं गौतमोऽब्रवीत्}
{दुष्टे त्वं तिष्ठ दुर्वृत्ते शिलायामाश्रमे मम} %5-27

\twolineshloka
{निराहारा दिवारात्रं तपः परममास्थिता}
{आतपानिलवर्षादिसहिष्णुः परमेश्वरम्} %5-28

\twolineshloka
{ध्यायन्ती राममेकाग्रमनसा हृदि संस्थितम्}
{नानाजन्तुविहीनोऽयमाश्रमो मे भविष्यति} %5-29

\twolineshloka
{एवं वर्षसहस्रेषु ह्यनेकेषु गतेषु च}
{रामो दाशरथिः श्रीमानागमिष्यति सानुजः} %5-30

\twolineshloka
{यदा त्वदाश्रयशिलां पादाभ्यामाक्रमिष्यति}
{तदैव धूतपापा त्वं रामं सम्पूज्य भक्तितः} %5-31

\twolineshloka
{परिक्रम्य नमस्कृत्य स्तुत्वा शापाद्विमोक्ष्यसे}
{पूर्ववन्मम शुश्रूषां करिष्यसि यथासुखम्} %5-32

\twolineshloka
{इत्युक्त्वा गौतमः प्रागाद्धिमवन्तं नगोत्तमम्}
{तदाद्यहल्या भूतानामदृश्या स्वाश्रमे शुभे} %5-33

\twolineshloka
{तव पादरजःस्पर्शं काङ्क्षते पवनाशना}
{आस्तेऽद्यापि रघुश्रेष्ठ तपो दुष्करमास्थिता} %5-34

\twolineshloka
{पावयस्व मुनेर्भार्यामहल्यां ब्रह्मणः सुताम्}
{इत्युक्त्वा राघवं हस्ते गृहीत्वा मुनिपुङ्गवः} %5-35

\twolineshloka
{दर्शयामास चाहल्यामुग्रेण तपसा स्थिताम्}
{रामः शिलां पदा स्पृष्ट्वा तां चापश्यत्तपोधनाम्} %5-36

\twolineshloka
{ननाम राघवोऽहल्यां रामोऽहमिति चाब्रवीत्}
{ततो दृष्ट्वा रघुश्रेष्ठं पीतकौशेयवाससम्} %5-37

\twolineshloka
{चतुर्भुजं शङ्खचक्रगदापङ्कजधारिणम्}
{धनुर्बाणधरं रामं लक्ष्मणेन समन्वितम्} %5-38

\twolineshloka
{स्मितवक्त्रं पद्मनेत्रं श्रीवत्साङ्कितवक्षसम्}
{नीलमाणिक्यसङ्काशं द्योतयन्तं दिशो दश} %5-39

\twolineshloka
{दृष्ट्वा रामं रमानाथं हर्षविस्फारितेक्षणा}
{गौतमस्य वचः स्मृत्वा ज्ञात्वा नारायणं वरम्} %5-40

\twolineshloka
{सम्पूज्य विधिवद्राममर्घ्यादिभिरनिन्दिता}
{हर्षाश्रुजलनेत्रान्ता दण्डवत्प्रणिपत्य सा} %5-41

\twolineshloka
{उत्थाय च पुनर्दृष्ट्वा रामं राजीवलोचनम्}
{पुलकाङ्कितसर्वाङ्गा गिरा गद्गदयैडत} %5-42

\uvacha{अहल्योवाच}

\fourlineindentedshloka
{अहो कृतार्थास्मि जगन्निवास ते}
{पादाब्जसंलग्नरजःकणादहम्}
{स्पृशामि यत्पद्मजशङ्करादिभिर्\-}
{विमृग्यते रन्धितमानसैः सदा} %5-43

\fourlineindentedshloka
{अहो विचित्रं तव राम चेष्टितम्}
{मनुष्यभावेन विमोहितं जगत्}
{चलस्यजस्रं चरणादिवर्जितः}
{सम्पूर्ण आनन्दमयोऽतिमायिकः} %5-44

\fourlineindentedshloka
{यत्पादपङ्कजपरागपवित्रगात्रा}
{भागीरथी भवविरिञ्चिमुखान् पुनाति}
{साक्षात्स एव मम दृग्विषयो यदास्ते}
{किं वर्ण्यते मम पुराकृतभागधेयम्} %5-45

\fourlineindentedshloka
{मर्त्यावतारे मनुजाकृतिं हरिम्}
{रामाभिधेयं रमणीयदेहिनम्}
{धनुर्धरं पद्मविशाललोचनम्}
{भजामि नित्यं न परान् भजिष्ये} %5-46

\fourlineindentedshloka
{यत्पादपङ्कजरजः श्रुतिभिर्विमृग्यम्}
{यन्नाभिपङ्कजभवः कमलासनश्च}
{यन्नामसाररसिको भगवान् पुरारिः}
{तं रामचन्द्रमनिशं हृदि भावयामि} %5-47

\fourlineindentedshloka
{यस्यावतारचरितानि विरिञ्चिलोके}
{गायन्ति नारदमुखा भवपद्मजाद्याः}
{आनन्दजाश्रुपरिषिक्तकुचाग्रसीमा}
{वागीश्वरी च तमहं शरणं प्रपद्ये} %5-48

\fourlineindentedshloka
{सोऽयं परात्मा पुरुषः पुराण}
{एकः स्वयञ्ज्योतिरनन्त आद्यः}
{मायातनुं लोकविमोहनीयाम्}
{धत्ते परानुग्रह एष रामः} %5-49

\fourlineindentedshloka
{अयं हि विश्वोद्भवसंयमानाम्}
{एकः स्वमायागुणबिम्बितो यः}
{विरिञ्चिविष्ण्वीश्वरनामभेदान्}
{धत्ते स्वतन्त्रः परिपूर्ण आत्मा} %5-50

\fourlineindentedshloka
{नमोऽस्तु ते राम तवाङ्घ्रिपङ्कजम्}
{श्रिया धृतं वक्षसि लालितं प्रियात्}
{आक्रान्तमेकेन जगत्त्रयं पुरा}
{ध्येयं मुनीन्द्रैरभिमानवर्जितैः} %5-51

\twolineshloka
{जगतामादिभूतस्त्वं जगत्त्वं जगदाश्रयः}
{सर्वभूतेष्वसंयुक्त एको भाति भवान् परः} %5-52

\twolineshloka
{ओङ्कारवाच्यस्त्वं राम वाचामविषयः पुमान्}
{वाच्यवाचकभेदेन भवानेव जगन्मयः} %5-53

\twolineshloka
{कार्यकारणकर्तृत्वफलसाधनभेदतः}
{एको विभासि राम त्वं मायया बहुरूपया} %5-54

\twolineshloka
{त्वन्मायामोहितधियस्त्वां न जानन्ति तत्त्वतः}
{मानुषं त्वाऽभिमन्यन्ते मायिनं परमेश्वरम्} %5-55

\twolineshloka
{आकाशवत्त्वं सर्वत्र बहिरन्तर्गतोऽमलः}
{असङ्गो ह्यचलो नित्यः शुद्धो बुद्धः सदव्ययः} %5-56

\twolineshloka
{योषिन्मूढाहमज्ञा ते तत्त्वं जाने कथं विभो}
{तस्मात्ते शतशो राम नमस्कुर्यामनन्यधीः} %5-57

\twolineshloka
{देव मे यत्र कुत्रापि स्थिताया अपि सर्वदा}
{त्वत्पादकमले सक्ता भक्तिरेव सदास्तु मे} %5-58

\twolineshloka
{नमस्ते पुरुषाध्यक्ष नमस्ते भक्तवत्सल}
{नमस्तेऽस्तु हृषीकेश नारायण नमोऽस्तुते} %5-59

\fourlineindentedshloka
{भवभयहरमेकं भानुकोटिप्रकाशम्}
{करधृतशरचापं कालमेघावभासम्}
{कनकरुचिरवस्त्रं रत्नवत्कुण्डलाढ्यम्}
{कमलविशदनेत्रं सानुजं राममीडे} %5-60

\twolineshloka
{स्तुत्वैवं पुरुषं साक्षाद्राघवं पुरतः स्थितम्}
{परिक्रम्य प्रणम्याऽऽशु साऽनुज्ञाता ययौ पतिम्} %5-61

\twolineshloka
{अहल्यया कृतं स्तोत्रं यः पठेद्भक्तिसंयुतः}
{स मुच्यतेऽखिलैः पापैः परं ब्रह्माधिगच्छति} %5-62

\twolineshloka
{पुत्राद्यर्थे पठेद्भक्त्या रामं हृदि निधाय च}
{संवत्सरेण लभते वन्ध्या अपि सुपुत्रकम्} %5-63

{सर्वान् कामानवाप्नोति रामचन्द्रप्रसादतः॥६४॥} %5-64
\refstepcounter{shlokacount}


\fourlineindentedshloka
{ब्रह्मघ्नो गुरुतल्पगोऽपि पुरुषः स्तेयी सुरापोऽपि वा}
{मातृभ्रातृविहिंसकोऽपि सततं भोगैकबद्धातुरः}
{नित्यं स्तोत्रमिदं जपन् रघुपतिं भक्त्या हृदिस्थं स्मरन्}
{ध्यायन्मुक्तिमुपैति किं पुनरसौ स्वाचारयुक्तो नरः} %5-65

{॥इति श्रीमदध्यात्मरामायणे उमामहेश्वरसंवादे बालकाण्डे
अहल्योद्धरणं नाम पञ्चमः सर्गः॥५॥
}
%%%%%%%%%%%%%%%%%%%%



\sect{षष्ठः सर्गः}

\twolineshloka
{विश्वामित्रोऽथ तं प्राह राघवं सहलक्ष्मणम्}
{गच्छामो वत्स मिथिलां जनकेनाभिपालिताम्} %6-1

\threelineshloka
{दृष्ट्वा क्रतुवरं पश्चादयोध्यां गन्तुमर्हसि}
{इत्युक्त्वा प्रययौ गङ्गामुत्तर्तुं सहराघवः}
{तस्मिन् काले नाविकेन निषिद्धो रघुनन्दनः} %6-2

\uvacha{नाविक उवाच}

\fourlineindentedshloka
{क्षालयामि तव पादपङ्कजम्}
{नाथ दारुदृषदोः किमन्तरम्}
{मानुषीकरणचूर्णमस्ति ते}
{पादयोरिति कथा प्रथीयसी} %6-3

\fourlineindentedshloka
{पादाम्बुजं ते विमलं हि कृत्वा}
{पश्चात्परं तीरमहं नयामि}
{नो चेत्तरी सद्युवती मलेन}
{स्याच्चेद्विभो विद्धि कुटुम्बहानिः} %6-4

\twolineshloka
{इत्युक्त्वा क्षालितौ पादौ परं तीरं ततो गताः}
{कौशिको रघुनाथेन सहितो मिथिलां ययौ} %6-5

\twolineshloka
{विदेहस्य पुरं प्रातरृषिवाटं समाविशत्}
{प्राप्तं कौशिकमाकर्ण्य जनकोऽतिमुदान्वितः} %6-6

\twolineshloka
{पूजाद्रव्याणि सङ्गृह्य सोपाध्यायः समाययौ}
{दण्डवत्प्रणिपत्याथ पूजयामास कौशिकम्} %6-7

\twolineshloka
{पप्रच्छ राघवौ दृष्ट्वा सर्वलक्षणसंयुतौ}
{द्योतयन्तौ दिशः सर्वाश्चन्द्रसूर्याविवापरौ} %6-8

\twolineshloka
{कस्यैतौ नरशार्दूलौ पुत्रौ देवसुतोपमौ}
{मनःप्रीतिकरौ मेऽद्य नरनारायणाविव} %6-9

\twolineshloka
{प्रत्युवाच मुनिः प्रीतो हर्षयन् जनकं तदा}
{पुत्रौ दशरथस्यैतौ भ्रातरौ रामलक्ष्मणौ} %6-10

\twolineshloka
{मखसंरक्षणार्थाय मयाऽऽनीतौ पितुः पुरात्}
{आगच्छन् राघवो मार्गे ताटकां विश्वघातिनीम्} %6-11

\twolineshloka
{शरेणैकेन हतवान्नोदितो मेऽतिविक्रमः}
{ततो ममाश्रमं गत्वा मम यज्ञविहिंसकान्} %6-12

\twolineshloka
{सुबाहुप्रमुखान् हत्वा मारीचं सागरेऽक्षिपत्}
{ततो गङ्गातटे पुण्ये गौतमस्याश्रमं शुभम्} %6-13

\twolineshloka
{गत्वा तत्र शिलारूपा गौतमस्य वधूः स्थिता}
{पादपङ्कजसंस्पर्शात्कृता मानुषरूपिणी} %6-14

\twolineshloka
{दृष्ट्वाऽहल्यां नमस्कृत्य तया सम्यक्प्रपूजितः}
{इदानीं द्रष्टुकामस्ते गृहे माहेश्वरं धनुः} %6-15

\threelineshloka
{पूजितं राजभिः सर्वैर्दृष्टमित्यनुशुश्रुवे}
{अतो दर्शय राजेन्द्र शैवं चापमनुत्तमम्}
{दृष्ट्वाऽयोध्यां जिगमिषुः पितरं द्रष्टुमिच्छति} %6-16

\threelineshloka
{इत्युक्तो मुनिना राजा पूजार्हाविति पूजया}
{पूजयामास धर्मज्ञो विधिदृष्टेन कर्मणा}
{ततः सम्प्रेषयामास मन्त्रिणं बुद्धिमत्तरम्} %6-17

\uvacha{जनक उवाच}

{शीघ्रमानय विश्वेशचापं रामाय दर्शय॥१८॥} %6-18
\refstepcounter{shlokacount}


\twolineshloka
{ततो गते मन्त्रिवरे राजा कौशिकमब्रवीत्}
{यदि रामो धनुर्धृत्वा कोट्यामारोपयेद्गुणम्} %6-19

\twolineshloka
{तदा मयाऽऽत्मजा सीता दीयते राघवाय हि}
{तथेति कौशिकोऽप्याह रामं संवीक्ष्य सस्मितम्} %6-20

\twolineshloka
{शीघ्रं दर्शय चापाग्र्यं रामायामिततेजसे}
{एवं ब्रुवति मौनीशे आगताश्चापवाहकाः} %6-21

\twolineshloka
{चापं गृहीत्वा बलिनः पञ्चसाहस्रसङ्ख्यकाः}
{घण्टाशतसमायुक्तं मणिवज्रादिभूषितम्} %6-22

\twolineshloka
{दर्शयामास रामाय मन्त्री मन्त्रयतां वरः}
{दृष्ट्वा रामः प्रहृष्टात्मा बद्ध्वा परिकरं दृढम्} %6-23

\twolineshloka
{गृहीत्वा वामहस्तेन लीलया तोलयन् धनुः}
{आरोपयामास गुणं पश्यत्स्वखिलराजसु} %6-24

\twolineshloka
{ईषदाकर्षयामास पाणिना दक्षिणेन सः}
{बभञ्जाखिलहृत्सारो दिशः शब्देन पूरयन्} %6-25

\twolineshloka
{दिशश्च विदिशश्चैव स्वर्गं मर्त्यं रसातलम्}
{तदद्भुतमभूत्तत्र देवानां दिवि पश्यताम्} %6-26

\twolineshloka
{आच्छादयन्तः कुसुमैर्देवाः स्तुतिभिरीडिरे}
{देवदुन्दुभयो नेदुर्ननृतुश्चाप्सरोगणाः} %6-27

\twolineshloka
{द्विधा भग्नं धनुर्दृष्ट्वा राजालिङ्ग्य रघूद्वहम्}
{विस्मयं लेभिरे सीतामातरोऽन्तःपुराजिरे} %6-28

\twolineshloka
{सीता स्वर्णमयीं मालां गृहीत्वा दक्षिणे करे}
{स्मितवक्त्रा स्वर्णवर्णा सर्वाभरणभूषिता} %6-29

\twolineshloka
{मुक्ताहारैः कर्णपत्रैः क्वणच्चरणनूपुरा}
{दुकूलपरिसंवीता वस्त्रान्तर्व्यञ्जितस्तनी} %6-30

\twolineshloka
{रामस्योपरि निक्षिप्य स्मयमाना मुदं ययौ}
{ततो मुमुदिरे सर्वे राजदाराः स्वलङ्कृतम्} %6-31

\twolineshloka
{गवाक्षजालरन्ध्रेभ्यो दृष्ट्वा लोकविमोहनम्}
{ततोऽब्रवीन्मुनिं राजा सर्वशास्त्रविशारदः} %6-32

\twolineshloka
{भो कौशिक मुनिश्रेष्ठ पत्रं प्रेषय सत्वरम्}
{राजा दशरथः शीघ्रमागच्छतु सपुत्रकः} %6-33

\twolineshloka
{विवाहार्थं कुमाराणां सदारः सहमन्त्रिभिः}
{तथेति प्रेषयामास दूतान्स्त्वरितविक्रमान्} %6-34

\twolineshloka
{ते गत्वा राजशार्दूलं रामश्रेयो न्यवेदयन्}
{श्रुत्वा रामकृतं राजा हर्षेण महताऽऽप्लुतः} %6-35

\twolineshloka
{मिथिलागमनार्थाय त्वरयामास मन्त्रिभिः}
{गच्छन्तु मिथिलां सर्वे गजाश्वरथपत्तयः} %6-36

\twolineshloka
{रथमानय मे शीघ्रं गच्छाम्यद्यैव मा चिरम्}
{वसिष्ठस्त्वग्रतो यातु सदारः सहितोऽग्निभिः} %6-37

\twolineshloka
{राममातॄः समादाय मुनिर्मे भगवान् गुरुः}
{एवं प्रस्थाप्य सकलं राजर्षिर्विपुलं रथम्} %6-38

\twolineshloka
{महत्या सेनया सार्धमारुह्य त्वरितो ययौ}
{आगतं राघवं श्रुत्वा राजा हर्षसमाकुलः} %6-39

\twolineshloka
{प्रत्युज्जगाम जनकः शतानन्दपुरोधसा}
{यथोक्तपूजया पूज्यं पूजयामास सत्कृतम्} %6-40

\twolineshloka
{रामस्तु लक्ष्मणेनाशु ववन्दे चरणौ पितुः}
{ततो हृष्टो दशरथो रामं वचनमब्रवीत्} %6-41

\twolineshloka
{दिष्ट्या पश्यामि ते राम मुखं फुल्लाम्बुजोपमम्}
{मुनेरनुग्रहात्सर्वं सम्पन्नं मम शोभनम्} %6-42

\twolineshloka
{इत्युक्त्वाऽऽघ्राय मूर्धानमालिङ्ग्य च पुनः पुनः}
{हर्षेण महताऽऽविष्टो ब्रह्मानन्दं गतो यथा} %6-43

\twolineshloka
{ततो जनकराजेन मन्दिरे सन्निवेशितः}
{शोभने सर्वभोगाढ्ये सदारः ससुतः सुखी} %6-44

\twolineshloka
{ततः शुभे दिने लग्ने सुमुहूर्ते रघूत्तमम्}
{आनयामास धर्मज्ञो रामं सभ्रातृकं तदा} %6-45

\twolineshloka
{रत्नस्तम्भसुविस्तारे सुविताने सुतोरणे}
{मण्डपे सर्वशोभाढ्ये मुक्तापुष्पफलान्विते} %6-46

\twolineshloka
{वेदविद्भिः सुसम्बाधे ब्राह्मणैः स्वर्णभूषितैः}
{सुवासिनीभिः परितो निष्ककण्ठीभिरावृते} %6-47

\twolineshloka
{भेरीदुन्दुभिनिर्घोषैर्गीतनृत्यैः समाकुले}
{दिव्यरत्नाञ्चिते स्वर्णपीठे रामं न्यवेशयत्} %6-48

\twolineshloka
{वसिष्ठं कौशिकं चैव शतानन्दः पुरोहितः}
{यथाक्रमं पूजयित्वा रामस्योभयपार्श्वयोः} %6-49

\twolineshloka
{स्थापयित्वा स तत्राग्निं ज्वालयित्वा यथाविधि}
{सीतामानीय शोभाढ्यां नानारत्नविभूषिताम्} %6-50

\twolineshloka
{सभार्यो जनकः प्रायाद्रामं राजीवलोचनम्}
{पादौ प्रक्षाल्य विधिवत्तदपो मूर्ध्न्यधारयत्} %6-51

\twolineshloka
{या धृता मूर्ध्नि शर्वेण ब्रह्मणा मुनिभिः सदा}
{ततः सीतां करे धृत्वा साक्षतोदकपूर्वकम्} %6-52

\twolineshloka
{रामाय प्रददौ प्रीत्या पाणिग्रहविधानतः}
{सीता कमलपत्राक्षी स्वर्णमुक्तादिभूषिता} %6-53

\twolineshloka
{दीयते मे सुता तुभ्यं प्रीतो भव रघूत्तम}
{इति प्रीतेन मनसा सीतां रामकरेऽर्पयन्} %6-54

\twolineshloka
{मुमोद जनको लक्ष्मीं क्षीराब्धिरिव विष्णवे}
{उर्मिलां चौरसीं कन्यां लक्ष्मणाय ददौ मुदा} %6-55

\twolineshloka
{तथैव श्रुतिकीर्तिं च माण्डवीं भ्रातृकन्यके}
{भरताय ददावेकां शत्रुघ्नायापरां ददौ} %6-56

\twolineshloka
{चत्वारो दारसम्पन्ना भ्रातरः शुभलक्षणाः}
{विरेजुः प्रजया सर्वे लोकपाला इवापरे} %6-57

\twolineshloka
{ततोऽब्रवीद्वसिष्ठाय विश्वामित्राय मैथिलः}
{जनकः स्वसुतोदन्तं नारदेनाभिभाषितम्} %6-58

\twolineshloka
{यज्ञभूमिविशुद्ध्यर्थं कर्षतो लाङ्गलेन मे}
{सीतामुखात्समुत्पन्ना कन्यका शुभलक्षणा} %6-59

\twolineshloka
{तामद्राक्षमहं प्रीत्या पुत्रिकाभावभाविताम्}
{अर्पिता प्रियभार्यायै शरच्चन्द्रनिभानना} %6-60

\twolineshloka
{एकदा नारदोऽभ्यागाद्विविक्ते मयि संस्थिते}
{रणयन्महतीम् वीणां गायन्नारायणं विभुम्} %6-61

\twolineshloka
{पूजितः सुखमासीनो मामुवाच सुखान्वितः}
{शृणुष्व वचनं गुह्यं तवाभ्युदयकारणम्} %6-62

\twolineshloka
{परमात्मा हृषीकेशो भक्तानुग्रहकाम्यया}
{देवकार्यार्थसिद्ध्यर्थं रावणस्य वधाय च} %6-63

\twolineshloka
{जातो राम इति ख्यातो मायामानुषवेषधृक्}
{आस्ते दाशरथिर्भूत्वा चतुर्धा परमेश्वरः} %6-64

\twolineshloka
{योगमायाऽपि सीतेति जाता वै तव वेश्मनि}
{अतस्त्वं राघवायैव देहि सीतां प्रयत्नतः} %6-65

\twolineshloka
{नान्येभ्यः पूर्वभार्यैषा रामस्य परमात्मनः}
{इत्युक्त्वा प्रययौ देवगतिं देवमुनिस्तदा} %6-66

\twolineshloka
{तदारभ्य मया सीता विष्णोर्लक्ष्मीर्विभाव्यते}
{कथं मया राघवाय दीयते जानकी शुभा} %6-67

\twolineshloka
{इति चिन्तासमाविष्टः कार्यमेकमचिन्तयम्}
{मत्पितामहगेहे तु न्यासभूतमिदं धनुः} %6-68

\twolineshloka
{ईश्वरेण पुरा क्षिप्तं पुरदाहादनन्तरम्}
{धनुरेतत्पणं कार्यमिति चिन्त्य कृतं तथा} %6-69

\twolineshloka
{सीतापाणिग्रहार्थाय सर्वेषां माननाशनम्}
{त्वत्प्रसादान्मुनिश्रेष्ठ रामो राजीवलोचनः} %6-70

\twolineshloka
{आगतोऽत्र धनुर्द्रष्टुं फलितो मे मनोरथः}
{अद्य मे सफलं जन्म राम त्वां सह सीतया} %6-71

{एकासनस्थं पश्यामि भ्राजमानं रविं यथा॥७२॥} %6-72
\refstepcounter{shlokacount}


\twolineshloka
{त्वत्पादाम्बुधरो ब्रह्मा सृष्टिचक्रप्रवर्तकः}
{बलिस्त्वत्पादसलिलं धृत्वाऽभूद्दिविजाधिपः} %6-73

\twolineshloka
{त्वत्पादपांसुसंस्पर्शादहल्या भर्तृशापतः}
{सद्य एव विनिर्मुक्ता कोऽन्यस्त्वत्तोऽधिरक्षिता} %6-74

\fourlineindentedshloka
{यत्पादपङ्कजपरागसुरागयोगि-}
{वृन्दैर्जितं भवभयं जितकालचक्रैः}
{यन्नामकीर्तनपरा जितदुःखशोका}
{देवास्तमेव शरणं सततं प्रपद्ये} %6-75

\twolineshloka
{इति स्तुत्वा नृपः प्रादाद्राघवाय महात्मने}
{दीनाराणां कोटिशतं रथानामयुतं तदा} %6-76

\twolineshloka
{अश्वानां नियुतं प्रादाद्गजानां षट्शतं तथा}
{पत्तीनां लक्षमेकं तु दासीनां त्रिशतं ददौ} %6-77

\twolineshloka
{दिव्याम्बराणि हारान्श्च मुक्तारत्नमयोज्ज्वलान्}
{सीतायै जनकः प्रादात्प्रीत्या दुहितृवत्सलः} %6-78

\twolineshloka
{वसिष्ठादीन् सुसम्पूज्य भरतं लक्ष्मणं तथा}
{पूजयित्वा यथान्यायं तथा दशरथं नृपम्} %6-79

\twolineshloka
{प्रस्थापयामास नृपो राजानं रघुसत्तमम्}
{सीतामालिङ्ग्य रुदतीं मातरः साश्रुलोचनाः} %6-80

\twolineshloka
{श्वश्रूशुश्रूषणपरा नित्यं राममनुव्रता}
{पातिव्रत्यमुपालम्ब्य तिष्ठ वत्से यथा सुखम्} %6-81

\twolineshloka
{प्रयाणकाले रघुनन्दनस्य भेरीमृदङ्गानकतूर्यघोषः}
{स्वर्वासिभेरीघनतूर्यशब्दैः सम्मूर्च्छितो भूतभयङ्करोऽभूत्} %6-82

{॥इति श्रीमदध्यात्मरामायणे उमामहेश्वरसंवादे
बालकाण्डे षष्ठः सर्गः॥६॥
}
%%%%%%%%%%%%%%%%%%%%



\sect{सप्तमः सर्गः}

\twolineshloka
{अथ गच्छति श्रीरामे मैथिलाद्योजनत्रयम्}
{निमित्तान्यतिघोराणि ददर्श नृपसत्तमः} %7-1

\twolineshloka
{नत्वा वसिष्ठं पप्रच्छ किमिदं मुनिपुङ्गव}
{निमित्तानीह दृश्यन्ते विषमाणि समन्ततः} %7-2

\twolineshloka
{वसिष्ठस्तमथ प्राह भयमागामि सूच्यते}
{पुनरप्यभयं तेऽद्य शीघ्रमेव भविष्यति} %7-3

\twolineshloka
{मृगाः प्रदक्षिणं यान्ति पश्य त्वां शुभसूचकाः}
{इत्येवं वदतस्तस्य ववौ घोरतरोऽनिलः} %7-4

\twolineshloka
{मुष्णन्श्चक्षूंषि सर्वेषां पांसुवृष्टिभिरर्दयन्}
{ततो व्रजन् ददर्शाग्रे तेजोराशिमुपस्थितम्} %7-5

\twolineshloka
{कोटिसूर्यप्रतीकाशं विद्युत्पुञ्जसमप्रभम्}
{तेजोराशिं ददर्शाथ जामदग्न्यं प्रतापवान्} %7-6

\twolineshloka
{नीलमेघनिभं प्रांशुं जटामण्डलमण्डितम्}
{धनुः परशुपाणिं च साक्षात्कालमिवान्तकम्} %7-7

\twolineshloka
{कार्तवीर्यान्तकं रामं दृप्तक्षत्रियमर्दनम्}
{प्राप्तं दशरथस्याग्रे कालमृत्युमिवापरम्} %7-8

\twolineshloka
{तं दृष्ट्वा भयसन्त्रस्तो राजा दशरथस्तदा}
{अर्घ्यादिपूजां विस्मृत्य त्राहि त्राहीति चाब्रवीत्} %7-9

\twolineshloka
{दण्डवत्प्रणिपत्याह पुत्रप्राणं प्रयच्छ मे}
{इति ब्रुवन्तं राजानमनादृत्य रघूत्तमम्} %7-10

\twolineshloka
{उवाच निष्ठुरं वाक्यं क्रोधात्प्रचलितेन्द्रियः}
{त्वं राम इति नाम्ना मे चरसि क्षत्रियाधम} %7-11

\twolineshloka
{द्वन्द्वयुद्धं प्रयच्छाशु यदि त्वं क्षत्रियोऽसि वै}
{पुराणं जर्जरं चापं भङ्क्त्वा त्वं कत्थसे मुधा} %7-12

\twolineshloka
{अस्मिन्स्तु वैष्णवे चापे आरोपयसि चेद्गुणम्}
{तदा युद्धं त्वया सार्धं करोमि रघुवंशज} %7-13

\twolineshloka
{नो चेत्सर्वान् हनिष्यामि क्षत्रियान्तकरो ह्यहम्}
{इति ब्रुवति वै तस्मिन्श्चचाल वसुधा भृशम्} %7-14

\twolineshloka
{अन्धकारो बभूवाथ सर्वेषामपि चक्षुषाम्}
{रामो दाशरथिर्वीरो वीक्ष्य तं भार्गवं रुषा} %7-15

\twolineshloka
{धनुराच्छिद्य तद्धस्तादारोप्य गुणमञ्जसा}
{तूणीराद्बाणमादाय सन्धायाकृष्य वीर्यवान्} %7-16

\twolineshloka
{उवाच भार्गवं रामं शृणु ब्रह्मन् वचो मम}
{लक्ष्यं दर्शय बाणस्य ह्यमोघो मम सायकः} %7-17

\twolineshloka
{लोकान् पादयुगं वाऽपि वद शीघ्रं ममाऽऽज्ञया}
{अयं लोकः परो वाथ त्वया गन्तुं न शक्यते} %7-18

\twolineshloka
{एवं त्वं हि प्रकर्तव्यं वद शीघ्रं ममऽऽज्ञया}
{एवं वदति श्रीरामे भार्गवो विकृताननः} %7-19

\twolineshloka
{संस्मरन् पूर्ववृत्तान्तमिदं वचनमब्रवीत्}
{राम राम महाबाहो जाने त्वां परमेश्वरम्} %7-20

\twolineshloka
{पुराणपुरुषं विष्णुं जगत्सर्गलयोद्भवम्}
{बाल्येऽहं तपसा विष्णुमाराधयितुमञ्जसा} %7-21

\twolineshloka
{चक्रतीर्थं शुभं गत्वा तपसा विष्णुमन्वहम्}
{अतोषयं महात्मानं नारायणमनन्यधीः} %7-22

\twolineshloka
{ततः प्रसन्नो देवेशः शङ्खचक्रगदाधरः}
{उवाच मां रघुश्रेष्ठ प्रसन्नमुखपङ्कजः} %7-23

\uvacha{श्रीभगवानुवाच}

\twolineshloka
{उत्तिष्ठ तपसो ब्रह्मन् फलितं ते तपो महत्}
{मच्चिदंशेन युक्तस्त्वं जहि हैहयपुङ्गवम्} %7-24

\twolineshloka
{कार्तवीर्यं पितृहणं यदर्थं तपसः श्रमः}
{ततस्त्रिःसप्तकृत्वस्त्वं हत्वा क्षत्रियमण्डलम्} %7-25

\twolineshloka
{कृत्स्नां भूमिं कश्यपाय दत्त्वा शान्तिमुपावह}
{त्रेतामुखे दाशरथिर्भूत्वा रामोऽहमव्ययः} %7-26

\twolineshloka
{उत्पत्स्ये परया शक्त्या तदा द्रक्ष्यसि मां ततः}
{मत्तेजः पुनरादास्ये त्वयि दत्तं मया पुरा} %7-27

\twolineshloka
{तदा तपश्चरन्ल्लोके तिष्ठ त्वं ब्रह्मणो दिनम्}
{इत्युक्त्वाऽन्तर्दधे देवस्तथा सर्वं कृतं मया} %7-28

\twolineshloka
{स एव विष्णुस्त्वं राम जातोऽसि ब्रह्मणार्थितः}
{मयि स्थितं तु त्वत्तेजस्त्वयैव पुनराहृतम्} %7-29

\twolineshloka
{अद्य मे सफलं जन्म प्रतीतोऽसि मम प्रभो}
{ब्रह्मादिभिरलभ्यस्त्वं प्रकृतेः पारगो मतः} %7-30

\twolineshloka
{त्वयि जन्मादिषड्भावा न सन्त्यज्ञानसम्भवाः}
{निर्विकारोऽसि पूर्णस्त्वं गमनादिविवर्जितः} %7-31

\twolineshloka
{यथा जले फेनजालं धूमो वह्नौ तथा त्वयि}
{त्वदाधारा त्वद्विषया माया कार्यं सृजत्यहो} %7-32

\twolineshloka
{यावन्मायावृता लोकास्तावत्त्वां न विजानते}
{अविचारितसिद्धैषाऽविद्या विद्याविरोधिनी} %7-33

\twolineshloka
{अविद्याकृतदेहादिसङ्घाते प्रतिबिम्बिता}
{चिच्छक्तिर्जीवलोकेऽस्मिन् जीव इत्यभिधीयते} %7-34

\twolineshloka
{यावद्देहमनःप्राणबुद्ध्यादिष्वभिमानवान्}
{तावत्कर्तृत्वभोक्तृत्वसुखदुःखादिभाग्भवेत्} %7-35

\twolineshloka
{आत्मनःसंसृतिर्नास्ति बुद्धेर्ज्ञानं न जात्विति}
{अविवेकाद्द्वयं युङ्क्त्वा संसारीति प्रवर्तते} %7-36

\twolineshloka
{जडस्य चित्समायोगाच्चित्त्वं भूयाच्चितेस्तथा}
{जडसङ्गाज्जडत्वं हि जलाग्न्योर्मेलनं यथा} %7-37

\twolineshloka
{यावत्त्वत्पादभक्तानां सङ्गसौख्यं न विन्दति}
{तावत्संसारदुःखौघान्न निवर्तेन्नरः सदा} %7-38

\twolineshloka
{तत्सङ्गलब्धया भक्त्या यदा त्वां समुपासते}
{तदा माया शनैर्याति तानवं प्रतिपद्यते} %7-39

\twolineshloka
{ततस्त्वज्ज्ञानसम्पन्नः सद्गुरुस्तेन लभ्यते}
{वाक्यज्ञानं गुरोर्लब्ध्वा त्वत्प्रसादाद्विमुच्यते} %7-40

\twolineshloka
{तस्मात्त्वद्भक्तिहीनानां कल्पकोटिशतैरपि}
{न मुक्तिशङ्का विज्ञानशङ्का नैव सुखं तथा} %7-41

\twolineshloka
{अतस्त्वत्पादयुगले भक्तिर्मे जन्मजन्मनि}
{स्यात्त्वद्भक्तिमतां सङ्गोऽविद्या याभ्यां विनश्यति} %7-42

\twolineshloka
{लोके त्वद्भक्तिनिरतास्त्वद्धर्मामृतवर्षिणः}
{पुनन्ति लोकमखिलं किं पुनः स्वकुलोद्भवान्} %7-43

\twolineshloka
{नमोऽस्तु जगतां नाथ नमस्ते भक्तिभावन}
{नमः कारुणिकानन्त रामचन्द्र नमोऽस्तु ते} %7-44

\twolineshloka
{देव यद्यत्कृतं पुण्यं मया लोकजिगीषया}
{तत्सर्वं तव बाणाय भूयाद्राम नमोऽस्तु ते} %7-45

\twolineshloka
{ततः प्रसन्नो भगवान् श्रीरामः करुणामयः}
{प्रसन्नोऽस्मि तव ब्रह्मन् यत्ते मनसि वर्तते} %7-46

\twolineshloka
{दास्ये तदखिलं कामं मा कुरुष्वात्र संशयम्}
{ततः प्रीतेन मनसा भार्गवो राममब्रवीत्} %7-47

\twolineshloka
{यदि मेऽनुग्रहो राम तवास्ति मधुसूदन}
{त्वद्भक्तसङ्गस्त्वत्पादे दृढा भक्तिः सदास्तु मे} %7-48

\twolineshloka
{स्तोत्रमेतत्पठेद्यस्तु भक्तिहीनोऽपि सर्वदा}
{त्वद्भक्तिस्तस्य विज्ञानं भूयादन्ते स्मृतिस्तव} %7-49

\twolineshloka
{तथेति राघवेणोक्तः परिक्रम्य प्रणम्य तम्}
{पूजितस्तदनुज्ञातो महेन्द्राचलमन्वगात्} %7-50

\twolineshloka
{राजा दशरथो हृष्टो रामं मृतमिवागतम्}
{आलिङ्ग्यालिङ्ग्य हर्षेण नेत्राभ्यां जलमुत्सृजत्} %7-51

\twolineshloka
{ततः प्रीतेन मनसा स्वस्थचित्तः पुरं ययौ}
{रामलक्ष्मणशत्रुघ्नभरता देवसम्मिताः} %7-52

\threelineshloka
{स्वां स्वां भार्यामुपादाय रेमिरे स्वस्वमन्दिरे}
{मातापितृभ्यां संहृष्टो रामः सीतासमन्वितः}
{रेमे वैकुण्ठभवने श्रिया सह यथा हरिः} %7-53

\twolineshloka
{युधाजिन्नाम कैकेयीभ्राता भरतमातुलः}
{भरतं नेतुमागच्छत्स्वराज्यं प्रीतिसंयुतः} %7-54

\twolineshloka
{प्रेषयामास भरतं राजा स्नेहसमन्वितः}
{शत्रुघ्नं चापि सम्पूज्य युधाजितमरिन्दमः} %7-55

\twolineshloka
{कौसल्या शुशुभे देवी रामेण सह सीतया}
{देवमातेव पौलोम्या शच्या शक्रेण शोभना} %7-56

\fourlineindentedshloka
{साकेते लोकनाथप्रथितगुणगणो लोकसङ्गीतकीर्तिः}
{श्रीरामः सीतयास्तेऽखिलजननिकरानन्दसन्दोहमूर्तिः}
{नित्यश्रीर्निर्विकारो निरवधिविभवो नित्यमायानिरासो}
{मायाकार्यानुसारी मनुज इव सदा भाति देवोऽखिलेशः} %7-57

{॥इति श्रीमदध्यात्मरामायणे उमामहेश्वरसंवादे
बालकाण्डे सप्तमः सर्गः॥७॥
}
%%%%%%%%%%%%%%%%%%%%

इति श्रीमदध्यात्मरामायणे बालकाण्डः समाप्तः॥